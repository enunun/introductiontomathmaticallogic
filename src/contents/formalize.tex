\chapter{論理の形式化}
\label[chapter]{chap:formulize}

数理論理学が数学の一分野として成功を収めた要因のひとつとして,
素朴的直観を伴わない形式的な記号列を主役に据えたことが挙げられる.
ともすれば,このことは「数学で用いる論理について研究する分野」
という一般的な認識と矛盾するように見えることだろう.
実際,この「形式的な記号列」に関する議論の結果を根拠に
我々が普段使っている「論理」について何かを主張したい場合,
これらの間の橋渡しを行うのは主張したい当人の責任であり,
数理論理学の諸定理がその橋渡しについて何かを保証してくれることはない.
これは,理論物理学で得られた結果から現実世界について言及したり,
統計モデルの性質をもとに現実で得られたデータ(あるいはその生成元)に
ついて言及したりする営みに非常によく似ている.

もし読者にこのような「それそのものではないが何らかの視点で
類似性をもつと期待される概念の性質をもとに,目的の対象について議論する」
という営みに親しみがあれば,本章の内容は極めて身近に思えるに違いない.
そして,この手法が科学においていかに強力であるかを知っていれば,
数理論理学の手法がいかに強力であるかも予想できるだろう.


\section{数学における論理}
\label[section]{sec:logic}

数学が世界的に広く学ばれていることが示すように,
数学(および算数)を学ぶことは有益であるというのが一般的な認識である.
「なぜ数学を学ぶのか」という問いの答えは個々人によってさまざまだろうが,
少なくとも日本の数学教育学の分野では,
数学教育の目的は
\begin{enumerate}
	\item 陶冶的目的
	\item 実用的目的
	\item 文化的目的
\end{enumerate}
の3つの観点から論じられるのが一般的である%
\footnote{%
	念のため述べておくが,これは観点別評価とはまったく別の話である.
}%
.
このうち,実用的目的と文化的目的については字面から容易に想像できる通りの意味である.
すなわち,実用的目的は教科教育の内容そのものの修得と実社会における活用を志向したものであり,
文化的目的は文化としての学問を継承・発展させていくことを志向したものである.

一方で,1つ目の陶冶的目的についてはやや聞きなれない言葉である.
ここでの「陶冶」という言葉には「人の性質や能力を円満に育てること」という意味である.
すなわち,陶冶的目的というのは人間形成や価値観・(教科教育の内容以外での)能力を養成することを志向したものである.
具体例を挙げていけばキリがないが,「論理的思考力の養成」という目的は数学教育に明るくない人からでも
頻繁に挙がるものである%
\footnote{%
	ちなみに,価値観的な側面では「合理性を重んじる態度の養成」が挙げられる.
}%
.
なぜ数学を学ぶことで論理的思考力を養成できるのだと考えられているかといえば,
「数学」と「論理」の間に切っても切れない深いつながりがあるからに他ならない.
数学では,ある言明が「正しい」と主張したいとき,
なぜそうなるのかということの説明,すなわち「証明」が求められる.

\begin{Ex} \label[Ex]{Ex:simpleLogic}
	「6は偶数である」という言明が正しいと主張したいとする.
	この言明は以下のように「証明」できる:
	\begin{enumerate}
		\item 「偶数」とは「2の倍数,すなわち2の整数倍として表される整数」のことである.
		\item \(6 = 2 \times 3\)と表わされる.
		\item 3は整数である.
		\item 6は2の整数倍として表わされる.
		\item 従って,6は偶数である.
	\end{enumerate}
\end{Ex}

我々は\cref{Ex:simpleLogic}のような議論でもって「6は偶数である」という主張を「正しい」と認識する.
そして\cref{Ex:simpleLogic}のような議論ができないとき,我々は「それはおそらく正しくないのだろう」と認識する%
\footnote{%
	「誰がどうやってもこのような議論はできないのだ」ということを主張したければ,そのこともまた「証明」が求められる.
}%
.
また,数学においてはこの「証明」からあいまいさを極力排除することを求めるのも特徴的である.

\begin{Ex} \label[Ex]{Ex:ambiguousLogic}
	「円と楕円は位相同型である」という言明が正しいを主張したいとする.
	しかし,以下のような議論は通常「証明」とは認められない:
	\begin{enumerate}
		\item 2つの図形が位相同型であるというのは,一方を連続的に変形して他方と一致させることができることをいう.
		\item 円を少しつぶすことで楕円と一致させることができる.
		\item よって,円と楕円は位相同型である.
	\end{enumerate}
\end{Ex}

位相同型というものがどういうものか知らずとも,\cref{Ex:ambiguousLogic}のような議論が「うさんくさい」ことに気づくであろう%
\footnote{%
	このような体験をもとにして「数学では厳格な証明のみが許容されるのだ」などとは思ってはいけない.
	むしろ,\cref{Ex:ambiguousLogic}のような素朴的直観を精密化することによって厳格な証明を与えることも多い.
	許容されないのは,このようなラフな議論によって正しさの検証が完全に完結したかのように考えることである.
}%
.
例えば,
\begin{itemize}
	\item 「連続的に変形」とはいったい何をどうすることなのか
	\item 「少しつぶす」とはいったい何をどうすることなのか
\end{itemize}
あたりであろう.いずれも議論の中で使われている言葉の定義にあいまいさがあることに起因している.

言葉の定義にあいまいさがあること以外に,数学においては不適切であるとみなされる議論の例も挙げよう.

\begin{Ex} \label[Ex]{Ex:insufficientLogic}
	「すべての整数\(n\)に対して,整数\(n^2\)を3で割った余りは0か1である」という言明が正しいと主張したいとする.
	しかし,以下のような議論は通常「証明」とは認められない:
	\begin{enumerate}
		\item \(n = 2\)とする.\(n^2 = 4\)を3で割った余りは1である.
		\item \(n = 11\)とする.\(n^2 = 121\)を3で割った余りは1である.
		\item \(n = 30\)とする.\(n^2 = 900\)を3で割った余りは0である.
		\item 以上より,「すべての整数\(n\)に対して,整数\(n^2\)を3で割った余りは0か1である」ことが確かめられた.
	\end{enumerate}
\end{Ex}

\Cref{Ex:insufficientLogic}での議論では,それぞれの場面で言葉の定義があいまいさがある場所はなかった.
この議論が不適切であるとみなされるのは,ひとえに検証が不十分であることが要因である.
整数というのは無限に多く存在するのにもかかわらず,\(2, 11, 30\)の3つでしか検証していない.
残りの整数に対して一切言及していないにもかかわらず,
あたかもすべての整数に対して検証が終わったかのように議論を進めていることが問題である.
これらの実験は,もとの言明の「正しさの根拠」とはなりえない%
\footnote{%
	当然のことであるが,正しさの根拠にならないからといって「これらの実験は無価値である」などと思ってはいけない.
	このような実験は,広大な数学という世界を渡り歩いていくうえで学術的・教育的に極めて高い価値を有する.
	学習者にとって対象が未知であればなおさらである.
}%
.
すべての整数に対してもれなく検証を終えて初めて正しさが検証されたといえる%
\footnote{%
	愚直に行うのは当然不可能なので,検証には別の方法を考える必要がある.
	ポピュラーなのは,特定の整数に限定しない一般的な整数\(n\)を「任意に」とって,
	この\(n\)に対してだけ主張の正しさを検証することである.
}.

以上のように数学と論理との関係について振り返ってみると,例えば次のような疑問が浮かび上がってくる:
\begin{enumerate}
	\item 我々は,数学において「正しい」ことと「証明できる」ことを自然に同一視してしまっているが,それは適切なのだろうか?
	\item 我々は,数学における議論の進め方に適切なものとそうでないものがあることを知っている.その境界になっているものは何か?
	\item 数学における論理について,通常の数学と同じように何か一般的な法則や定理を見いだせないだろうか?
\end{enumerate}
これらの問いに完全な解答を与えるのは極めて困難であろう.
何を主張しても「そういう意見もあるよね」程度の立ち位置に落ち着いてしまいそうである.
「証明」や「正しさ」の意味するところがあいまいであることが解答の難しさに拍車をかけている.

数理論理学では,このあいまいさに対して一定の解決策を見出すことができる.
それは「論理そのものに対して直接議論することなく,代わりに形式的な記号列について議論すること」である.
これがいったいどういうことなのかを次節以降で学んでいく.

\section{記号論理学} \label[section]{sec:symbolicLogic}

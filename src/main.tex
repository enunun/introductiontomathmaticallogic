\documentclass[10pt, a5j, ]{ltjsbook} % lualatex

\usepackage{caption}
\usepackage{subcaption}
\renewcommand{\thesubfigure}{\alph{subfigure}}

\makeatletter
\captionsetup{compatibility=false}
\captionsetup{%
	format = plain,
	labelsep = quad,
	font = small,
	skip = .2\baselineskip,
	width = .85\linewidth,
	subrefformat = parens,
	skip = 5\jsc@mpt
}
\captionsetup[subfigure]{%
	labelformat = parens,
	labelsep = space
}
\makeatother
%%%%%%%%=========package=============%%%%%%%%%%%%%
%
\usepackage[no-math]{luatexja-fontspec}
\usepackage{fontspec}
\usepackage{mathtools}
\usepackage{unicode-math}
\setmainfont[Ligatures=TeX]{XITS}
\setsansfont[Ligatures=TeX]{Fira sans}
\setmonofont[Ligatures=TeX]{Fira mono}
\setmathfont{XITS Math}[math-style=ISO, bold-style=ISO]
\setmathfont{XITS Math}[range=cal, StylisticSet=1]
\setmathfont{TeX Gyre Pagella Math}[range=bb]
%\setmathfont{TeX Gyre Pagella Math}[range={\DashVDash}]
%\setmathfont{TeX Gyre Pagella Math}[range={\dashVdash}]
%\setmathfont{Garamond Math}[range={\Coloneq}]
\usepackage[HaranoAji]{luatexja-preset}
\ltjsetparameter{jacharrange={-2,-3}}
\usepackage{amsthm}
\usepackage[draft]{graphicx}
\usepackage{booktabs}
\usepackage{enumitem}
\usepackage{multienum}
\usepackage{index}
\usepackage{pxrubrica}
\usepackage{luatexja-ruby}
\usepackage{tikz}
\usepackage{xcolor}
\usepackage{braket}
\usepackage{bussproofs}
\usepackage{multicol}
\usepackage{framed}
\usepackage{numbersets}

\numberwithin{equation}{section}

%\delimitershortfall=-.3pt

\graphicspath{{./fig/}}

\DeclareFontFamily{OML}{nxlmi}{\skewchar \font =127}
\DeclareFontShape{OML}{nxlmi}{m}{it}{
	<-6.3>  nxlmi035
	<6.3-8.6> nxlmi037
	<8.6->  nxlmi03
}{}
\DeclareSymbolFont{gletters}{OML}{nxlmi}{m}{it}
\DeclareMathSymbol{g}{\mathalpha}{gletters}{`g}

%%
%%%%%%%%%%%%%%%%%=============macro=======================%%%%%%%%%%%%%%
%

\NewDocumentCommand{\paren}{sm}{%
	\IfBooleanTF{#1}{%
	\mathopen{} (#2) \mathclose{}%
	}%
	{%
	\mathopen{} \left( #2 \right) \mathclose{}%
	}%
}

\NewDocumentCommand{\apply}{smm}{%
	\IfBooleanTF{#1}{#2\paren*{#3}%
	}%
	{%
	#2\paren{#3}%
	}%
}

\NewDocumentCommand{\sequent}{}{\Rightarrow}

\NewDocumentCommand{\standardmodel}{}{\symcal{N}}

\NewDocumentCommand{\numeral}{m}{\underline{#1}}

\NewDocumentCommand{\term}{m}{\textbf{#1}}
\RenewDocumentCommand{\emph}{m}{\textbf{#1}}

\NewDocumentCommand{\obj}{m}{\symtt{#1}}

\NewDocumentCommand{\entireminus}{}{\mathbin{\dot{-}}}

\DeclareMathOperator{\FV}{FV}
\DeclareMathOperator{\BV}{BV}
\DeclareMathOperator{\Var}{Var}
\DeclareMathOperator{\depth}{depth}
\DeclareMathOperator{\size}{size}
\DeclareMathOperator{\Th}{Th}
\DeclareMathOperator{\eDiag}{eDiag}
\DeclareMathOperator{\Diag}{Diag}
\DeclareMathOperator{\zero}{zero}
\DeclareMathOperator{\proj}{proj}
\DeclareMathOperator{\suc}{suc}
\DeclareMathOperator{\truefunction}{true}
\DeclareMathOperator{\falsefunction}{false}
\NewDocumentCommand{\pairfunc}{}{J}
\NewDocumentCommand{\connectionfunc}{}{\mathit{jp}}
\DeclareMathOperator{\lh}{lh}
\NewDocumentCommand{\codefunctioninv}{mm}{\paren{#1}_{#2}}
\DeclareMathOperator{\rem}{rem}
\DeclareMathOperator{\quo}{quo}
\NewDocumentCommand{\characteristicfunction}{m}{\chi_{#1}}
\DeclareMathOperator{\ProofPredicate}{Proof}
\DeclareMathOperator{\Bew}{Bew}
\DeclareMathOperator{\Bewstar}{Bew^{\ast}}
\DeclareMathOperator{\Con}{Con}

\NewDocumentCommand{\subst}{smm}{%
\IfBooleanTF{#1}{#2[{#3}]%
	}%
	{%
	#2\mathopen{} \left[ #3 \right] \mathclose{}%
	}%
}

\NewDocumentCommand{\LK}{}{LK}

\NewDocumentEnvironment{naiveproof}{}{\oframed}{\endoframed}

\NewDocumentCommand{\pair}{m}{\langle #1 \rangle}

\NewDocumentCommand{\provable}{}{\vdash}
\NewDocumentCommand{\notprovable}{}{\mathbin{\not\vdash}}
\NewDocumentCommand{\interderivable}{}{\dashVdash}

\NewDocumentCommand{\KleeneClosure}{m}{{#1}^{\ast}}

\NewDocumentCommand{\vertbracket}{sm}{%
	\IfBooleanTF{#1}{%
		\mathopen{} \lvert #2\rvert \mathclose{}%
	}%
	{%
		\mathopen{} \left\lvert #2 \right\rvert \mathclose{}%
	}%
}

\NewDocumentCommand{\length}{sm}{%
	\IfBooleanTF{#1}{%
		\vertbracket*{#2}
	}%
	{%
		\vertbracket{#2}%
	}%
}

\NewDocumentCommand{\absolute}{sm}{%
	\IfBooleanTF{#1}{%
		\vertbracket*{#2}
	}%
	{%
		\vertbracket{#2}%
	}%
}

\NewDocumentCommand{\squarebrackets}{sm}{%
	\IfBooleanTF{#1}{%
		\mathopen{} [#2] \mathclose{}%
	}%
	{%
		\mathopen{} \left[ #2 \right] \mathclose{}%
	}%
}

\NewDocumentCommand{\equivclass}{sm}{%
	\IfBooleanTF{#1}{%
		\squarebrackets*{#2}
	}%
	{%
		\squarebrackets{#2}
	}%
}

\NewDocumentCommand{\polynomial}{smm}{%
	\IfBooleanTF{#1}{%
		#2 \squarebrackets*{#3}
	}%
	{%
		#2 \squarebrackets{#3}
	}%
}

\NewDocumentCommand{\anglebrackets}{sm}{%
	\IfBooleanTF{#1}{%
		\mathopen{} \langle #2\rangle \mathclose{}%
	}%
	{%
		\mathopen{} \left\langle #2\right\rangle \mathclose{}%
	}%
}

\NewDocumentCommand{\emptystring}{}{\varepsilon}

\NewDocumentCommand{\restriction}{}{\upharpoonright}

\NewDocumentCommand{\objeq}{}{\equiv}

\NewDocumentCommand{\formulaequiv}{}{\leftrightarrow}

\NewDocumentCommand{\uexists}{}{{\exists !}}

\NewDocumentCommand{\powerset}{m}{\apply{\symfrak{P}}{#1}}

\NewDocumentCommand{\interpretation}{mm}{{#2}^{#1}}

\NewDocumentCommand{\languagewithname}{mm}{\apply{#1}{#2}}

\NewDocumentCommand{\metaimplies}{}{\implies}

\NewDocumentCommand{\metaequivalent}{}{\iff}

\NewDocumentCommand{\satisfy}{}{\vDash}
\NewDocumentCommand{\notsatisfy}{}{\mathbin{\not\vDash}}

\NewDocumentCommand{\logicallyequivalent}{}{\DashVDash}

\NewDocumentCommand{\isomorphic}{}{\cong}

\NewDocumentCommand{\elementarilyequivalent}{}{\equiv}

\NewDocumentCommand{\theoryset}{m}{\symsfup{#1}}

\NewDocumentCommand{\GP}{}{\theoryset{GP}}

\NewDocumentCommand{\Ring}{}{\theoryset{Ring}}

\NewDocumentCommand{\CRing}{}{\theoryset{CRing}}

\NewDocumentCommand{\OrderedRing}{}{\theoryset{OR}}

\NewDocumentCommand{\ProjectiveGeometryPart}{}{\theoryset{B}}

\NewDocumentCommand{\ZF}{}{\theoryset{ZF}}

\NewDocumentCommand{\ZFC}{}{\theoryset{ZFC}}

\NewDocumentCommand{\RArithmetic}{}{\theoryset{R}}

\NewDocumentCommand{\Robinson}{}{\theoryset{Q}}

\NewDocumentCommand{\PA}{}{\theoryset{PA}}

\NewDocumentCommand{\POSET}{}{\theoryset{POSET}}

\NewDocumentCommand{\TOSET}{}{\theoryset{TOSET}}

\NewDocumentCommand{\OpenFormulaSet}{}{\theoryset{Open}}

\NewDocumentCommand{\SigmaFormula}{}{\symup{\Sigma}}

\NewDocumentCommand{\PiFormula}{}{\symup{\Pi}}

\NewDocumentCommand{\DeltaFormula}{}{\symup{\Delta}}

\NewDocumentCommand{\InductionLimitedTheory}{m}{\symup{I}#1}

\NewDocumentCommand{\OpenFormulaInduction}{}{\InductionLimitedTheory{\OpenFormulaSet}}

\NewDocumentCommand{\SigmaFormulaInduction}{}{\InductionLimitedTheory{\SigmaFormula}}

\NewDocumentCommand{\PiFormulaInduction}{}{\InductionLimitedTheory{\PiFormula}}

\NewDocumentCommand{\DeltaFormulaInduction}{}{\InductionLimitedTheory{\DeltaFormula}}

\NewDocumentCommand{\Ar}{}{\symup{Ar}}

\NewDocumentCommand{\muoperator}{}{\mu}

\NewDocumentCommand{\GodelNumber}{m}{\mathopen{}\left\ulcorner #1 \right\urcorner\mathclose{}}

\NewDocumentCommand{\symbolcode}{m}{\GodelNumber{#1}}

\NewDocumentCommand{\finitesequencecode}{m}{\langle #1 \rangle}

\NewDocumentCommand{\codeconnection}{}{\mathbin{\ast}}

\NewDocumentCommand{\complementset}{m}{{#1}^{c}}

\NewDocumentCommand{\partialequal}{}{\sim}
%
%%%%%%%%%%%%%%=======================theorem==================%%%%%%%%%%%%%
%
%
\theoremstyle{definition}
\newtheorem{Def}{定義}[section]
\newtheorem{Thm}[Def]{定理}
\newtheorem{Ex}[Def]{例}
\newtheorem{Que}{演習}[section]
\newtheorem{Lemma}[Def]{補題}
\newtheorem{Corollary}[Def]{系}
\newtheorem{Note}[Def]{注意}
%
%
%%%%%%%%%%%%%=====================index==================================%%%%%%%%%%%%%%
%
\newindex{sidx}{sidx}{sind}{記号索引}
\newindex{widx}{widx}{wind}{用語索引}
%
%
%%%%%%%%%===================foot note numbering=================%%%%%%%%%%%%%%%%%%%%
%
\makeatletter
\@addtoreset{footnote}{section}
\makeatother

%%
%%%%%%%%%%%%%%========hyperref============%%%%%%%%%%
%
\usepackage[unicode, pdfusetitle, hidelinks, draft = false]{hyperref} % hyperlink

\hypersetup{% setting hyperref
	bookmarksnumbered=true,
	bookmarksopen=true,
	bookmarkstype=toc,
	pdfborder={0 0 0},
	colorlinks = true,
	draft=false,
}
\usepackage{footnotebackref}
%
%
%%%%%%%==============crossreference================%%%%%%%%%%
%
\usepackage[nameinlink]{cleveref}
\crefname{figure}{図}{図}
\Crefname{figure}{図}{図}
\crefname{equation}{式}{式}
\Crefname{equation}{式}{式}
\crefformat{chapter}{#2第#1章#3}
\Crefformat{chapter}{#2第#1章#3}
\crefrangeformat{chapter}{#3第#1#4章から#5第#2#6章}
\Crefrangeformat{chapter}{#3第#1#4章から#5第#2#6章}
\crefname{page}{p.}{p.}
\Crefname{page}{p.}{p.}
\crefname{Ex}{例}{例}
\Crefname{Ex}{例}{例}
\Crefname{Def}{定義}{定義}
\crefname{Thm}{定理}{定理}
\Crefname{Thm}{定理}{定理}
\crefname{table}{表}{表}
\Crefname{table}{表}{表}
\Crefname{Que}{演習}{演習}
\crefname{Que}{演習}{演習}
\Crefname{Lemma}{補題}{補題}
\crefname{Lemma}{補題}{補題}
\Crefname{Corollary}{系}{系}
\crefname{Corollary}{系}{系}
\Crefname{Note}{注意}{注意}
\crefname{Note}{注意}{注意}
\Crefname{enumi}{}{}
\crefname{enumi}{}{}
\crefname{section}{\presectionname}{\presectionname}
\Crefname{section}{\presectionname}{\presectionname}
\crefname{subsection}{\presubsectionname}{\presubsectionname}
\Crefname{subsection}{\presubsectionname}{\presubsectionname}
\crefrangeformat{section}{#3\presectionname#1#4から#5\presectionname#2#6}
\Crefrangeformat{section}{#3\presectionname#1#4から#5\presectionname#2#6}
\crefrangeformat{subsection}{#3\presubsectionname#1#4から#5\presubsectionname#2#6}
\Crefrangeformat{subsection}{#3\presubsectionname#1#4から#5\presubsectionname#2#6}

%
%
%%%%%%%======biblatex=============%%%%%%%%%%
%
\usepackage[%
	backend = biber,%
	url = false,%
	doi = false,%
	eprint = false,%
	isbn = false,%
	sorting = none,%
	style = numeric-comp%
]{biblatex} % use "biblatex"
\DeclareFieldFormat*{title}{\mkbibquote{#1}}
\DeclareDelimFormat{labelnamepunct}{\addcomma\addspace}
\DeclareFieldFormat{url}{\url{#1}}
\addbibresource{reference/book.bib}
%\addbibresource{reference/online.bib}

\makeatletter
\renewcommand{\section}{%
	\if@slide\clearpage\fi
	\@startsection{section}{1}{\z@}%
	{\Cvs \@plus.5\Cdp \@minus.2\Cdp}% 前アキ
	{.5\Cvs \@plus.3\Cdp}% 後アキ
	%   {\normalfont\Large\headfont\@secapp}}
	{\normalfont\Large\headfont\presectionname\raggedright}%
}
\renewcommand{\subsection}{%
	\@startsection{subsection}{2}{\z@}%
	{\Cvs \@plus.5\Cdp \@minus.2\Cdp}% 前アキ
	{.5\Cvs \@plus.3\Cdp}% 後アキ
	{\normalfont\large\headfont\presubsectionname}%
}
\makeatother
\renewcommand{\presectionname}{\S}
\newcommand{\presubsectionname}{\S\S}
\setcounter{tocdepth}{2}
%
%


%%%%%%%%%%%%%%%===========index============%%%%%%%%%%%

\usepackage{needspace}

\makeatletter
\NewDocumentCommand\idxhead{m}{%
	\needspace{2\baselineskip}%
	\vspace{\baselineskip}%
	\DeclareDocumentCommand\hrulefill{}{\leavevmode\leaders\hrule height 0.8pt\hfill\kern\z@}%
	\hbox to \columnwidth{\hfil%
	\normalsize%
	\textcolor{gray}{\raisebox{.15ex}{■}}\hspace{.5\zw}\textsf{#1}\hspace{.5\zw}\textcolor{gray}{\raisebox{.15ex}{■}}\hfil}\vspace{-4.5mm}\hrulefill\par%
	\nopagebreak%
}
\makeatother
%

\begin{document}

\begin{titlepage}
	\title{0から始める数理論理学入門 第3版}
	\author{野口匠}
	\date{2024年12月30日}
	\maketitle
\end{titlepage}

\frontmatter

\chapter{はじめに}

数理論理学は,ゲーデルの不完全性定理の存在もあってか世間一般の認知度は比較的高い.
しかしながら,数理論理学の基本コンセプトである「論理を形式化してその性質を探っていく」という考え方は広く浸透しているとはいいがたい.
特にゲーデルの不完全性定理は,なんとなく日常用語として理解できそうな言い回しや字面のインパクトの強さからか
「ゲーデルは数学が万能とはなりえないことを証明した」だの「完璧な数学理論は存在しない」だのと意味不明な誤解が発生しがちである.
とはいえ,最近では「これは誤解である」という認識自体は広まっており,状況は改善しつつあるといえる.

一方で,いわゆる「通常の数学理論」と比較すると数理論理学はあまり広く学ばれてはいないのが現状である.
現在ではわかりやすい和書も多数出版されており,学ぶハードルはそれなりに低いけども,学んでいる人の数はほかの数学理論と比較すれば少ないと言わざるを得ないのではないだろうか.
実際,数学をそれなりに学んでいる人であっても数理論理学については何も知らないという人が少なくない.数学を専門とせず,道具として使っている物理学や統計学を学んでいる人であればなおさらである.

しかし,特に数学を学んでいる人にとっては「論理記号」については身近だと感じる人は多い.彼らにとって論理記号は普段使っている数学記号と同じく「なんらかの数学的主張を表現したもの」であり,
その認識のままで大きな問題が生じることはない.
むろん数理論理学を学んだ人からすればその認識は厳密には誤りである.
誤っている部分が数理論理学にとっても些事であればよいのだが,残念なことに論理を形式化するという数理論理学の基本的な方法について理解できていない致命的な誤りである.

本書はそのような誤解を払拭することを第一の目標とした.
すなわち「論理を形式化する」ことがいったいどういうことなのかを実感をもって学ぶことが目標である.
そのため,入門書で多く取り上げられているであろう数理論理学に関する結果の多くは取り上げない.
特に不完全性定理については取り上げないので,それをめあてにして本書を手に取るとがっかりするであろう.
その代わり,通常の数学理論との接点を多く紹介することで「論理の形式化」についての理解を深めたい.

とりわけ「\(A \land B\)は\(A\)かつ\(B\)という意味」\emph{ではない}というジャーゴンの意味が理解できれば目標達成である.
このことが理解できれば,数理論理学の世界に飛び込む準備は完全に整ったといってよい.要するに,本書は数理論理学入門の前段階という位置づけで活用するとよい.
本書の後,あるいは並行して読むことになるであろう1冊目の入門書としては前原\cite{maehara2005}や鹿島\cite{kashima2009}がおすすめである.
数理論理学入門としての色が強いのは後者であるが,やや難しいと感じた場合には前者を読んでみるのもよいだろう.
また,数理論理学が数学の一分野である以上,集合や写像といった「道具」の修得は欠かせない.本書でも説明抜きに用いている.
自信がない人は嘉田\cite{kada2008}を手に取るとよい.
ユークリッド幾何学の基礎づけについて気になる人は足立\cite{adachi2019}を,数の体系について知りたい人は田中\cite{tanaka2019}を読むとよい.

本書の原稿やソースファイルは,以下のGitHubリポジトリにて閲覧可能である.

\begin{center}
	\url{https://github.com/enunun/introductiontomathmaticallogic}
\end{center}

執筆時間の関係で書ききれなかった内容や演習問題の解答は,ここに随時追加予定である.

\begin{flushright}
	2023年12月31日
\end{flushright}

\tableofcontents

\mainmatter

\chapter{論理の形式化}
\label[chapter]{chap:formulize}

数理論理学が数学の一分野として成功を収めた要因のひとつとして,
素朴的直観を伴わない形式的な記号列を主役に据えたことが挙げられる.
ともすれば,このことは「数学で用いる論理について研究する分野」
という一般的な認識と矛盾するように見えることだろう.
実際,この「形式的な記号列」に関する議論の結果を根拠に
我々が普段使っている「論理」について何かを主張したい場合,
これらの間の橋渡しを行うのは主張したい当人の責任であり,
数理論理学の諸定理がその橋渡しについて何かを保証してくれることはない.
これは,理論物理学で得られた結果から現実世界について言及したり,
統計モデルの性質をもとに現実で得られたデータ(あるいはその生成元)に
ついて言及したりする営みに非常によく似ている.

もし読者にこのような「それそのものではないが何らかの視点で
類似性をもつと期待される概念の性質をもとに,目的の対象について議論する」
という営みに親しみがあれば,本章の内容は極めて身近に思えるに違いない.
そして,この手法が科学においていかに強力であるかを知っていれば,
数理論理学の手法がいかに強力であるかも予想できるだろう.


\section{数学における論理}
\label[section]{sec:logic}

数学が世界的に広く学ばれていることが示すように,
数学(および算数)を学ぶことは有益であるというのが一般的な認識である.
「なぜ数学を学ぶのか」という問いの答えは個々人によってさまざまだろうが,
少なくとも日本の数学教育学の分野では,
数学教育の目的は
\begin{enumerate}
	\item 陶冶的目的
	\item 実用的目的
	\item 文化的目的
\end{enumerate}
の3つの観点から論じられるのが一般的である%
\footnote{%
	念のため述べておくが,これは観点別評価とはまったく別の話である.
}%
.
このうち,実用的目的と文化的目的については字面から容易に想像できる通りの意味である.
すなわち,実用的目的は教科教育の内容そのものの修得と実社会における活用を志向したものであり,
文化的目的は文化としての学問を継承・発展させていくことを志向したものである.

一方で,1つ目の陶冶的目的についてはやや聞きなれない言葉である.
ここでの「陶冶」という言葉には「人の性質や能力を円満に育てること」という意味である.
すなわち,陶冶的目的というのは人間形成や価値観・(教科教育の内容以外での)能力を養成することを志向したものである.
具体例を挙げていけばキリがないが,「論理的思考力の養成」という目的は数学教育に明るくない人からでも
頻繁に挙がるものである%
\footnote{%
	ちなみに,価値観的な側面では「合理性を重んじる態度の養成」が挙げられる.
}%
.
なぜ数学を学ぶことで論理的思考力を養成できるのだと考えられているかといえば,
「数学」と「論理」の間に切っても切れない深いつながりがあるからに他ならない.
数学では,ある言明が「正しい」と主張したいとき,
なぜそうなるのかということの説明,すなわち「証明」が求められる.

\begin{Ex} \label[Ex]{Ex:simpleLogic}
	「6は偶数である」という言明が正しいと主張したいとする.
	この言明は以下のように「証明」できる:
	\begin{enumerate}
		\item 「偶数」とは「2の倍数,すなわち2の整数倍として表される整数」のことである.
		\item \(6 = 2 \times 3\)と表わされる.
		\item 3は整数である.
		\item 6は2の整数倍として表わされる.
		\item 従って,6は偶数である.
	\end{enumerate}
\end{Ex}

我々は\cref{Ex:simpleLogic}のような議論でもって「6は偶数である」という主張を「正しい」と認識する.
そして\cref{Ex:simpleLogic}のような議論ができないとき,我々は「それはおそらく正しくないのだろう」と認識する%
\footnote{%
	「誰がどうやってもこのような議論はできないのだ」ということを主張したければ,そのこともまた「証明」が求められる.
}%
.
また,数学においてはこの「証明」からあいまいさを極力排除することを求めるのも特徴的である.

\begin{Ex} \label[Ex]{Ex:ambiguousLogic}
	「円と楕円は位相同型である」という言明が正しいを主張したいとする.
	しかし,以下のような議論は通常「証明」とは認められない:
	\begin{enumerate}
		\item 2つの図形が位相同型であるというのは,一方を連続的に変形して他方と一致させることができることをいう.
		\item 円を少しつぶすことで楕円と一致させることができる.
		\item よって,円と楕円は位相同型である.
	\end{enumerate}
\end{Ex}

位相同型というものがどういうものか知らずとも,\cref{Ex:ambiguousLogic}のような議論が「うさんくさい」ことに気づくであろう%
\footnote{%
	このような体験をもとにして「数学では厳格な証明のみが許容されるのだ」などとは思ってはいけない.
	むしろ,\cref{Ex:ambiguousLogic}のような素朴的直観を精密化することによって厳格な証明を与えることも多い.
	許容されないのは,このようなラフな議論によって正しさの検証が完全に完結したかのように考えることである.
}%
.
例えば,
\begin{itemize}
	\item 「連続的に変形」とはいったい何をどうすることなのか
	\item 「少しつぶす」とはいったい何をどうすることなのか
\end{itemize}
あたりであろう.いずれも議論の中で使われている言葉の定義にあいまいさがあることに起因している.

言葉の定義にあいまいさがあること以外に,数学においては不適切であるとみなされる議論の例も挙げよう.

\begin{Ex} \label[Ex]{Ex:insufficientLogic}
	「すべての整数\(n\)に対して,整数\(n^2\)を3で割った余りは0か1である」という言明が正しいと主張したいとする.
	しかし,以下のような議論は通常「証明」とは認められない:
	\begin{enumerate}
		\item \(n = 2\)とする.\(n^2 = 4\)を3で割った余りは1である.
		\item \(n = 11\)とする.\(n^2 = 121\)を3で割った余りは1である.
		\item \(n = 30\)とする.\(n^2 = 900\)を3で割った余りは0である.
		\item 以上より,「すべての整数\(n\)に対して,整数\(n^2\)を3で割った余りは0か1である」ことが確かめられた.
	\end{enumerate}
\end{Ex}

\Cref{Ex:insufficientLogic}での議論では,それぞれの場面で言葉の定義があいまいさがある場所はなかった.
この議論が不適切であるとみなされるのは,ひとえに検証が不十分であることが要因である.
整数というのは無限に多く存在するのにもかかわらず,\(2, 11, 30\)の3つでしか検証していない.
残りの整数に対して一切言及していないにもかかわらず,
あたかもすべての整数に対して検証が終わったかのように議論を進めていることが問題である.
これらの実験は,もとの言明の「正しさの根拠」とはなりえない%
\footnote{%
	当然のことであるが,正しさの根拠にならないからといって「これらの実験は無価値である」などと思ってはいけない.
	このような実験は,広大な数学という世界を渡り歩いていくうえで学術的・教育的に極めて高い価値を有する.
	学習者にとって対象が未知であればなおさらである.
}%
.
すべての整数に対してもれなく検証を終えて初めて正しさが検証されたといえる%
\footnote{%
	愚直に行うのは当然不可能なので,検証には別の方法を考える必要がある.
	ポピュラーなのは,特定の整数に限定しない一般的な整数\(n\)を「任意に」とって,
	この\(n\)に対してだけ主張の正しさを検証することである.
}.

以上のように数学と論理との関係について振り返ってみると,例えば次のような疑問が浮かび上がってくる:
\begin{enumerate}
	\item 我々は,数学において「正しい」ことと「証明できる」ことを自然に同一視してしまっているが,それは適切なのだろうか?
	\item 我々は,数学における議論の進め方に適切なものとそうでないものがあることを知っている.その境界になっているものは何か?
	\item 数学における論理について,通常の数学と同じように何か一般的な法則や定理を見いだせないだろうか?
\end{enumerate}
これらの問いに完全な解答を与えるのは極めて困難であろう.
何を主張しても「そういう意見もあるよね」程度の立ち位置に落ち着いてしまいそうである.
「証明」や「正しさ」の意味するところがあいまいであることが解答の難しさに拍車をかけている.

数理論理学では,このあいまいさに対して一定の解決策を見出すことができる.
それは「論理そのものに対して直接議論することなく,代わりに形式的な記号列について議論すること」である.
これがいったいどういうことなのかを次節以降で学んでいく.

\section{記号論理学} \label[section]{sec:symbolicLogic}

\chapter{1階述語論理の統語論}
\label[chapter]{chap:syntax}

本章では,1階述語論理と呼ばれる体系の統語論について述べる.
統語論とは,雑に述べると「文の構造」についての理論である.
この「文」は,ここでは「数学における何らかの主張」に対応する.
つまり,ここで議論したいのは
「数学における何らかの主張はどのような要素がどう組み合わさってできているのか」
ということである.
このことを議論するための足掛かりとして,我々は\cref{chap:formulize}で述べた方針に従い
「何ら素朴的直観が関与しない形式的な記号列の世界」において
論理式という概念を構成していく.

形式的記号列の世界での定義により,何を議論の対象とし,何を議論の対象としないのかが明瞭となる.
これにより,通常の数学と同じく一般的な法則や定理を研究することが可能となる.
本章の内容はその前準備に相当する.

\newpage

\section{言語} \label[seciton]{sec:language}

議論を始めるにあたり,どのような記号を使用するのかを明瞭にする必要がある.
まず,通常の数学においてはどうなっているかを振り返ろう.

\begin{Ex} \label[Ex]{Ex:informalsymbol}
	群論においては,単位元を表す記号「\(\obj{e}\)」,2項演算を表す記号
	「\(\obj{\mathord{\ast}}\)」,逆元を表す記号「\({}^{\obj{-1}}\)」が用いられる.
	また,順序の理論においては,順序を表す記号「\(\obj{\le}\)」が用いられる.

	この他,「\(\obj{x}\)」や「\(\obj{y}\)」等の変数を表す記号やカッコ「\(\lparen\)」「\(\rparen\)」や
	カンマ「\(,\)」等は共通して用いられる.
\end{Ex}

\Cref{Ex:informalsymbol}では,対象の理論に依存して必要であったりそうでなかったりする記号と
共通して用いられる記号があった.
そのような記号を\cref{tab:commonsymbol}に示し,今後逐一言及しないものとする%
\footnote{%
	ここに示す記号とは違う記号を採用する場合もあるが,それは単に流儀や議論の対象の違いである.
	その違いによって質的に大きな差異が生じることもあればそうでないこともある.
}%
.

\begin{table}[htbp]
	\centering
	\caption{数学理論で共通して用いられる記号}
	\label{tab:commonsymbol}
	\begin{tabular}{ccc}
		\toprule
		種別   & 一覧                                                                   & 備考              \\
		\midrule
		変数記号 & \(\obj{x}, \obj{y}, \obj{z}, \dotsc\)                                & 無限に多く存在する(可算無限) \\
		論理記号 & \(\bot, \lnot, \land, \lor, \to, \forall, \exists\)                                    \\
		等号   & \(\obj{\mathord{\objeq}}\)                                           & オブジェクト側の意味での等号  \\
		補助記号 & \(\text{``\(\lparen\)''}, \text{``\(\rparen\)''}, \text{``\(,\)''}\) & カッコやカンマ         \\
		\bottomrule
	\end{tabular}
\end{table}

「名称」列に「変数記号」や「論理記号」等の名前が定義なしに入っているが,これは単にそういう区分けができることのみが要請される.
また,変数記号については無限に多く存在することが要請されているが,これについてはすでにいくつか記号が登場している状況下において
そのいずれとも異なる変数記号をいつでも用意できることを期待しての要請である.

さて,\cref{tab:commonsymbol}に追加で記号を付け加えることにより,各理論の特色が現れる.

\begin{Def} \label{def:language}
	記号の集合\(\symcal{L}\)が%
	\index[widx]{げんご@言語}%
	\term{言語}であるとは,\(\symcal{L}\)の元が以下の3種類に区分けされていることをいう:
	\begin{itemize}
		\item \index[widx]{ていすうきごう@定数記号}定数記号
		\item \index[widx]{かんすうきごう@関数記号}関数記号,\index[widx]{ありてぃ@アリティ}\term{アリティ}と呼ばれる正の整数\(n\)をもつ
		\item \index[widx]{かんけいきごう@関係記号}関係記号,\index[widx]{ありてぃ@アリティ}アリティと呼ばれる正の整数\(n\)をもつ
	\end{itemize}
\end{Def}

「定数記号」や「関数記号」という字面はいかにも我々の素朴的直観を呼び起こしそうであるが,
ここでは単にそういう名称で区分けできるということだけを要請しているに過ぎない.
アリティについても同様である.「この関数記号のアリティはいくつですか?」に対して「2です」のような
解答を返せることを要請しているに過ぎない.

言語については,例を述べるのがわかりやすいだろう.

\begin{Ex} \label[Ex]{Ex:languageexample}
	群論の言語\(\symcal{L}_1\)は\(\symcal{L}_1 = \set{\obj{\ast}, \obj{e}, {}^{\obj{-1}}}\)
	と与えることができる.ここで,\(\obj{\ast}\)はアリティ2の関数記号,\(\obj{e}\)は定数記号,
	\({}^{\obj{-1}}\)はアリティ1の関数記号である.
	また,順序の理論の言語\(\symcal{L}_2\)は\(\symcal{L}_2 = \set{\obj{\le}}\)と与えることができる.
	ここで,\(\obj{\le}\)はアリティ2の関係記号である.
\end{Ex}

アリティというのは,素朴的直観における「引数の数」の対応物だと考えられる.
こう考えてみれば,関数記号や関係記号にアリティが定まっていることを要請するのはごく自然であろう.


\section{項と論理式} \label[section]{sec:logicalexpression}

使う記号を定義したことで,これらを組み合わせて「モノ」や「主張」に相当する概念を構築していくことができる.
これは言語に対してある種の「文法」を定めることに相当する.
文法を定めることにより,形式的記号の集合でしかなかった言語が一気に「数学っぽい」性格を帯びてくる.

\index[widx]{こう@項}
\begin{Def} \label[Def]{Def:term}
	言語\(\symcal{L}\)に対し,\(\symcal{L}\)項を以下のように帰納的に定義する:
	\begin{enumerate}
		\item 変数記号は\(\symcal{L}\)項である.
		\item 定数記号は\(\symcal{L}\)項である.
		\item \(f\)が\(\symcal{L}\)におけるアリティ\(n\)の関数記号であり%
		      \footnote{%
			      ここで\(f\)がイタリック体なのは,「\(f\)」という記号そのものが\(\symcal{L}\)の関数記号というわけではなく
			      \(\symcal{L}\)の関数記号のうちのどれかであるということを明示することを意図している.
		      }%
		      ,
		      \(t_1, t_2, \dotsc, t_n\)が\(\symcal{L}\)項であるならば,記号列
		      \begin{equation}
			      \apply{f}{t_1, t_2, \dotsc, t_n}
		      \end{equation}
		      は\(\symcal{L}\)項である.
		\item 以上の規則を有限回適用して得られるもののみが項である.
	\end{enumerate}
\end{Def}


\begin{Ex} \label[Ex]{Ex:groupterm}
	\Cref{Ex:languageexample}で述べた群論の言語\(\symcal{L}_1\)において,
	\(\apply{\obj{\ast}}{\obj{x}, \obj{y}}, \obj{e}, \apply{\obj{\ast}}{\apply{\obj{\ast}}{\obj{e}, \obj{x}}, \obj{y}}\)は
	いずれも\(\symcal{L}_1\)項である.ここで,\(\obj{x}, \obj{y}\)は変数記号である.
	なお,順序の理論の言語\(\symcal{L}_2\)は定数記号も関数記号ももたないため,
	\(\symcal{L}_2\)項は変数記号のみである.
\end{Ex}

\begin{Que} \label{Que:termexample}
	\Cref{Def:term}に基づき,\cref{Ex:groupterm}における
	「\(\apply{\obj{\ast}}{\apply{\obj{\ast}}{\obj{e}, \obj{x}}, \obj{y}}\)」
	が\(\symcal{L}_1\)項であることを確かめよ.
	また,記号列「\(\obj{e} \apply{\mathord{\obj{\ast}}}{\obj{e}, \obj{e}}\)」が\(\symcal{L}_1\)項でないことを確かめよ.
\end{Que}


\index[widx]{ろんりしき@論理式}
\begin{Def} \label[Def]{Def:logicalexpression}
	言語\(\symcal{L}\)に対し,\(\symcal{L}\)論理式を以下のように帰納的に定義する:
	\begin{enumerate}
		\item \(\bot\)は論理式である.
		\item \(t_1, t_2\)が\(\symcal{L}\)項であるならば,記号列
		      \begin{equation}
			      \paren{t_1 \obj{\objeq} t_2}
		      \end{equation}
		      は\(\symcal{L}\)論理式である.
		\item \(r\)が\(\symcal{L}\)におけるアリティ\(n\)の関係記号であり%
		      \footnote{%
			      \Cref{Def:term}のときと同様の理由で\(r\)はイタリック体としている.
		      }%
		      ,\(t_1, t_2, \dotsc, t_n\)が\(\symcal{L}\)項であるならば,
		      記号列
		      \begin{equation}
			      \apply{r}{t_1, t_2, \dotsc, t_n}
		      \end{equation}
		      は\(\symcal{L}\)論理式である.
		\item \(\varphi, \psi\)が論理式で\(x\)が変数記号であるならば,
		      記号列
		      \begin{align}
			      \paren{\lnot \varphi},      \\
			      \paren{\varphi \land \psi}, \\
			      \paren{\varphi \lor \psi},  \\
			      \paren{\varphi \to \psi},   \\
			      \paren{\forall x \varphi},  \\
			      \paren{\exists x \varphi}
		      \end{align}
		      はいずれも\(\symcal{L}\)論理式である.
		\item 以上の規則を有限回適用して得られるもののみが論理式である.
	\end{enumerate}
\end{Def}

\begin{Def} \label[Def]{Def:atomiclogicalexpression}
	\Cref{Def:logicalexpression}において,最後に規則1, 2, 3を適用して得られる\(\symcal{L}\)論理式,すなわち
	\begin{align*}
		\bot,                   \\
		\paren{t_1 \objeq t_2}, \\
		\apply{r}{t_1, t_2, \dotsc, t_n}
	\end{align*}
	の形の\(\symcal{L}\)論理式を%
	\index[widx]{ろんりしき@論理式!げんしろんりしき@原子---}%
	\(\symcal{L}\)\term{原子論理式}といい,
	\(\symcal{L}\)原子論理式でない\(\symcal{L}\)論理式,
	すなわち最後に規則4を適用して得られる\(\symcal{L}\)論理式を%
	\index[widx]{ろんりしき@論理式!ふくごうろんりしき@複合---}%
	\(\symcal{L}\)\term{複合論理式}という.
	さらに,
	\begin{align*}
		\forall x \varphi, \\
		\exists x \varphi
	\end{align*}
	の形の\(\symcal{L}\)論理式をそれぞれ%
	\index[widx]{ろんりしき@論理式!ぜんしょうろんりしき@全称---}%
	\(\symcal{L}\)\term{全称論理式},
	\index[widx]{ろんりしき@論理式!そんざいろんりしき@存在---}%
	\(\symcal{L}\)\term{存在論理式}という.
\end{Def}

\begin{Def} \label[Def]{Def:universalclosure}
	\(\symcal{L}\)を言語とし,\(\varphi\)を\(\symcal{L}\)論理式,\(\varphi\)に自由出現する変数記号全体が
	\[
		\apply{\FV}{\varphi} = \Set{x_1, x_2, \dots, x_n}
	\]
	であるとする.このとき,\(\symcal{L}\)論理式
	\[
		\forall x_1 \forall x_2 \dotsb \forall x_n \varphi
	\]
	は\(\symcal{L}\)閉論理式となる.この論理式を\(\varphi\)の%
	\index[widx]{ぜんしょうへいほう@全称閉包}%
	\term{全称閉包}という.
\end{Def}

\begin{Ex} \label[Ex]{Ex:logicalexpression}
	群論の言語\(\symcal{L}_1\)において,
	\(\paren{\apply{\obj{\ast}}{\obj{x}, \obj{y}} \obj{\objeq} \apply{\obj{\ast}}{\obj{y}, \obj{x}}}\)は
	\(\symcal{L}_1\)論理式である.また,順序の理論の言語\(\symcal{L}_2\)において,
	\(\paren{ \forall \obj{y} \mathord{\obj{\le}}\paren{ \obj{x}, \obj{y}} }\)や
	\(\paren{ \forall \obj{x} \paren{ \forall \obj{y}\apply{\mathord{\obj{\le}}}{\obj{x}, \obj{y}}} }\)は
	いずれも\(\symcal{L_2}\)論理式である.ここで,\(\obj{x}, \obj{y}\)は変数記号である.
\end{Ex}

論理式でない記号列の例も挙げておこう.

\begin{Ex} \label[Ex]{Ex:nologicalexpression}
	言語\(\symcal{L}\)において,\(\obj{x}\)が変数記号であるとき,
	記号列\(\paren{\exists \obj{x} \paren{\obj{x}}}\)は\(\symcal{L}\)論理式ではない.
	一方で,\(\paren{\exists \obj{x} \paren{\obj{x} \obj{\objeq} \obj{x}}}\)は\(\symcal{L}\)論理式である.
	また,群論の言語\(\symcal{L}_1\)において,\(\paren{\forall \obj{x} \paren{\obj{x} \obj{\ast} \obj{e} \objeq \obj{x}}}\)や
	\(\forall \obj{x} \paren{ \obj{x} \obj{\objeq} \obj{x}}\)はいずれも
	\emph{\cref{Def:logicalexpression}で述べた意味においては}\(\symcal{L}_1\)論理式ではない.
\end{Ex}

\begin{Que} \label[Que]{Que:logicalexpression}
	\Cref{Def:logicalexpression}に基づき,\cref{Ex:logicalexpression}と\cref{Ex:nologicalexpression}で
	挙げた各式について,それが実際に論理式であることやそうでないことを確かめよ.
\end{Que}

\begin{Note} \label[Note]{Note:logicalexpression}
	\Cref{Ex:nologicalexpression}の後半で「\cref{Def:logicalexpression}で述べた意味においては」
	と述べたのは,
	それなりに妥当性のある略記表現に関する約束事を適切に定めることにより,
	これらが論理式であるようにみなせるからである.
	例えば,以下のように約束することが多い:
	\begin{itemize}
		\item 誤解が生じない範囲で論理式の構成順序を表すかっこは省略してよい.
		\item 論理記号\(\lnot, \land, \lor, \to, \forall, \exists\)たちの結合の優先順位については,
		      \(\lnot, \forall, \exists\)がもっとも高く,次に\(\land, \lor\)が高く,
		      もっとも低いのが\(\to\)であると約束する.
		\item アリティ2の関数記号や関係記号については,\(\obj{x} \obj{\ast} \obj{y}\)や\(\obj{x} \le \obj{y}\)のように
		      その記号が真ん中に来るように配置して表記してよい.
	\end{itemize}
	こうすると,\(\paren{\paren{\varphi \land \paren{\lnot \psi}} \to \xi}\)は\(\varphi \land \lnot \psi \to \xi\)のように見やすくできる.
	これは,通常の数学における数の演算において,\(\mathord{\times}\)が\(\mathord{+}\)よりも優先度が高いとみなして
	\(\paren{\paren{2 \times 3} + 4} = 2 \times 3 + 4\)と略記することで可読性の向上を図るのとまったく同じである.
	これは人間が目で見る際に楽をするための約束事であって,数学的な議論の帰結ではないことに注意しておこう.
	本書でも,これらの略記表現を積極的に利用する.
\end{Note}

\begin{Note}
	例えば論理式\(\varphi, \psi, \chi\)に対する以下の2つの論理式
	\begin{align*}
		\varphi \land \paren{\psi \land \chi}, \\
		\paren{\varphi \land \psi} \land \chi
	\end{align*}
	は,それが表すと期待される主張を考えればどちらも「同じ論理式」であって,これらを
	\[
		\varphi \land \psi \land \chi
	\]
	と表記してしまってもよさそうに見える.
	本書では,これらの「素朴に考えれば問題は生じなさそう」という略記表現も積極的に使用することとする.
	確かに,素朴には問題なさそうでも本当に以降の議論に支障が生じないかどうかまでは実際に確かめるまではわからないというのは事実である.
	しかし,少なくとも本書の範囲内ではそれによって問題が生じることはまずない.
\end{Note}

\begin{Note} \label[Note]{Note:languageomission}
	ここまで「言語\(\symcal{L}\)において」とか「\(\symcal{L}\)項」のように,用いる言語を明示して議論を進めてきた.
	しかし,以下で行われるのは特定の言語に依存しない議論がほとんどである.
	そのため,特に断りがない限りは「言語\(\symcal{L}\)」の表記は省略することとする.
	「\(\symcal{L}\)項」や「\(\symcal{L}\)論理式」は単に「項」や「論理式」と呼称する.
\end{Note}


\section{変数の出現と代入可能性} \label{sec:substitution}

数学においては,一般論に具体例を当てはめることによって議論を進めることが多い.
これの形式的記号列の世界での対応物は,
項や論理式に登場する変数記号に別のものを当てはめることである.
この操作を定式化するためには,いくつかのステップを踏む必要がある.
なお,正確に書くとそれなりに議論が長くなるため,以下では相当にラフに記述していることに注意されたい.
ラフな部分はいずれも帰納的定義によって精密化できる.


\begin{Def} \label[Def]{Def:occurence}
	論理式
	\begin{align*}
		\varphi \colon \paren{\cdots \paren{\forall x \paren{\cdots x \cdots } \cdots}}, \\
		\psi \colon \paren{\cdots \paren{\exists x \paren{\cdots x \cdots } \cdots}}
	\end{align*}
	の変数記号\(x\)のように,\(\forall, \exists\)とともに出現している変数記号は
	その論理式に%
	\index[widx]{そくばくしゅつげん@束縛出現}%
	\term{束縛出現}しているといい,
	\(\forall, \exists\)をともなわずに出現している変数記号は
	その論理式に%
	\index[widx]{じゆうしゅつげん@自由出現}%
	\term{自由出現}しているという.
	また,論理式\(\varphi\)に束縛出現する変数記号全体の集合と\(\varphi\)に自由出現する変数記号全体の集合を,それぞれ%
	\index[sidx]{\(\apply{\BV}{\varphi}\):論理式に束縛出現する変数記号全体の集合}%
	\index[sidx]{\(\apply{\FV}{\varphi}\):論理式に自由出現する変数記号全体の集合}%
	\begin{align}
		\apply{\BV}{\varphi}, \\
		\apply{\FV}{\varphi}
	\end{align}
	と表す.%
	\index[sidx]{\(\apply{\Var}{t}\):項に出現する変数記号全体の集合}%
	さらに,項\(t\)に出現する変数記号全体の集合を
	\begin{equation}
		\apply{\Var}{t}
	\end{equation}
	と表す.%
	\(\apply{\FV}{\varphi} = \emptyset\)であるような論理式\(\varphi\)は%
	\index[widx]{ろんりしき@論理式!へいろんりしき@閉---}%
	\term{閉論理式},あるいは%
	\index[widx]{ぶん@文|see{閉論理式}}%
	\term{文}と呼ぶ.
	\(\apply{\Var}{t} = \emptyset\)となる項\(t\)は%
	\index[widx]{こう@項!へいこう@閉---}%
	\term{閉項}であるという.
\end{Def}

\begin{Note} \label[Note]{Note:languagesentence}
	言語\(\symcal{L}\)を明示する文脈においては
	閉論理式や文,閉項はそれぞれ\(\symcal{L}\)閉論理式,\(\symcal{L}\)文,\(\symcal{L}\)閉項と呼ぶ.
\end{Note}

\begin{Ex} \label[Ex]{Ex:occurence}
	群論の言語\(\symcal{L} = \set{\obj{\ast}, \obj{e}, {}^{\obj{-1}}}\)において,
	論理式
	\begin{equation*}
		\varphi \colon \paren{\forall \obj{a} \paren{\forall y \paren{
					\apply{\mathord{\obj{\ast}}}{\obj{x}, \obj{y}} \obj{\objeq} \apply{\mathord{\ast}}{\obj{y}, \obj{x}}
				}}}
	\end{equation*}
	に対しては
	\begin{align*}
		\apply{\BV}{\varphi} = \set{\obj{a}, \obj{y}}, \\
		\apply{\FV}{\varphi} = \set{\obj{x}}
	\end{align*}
	が成り立つ.この\(\obj{a}\)のように,
	\(\forall, \exists\)記号の直後でのみ出現する変数記号も束縛出現するとみなす.
\end{Ex}

以上の準備のもと,代入操作を定式化したいのだが,先に代入可能性について論ずる必要がある.

\index[widx]{だいにゅう@代入}
\begin{Def} \label[Def]{Def:cansubstitution}
	\(\varphi\)を論理式,\(x\)を変数記号,\(t\)を項とする.
	以下の2条件をともに満たす変数記号\(y\)が存在するとき,
	\(t\)は\(\varphi\)中の\(x\)に%
	\index[widx]{だいにゅう@代入!だいにゅうふかのう@---不可能}%
	\term{代入不可能}であるといい,
	そのような\(y\)が存在しないとき,
	\(t\)は\(\varphi\)中の\(x\)に%
	\index[widx]{だいにゅう@代入!だいにゅうかのう@---可能}%
	\term{代入可能}%
	であるという:
	\begin{itemize}
		\item \(\varphi\)が
		      \(\paren{\cdots \paren{\forall y \paren{\cdots x \cdots} \cdots} \cdots}\)
		      か
		      \(\paren{\cdots \paren{\exists y \paren{\cdots x \cdots} \cdots} \cdots}\)
		      の形の論理式である.ただし,この\(x\)は\(\varphi\)に自由出現しているものとする.
		\item \(y \in \apply{\Var}{t}\)である.
	\end{itemize}
\end{Def}

代入可能性については,例を見るのが手っ取り早い.

\begin{Ex} \label{Ex:cansubstitution}
	群論の言語\(\symcal{L} = \set{\obj{\ast}, \obj{e}, {}^{\obj{-1}}}\)において,
	論理式
	\begin{equation*}
		\varphi \colon \paren{\forall \obj{a} \paren{\forall y \paren{
					\apply{\mathord{\obj{\ast}}}{\obj{x}, \obj{y}} \obj{\objeq} \apply{\mathord{\ast}}{\obj{y}, \obj{x}}
				}}}
	\end{equation*}
	中の\(\obj{x}\)に項\(\apply{\obj{\ast}}{\obj{y}, \obj{e}}\)は代入不可能である.
	一方で,\(\obj{x}\)に項\(\obj{e}\)は代入可能である.
	また,\(\obj{a}, \obj{y}\)だけでなく
	\(\varphi\)に出現しない変数記号すべてに対してあらゆる項が代入可能である.
\end{Ex}

\index[sidx]{\(\subst{\varphi}{t/x}\):論理式への代入}
\index[sidx]{\(\subst{t}{s/x}\):項への代入}
\begin{Def} \label[Def]{Def:substitution}
	論理式\(\varphi\)中の変数記号\(x\)に項\(t\)が代入可能であるとき,
	\(\varphi\)に自由出現している\(x\)すべてを\(t\)に置き換えて得られる論理式を
	\begin{equation}
		\subst{\varphi}{t/x}
	\end{equation}
	と表す.

	また,項\(t\)に現れる変数記号\(x\)すべてを項\(s\)に置き換えて得られる項も同様に
	\begin{equation}
		\subst{t}{s/x}
		\label{eq:substterm}
	\end{equation}
	と表す.
\end{Def}

\begin{Ex} \label[Ex]{Ex:substitution}
	\Cref{Ex:cansubstitution}における\(\varphi\)において
	\begin{equation*}
		\subst{\varphi}{\obj{e} / \obj{x}} \colon
		\paren{\forall \obj{a} \paren{\forall \obj{y} \paren{
					\apply{\mathord{\obj{\ast}}}{\obj{e}, \obj{y}} \obj{\objeq} \apply{\mathord{\ast}}{\obj{y}, \obj{e}}
				}}}
	\end{equation*}
	である.また,\(\subst{\varphi}{\obj{e} / \obj{a}}\)と\(\subst{\varphi}{\obj{e} / \obj{y}}\)はいずれも\(\varphi\)そのものである.
\end{Ex}

\begin{Note}
	\(\varphi\)を論理式,\(x_1, x_2, \dots, x_n\)を\(\varphi\)に束縛出現しない変数記号とするとき,
	\(\varphi\)のことを
	\begin{equation}
		\apply{\varphi}{x_1, x_2, \dots, x_n}
		\label{eq:freevariablefunction}
	\end{equation}
	と表すことがある.このとき,\(\varphi\)中の\(x_1, x_2, \dots, x_n\)に代入可能な項\(t_1, t_2, \dots, t_n\)を代入して得られる論理式は
	\begin{equation}
		\apply{\varphi}{t_1, t_2, \dots, t_n}
		\label{eq:freevariablefunctionsubstitution}
	\end{equation}
	と表すことが多い.
	ここで,\(t_1, t_2, \dots, t_n\)に現れる変数記号が\(\varphi\)に束縛出現する場合,その束縛出現する記号を別の変数記号に置き換え,
	\(t_1, t_2, \dots, t_n\)に現れる変数記号が\(\varphi\)に束縛出現しないようにしておく.
	このような置き換えは,しばしば暗黙的に行われる.
	そのようにしても議論に影響がないことは\cref{chap:semantics}の\cref{Thm:alphaequivalent}による.
\end{Note}

記述を簡素化するために,追加で略記表現をを導入する.

\index[sidx]{\(\formulaequiv\):(論理式の)同値}
\index[sidx]{\(\uexists\):一意存在}
\begin{Def} \label{Def:AbbreviatioForFormula}
	論理式\(\varphi, \psi\)に対し,論理式
	\begin{align*}
		\paren{\varphi \to \psi} \land \paren{\psi \to \varphi}
	\end{align*}
	を
	\begin{equation}
		\varphi \formulaequiv \psi
		\label{eq:formulaequiv}
	\end{equation}
	と略記する.
	また,変数記号\(x_1, x_2\)が自由出現しない論理式\(\varphi\)に対し,論理式
	\begin{align*}
		\exists x \varphi \land \forall x \forall y \paren{\subst{\varphi}{x_1/x} \land \subst{\varphi}{x_2} \to x_1 \objeq x_2}
	\end{align*}
	を
	\begin{equation}
		\uexists x \varphi
		\label{eq:uexists}
	\end{equation}
	と略記する.

	\Cref{eq:formulaequiv}は「\(\varphi\)と\(\psi\)は同値である」
	ことを表現することが期待される論理式で,
	\cref{eq:uexists}は「\(\varphi\)となる\(x\)がただ1つ存在する」
	ことを表現することが期待される論理式である.
\end{Def}


\section{理論とその例} \label[section]{sec:Theory}

前節までは項や論理式に関する一般論を述べた.
ここでは,実際の数学理論がどう形式化されるかについて少しだけ触れる.

\begin{Def} \label[Def]{Def:theory}
	\(\symcal{L}\)を言語とするとき,\(\symcal{L}\)文からなる集合を%
	\index[widx]{りろん@理論}%
	\(\symcal{L}\)\term{理論},あるいは%
	\index[widx]{こうりけい@公理系}%
	\(\symcal{L}\)\term{公理系}
	という.
	言語\(\symcal{L}\)を明示しない文脈では,\(\symcal{L}\)理論や\(\symcal{L}\)公理系は単に理論や公理系と呼ばれる.
\end{Def}

今後,例として登場する言語や理論を先んじていくつか挙げておく.

\index[sidx]{\(\GP\):群の理論}
\index[sidx]{\(\symcal{L}_{\GP}\):群の言語}
\begin{Ex} \label[Ex]{Ex:grouptheory}
	群の言語を\(\symcal{L}_{\GP} = \set{\obj{\ast}, \obj{e}, {}^{\obj{-1}}}\)とする.
	ここで,\(\obj{\ast}\)はアリティ2の関数記号,\(\obj{e}\)は定数記号,\({}^{\obj{-1}}\)はアリティ1の関係記号である.
	群の理論\(\GP\)は,次の3つの文からなる\(\symcal{L}_{\GP}\)理論であると考えることができる:
	\begin{enumerate}
		\item \(\forall \obj{x} \forall \obj{y} \forall \obj{z}
		      \paren{\paren{\obj{x} \obj{\ast} \obj{y}} \obj{\ast} \obj{z} \objeq \obj{x} \obj{\ast} \paren{\obj{y} \obj{\ast} \obj{z}}}\),
		\item \(\forall \obj{x} \paren{\obj{e} \obj{\ast} \obj{x} \objeq \obj{x}}\),
		\item \(\forall \obj{x} \paren{\obj{x}^{\obj{-1}} \obj{\ast} \obj{x} \objeq \obj{e}}\).
	\end{enumerate}
	ここで,\(\apply{{}^{\obj{-1}}}{\obj{x}}\)を\(\obj{x} ^ {\obj{-1}}\)と略記した.
\end{Ex}

\begin{Que} \label[Que]{Que:invalidtheory}
	群の言語を\(\symcal{L} = \set{\obj{\ast}}\)だと考えて,群の理論を次のような\(\symcal{L}\)理論として書き下そうとする場合がある.
	\begin{enumerate}
		\item \(\forall \obj{x} \forall \obj{y} \forall \obj{z}
		      \paren{\paren{\obj{x} \obj{\ast} \obj{y}} \obj{\ast} \obj{z} \objeq \obj{x} \obj{\ast} \paren{\obj{y} \obj{\ast} \obj{z}}},\)
		\item \(\exists \obj{e} \forall \obj{x} \paren{\obj{e} \obj{\ast} \obj{x} \objeq \obj{x}},\)
		\item \(\forall \obj{x} \exists \obj{y} \paren{\obj{y} \obj{\ast} \obj{x} \objeq \obj{e}}.\)
	\end{enumerate}
	しかし,この書き下し方は構文的に不適切である.その理由を述べよ.
\end{Que}

\index[sidx]{\(\Ring\):環の理論}
\index[sidx]{\(\symcal{L}_{\Ring}\):環の言語}
\begin{Ex} \label[Ex]{Ex:Ring}
	環の言語を\(\symcal{L}_{\Ring} = \Set{\obj{+}, \obj{\cdot}, \obj{0}, \obj{1}}\)とする.
	環の理論\(\Ring\)は,以下の文からなる\(\symcal{L}_{\Ring}\)理論であると考えることができる:
	\begin{enumerate}
		\item \(\forall \obj{x} \forall \obj{y} \forall \obj{z} \paren{\paren{\obj{x} \obj{+} \obj{y}} \obj{+} \obj{z} \objeq \obj{x} \obj{+} \paren{\obj{y} \obj{+} \obj{z}}},\)
		\item \(\forall \obj{x} \paren{\obj{0} \obj{+} \obj{x} \objeq \obj{x}},\)
		\item \(\forall \obj{x} \exists \obj{y} \paren{\obj{y} \obj{+} \obj{x} \objeq \obj{0}},\)
		\item \(\forall \obj{x} \forall \obj{y} \paren{\obj{x} \obj{+} \obj{y} \objeq \obj{y} \obj{+} \obj{x}},\)
		\item \(\forall \obj{x} \forall \obj{y} \forall \obj{z}
		      \paren{\paren{\obj{x} \obj{\cdot} \obj{y}} \obj{\cdot} \obj{z} \objeq \obj{x} \obj{\cdot} \paren{\obj{y} \obj{\cdot} \obj{z}}},\)
		\item \(\forall \obj{x} \paren{\paren{\obj{1} \obj{\cdot} \obj{x} \objeq \obj{x}} \land \paren{\obj{x} \obj{\cdot} \obj{1} \objeq \obj{x}}},\)
		\item \(\forall \obj{x} \forall \obj{y} \forall \obj{z}
		      \paren{\obj{x} \obj{\cdot} \paren{\obj{y} \obj{+} \obj{z}} \objeq \paren{\obj{x} \obj{\cdot} \obj{y}} \obj{+} \paren{\obj{x} \obj{\cdot} \obj{z}}},\)
		\item \(\forall \obj{x} \forall \obj{y} \forall \obj{z}
		      \paren{ \paren{\obj{x} \obj{+} \obj{y}} \obj{\cdot} \obj{z} \objeq \paren{\obj{x} \obj{\cdot} \obj{z}} \obj{+} \paren{\obj{y} \obj{\cdot} \obj{z}}}.\)
	\end{enumerate}
	最初の文3つは,\(\symcal{L}_{\GP}\)における\(\obj{\ast}\)を\(\obj{+}\)に置き換えたものであることに注意せよ.
\end{Ex}

\index[sidx]{\(\POSET\):狭義の半順序の理論}
\index[sidx]{\(\TOSET\):狭義の全順序の理論}
\index[sidx]{\(\symcal{L}_{\POSET}\):狭義の半順序の言語}
\begin{Ex} \label{Ex:orderedset}
	狭義の半順序の言語を\(\symcal{L}_{\POSET} = \Set{\obj{<}}\)とする.
	狭義の半順序の理論\(\POSET\)は,以下の2つの文からなる\(\symcal{L}_{\POSET}\)理論と考えることができる.
	\begin{enumerate}
		\item \(\forall \obj{x} \lnot \paren{x \obj{<} \obj{x}},\)
		\item \(\forall \obj{x} \forall \obj{y} \forall \obj{z}
		      \paren{\obj{x} \obj{<} \obj{y} \land \obj{y} \obj{<} \obj{z} \to \obj{x} \obj{<} \obj{z}}\)
	\end{enumerate}
	なお,理論\(\POSET\)に
	\[
		\forall \obj{x} \forall \obj{y}
		\paren{\obj{x} \obj{<} \obj{y} \lor \obj{y} \obj{<} \obj{x} \lor \obj{x} \objeq \obj{y}}
	\]
	を加えた理論は,狭義の全順序の理論を表していると考えることができる.
	この理論を\(\TOSET\)と表すこととする.
\end{Ex}

\index[sidx]{\(\OrderedRing\):順序環の理論}
\index[sidx]{\(\symcal{L}_{\OrderedRing}\):順序環の言語}
\begin{Ex} \label[Ex]{Ex:OrderedRing}
	順序環の言語を\(\symcal{L}_{\OrderedRing} = \Set{\obj{+}, \obj{\cdot}, \obj{0}, \obj{1}, \obj{<}}\)とする.
	順序環の理論\(\OrderedRing\)は,以下の文からなる\(\symcal{L}_{\OrderedRing}\)理論と考えることができる:
	\begin{enumerate}
		\item \Cref{Ex:Ring}で挙げた8つの論理式すべて,
		\item \Cref{Ex:orderedset}で挙げた3つの論理式すべて,
		\item \(\forall \obj{x} \forall \obj{y} \paren{\obj{x} \obj{\cdot} \obj{y} \objeq \obj{y} \obj{\cdot} \obj{x}},\)
		\item \(\forall \obj{x} \forall \obj{y} \forall \obj{z}
		      \paren{\obj{x} \obj{<} \obj{y} \to \obj{x} \obj{+} \obj{z} \obj{<} \obj{y} \obj{+} \obj{z}},\)
		\item \(\forall \obj{x} \forall \obj{y} \forall \obj{z}
		      \paren{\obj{x} \obj{<} \obj{y} \land \obj{0} \obj{<} \obj{z} \to \obj{x} \obj{\cdot} \obj{z} \obj{<} \obj{y} \obj{\cdot} \obj{z}}.\)
	\end{enumerate}
\end{Ex}

\index[sidx]{\(\ProjectiveGeometryPart\):射影平面幾何の部分体系}
\index[sidx]{\(\symcal{L}_{\ProjectiveGeometryPart}\):射影平面幾何の言語}
\begin{Ex} \label[Ex]{Ex:ProjectiveGeometry}
	\NewDocumentCommand{\LiesOn}{}{\mathrel{\obj{\varepsilon}}}
	射影平面幾何の言語を\(\symcal{L}_{\ProjectiveGeometryPart}\)を\(\symcal{L}_{\ProjectiveGeometryPart} = \Set{\obj{P}, \obj{L}, \LiesOn}\)
	とする.ここで,\(\obj{P}, \obj{L}\)はアリティ1の関係記号,\(\obj{\varepsilon}\)はアリティ2の関係記号である.
	このとき,射影平面幾何の理論の部分体系\(\ProjectiveGeometryPart\)として,以下の文からなる\(\symcal{L}_{\ProjectiveGeometryPart}\)理論を考えることができる:
	\begin{enumerate}
		\item \(\forall \obj{x} \paren{\apply{\obj{P}}{\obj{x}} \formulaequiv \lnot \apply{\obj{L}}{\obj{x}}},\)
		\item \(\forall \obj{x} \forall \obj{y} \paren{\obj{x} \LiesOn \obj{y} \to \apply{\obj{P}}{\obj{x}} \land \apply{\obj{L}}{\obj{y}}},\)
		\item \(\forall \obj{x} \forall \obj{y} \paren{\apply{\obj{P}}{\obj{x}} \land \apply{\obj{P}}{\obj{y}} \land \lnot \paren{\obj{x} \objeq \obj{y}}
			      \to \uexists z \paren{\obj{x} \LiesOn \obj{z} \land \obj{y} \LiesOn \obj{z}}},\)
		\item \(\forall \obj{x} \forall \obj{y} \paren{\apply{\obj{L}}{\obj{x}} \land \apply{\obj{L}}{y} \land \lnot \paren{\obj{x} \objeq \obj{y}}
			      \to \uexists z \paren{\obj{z} \LiesOn \obj{x} \land \obj{z} \LiesOn \obj{y}}},\)
		\item \(\apply{C}{x, y, z}\)を
		      \[
			      \lnot \exists \obj{w} \paren{x \LiesOn \obj{w} \land y \LiesOn \obj{w} \land z \LiesOn \obj{w}}
		      \]
		      の略記,
		      \(\apply{n}{a, b, c, d}\)を\[
			      \lnot \paren{a \objeq b} \land \lnot \paren{a \objeq c} \land \lnot \paren{a \objeq d} \land \lnot \paren{b \objeq c}
			      \land \lnot \paren{c \objeq d}\]
		      の略記としたときの
		      \[
			      \exists \obj{a} \exists \obj{b} \exists \obj{c} \exists \obj{d}
			      \paren{
				      \apply{n}{\obj{a}, \obj{b}, \obj{c}, \obj{d}}
				      \land \apply{C}{\obj{a}, \obj{b}, \obj{c}}
				      \land \apply{C}{\obj{a}, \obj{b}, \obj{d}}
				      \land \apply{C}{\obj{a}, \obj{c}, \obj{d}}
				      \land \apply{C}{\obj{b}, \obj{c}, \obj{d}}
			      }.
		      \]
	\end{enumerate}
	ただし,\(\obj{P}, \obj{L}\)は\(\apply{\obj{P}}{x}, \apply{\obj{L}}{x}\)でそれぞれ「\(x\)は点である」ことと「\(x\)は直線である」
	ことを表現する意図で導入した記号であり,
	\(\LiesOn\)は\(x \LiesOn y\)で「\(x\)は\(y\)上にある」ことを表現する意図で導入した記号である.
	すると,上記の各文は順に
	\begin{enumerate}
		\item 点と直線は異なる対象である.
		\item \(x\)が\(y\)上にあるならば,\(x\)は点であり,\(y\)は直線である.
		\item 任意の相異なる2点を通る直線がただ1つ存在する.
		\item 相異なる2直線はつねに1点で交わる.
		\item どの3つも共線でないような4点が存在する.
	\end{enumerate}
	を表現することを意図した論理式であると解釈できる.
	上記\(\apply{C}{x, y, z}\)は「3点\(x, y, z\)は共線ではない」という主張に対応する論理式であると考えることができる.
\end{Ex}

\begin{Note}
	\cref{Ex:ProjectiveGeometry}では\(c\)や\(n\)のようなもとの言語になかった記号が理論を記述するために使用されている.
	しかし,ここでの\(\apply{c}{x, y, y}\)や\(\apply{n}{a, b, c, d}\)は単に既知の論理式の略記表現であり,
	アリティ3, 4の新しい関係記号が導入されたわけではないことに注意しよう.
	\(c, n\)を使った論理式は,単に\(\apply{c}{x, y, z}, \apply{n}{a, b, c, d}\)をその定義に置き換えることによって
	いつでも\(c, n\)があらわれない形に書き直すことができる.
\end{Note}

\begin{Note}
	ここで導入した体系\(\ProjectiveGeometryPart\)は射影平面幾何の部分体系であり,射影平面幾何そのものではない.
	例えばこの体系\(\ProjectiveGeometryPart\)では,射影平面幾何の定理として有名なPappusの定理は証明できないことが知られている.

	また,平面幾何においては「点」と「直線」を区別して扱いたいので,対応する一階述語論理の理論においてそれを陽に提示する場合には
	\(\obj{P}\)や\(\obj{L}\)のような関係記号が必要となる.
	そのため,射影空間幾何の理論をこのやり方で作ろうとした場合は「平面」に対応する新しい記号が必要となる.
	すなわち,次元を1つ増やすごとにアリティ1の関係記号が1つ増えることになる.
\end{Note}


\begin{Que} \label[Que]{Que:ProjectiveGeometrydual}
	\NewDocumentCommand{\LiesOn}{}{\mathrel{\obj{\varepsilon}}}
	\Cref{Ex:ProjectiveGeometry}における言語\(\symcal{L}_{\ProjectiveGeometryPart}\)を考える.
	\(\sigma\)を\(\symcal{L}_{\ProjectiveGeometryPart}\)論理式とするとき,\(\sigma\)に対して以下の操作を施して得られる
	\(\symcal{L}_{\ProjectiveGeometryPart}\)論理式を\(\tilde{\sigma}\)とする%
	\footnote{%
		論理式における代入と同じように,この操作は帰納的定義によって精密化できる.%
	}%
	:
	\begin{enumerate}
		\item \(\sigma\)にあらわれる\(\obj{P}, \obj{L}\)をすべて入れ替える.
		\item \(\sigma\)にあらわれる\(x \LiesOn y\)という形の部分をすべて\(y \LiesOn x\)に変更する.
	\end{enumerate}
	この\(\tilde{\sigma}\)を\(\sigma\)の双対と呼ぶ.

	このとき,\cref{Ex:ProjectiveGeometry}で述べた体系\(\ProjectiveGeometryPart\)の各公理の双対を求めよ.
\end{Que}
\chapter{1階述語論理の意味論}
\label[chapter]{chap:semantics}

本章では,1階述語論理の意味論について述べる.
\Cref{chap:formulize}から繰り返し「数理論理学で扱うのは形式的記号列である」と
繰り返し述べてきたが,
それと同時にそれらの記号列は通常の数学において何らかの「意味」を見出すことも期待していたのであった.
\Cref{chap:syntax}で定義した項や論理式に対して数学的な「意味」を付与するというのが本章での主題である.
このことは,形式的記号列の世界と通常の数学の世界との橋渡しを行っていると考えることもできる.
理論のモデルや同型の概念に覚えがある読者は多いだろう.
普段使っている概念を一歩引いた形で見直すことにより得られるものも多いはずである.

\newpage

\section{理論とそのモデル} \label[section]{sec:model}

我々は,項や論理式を純粋な形式的記号列として導入し,
数学理論の形式化を試みた.
例として群論や順序の理論を取り扱ったが,通常の数学においては
これらは集合や写像の言葉で書かれることがほとんどである.
両者の間の関係を考えよう.

\begin{Def} \label[Def]{Def:structure}
	\(\symcal{L}\)を言語とする.このとき,空でない集合\(M\)と写像
	\(F \colon L \to M \cup \bigcup_{n > 0} \paren{\powerset{M^n} \cup M^{M^n}}\)の%
	\footnote{%
		集合\(X\)が与えられたとき,\(X\)の部分集合全体からなる集合を\(X\)の
		\index[sidx]{\(\powerset{X}\):べき集合}%
		\index[widx]{べきしゅうごう@べき集合}%
		\term{べき集合}%
		といい,\(\powerset{X}\)と表す.
	}%
	対\(\symcal{M} = \pair{M, F}\)が\(\symcal{L}\)%
	\index[widx]{こうぞう@\(\symcal{L}\)構造}%
	\term{構造}%
	であるとは,\(F\)が以下の条件をすべて満たすことをいう:
	\begin{enumerate}
		\item \(c \in L\)が定数記号ならば\(\apply{F}{c} \in M\),すなわち\(\apply{F}{c}\)は\(M\)の元である.
		\item \(f \in L\)がアリティ\(n\)の関数記号ならば\(\apply{F}{f} \in M^{M^n}\),すなわち\(\apply{F}{f}\)は\(M^n\)から\(M\)への写像である.
		\item \(r \in L\)がアリティ\(n\)の関係記号ならば\(\apply{F}{r} \in \powerset{M^n}\),すなわち\(\apply{F}{r}\)は\(M\)上の\(n\)項関係である.
	\end{enumerate}
	また\(\xi \in L\)の\(F\)による像\(\apply{F}{\xi}\)を
	\index[sidx]{\(\interpretation{\symcal{M}}{\xi}\):解釈}
	\begin{equation}
		\interpretation{\symcal{M}}{\xi}
		\label{eq:interpretation}
	\end{equation}
	と表記し,\(\xi\)の\(\symcal{L}\)構造\(\symcal{M}\)による%
	\index[widx]{かいしゃく@解釈}%
	\term{解釈}%
	と呼ぶことが多い.

	\(\symcal{M} = \pair{M, F}\)が\(\symcal{L}\)構造のとき,\(M\)を\(\symcal{M}\)の%
	\index[widx]{たいしょうりょういき@対象領域}%
	\term{対象領域}%
	と呼ぶ.
\end{Def}

\Cref{Def:structure}は,形式的記号列として与えられる言語\(L\)の元が
通常の数学においてはどういうものに相当するかを定義するものである.
項や論理式についても同様の定義をしたいのだが,まずその準備として言語の拡張を定義しておく.

\begin{Def} \label{Def:namelanguage}
	\(\symcal{L}\)を言語とし,\(\symcal{M} = \pair{M, F}\)を\(\symcal{L}\)構造とする.
	このとき,集合\(M\)の元\(a\)ごとに新たな定数記号\(c_a\)を用意し,
	これを\(\symcal{L}\)に付け加えた言語\(\symcal{L} \cup \Set{c_a | a \in M}\)を考えることができる.
	この言語を%
	\index[sidx]{\(\languagewithname{\symcal{L}}{\symcal{M}}\):構造による言語の拡張}%
	\begin{equation}
		\languagewithname{\symcal{L}}{\symcal{M}}
		\label{eq:languagewithname}
	\end{equation}
	と表すことにする.各\(a \in M\)に対する\(c_a\)を\(a\)の%
	\index[sidx]{\(c_a\):\(a\)の名前}%
	\index[widx]{なまえ@名前}%
	\term{名前}という.
\end{Def}

\begin{Note}
	\(\symcal{L}\)を言語とする.
	\(\symcal{L}\)構造\(\symcal{M}\)は,の元のうち
	\(a \in M\)の名前\(c_a\)の\(\symcal{M}\)による解釈を
	\begin{equation}
		\interpretation{\symcal{M}}{c_a} = a
		\label{eq:languagewithnameinterpretation}
	\end{equation}
	と定めることにより,自然に\(\languagewithname{\symcal{L}}{\symcal{M}}\)構造とみなせる.
	以後,\(\symcal{L}\)構造\(\symcal{M}\)はこの解釈によって
	\(\languagewithname{\symcal{L}}{\symcal{M}}\)構造でもあるとみなす.
\end{Note}

まずは項に対する解釈を定義しよう.

\begin{Def} \label{Def:interpretationforterm}
	\(\symcal{L}\)を言語とし,\(\symcal{M}\)を\(\symcal{L}\)構造とする.
	\(t\)を\(\languagewithname{\symcal{L}}{\symcal{M}}\)閉項として,\(t\)の\(\symcal{M}\)による解釈\(\interpretation{\symcal{M}}{t}\)
	を以下のように帰納的に定義する:
	\begin{enumerate}
		\item \(t\)が言語\(\languagewithname{\symcal{L}}{\symcal{M}}\)における定数記号\(c\)であるならば
		      \(\interpretation{\symcal{M}}{t} = \interpretation{\symcal{M}}{c}\)と定める.
		\item \(t\)がアリティ\(n\)の関数記号\(f\)と\(\languagewithname{\symcal{L}}{\symcal{M}}\)閉項
		      \(t_1, t_2, \dots, t_n\)を用いて\(\apply{f}{t_1, t_2, \dots, t_n}\)と表されるならば
		      \begin{equation}
			      \interpretation{\symcal{M}}{t}
			      = \apply{\interpretation{\symcal{M}}{f}}{
				      \interpretation{\symcal{M}}{t_1},
				      \interpretation{\symcal{M}}{t_2},
				      \dots,
				      \interpretation{\symcal{M}}{t_n}
			      }
			      \label{eq:terminterpretation}
		      \end{equation}
		      と定める.
	\end{enumerate}
\end{Def}

論理式においても\cref{Def:structure}や\cref{Def:interpretationforterm}
と同じような定義をしたい.しかし,与えられた集合の元や写像,関係として定式化できる
言語や項とは違い,論理式に対応する通常の数学における概念は「数学的な主張」である.
これはすでにあいまいさなく定式化されているとはいいがたいので,代わりに充足関係と呼ばれる関係を定義する.

\begin{Def} \label{Def:semanticimplies}
	\(\symcal{L}\)を言語とし,\(\symcal{M} = \pair{M, F}\)を\(\symcal{L}\)構造とする.
	このとき,\(\languagewithname{\symcal{L}}{\symcal{M}}\)閉論理式\(\varphi\)に対する
	\index[sidx]{\(\symcal{M} \satisfy \varphi\):充足関係}
	\begin{equation}
		\symcal{M} \satisfy \varphi
		\label{eq:structuresatisfy}
	\end{equation}
	を%
	\footnote{%
		この記号「\(\satisfy\)」は「ダブルターンスタイル記号」だとか「ゲタ記号」だとか呼ばれているようである.
	}%
	,以下のように帰納的に定義する:
	\begin{enumerate}
		\item \(t_1, t_2\)を\(\languagewithname{\symcal{L}}{\symcal{M}}\)閉項とするとき,
		      \[
			      \symcal{M} \satisfy t_1 \objeq t_2                                       \metaequivalent \interpretation{\symcal{M}}{t_1} = \interpretation{\symcal{M}}{t_2}
		      \]
		      とする.
		\item \(r \in \symcal{L}\)をアリティ\(n\)の関係記号,
		      \(t_1, t_2, \dots, t_n\)を\(\languagewithname{\symcal{L}}{\symcal{M}}\)閉項とするとき,
		      \(\symcal{M} \satisfy \apply{r}{t_1, t_2, \dots, t_n}
		      \metaequivalent \pair{
			      \interpretation{\symcal{M}}{t_1},
			      \interpretation{\symcal{M}}{t_2},
			      \dots,
			      \interpretation{\symcal{M}}{t_n}
		      } \in \interpretation{\symcal{M}}{r}
		      \)
		      とする.
		\item \(\symcal{M} \satisfy \bot\)は成り立たないとする.
		\item \(\varphi, \psi\)を\(\languagewithname{\symcal{L}}{\symcal{M}}\)閉論理式とするとき,
		      \[
			      \symcal{M} \satisfy \lnot \varphi      \metaequivalent \text{\(\symcal{M} \satisfy \varphi\)でない}
		      \]
		      とする.
		\item \(\varphi, \psi\)を\(\languagewithname{\symcal{L}}{\symcal{M}}\)閉論理式とするとき,
		      \[
			      \symcal{M} \satisfy \varphi \lor \psi  \metaequivalent \text{\(\symcal{M} \satisfy \varphi\)または\(\symcal{M} \satisfy \psi\)}
		      \]
		      とする.
		\item \(\varphi, \psi\)を\(\languagewithname{\symcal{L}}{\symcal{M}}\)閉論理式とするとき,
		      \[
			      \symcal{M} \satisfy \varphi \land \psi \metaequivalent \text{\(\symcal{M} \satisfy \varphi\)かつ\(\symcal{M} \satisfy \psi\)}
		      \]
		      とする.
		\item \(\varphi, \psi\)を\(\languagewithname{\symcal{L}}{\symcal{M}}\)閉論理式とするとき,
		      \[
			      \symcal{M} \satisfy \varphi \to \psi \metaequivalent \text{\(\symcal{M} \satisfy \lnot \varphi\)または\(\symcal{M} \satisfy \psi\)}
		      \]
		      とする%
		      \footnote{%
			      ここでは\(\symcal{M} \satisfy \varphi\)ならば\(\symcal{M} \satisfy \psi\)と書きたくなるところだが,あえてそうしていない.
			      これは\(\symcal{M} \satisfy \varphi\)でないばあいに\(\symcal{M} \satisfy \varphi \to \psi\)といえるかどうかに
			      若干のあいまいさが残ってしまうことを避けるためである.
			      通常の数学では,ある主張\(A\)が成り立たない場合は「\(A\)ならば\(B\)」は成り立つものとして考えるためそれでも問題ないと思われるが,
			      ここではあいまいさを極力排することを意図してこのような記述とした.
			      言い換えれば『\(A\)が成り立たない場合は「\(A\)ならば\(B\)」は成り立つと決めた』ということとなる.
		      }%
		      .
		\item \(\varphi\)を\(\languagewithname{\symcal{L}}{\symcal{M}}\)論理式,\(x\)を変数記号とするとき,
		      \(\exists x \varphi\)が\(\languagewithname{\symcal{L}}{\symcal{M}}\)閉論理式(つまり\(\varphi\)に自由出現する変数記号が\(x\)以外にない)であれば
		      \[
			      \symcal{M} \satisfy \exists x \varphi  \metaequivalent \text{\(\symcal{M} \satisfy \subst{\varphi}{c_a/x}\)となる\(a \in M\)が存在する}
		      \]
		      とする.ただし,\(c_a\)は\(a \in M\)の名前である.
		\item \(\varphi\)を\(\languagewithname{\symcal{L}}{\symcal{M}}\)論理式,\(x\)を変数記号とするとき,
		      \(\forall x \varphi\)が\(\languagewithname{\symcal{L}}{\symcal{M}}\)閉論理式(つまり\(\varphi\)に自由出現する変数記号が\(x\)以外にない)であれば
		      \[
			      \symcal{M} \satisfy \forall x \varphi  \metaequivalent \text{任意の\(a \in M\)に対して\(\symcal{M} \satisfy \subst{\varphi}{c_a/x}\)となる}
		      \]
		      とする.ただし,\(c_a\)は\(a \in M\)の名前である.
	\end{enumerate}

	閉論理式とは限らない論理式\(\varphi\)については,その全称閉包\(\hat{\varphi}\)を用いて
	\begin{equation}
		\symcal{M} \satisfy \varphi \metaequivalent \symcal{M} \satisfy \hat{\varphi}
		\label{eq:structuresatisfyclosure}
	\end{equation}
	と定める.

	\(\symcal{L}\)構造\(\symcal{M}\)と\(\symcal{L}\)論理式\(\varphi\)について,
	\(\symcal{M} \satisfy \varphi\)であるとき,
	\(\symcal{M}\)は\(\varphi\)を%
	\index[widx]{じゅうそく@充足}%
	\term{充足}するという.
	論理式\(\varphi\)が与えられたとき,\(\symcal{M} \satisfy \varphi\)となる\(\symcal{L}\)構造\(\symcal{M}\)が存在するならば,
	\(\varphi\)は%
	\index[widx]{じゅうそく@充足!じゅうそくかのう@---可能}%
	\term{充足可能}であるという.

\end{Def}

\begin{Note}
	\(a \in M\)の名前\(c_a\)は\(\apply{\Var}{c_a} = \emptyset\)を満たすので,
	任意の論理式中の任意の変数記号に代入可能である.
\end{Note}

\begin{Ex} \label[Ex]{Ex:satisfy}
	\(\symcal{L}\)を\cref{Ex:grouptheory}で述べた群論の言語とする.
	整数全体の集合\(\Integers\)に対し,\(\Integers\)を対象領域とする\(\symcal{L}\)構造\(\symcal{M}\)を以下によって定義する:
	\begin{enumerate}
		\item \(\interpretation{\symcal{M}}{\obj{e}} = 0,\)
		\item \(\interpretation{\symcal{M}}{\obj{\ast}} = \mathord{+},\)
		\item \(\interpretation{\symcal{M}}{\obj{{}^{-1}}} = \mathord{-}.\)
	\end{enumerate}
	ここで,3つ目の式における右辺の「\(\mathord{-}\)」は\(x \mapsto -x\)で定義される写像である.

	この\(\symcal{L}\)構造\(\symcal{M}\)は\cref{Ex:grouptheory}で挙げた3つの\(\symcal{L}\)文をすべて充足する.
	一方で,\(\symcal{M}\)による解釈のうち\(\symcal{M}\)による\(\obj{e}\)の解釈を\(\interpretation{\symcal{M}}{\obj{e}} = 1\)
	と置き換えて得られる\(\symcal{L}\)構造\(\symcal{M}'\)は,
	\cref{Ex:grouptheory}で挙げた3つの\(\symcal{L}\)文のうち1つ目のみを充足する.
\end{Ex}

\begin{Que} \label{Que:satisfy}
	\Cref{Ex:satisfy}で述べた事実が正しいことを確かめよ.
\end{Que}

\begin{Def} \label{Def:model}
	\(\symcal{L}\)を言語,\(T\)を\(\symcal{L}\)理論とする.
	このとき,\(\symcal{L}\)構造\(\symcal{M}\)が\(T\)の%
	\index[widx]{もでる@モデル}%
	\term{モデル}であるとは,任意の\(\varphi \in T\)に対して
	\(\symcal{M} \satisfy \varphi\)
	が成り立つことをいう.\(\symcal{M}\)が\(T\)のモデルであることを%
	\index[sidx]{\(\symcal{M} \satisfy T\):モデル}%
	\begin{equation}
		\symcal{M} \satisfy T
		\label{eq:model}
	\end{equation}
	と表す.
	また,\(T\)がモデルをもつとき,\(T\)は%
	\index[widx]{じゅうそく@充足!じゅうそくかのう@---可能}%
	\term{充足可能}であるという.

	また,\(\varphi\)を\(\symcal{L}\)閉論理式とするとき,\(\symcal{L}\)理論\(T\)のすべてのモデルが
	\(\varphi\)を充足するならば,\(\varphi\)は\(T\)の%
	\index[widx]{ろんりてききけつ@論理的帰結}%
	\term{論理的帰結}である,または\(\varphi\)は\(T\)における%
	\index[widx]{ていり@定理}%
	\term{定理}であるといい,
	\begin{equation}
		T \satisfy \varphi
		\label{eq:logicalconsequence}
	\end{equation}
	と表す.特に,\(T = \emptyset\)のときには
	\begin{equation}
		\satisfy \varphi
		\label{eq:nulllogicalconsequence}
	\end{equation}
	と表記する.
\end{Def}

\begin{Ex} \label[Ex]{Ex:model}
	\Cref{Ex:grouptheory}で述べた3つの\(\symcal{L}\)文からなる\(\symcal{L}\)理論を\(T\)とする.
	このとき,\cref{Ex:satisfy}で述べた\(\symcal{L}\)構造\(\symcal{M}\)は\(T\)のモデルである.
	一方で,\cref{Ex:satisfy}で述べた\(\symcal{L}\)構造\(\symcal{M}'\)は\(T\)のモデルではない.
\end{Ex}

\begin{Note}
	通常の通学では,\cref{Ex:model}における「\(\symcal{M}\)が\(T\)のモデルである」
	ことを「\(\symcal{M}\)が群である」と表現している.
	通常の数学では理論そのものと構造がその理論のモデルであるという主張を区別する必要性は薄いのだが,
	数理論理学の文脈では当然厳格に区別する必要がある.
\end{Note}

\begin{Note}
	理論\(T\)に対して,
	\[
		T \satisfy \bot
	\]
	は,\(T\)にモデルが存在しない場合,かつその場合にのみ成り立つ.
\end{Note}

\begin{Que} \label{Que:model}
	\Cref{Ex:grouptheory}で挙げた群の理論\(T\)について,次の論理式を\(\varphi\)とする:
	\[
		\forall \obj{x} \forall \obj{y} \paren{\obj{x} \obj{\ast} \obj{y} = \obj{y} \obj{\ast} \obj{x}}.
	\]
	このとき,\(\varphi\)を充足する\(T\)のモデルと\(\varphi\)を充足しない\(T\)のモデルを1つずつ挙げよ.
\end{Que}

\begin{Que} \label{Que:grouptheoryaxiomize}
	\Cref{Que:invalidtheory}に関連して,アリティ2の関数記号1つだけからなる
	言語\(\symcal{L}_1 = \Set{\obj{\ast}}\)に関する以下の
	3つの論理式からなる\(\symcal{L}_1\)理論を\(T_1\)とする:
	\begin{align*}
		\varphi_1 \colon & \forall \obj{x} \forall \obj{y} \forall \obj{z}
		\paren{\paren{\obj{x} \obj{\ast} \obj{y}} \obj{\ast} \obj{z} \objeq \obj{x} \obj{\ast} \paren{\obj{y} \obj{\ast} \obj{z}}}, \\
		\varphi_2 \colon & \exists \obj{e'} \forall \obj{x} \paren{\obj{e'} \obj{\ast} \obj{x} \objeq \obj{x}},                     \\
		\varphi_3 \colon & \exists \obj{e'}\forall \obj{x} \exists \obj{y} \paren{\obj{y} \obj{\ast} \obj{x} \objeq \obj{e'}}.
	\end{align*}
	このとき,\cref{Ex:grouptheory}で挙げた群の言語\(\symcal{L}\)に対して\(\symcal{L}_1 \subset \symcal{L}\)が成り立つので,
	\(T_1\)は自然に\(\symcal{L}\)理論ともみなせる.
	さて,\(\symcal{L}_1\)理論\(T_1\)のモデルだが,\(\symcal{L}_1\)が有していない記号\(\obj{e}, {}^{\obj{-1}}\)
	の解釈をどのように定めても\cref{Ex:grouptheory}で挙げた
	群の理論\(T\)のモデルとみなすことはできない\(\symcal{L}_1\)構造の例を挙げよ.

	上記のような例が存在することは,\(\symcal{L}_1\)に記号を追加して
	言語を拡張しても,\(T_1\)から\(T\)を自然に得ることはできないことを意味している.
\end{Que}



\section{構造の同型} \label{sec:isomorphic}

\Cref{sec:model}では,構造やモデルについて基本的な定義を行った.
\Cref{sec:model}は理論と構造との間の関係に着目した議論であったが,ここでは構造同士の関係について考えよう.

\begin{Def} \label{Def:isomorphic}
	\(\symcal{L}\)を言語とし,\(\symcal{M}, \symcal{N}\)を対象領域が
	それぞれ\(M, N\)であるような\(\symcal{L}\)構造とする.
	このとき,写像\(\sigma \colon M \to N\)が\(\symcal{M}\)から\(\symcal{N}\)への%
	\index[widx]{どうけい@同型!じゅうどうけい@準---写像}%
	\term{準同型写像}であるとは,
	\(\sigma\)が以下の条件を満たすことをいう:
	\begin{enumerate}
		\item すべての定数記号\(c \in L\)に対して
		      \begin{equation}
			      \apply{\sigma}{\interpretation{\symcal{M}}{c}} = \interpretation{\symcal{N}}{c}
			      \label{eq:constanthomorphism}
		      \end{equation}
		      となる.
		\item アリティ\(n\)の任意の関数記号\(f\)と任意の\(a_1, a_2, \dots, a_n \in M\)に対して
		      \begin{equation}
			      \apply{\sigma}{\apply{\interpretation{\symcal{M}}{f}}{a_1, a_2, \dots, a_n}}
			      = \apply{\interpretation{\symcal{N}}{f}}{
				      \apply{\sigma}{a_1}, \apply{\sigma}{a_2}, \dots, \apply{\sigma}{a_n}
			      }
			      \label{eq:functionhomorphism}
		      \end{equation}
		      が成り立つ.
		\item アリティ\(n\)の任意の関係記号\(r\)と任意の\(a_1, a_2, \dots, a_n\)に対して
		      \begin{equation}
			      \pair{a_1, a_2, \dots, a_n} \in \interpretation{\symcal{M}}{r}
			      \metaimplies \pair{\apply{\sigma}{a_1}, \apply{\sigma}{a_2}, \dots, \apply{\sigma}{a_n}} \in \interpretation{\symcal{N}}{r}
			      \label{eq:relationhomorphism}
		      \end{equation}
		      が成り立つ.
	\end{enumerate}

	また,準同型写像\(\sigma \colon M \to N\)が\(\symcal{M}\)から\(\symcal{N}\)への%
	\index[widx]{うめこみ@埋め込み}%
	\term{埋め込み}であるとは,\(\sigma\)が写像として単射であって,さらに
	\begin{equation}
		\pair{a_1, a_2, \dots, a_n} \in \interpretation{\symcal{M}}{r}
		\metaequivalent \pair{\apply{\sigma}{a_1}, \apply{\sigma}{a_2}, \dots, \apply{\sigma}{a_n}} \in \interpretation{\symcal{N}}{r}
		\label{eq:relationembedding}
	\end{equation}
	が成り立つことをいう.

	埋め込み\(\sigma \colon M \to N\)が写像として全単射である場合,\(\sigma\)を\(\symcal{M}\)から\(\symcal{N}\)への
	\index[widx]{どうけい@同型!どうけいしゃぞう@---写像}%
	\term{同型写像}と呼ぶ.
	2つの構造\(\symcal{M}, \symcal{N}\)について,\(\symcal{M}\)から\(\symcal{N}\)への同型写像が存在する場合,
	\(\symcal{M}\)と\(\symcal{N}\)は%
	\index[widx]{どうけい@同型}%
	\term{同型}であるといい,%
	\index[sidx]{\(\symcal{M} \isomorphic \symcal{N}\):同型}%
	\begin{equation}
		\symcal{M} \isomorphic \symcal{N}
		\label{eq:isomorphic}
	\end{equation}
	と表す.
\end{Def}

簡単な議論によって,以下の補題が成り立つことがわかる.

\begin{Lemma} \label[Lemma]{Lemma:equivrelationforisomorphic}
	\(\symcal{L}\)を言語とし,\(\symcal{M_1}, \symcal{M_2}, \symcal{M_3}\)を\(\symcal{L}\)構造とする.
	このとき
	\begin{enumerate}
		\item \(\symcal{M_1} \isomorphic \symcal{M_1}\)
		\item \(\symcal{M_1} \isomorphic \symcal{M_2} \metaimplies \symcal{M_2} \isomorphic \symcal{M_1}\)
		\item \(\text{\(\symcal{M_1} \isomorphic \symcal{M_2}\)かつ\(\symcal{M_2} \isomorphic \symcal{M_3}\)}
		      \metaimplies \symcal{M_1} \isomorphic \symcal{M_3}\)
	\end{enumerate}
	が成り立つ.
\end{Lemma}

\begin{proof}
	1は恒等写像が同型写像であることから,
	2は同型写像の逆写像もまた同型写像であることから,
	3は同型写像の合成もまた同型写像であることからそれぞれ従う.
\end{proof}

2つの\(\symcal{L}\)構造\(\symcal{M}, \symcal{N}\)が同型であることは,
定数記号,関数記号,関係記号のすべてが同型写像\(\sigma\)を通して「翻訳」できることを主張している.
通常の数学では,2つの構造が同型であることは「その理論の内部ではその2つの構造は同一視できる」
ことと解釈される.この認識が適切であることを以下で検証しよう.

まずは「その理論の内部ではその2つの構造は同一視できる」ことの定式化から行う.
この主張は「その2つの構造はまったく同じ性質を満たす」ここと解釈できる.
今までの議論の中で類似した概念として充足関係が登場している.
そこで,以下のように定義しよう.

\begin{Def} \label{Def:elementarilyequivalent}
	\(\symcal{L}\)を言語とし,\(\symcal{M}, \symcal{N}\)を\(\symcal{L}\)構造とする.
	このとき,
	任意の\(\symcal{L}\)閉論理式\(\varphi\)に対して
	\begin{equation}
		\symcal{M} \satisfy \varphi \metaequivalent \symcal{N} \satisfy \varphi
		\label{eq:elementarilyequivalentcondition}
	\end{equation}
	が成り立つとき,\(\symcal{M}\)と\(\symcal{N}\)は
	\index[widx]{しょとうどうち@初等同値}%
	\term{初等同値}であるといい,%
	\index[sidx]{\(\symcal{M} \elementarilyequivalent \symcal{N}\):初等同値}
	\begin{equation}
		\symcal{M} \elementarilyequivalent \symcal{N}
		\label{eq:elementarilyequivalent}
	\end{equation}
	と表す%
	\footnote{%
		オブジェクト側の等号と記号が被っているが,文脈上混乱することはまずないと考えられる.
	}%
	.
\end{Def}

以下の補題は\cref{Def:elementarilyequivalent}から容易に得られる.

\begin{Lemma} \label[Lemma]{Lemma:equivrelationforelementarilyequivalent}
	\(\symcal{L}\)を言語とし,\(\symcal{M_1}, \symcal{M_2}, \symcal{M_3}\)を\(\symcal{L}\)構造とする.
	このとき
	\begin{enumerate}
		\item \(\symcal{M_1} \elementarilyequivalent \symcal{M_1}\)
		\item \(\symcal{M_1} \elementarilyequivalent \symcal{M_2} \metaimplies \symcal{M_2} \elementarilyequivalent \symcal{M_1}\)
		\item \(\text{\(\symcal{M_1} \elementarilyequivalent \symcal{M_2}\)かつ\(\symcal{M_2} \elementarilyequivalent \symcal{M_3}\)} \metaimplies \symcal{M_1} \elementarilyequivalent \symcal{M_3}\)
	\end{enumerate}

	が成り立つ.
\end{Lemma}

容易に予想できるように,同型である2つの構造は初等同値である.このことを検証しよう.

\begin{Lemma} \label[Lemma]{Lemma:isomorphicforterm}
	\(\symcal{L}\)を言語とし,\(\symcal{M}, \symcal{N}\)は対象領域がそれぞれ\(M, N\)であるような
	同型である\(\symcal{L}\)構造とし,\(\sigma \colon M \to N\)を同型写像とする.
	このとき,任意の\(\languagewithname{\symcal{L}}{\symcal{M}}\)閉項\(t\)に対して
	\begin{align}
		\apply{\sigma}{
			\interpretation{\symcal{M}}{t}
		}
		= \interpretation{\symcal{N}}{
			\subst{t}{\sigma}
		}
		\label{eq:isomorphicforterm}
	\end{align}
	が成り立つ.ここで,\(\subst{t}{\sigma}\)は\(t\)に出現する各\(a \in M\)の名前をすべて\(\apply{\sigma}{a} \in N\)
	の名前で置き換えて得られる\(\languagewithname{\symcal{L}}{\symcal{N}}\)閉項とする.
\end{Lemma}

\begin{proof}
	\(t\)の構成に関する帰納法によって示す.
	\(t\)が\(\symcal{L}\)における定数記号であるとき,\cref{eq:constanthomorphism}から\cref{eq:isomorphicforterm}が従う.
	\(t\)が\(M\)の元\(a\)の名前\(c_a\)であるとき,\cref{eq:isomorphicforterm}は
	\[
		\apply{\sigma}{\interpretation{\symcal{M}}{c_a}} = \interpretation{\symcal{N}}{c_{\apply{\sigma}{a}}}
	\]
	と書き換えられ,どちらも\(\apply{\sigma}{a}\)に等しいことから\cref{eq:isomorphicforterm}の成立がわかる.
	次に,\(t\)がアリティ\(n\)の関数記号\(f\)と,
	示すべき主張が成り立つ\(\languagewithname{\symcal{L}}{\symcal{M}}\)閉項\(t_1, t_2, \dots, t_n\)を用いて
	\[
		t = \apply{f}{t_1, t_2, \dots, t_n}
	\]
	と書ける場合を考える.このとき,
	\begin{align*}
		\apply{\sigma}{
			\interpretation{\symcal{M}}{t}
		}
		 & =
		\apply{\sigma}{
			\apply{\interpretation{\symcal{M}}{f}}{
				\interpretation{\symcal{M}}{t_1},
				\interpretation{\symcal{M}}{t_2},
				\dots,
				\interpretation{\symcal{M}}{t_n}
			}
		}    \\
		 & =
		\apply{\interpretation{\symcal{N}}{f}}{
			\apply{\sigma}{\interpretation{\symcal{M}}{t_1}},
			\apply{\sigma}{\interpretation{\symcal{M}}{t_2}},
			\dots,
			\apply{\sigma}{\interpretation{\symcal{M}}{t_n}}
		}    \\
		 & =
		\apply{\interpretation{\symcal{N}}{f}}{
			\interpretation{\symcal{N}}{\subst{t_1}{\sigma}},
			\interpretation{\symcal{N}}{\subst{t_2}{\sigma}},
			\dots,
			\interpretation{\symcal{N}}{\subst{t_n}{\sigma}}
		}    \\
		 & =
		\interpretation{\symcal{N}}{\paren{\apply{f}{
					\subst{t_1}{\sigma},
					\subst{t_2}{\sigma},
					\dots,
					\subst{t_n}{\sigma},
		}}}  \\
		 & =
		\interpretation{\symcal{N}}{\subst{t}{\sigma}}
	\end{align*}
	となるので\cref{eq:isomorphicforterm}が成り立つ.

	以上より,任意の\(\languagewithname{\symcal{L}}{\symcal{M}}\)閉項\(t\)に対してもとの主張が成り立つことがわかった.
\end{proof}

\begin{Thm} \label[Thm]{Thm:isomorphicforlogicalexpression}
	\(\symcal{L}\)を言語,\(\symcal{M}, \symcal{N}\)を\(\symcal{L}\)構造とする.
	このとき,\(\symcal{M}\)と\(\symcal{N}\)が同型であるならば,
	\(\symcal{M}\)と\(\symcal{N}\)は初等同値である.
\end{Thm}

\begin{proof}
	\(\symcal{M}, \symcal{N}\)の対象領域をそれぞれ\(M, N\)とし,
	\(\sigma \colon M \to N\)を同型写像とする.
	\Cref{Lemma:equivrelationforelementarilyequivalent}により,
	任意の\(\symcal{L}\)閉論理式\(\varphi\)に対して
	\(\symcal{M} \satisfy \varphi\)ならば\(\symcal{N} \satisfy \varphi\)となることを示せば十分.

	\(\varphi\)が\(\symcal{L}\)閉項\(t_1, t_2\)に対する\(t_1 \objeq t_2\)の形のとき,
	\(\symcal{M} \satisfy \varphi\)とすると\(\interpretation{\symcal{M}}{t_1} = \interpretation{\symcal{M}}{t_2}\)である.
	よって\cref{Lemma:isomorphicforterm}から任意の\(\symcal{L}\)閉項\(t\)に対して
	\[
		\apply{\sigma}{\interpretation{\symcal{M}}{t}}
		=
		\interpretation{\symcal{N}}{\subst{t}{\sigma}}
		=
		\interpretation{\symcal{N}}{t}
	\]
	なので,\(\apply{\sigma}{t_1} = \apply{\sigma}{t_2}\)から
	\(\interpretation{\symcal{N}}{t_1} = \interpretation{\symcal{N}}{t_2}\)が得られるため
	\(\symcal{N} \satisfy t_1 \objeq t_2\)となる.

	\(\varphi\)がアリティ\(n\)の関係記号\(n\)と\(\symcal{L}\)閉項\(t_1, t_2, \dots, t_n\)に対する
	\(\apply{r}{t_1, t_2, \dots, t_n}\)の形のとき,
	\(\symcal{M} \satisfy \varphi\)とすると
	\(\pair{
		\interpretation{\symcal{M}}{t_1},
		\interpretation{\symcal{M}}{t_2},
		\dots,
		\interpretation{\symcal{M}}{t_n}
	} \in \interpretation{\symcal{M}}{r}\)となる.
	よって,\cref{eq:functionhomorphism}と\cref{Lemma:isomorphicforterm}から
	\(\symcal{N} \satisfy \varphi\)が従う.

	次に,\(\varphi\)が定理の主張を満たす\(\symcal{L}\)閉論理式\(\psi\)に対する
	\(\lnot \psi\)の形の場合を考える.
	\(\symcal{M} \satisfy \varphi\)とすると\(\symcal{M} \satisfy \psi\)ではない.
	ここで,\(\symcal{N} \satisfy \psi\)と仮定すると,帰納法の仮定によって
	\(\symcal{M} \satisfy \psi\)となって矛盾する.
	よって\(\symcal{N} \satisfy \psi\)ではないので\(\symcal{N} \satisfy \varphi\)である.

	\(\varphi\)が定理の主張を満たす\(\symcal{L}\)閉論理式\(\varphi_1, \varphi_2\)に対する
	\(\varphi_1 \lor \varphi_2\)の形の場合を考える.
	\(\symcal{M} \satisfy \varphi\)だとすると\(\symcal{M} \satisfy \varphi_1\)または
	\(\symcal{M} \satisfy \varphi_2\)である.
	どちらの場合も帰納法の仮定によって
	「\(\symcal{N} \satisfy \varphi_1\)または\(\symcal{N} \satisfy \varphi_2\)」
	が成り立つので\(\symcal{N} \satisfy \varphi\)である.

	残りの場合も上と同様にして導ける(\cref{Def:semanticimplies}の全パターンを網羅すればよい).
\end{proof}

\begin{Ex} \label[Ex]{Ex:isomorphic}
	\Cref{Ex:grouptheory}で挙げた群の理論\(T\)の2つのモデル\(\symcal{M}, \symcal{N}\)が同型であることは,
	よく知られた群の同型に関する標準的な定義
	(\(M, N\)をそれぞれ\(\symcal{M}, \symcal{N}\)の対象領域とするとき,
	全単射\(f \colon M \to N\)で任意の\(x, y \in M\)に対して
	\(\apply{f}{x \interpretation{\symcal{M}}{\ast} y}
	= \apply{f}{x} \interpretation{\symcal{N}}{\ast} \apply{f}{y}\)が成り立つものが存在すること)と一致する.
	よって「群として同型」であるような2つの群は,\(\symcal{L}\)閉論理式で表現できる性質に差異はない.
	例えば「可換群である」という主張は\(\symcal{L}\)閉論理式として表現できる(\cref{Que:model}を参照)ので,
	同型な2つの群はどちらも可換であるか,どちらも可換でないかのいずれかである.
\end{Ex}


\section{同値な論理式} \label{sec:equivalentformula}

以降,言語\(\symcal{L}\)が重要視されない文脈では\(\symcal{L}\)構造のことを単に構造と表記することとする.

\Cref{sec:model}では,与えられた構造が各論理式を充足するかどうかという関係を定義した.
この充足関係に着目したとき,論理式は以下の3パターンに分類することができる:
\begin{enumerate}
	\item あらゆる構造が充足する論理式,
	\item ある構造は充足するが,別の構造は充足しない論理式,
	\item どんな構造も充足しない論理式.
\end{enumerate}
これらのパターンに着目して,以下のように定義しよう.

\begin{Def} \label[Def]{Def:validformula}
	論理式\(\varphi\)が%
	\index[widx]{こうしんしき@恒真式}%
	\term{恒真式}%
	であるとは,すべての構造\(\symcal{M}\)が\(\varphi\)を充足することをいう.
	また,\(\varphi\)が%
	\index[widx]{こうぎしき@恒偽式}%
	\index[widx]{むじゅんしき@矛盾式}%
	\term{恒偽式},あるいは\term{矛盾式}であるとは,
	すべての構造\(\symcal{M}\)が\(\varphi\)を充足しないことをいう.
	論理式\(\varphi\)が恒偽式であることは,\(\lnot \varphi\)が恒真式であることと同値である.
\end{Def}

\begin{Ex} \label[Ex]{Ex:validformula}
	\(\varphi\)を論理式とするとき,論理式\(\varphi \lor \lnot \varphi\)は恒真式である.
	実際,任意の構造\(\symcal{M}\)に対して,\(\symcal{M} \satisfy \varphi\)である場合には
	\(\symcal{M} \satisfy \varphi \lor \lnot \varphi\)が成り立ち,
	\(\symcal{M} \satisfy \varphi\)でない場合には\(\symcal{M} \satisfy \lnot \varphi\)となるので
	\(\symcal{M} \satisfy \varphi \lor \lnot \varphi\)となる.

	一方で,\cref{Ex:OrderedRing}で挙げた言語\(\symcal{L}_{\OrderedRing}\)における
	論理式\(\forall \obj{x} \lnot \paren{\obj{x} \obj{<} \obj{x}}\)を\(\varphi\)とするとき,\(\varphi\)は恒真式ではない.
	理論\(\OrderedRing\)のモデルはすべて\(\varphi\)を充足するが,\(\OrderedRing\)のモデルでない
	\(\symcal{L}_{\OrderedRing}\)構造で\(\varphi\)を充足しないものが存在するからである.
\end{Ex}

\begin{Que} \label{Que:validformula}
	\Cref{Ex:validformula}の後半で述べた事実を検証せよ.すなわち,
	\(\symcal{L}_{\OrderedRing}\)構造\(\symcal{M}\)で論理式\(\forall \obj{x} \lnot \paren{\obj{x} \obj{<} \obj{x}}\)を
	充足しないものを挙げよ.
\end{Que}



\index[sidx]{\(\varphi \logicallyequivalent \psi\):論理的に同値}
\index[widx]{ろんりてきにどうち@論理的に同値}
\begin{Def} \label{Def:equivalentformula}
	\(\varphi, \psi\)を論理式とする.このとき,論理式
	\[
		\varphi \formulaequiv \psi
	\]
	恒真式であるとき,\(\varphi\)と\(\psi\)は\term{論理的に同値}であるといい
	\begin{equation}
		\varphi \logicallyequivalent \psi
		\label{eq:logicallyequivalent}
	\end{equation}
	と表す.
\end{Def}

\begin{Note}
	論理式\(\varphi, \psi\)が論理的に同値である場合,通常は\(\varphi, \psi\)は「意味論的には同じ論理式である」とみなされる.
	\(\varphi, \psi\)が論理的に同値である場合,\(\varphi, \psi\)を充足する構造は一致するため
	「どのような構造によって充足されるか」という観点では\(\varphi, \psi\)に差を見出すことはできないからである.
\end{Note}


論理記号\(\forall, \exists\)や等号\(\objeq\)および関係記号が陽に現れない論理式の場合,
与えられた2つの論理式が論理的に同値であるかどうかを判定するのは容易である.
構造\(\symcal{M}\)を固定した場合,どのように複雑な論理式であっても結局は\(\symcal{M}\)によって充足されるかされないかの2通りでしかない.
そのため,ベースとなる論理式を決め,その論理式が固定した構造によって充足されるかどうかに応じて
同値であるかを判定したい論理式がその構造によって充足されるかどうかを調べればよい.
論理記号\(\forall, \exists\),等号\(\objeq\),および関係記号が陽に現れない場合,\cref{Def:semanticimplies}に基づいて
各論理式の充足性を検証する作業は有限回の機械的な操作で完了する.
このとき,充足される場合に1を,そうでない場合に0を割り当て表にしておくとわかりやすい.
この表を%
\index[widx]{しんりち@真理値!---表}%
\term{真理値表}といい,それぞれの論理式に割り当てられた0, 1の値をその論理式の%
\index[widx]{しんりち@真理値}%
\term{真理値}という.
論理的に同値であるかを判定したい論理式の真理値がつねに一致している場合,それらの論理式は論理的に同値であるといえる.
逆に,真理値が一致しない場合がある場合にはそれらの論理式は論理的に同値ではない.

なお,上記の文脈で与えられた論理式が恒真式であるかどうかを判定したい場合,その論理式の真理値表を書き真理値がつねに1になるかどうかを検証すればよい.
つねに1になるならばその論理式は恒真式であり,そうでないならばその論理式は恒真式ではない.

まずは\cref{Def:semanticimplies}における\(\lnot, \lor, \land, \to\)について考えよう.

\begin{Ex} \label[Ex]{Ex:truthtableforlnot}
	\(\varphi\)を閉論理式とする.このとき,\(\varphi\)が1つの構造\(\symcal{M}\)によって充足されるかどうかに応じて
	\(\lnot \varphi\)が充足されるかどうかを表にまとめたものを\cref{tab:lnottruthtable}に示す.

	\begin{table}[htbp]
		\centering
		\caption{
			\(\lnot \varphi\)の真理値表,1列目が\(\varphi\)が\(\symcal{M}\)に充足されるかどうか,2列目が\(\lnot \varphi\)が\(\symcal{M}\)に充足されるかどうかを示す.
			1行目において1列目が0で2列目が1であることは,\(\varphi\)が\(\symcal{M}\)に充足されない場合,\(\lnot \varphi\)が\(\symcal{M}\)に充足されることを示す.
			2行目も同様である.
		}
		\label{tab:lnottruthtable}
		\begin{tabular}{c|c}
			\hline
			\(\varphi\) & \(\lnot \varphi\) \\ \hline
			0           & 1                 \\
			1           & 0                 \\
			\hline
		\end{tabular}
	\end{table}
\end{Ex}

\begin{Ex} \label[Ex]{Ex:truthtableforbinsymbol}
	\(\varphi, \psi\)を閉論理式とする.このとき,\(\varphi, \psi\)が1つの構造\(\symcal{M}\)によって充足されるかどうかに応じて
	\(\varphi \lor \psi, \varphi \land \psi, \varphi \to \psi\)が充足されるかどうかを表にまとめたものを\cref{tab:binsymboltruthtable}に示す.

	\begin{table}[htbp]
		\centering
		\caption{
			\(\varphi \land \psi, \varphi \lor \psi, \varphi \to \psi\)の真理値表,
			\(\varphi \to \psi\)の真理値は\(\lnot \varphi\)と\(\psi\)の真理値によって決まるため,\(\lnot \varphi\)の真理値を記載した列が書かれている.
		}
		\label{tab:binsymboltruthtable}
		\begin{tabular}{cc|cccc}
			\hline
			\(\varphi\) & \(\psi\) & \(\varphi \lor \psi\) & \(\varphi \land \psi\) & \(\lnot \varphi\) & \(\varphi \to \psi\) \\ \hline
			0           & 0        & 0                     & 0                      & 1                 & 1                    \\
			0           & 1        & 1                     & 0                      & 1                 & 1                    \\
			1           & 0        & 1                     & 0                      & 0                 & 0                    \\
			1           & 1        & 1                     & 1                      & 0                 & 1                    \\
			\hline
		\end{tabular}
	\end{table}
\end{Ex}

\begin{Note}
	真理値表を書く際には,ベースと決めた論理式の内部構造は考えずにそれが与えられた構造によって充足されるかどうかを考えた.
	「与えられた構造によって充足されること」を「その構造においては真である(正しい)」と読み替えることで,
	真理値表を書く際は「ベースとなる論理式の内部構造は無視してその真偽のみに着目した」と考えることができる.
	本書では議論しないが,
	ベースとなる論理式の内部構造に踏み込まざるを得ない論理記号\(\forall, \exists\)や等号\(\objeq\)および言語を排した体系というのも考えられる.
	このような体系として典型的なのは命題論理と呼ばれる体系である.
	命題論理においては,この「ベースとなる論理式」は命題として提示される.
	本書で学ぶ1階述語論理は,命題論理で登場する命題の内部構造を議論できるよう拡張したものと考えることができる.
\end{Note}

\begin{Note}
	「命題」という術語については,よく「命題とは,真偽を判断する対象となる言明のことである」という
	「定義」らしきものが提示されることがある.
	むろんこのようなあいまいさの強い「定義」は定義とはいわない.
	「命題」という用語は「何らかの数学的意味を有する主張」程度の意味合いでラフに使われる言葉であり,
	何が命題で何が命題でないかを厳格に区別するような一般的な定義があるわけではない.

	なお,命題論理においては「命題」の定義が提示されることはない.
	述語論理において「関数記号」の定義が提示されないのと同じように,
	命題論理において「命題」を定義しなくても困ることはない.
\end{Note}

\begin{Que} \label{Que:truthtableforequivalent}
	\(\varphi, \psi\)を論理式とする.\(\varphi \formulaequiv \psi\)の真理値表を書き,
	\(\varphi \formulaequiv \psi\)の真理値が1になるのは\(\varphi, \psi\)の真理値が一致する場合であることを確かめよ.
\end{Que}

\begin{Thm} \label{Thm:semanticdne}
	\(\varphi\)を論理式とする.このとき,
	\begin{equation}
		\varphi \logicallyequivalent \lnot \lnot \varphi
		\label{eq:semanticdne}
	\end{equation}
	が成り立つ.
\end{Thm}

\begin{proof}
	\(\lnot \lnot \varphi\)の真理値表を書くことによって示す.
	\begin{table}[htbp]
		\centering
		\begin{tabular}{c|cc}
			\hline
			\(\varphi\) & \(\lnot \varphi\) & \(\lnot \lnot \varphi\) \\ \hline
			0           & 1                 & 0                       \\
			1           & 0                 & 1                       \\
			\hline
		\end{tabular}
	\end{table}

	真理値表により,\(\varphi\)と\(\lnot \lnot \varphi\)の真理値はつねに一致することがわかる.
	ゆえに\cref{eq:semanticdne}が成り立つ.
\end{proof}

\begin{Que} \label{Que:semanticlawofexcludedmiddle}
	\(\varphi\)を論理式とするとき,\(\varphi \lor \lnot \varphi\)の真理値表を書き,これが恒真式であることを確かめよ.
\end{Que}

以下,よく知られた同値な論理式のうち,真理値表を書くだけで簡単に検証できるものを提示しておく.

\begin{Thm} \label{Thm:semantictodisjunctivenormalform}
	\(\varphi, \psi\)を論理式とする.このとき
	\begin{equation}
		\varphi \to \psi \logicallyequivalent \lnot \varphi \lor \psi
		\label{eq:semantictodisjunctivenormalform}
	\end{equation}
	が成り立つ.
\end{Thm}

\begin{Thm} \label{Thm:semanticDemorganslawproposition}
	\(\varphi, \psi\)を論理式とする.このとき
	\begin{align}
		\lnot \paren{\varphi \lor \psi}  & \logicallyequivalent \lnot \varphi \land \lnot \psi,
		\label{eq:semanticDemorganslawpropositionlor}                                           \\
		\lnot \paren{\varphi \land \psi} & \logicallyequivalent \lnot \varphi \lor \lnot \psi
		\label{eq:semanticDemorganslawpropositionland}
	\end{align}
	が成り立つ.
\end{Thm}

\begin{Thm} \label{Thm:semanticcommunitativelaw}
	\(\varphi, \psi\)を論理式とする.
	このとき
	\begin{align}
		\varphi \land \psi & \logicallyequivalent \psi \land \psi,
		\label{eq:semanticcommunitativelawland}                     \\
		\varphi \lor \psi  & \logicallyequivalent \psi \lor \varphi
		\label{eq:semanticcommunitativelawlor}
	\end{align}
	が成り立つ.
\end{Thm}

\begin{Thm} \label{Thm:semanticassociativelaw}
	\(\varphi, \psi\)を論理式とする.
	このとき
	\begin{align}
		\varphi \land \paren{\psi \land \chi} & \logicallyequivalent \paren{\varphi \land \psi} \land \chi,
		\label{eq:semanticassosiativelawland}                                                               \\
		\varphi \lor \paren{\psi \lor \chi}   & \logicallyequivalent \paren{\varphi \lor \psi} \lor \chi
		\label{eq:semanticassosiativelawlor}
	\end{align}
	が成り立つ.
\end{Thm}

\Cref{Thm:semanticassociativelaw}により,\(\varphi \land \psi \land \chi\)や\(\varphi \lor \psi \lor \chi\)のような
略記表現について,意味論的には問題が生じないことがわかる.

\begin{Thm} \label{Thm:semantictolnotland}
	\(\varphi, \psi\)を論理式とする.
	このとき
	\begin{align}
		\lnot \paren{\varphi \to \psi}    & \logicallyequivalent \varphi \land \lnot \psi,
		\label{eq:semantictolnotland}                                                          \\
		\varphi \to \psi                  & \logicallyequivalent \lnot \psi \to \lnot \varphi,
		\label{eq:semanticcontraposition}                                                      \\
		\varphi \to \paren{\psi \to \chi} & \logicallyequivalent \varphi \land \psi \to \chi
		\label{eq:semantictonesting}
	\end{align}
	が成り立つ.
\end{Thm}

\(\forall, \exists\)や等号\(\objeq\)が陽に現れない文脈では,以上のように真理値表を書くだけで論理式の論理的同値性を確かめることができる.
一方で,\(\forall, \exists\)や等号\(\objeq\)が陽に現れる文脈ではこうはいかない.
ここでは,\(\forall, \exists\)が現れる文脈においてはどうすればいいか考えよう%
\footnote{%
	等号\(\objeq\)が陽に現れる場合については本書では取り上げない.ただし,\cref{chap:proof}においては
	等号\(\objeq\)が陽に現れる場合も含めて「同値な論理式」について本章とは違った考え方で議論する.
}%
.

以降,論理式\(\varphi_1, \varphi_2, \dots, \varphi_n, \psi\)に対する
\[
	\Set{\varphi_1, \varphi_2, \dots, \varphi_n} \satisfy \psi
\]
を
\[
	\varphi_1, \varphi_2, \dots, \varphi_n \satisfy \psi
\]
と略記する(要するに,集合を表すカッコを省略する).

\(\forall, \exists\)が陽に現れる文脈では,真理値表のような「全パターンを機械的に網羅する」という手が使えない.
そこで代わりに登場するのは,意味論的な「推論」という概念である.

\begin{Lemma}[背理法] \label{lemma:semanticcontradiction}
	\(\Gamma\)を論理式の集合,\(\varphi\)を論理式とする.
	このとき
	\[
		\Gamma \satisfy \varphi \metaequivalent \Set{\lnot \varphi} \cup \Gamma \satisfy \bot
	\]
	が成り立つ.
\end{Lemma}

\begin{proof}
	\[
		\Gamma \satisfy \varphi
	\]
	とする.
	このとき,任意の構造\(\symcal{M}\)に対し,\(\symcal{M} \satisfy \varphi\)が成り立つ.
	よって,\(\symcal{M} \satisfy \lnot \varphi\)が成り立つことはないので,\(\symcal{M}\)が
	\(\Set{\lnot \varphi} \cup \Gamma\)のモデルになることはない.
	従って
	\[
		\Set{\lnot \varphi} \cup \Gamma \satisfy \bot
	\]
	となる.

	\[
		\Set{\lnot \varphi} \cup \Gamma \satisfy \bot
	\]
	と仮定する.このとき,\(\Set{\lnot \varphi} \cup \Gamma\)のモデルは存在しない.
	任意の構造\(\symcal{M}\)に対し,\(\symcal{M} \satisfy \Gamma\)と仮定する.もし\(\symcal{M} \satisfy \lnot \varphi\)であれば
	\(\symcal{M}\)が\(\Set{\lnot \varphi} \cup \Gamma\)のモデルになるので,\(\symcal{M} \satisfy \varphi\)でなければならない.
	よって
	\[
		\Gamma \satisfy \varphi
	\]
	となる.
\end{proof}

\begin{Lemma}[全称除去律] \label{lemma:semanticforallelimination}
	\(\varphi, \psi\)を論理式,\(x\)を変数記号,\(a\)を\(\varphi\)中の\(x\)に代入可能な項とする.
	このとき,任意の構造\(\symcal{M}\)に対して
	\[
		\symcal{M} \satisfy \forall x \varphi \metaimplies \symcal{M} \satisfy \subst{\varphi}{a/x}
	\]
	が成り立つ.
\end{Lemma}

\begin{proof}
	\(\symcal{M}\)の対象領域を\(M\)とする.
	\(\symcal{M}\)による項\(a\)の解釈\(\interpretation{\symcal{M}}{a}\)は\(M\)の元である.
	従って,その元の名前を\(c\)とすると,
	\[
		\symcal{M} \satisfy \subst{\varphi}{a/x} \metaequivalent \symcal{M} \satisfy \subst{\varphi}{c/x}
	\]
	となる(厳密には\(\varphi\)の構造に関する帰納法で示せる).
	\(\symcal{M} \satisfy \forall x \varphi\)だから
	\(\symcal{M} \satisfy \subst{\varphi}{c/x}\)は成り立つので,
	\(\symcal{M} \satisfy \subst{\varphi}{a/x}\)となる.
\end{proof}


\section{等式理論とHorn理論} \label{sec:equationaltheory}

本節では,言語や公理に特別な制約を課した場合について考察する.

\begin{Def} \label[Def]{Def:horntheory}
	原子論理式のことを%
	\index[widx]{りてらる@リテラル!せいのりてらる@正の---}%
	\term{正のリテラル}といい,
	原子論理式\(p\)の否定\(\lnot p\)のことを%
	\index[widx]{りてらる@リテラル!ふのりてらる@負の---}%
	\term{負のリテラル}という.
	正のリテラルと負のリテラルを総称して%
	\index[widx]{りてらる@リテラル}%
	\term{リテラル}という.
	また,リテラル\(p_1, p_2, \dots, p_n\)に対し,論理式
	\begin{equation*}
		p_1 \lor p_2 \lor \dotsb \lor p_n
	\end{equation*}
	を%
	\index[widx]{せつ@節}%
	\term{節}%
	といい,節に含まれる正のリテラルが高々1つである場合にはその節のことを%
	\index[widx]{せつ@節!Hornせつ@Horn節}%
	\term{Horn節}という.

	理論\(T\)の各文がすべてHorn節の全称閉包である場合,\(T\)は%
	\index[widx]{りろん@理論!Hornりろん@全称Horn---}%
	\term{全称Horn理論}であるという.
\end{Def}

全称Horn理論の特別な場合として,以下に示す等式理論も重要である.

\begin{Def} \label[Def]{Def:equationaltheory}
	\(\symcal{L}\)を言語とし,\(T\)を\(\symcal{L}\)理論とする.
	このとき,\(T\)が%
	\index[widx]{りろん@理論!とうしきりろん@等式---}%
	\term{等式理論}であるとは,\(\symcal{L}\)が関係記号をもたず,
	さらに\(T\)の各文がすべて等式(\(s \objeq t\)の形の論理式)の全称閉包になっていることをいう.
\end{Def}

\begin{Ex} \label[Ex]{Ex:equationaltheory}
	\Cref{Ex:grouptheory}における群の理論\(\GP\)と\cref{Ex:Ring}における環の理論\(\Ring\)はいずれも等式理論であり,
	従って全称Horn理論でもある.
\end{Ex}
\chapter{1階述語論理の証明論}
\label[chapter]{chap:proof}

この章では,いよいよ「証明」という概念の形式化を試みる.
\Cref{chap:syntax}で定義された論理式は,数学における何らかの主張の
形式的記号列の世界での対応物であった.
数学における「証明」は,
形式的記号列の世界においてはある論理式から別の論理式を得る操作に対応する.
どのような操作を妥当なものとして認め,それをどのような記法で書き下すかによって
多種多様な体系が得られる.本書では,その中から自然演繹式シークエント計算と呼ばれる体系を取り上げる.

\newpage


\section{シークエント} \label[section]{sec:sequent}

自然演繹式シークエント計算において基本的なのは,シークエントと呼ばれる記号列である.

\begin{Def} \label[Def]{Def:sequent}
	0個以上の論理式の有限列\(\Gamma\)と論理式\(\varphi\)に対して,記号列%
	\index[sidx]{\(\Gamma \sequent \varphi\):シークエント}%
	\begin{equation}
		\Gamma \sequent \varphi
	\end{equation}
	を%
	\index[widx]{しーくえんと@シークエント}%
	\term{シークエント}という.
	このとき,\(\Gamma\)をこのシークエントの左辺,\(\varphi\)をこのシークエントの右辺という.
\end{Def}

\begin{Ex} \label[Ex]{Ex:sequent}
	論理式\(\varphi\)に対して,
	\begin{equation*}
		\varphi \sequent \varphi
	\end{equation*}
	という記号列はシークエントである.この形のシークエントを%
	\index[widx]{ししき@始式}%
	\term{始式}と呼ぶ.
	また,左辺に何もない
	\begin{equation*}
		\sequent \varphi, \\
	\end{equation*}
	といった記号列もシークエントである.
\end{Ex}

シークエント「\(\Gamma \sequent \varphi\)」は,
通常の数学における「\(\Gamma\)という仮定によって\(\varphi\)を証明することができる」という
主張の形式的記号列の世界での対応物であることを期待して導入されたものである.
左辺に何もないシークエント「\(\sequent \varphi\)」は,
何も仮定せずとも\(\varphi\)が証明できること,すなわち「\(\varphi\)は証明できる」ことの
形式的記号列の世界での対応物であることが期待される.
以下で行われるのは,このシークエントを操作していくルールをうまく定義して
「証明っぽいもの」を作り上げていくことである.


\section{公理と推論規則} \label[section]{sec:axiomandrule}

通常の数学においては,議論の出発点となる主張,すなわち公理を用意し,
そこから演繹的推論を重ねていくことによってさまざまな結果を得る.
自然演繹式シークエント計算においては,
出発点となるシークエントから特定の操作を行うことによって
さまざまな結果を得ることになる.
このとき,出発点となるシークエントのことを%
\index[widx]{こうり@公理}%
\term{公理}と呼ぶ%
\footnote{%
	ここでいう「公理」は自然演繹式シークエント計算という証明体系そのものの出発点となるシークエントであり,
	個々の理論の基礎となる主張のことではない.通常の数学における「公理」に対応するのは
	閉論理式の集合である.
}%
.
また,
あるシークエントから別のシークエントを得る操作をいくつか列挙し,
それを使って議論を進めていく.
この時に列挙した操作のことを%
\index[widx]{すいろんきそく@推論規則}%
\term{推論規則}
と呼ぶ.
以下に自然演繹式シークエント計算で用いる公理と推論規則の一覧を述べる.

公理や推論規則は,シークエントを横や縦に並べた
\begin{prooftree}
	\AxiomC{\(\Gamma \sequent \varphi \)}
	\AxiomC{\(\Delta \sequent \psi\)}
	\LeftLabel{(name)}
	\BinaryInfC{\(\Sigma \sequent \xi\)}
\end{prooftree}
のような形式で記述される.「(name)」は規則名である.
この記述は「上側に記述されているシークエントがすべて得られたら下段のシークエントを得てよい」
という意味ととらえてよい.
上段に何も書かれていない場合もあり,それが公理である.

\begin{Def}[公理] \label[Def]{Def:axiom}
	論理式\(\varphi\)と項\(t\)に対して,以下は自然演繹式シークエント計算における公理である:
	\begin{multicols}{2}
		\begin{prooftree}
			\AxiomC{}
			\LeftLabel{(ID)}
			\UnaryInfC{\(\varphi \sequent \varphi\)}
		\end{prooftree}
		\columnbreak
		\begin{prooftree}
			\AxiomC{}
			\LeftLabel{(REFL)}
			\UnaryInfC{\(t = t\)}
		\end{prooftree}
	\end{multicols}
\end{Def}

推論規則名になっている「ID」と「REFL」は同一律(law of identity)と反射律(reflexive law)に由来する.

推論規則の数はそれなりに多いので,いくつかのグループに分けて述べる.
なお,\cref{Def:structuralrule}における推論規則名「w」「e」「c」はそれぞれ
弱化(weakening),交換(exchange),縮約(contraction)に由来する.

\begin{Def}[構造規則] \label[Def]{Def:structuralrule}
	論理式の有限列\(\Gamma, \Delta\)と論理式\(\varphi, \psi, \xi\)に対し,
	下記の3つは自然演繹式シークエント計算における推論規則である.
	\begin{multicols}{3}
		\begin{prooftree}
			\AxiomC{\(\Gamma \sequent \xi\)}
			\LeftLabel{(w)}
			\UnaryInfC{\(\varphi, \Gamma \sequent \xi\)}
		\end{prooftree}
		\columnbreak
		\begin{prooftree}
			\AxiomC{\(\Gamma, \psi, \varphi, \Delta \sequent \xi\)}
			\LeftLabel{(e)}
			\UnaryInfC{\(\Gamma, \varphi, \psi, \Delta \sequent \xi\)}
		\end{prooftree}
		\columnbreak
		\begin{prooftree}
			\AxiomC{\(\varphi, \varphi, \Gamma \sequent \xi\)}
			\LeftLabel{(c)}
			\UnaryInfC{\(\varphi, \Gamma \sequent \xi\)}
		\end{prooftree}
	\end{multicols}
\end{Def}

論理記号に関する推論規則を述べる.
推論規則名の「I」や「E」はそれぞれ
導入(Introduction)と除去(Elimination)に由来する.
例えば,「\(\land\)I」は右辺に\(\land\)記号を新しく導入する推論規則であり,
「\(\land\)E」は右辺から\(\land\)記号を除去するような推論規則である.
すべての推論規則が導入と除去の対になっていることに着目しよう.
我々の素朴的直観において導入規則と除去規則はそれぞれ
「その論理記号を含む論理式に対応する主張を結論として得るためのルール」
「その論理記号を含む論理式に対応する主張が得られた場合に行える推論を表すルール」
に対応する.


\begin{Def}[論理規則その1] \label[Def]{Def:logicalrule1}
	論理式の有限列\(\Gamma, \Delta, \Sigma\)と論理式\(\varphi, \psi, \xi\)に対して,
	下記は自然演繹式シークエント計算における推論規則である.
	\begin{multicols}{2}
		\begin{prooftree}
			\AxiomC{\(\varphi, \Gamma \sequent \psi\)}
			\LeftLabel{(\(\to\)I)}
			\UnaryInfC{\(\Gamma \sequent \varphi \to \psi\)}
		\end{prooftree}
		\columnbreak
		\begin{prooftree}
			\AxiomC{\(\Gamma \sequent \varphi\)}
			\AxiomC{\(\Delta \sequent \varphi \to \psi\)}
			\LeftLabel{(\(\to\)E)}
			\BinaryInfC{\(\Gamma, \Delta \sequent \psi\)}
		\end{prooftree}
	\end{multicols}
	\begin{multicols}{2}
		\begin{prooftree}
			\AxiomC{\(\Gamma \sequent \varphi\)}
			\AxiomC{\(\Delta \sequent \psi\)}
			\LeftLabel{(\(\land\)I)}
			\BinaryInfC{\(\Gamma, \Delta \sequent \varphi \land \psi\)}
		\end{prooftree}
		\columnbreak
		\begin{prooftree}
			\AxiomC{\(\Gamma \sequent \varphi_1 \land \varphi_2\)}
			\LeftLabel{(\(\land\)E)}
			\RightLabel{(\(i \equiv 1,2\))}
			\UnaryInfC{\(\Gamma \sequent \varphi_i\)}
		\end{prooftree}
	\end{multicols}
	\begin{prooftree}
		\AxiomC{\(\Gamma \sequent \varphi_i\)}
		\LeftLabel{(\(\lor\)I)}
		\RightLabel{(\(i \equiv 1,2\))}
		\UnaryInfC{\(\Gamma \sequent \varphi_1 \lor \varphi_2\)}
	\end{prooftree}
	\begin{prooftree}
		\AxiomC{\(\Gamma \sequent \varphi \lor \psi\)}
		\AxiomC{\(\varphi, \Delta \sequent \xi\)}
		\AxiomC{\(\psi, \Sigma \sequent \xi\)}
		\LeftLabel{(\(\lor\)E)}
		\TrinaryInfC{\(\Gamma, \Delta, \Sigma \sequent \xi\)}
	\end{prooftree}
\end{Def}

\begin{Def}[論理規則その2] \label[Def]{Def:quantiferrule}
	論理式の有限列\(\Gamma, \Delta\)と論理式\(\varphi, \psi\)および変数記号\(x, a\)に対して,
	下記の4つは自然演繹式シークエント計算における推論規則である.
	ここで,\(a\)は\(\varphi\)中の\(x\)に代入可能であるものとする.
	\begin{multicols}{2}
		\begin{prooftree}
			\AxiomC{\(\Gamma \sequent \subst{\varphi}{a/x}\)}
			\LeftLabel{(\(\forall\)I)}
			\UnaryInfC{\(\Gamma \sequent \forall x \varphi\)}
		\end{prooftree}
		ただし,\(a\)は\(\Gamma\)中の各論理式や\(\varphi\)に自由出現しない.
		\columnbreak
		\begin{prooftree}
			\AxiomC{\(\Gamma \sequent \forall x \varphi\)}
			\LeftLabel{(\(\forall\)E)}
			\UnaryInfC{\(\Gamma \sequent \subst{\varphi}{a/x}\)}
		\end{prooftree}
	\end{multicols}
	\begin{multicols}{2}
		\begin{prooftree}
			\AxiomC{\(\Gamma \sequent \subst{\varphi}{a/x}\)}
			\LeftLabel{(\(\exists\)I)}
			\UnaryInfC{\(\Gamma \sequent \exists x \varphi\)}
		\end{prooftree}
		\columnbreak
		\begin{prooftree}
			\AxiomC{\(\Gamma \sequent \exists x \varphi\)}
			\AxiomC{\(\subst{\varphi}{a/x}, \Delta \sequent \psi\)}
			\LeftLabel{(\(\exists\)E)}
			\BinaryInfC{\(\Gamma, \Delta \sequent \psi\)}
		\end{prooftree}
		ただし,\(a\)は\(\Gamma, \Delta\)中の各論理式と\(\exists x \varphi, \psi\)のいずれにも自由出現しない.
	\end{multicols}
\end{Def}

\begin{Def}[論理規則その3] \label{Def:lnot}
	論理式の有限列\(\Gamma\)と論理式\(\varphi\)について,
	下記は自然演繹式シークエント計算における推論規則である.
	\begin{multicols}{2}
		\begin{prooftree}
			\AxiomC{\(\varphi, \Gamma \sequent \bot\)}
			\LeftLabel{(\(\lnot\)I)}
			\UnaryInfC{\(\Gamma \sequent \lnot \varphi\)}
		\end{prooftree}
		\columnbreak
		\begin{prooftree}
			\AxiomC{\(\Gamma \sequent \varphi\)}
			\AxiomC{\(\Gamma \sequent \lnot \varphi\)}
			\LeftLabel{(\(\lnot\)E)}
			\BinaryInfC{\(\Gamma \sequent \bot\)}
		\end{prooftree}
	\end{multicols}
\end{Def}

等号に関する推論規則も必要である.
ここで,規則名「SUBST」は代入(substitution)に由来する.

\begin{Def}[等号に関する推論規則] \label[Def]{Def:SUBST}
	\(\varphi\)を論理式,\(x\)を変数記号,\(s,t\)を\(\varphi\)中の\(x\)に代入可能な項とするとき,
	以下は自然演繹式シークエント計算における推論規則である.
	\begin{prooftree}
		\AxiomC{\(s = t\)}
		\AxiomC{\(\subst{\varphi}{t / x}\)}
		\LeftLabel{(SUBST)}
		\BinaryInfC{\(\subst{\varphi}{s / x}\)}
	\end{prooftree}
\end{Def}

最後に,いわゆる背理法の基礎となる推論規則を導入しておく.
規則名「DNE」は2重否定除去(Double Negation Elimination)に由来する.

\begin{Def}[2重否定除去] \label[type]{Def:DNE}
	論理式の有限列\(\Gamma\)と論理式\(\varphi\)に対し,
	下記は自然演繹式シークエント計算における推論規則である.
	\begin{prooftree}
		\AxiomC{\(\Gamma \sequent \lnot \lnot \varphi\)}
		\LeftLabel{(DNE)}
		\UnaryInfC{\(\Gamma \sequent \varphi\)}
	\end{prooftree}
\end{Def}

以上で公理と推論規則の準備が整った.あとは「証明」っぽく見えるようにいくつか定義を行うだけである.

\begin{Def} \label[Def]{Def:provable}
	シークエントが%
	\index[widx]{どうしゅつかのう@(シークエントが)導出可能}%
	導出可能であることを,以下のように定義する:
	\begin{enumerate}
		\item 任意の論理式\(\varphi\)に対し,始式\(\varphi \sequent \varphi\)は導出可能である.
		\item これまでに挙げた各推論規則における上段のシークエントがすべて導出可能であるならば,下段のシークエントも導出可能である.
		\item 以上の規則を有限回適用して得られるシークエントのみが導出可能である.
	\end{enumerate}
\end{Def}

あるシークエントが導出可能であることを
確かめる作業のことをそのシークエントの「導出」と呼ぶことがある.
「シークエント\(\Gamma \sequent \varphi\)を導出する」といった言い回しである.

さて,シークエントの導出可能性の次は論理式の証明可能性である.

\begin{Def} \label[Def]{Def:logicalexpressionprovable}
	\(\varphi\)を論理式とする.シークエント
	\begin{equation}
		\sequent \varphi
	\end{equation}
	が導出可能であるとき,\(\varphi\)は%
	\index[widx]{しょうめいかのう@(論理式が)証明可能}%
	\term{証明可能}であるという.
\end{Def}


ここで定義した体系が我々の素朴的直観における「証明」を模倣できているかどうかは,
ここから導かれるさまざまな結果を見るよりほかはない.
例を見てみよう.

\begin{Thm}[背理法] \label[Thm]{Thm:RAA}
	\(\Gamma\)を論理式の有限列,\(\varphi\)を論理式とするとき,
	シークエント
	\(\lnot \varphi, \Gamma \sequent \bot\)
	が導出可能であるならば,シークエント
	\(\Gamma \sequent \varphi\)
	は導出可能である.
\end{Thm}

\begin{proof}
	シークエント\(\Gamma \sequent \varphi\)は次のように得られる.
	\begin{prooftree}
		\AxiomC{\(\lnot \varphi, \Gamma \sequent \bot \)}
		\LeftLabel{(\(\lnot\)I)}
		\UnaryInfC{\(\Gamma \sequent \lnot \lnot \varphi\)}
		\LeftLabel{(DNF)}
		\UnaryInfC{\(\Gamma \sequent \varphi\)}
	\end{prooftree}
	よって,シークエント\(\Gamma \sequent \varphi\)は導出可能である.
\end{proof}

\begin{Thm} \label[Thm]{Thm:DNFsequent}
	\(\varphi\)を論理式とするとき,シークエント\(\lnot \lnot \varphi \sequent \varphi\)と
	\(\varphi \sequent \lnot \lnot \varphi\)はいずれも導出可能である.
\end{Thm}

\begin{proof}
	シークエント\(\lnot \lnot \varphi \sequent \varphi\)は次のようにして得られる.
	\begin{prooftree}
		\AxiomC{}
		\LeftLabel{(ID)}
		\UnaryInfC{\(\lnot \lnot \varphi \sequent \lnot \lnot \varphi\)}
		\LeftLabel{(DNF)}
		\UnaryInfC{\(\lnot \lnot \varphi \sequent \varphi\)}
	\end{prooftree}
	次に,シークエント\(\varphi \sequent \lnot \lnot \varphi\)を導出する.
	\begin{prooftree}
		\AxiomC{}
		\LeftLabel{(ID)}
		\UnaryInfC{\(\lnot \varphi \sequent \lnot \varphi\)}
		\AxiomC{}
		\LeftLabel{(ID)}
		\UnaryInfC{\(\varphi \sequent \varphi\)}
		\LeftLabel{(\(\lnot\)E)}
		\BinaryInfC{\(\lnot \varphi, \varphi \sequent \bot\)}
		\LeftLabel{(\(\lnot\)I)}
		\UnaryInfC{\(\varphi \sequent \lnot \lnot \varphi\)}
	\end{prooftree}
	よって,シークエント\(\lnot \lnot \varphi \sequent \varphi\)と
	\(\varphi \sequent \lnot \lnot \varphi\)はいずれも導出可能である.
\end{proof}

シークエントの導出過程を示すとき,\cref{Thm:RAA}の証明のように各推論規則の適用過程を
上から下に進む木構造として書き下すのが普通である.
体系がもつ一般的な性質を研究する上ではこの記法は非常に便利なのだが,
個々のシークエントの導出を人間の目にわかりやすい形で行うには扱いずらい.
そこで,推論規則の適用過程は
導出可能であることが分かったシークエントを順に書き並べる形で示すこととする.
次の\cref{Thm:lawofexcludedmiddle}の証明で実例を見せるとしよう.

\begin{Thm}[排中律] \label[Thm]{Thm:lawofexcludedmiddle}
	\(\varphi\)を論理式とするとき,論理式
	\begin{equation}
		\varphi \lor \lnot \varphi
	\end{equation}
	は証明可能である.
\end{Thm}

\begin{proof}
	シークエント\(\sequent \varphi \lor \lnot \varphi\)
	を以下のようにして導く.
	\begin{enumerate}
		\item \(\lnot \paren{\varphi \lor \lnot \varphi} \sequent \lnot \paren{\varphi \lor \lnot \varphi}\)\quad (ID)
		\item \(\varphi \sequent \varphi\)\quad (ID)
		\item \(\varphi \sequent \varphi \lor \lnot \varphi\)\quad (2から(\(\lor\)I)による)
		\item \(\varphi, \lnot \paren{\varphi, \lnot \varphi} \sequent \lnot \paren{\varphi \lor \lnot \varphi}\)\quad (1から(w)による)
		\item \(\lnot \paren{\varphi \lor \lnot \varphi}, \varphi \sequent \varphi \lor \lnot \varphi\)\quad (3から(w)による)
		\item \(\varphi,\lnot \paren{\varphi \lor \lnot \varphi} \sequent \varphi \lor \lnot \varphi\)\quad (4から(e)による)
		\item \(\varphi, \lnot \paren{\varphi \lor \lnot \varphi} \sequent \bot\)\quad (\(4,6\)から(\(\lnot\)E)による)
		\item \(\lnot \paren{\varphi \lor \lnot \varphi} \sequent \lnot \varphi\)\quad (7から(\(\lnot\)I)による)
		\item \(\lnot \paren{\varphi \lor \lnot \varphi} \sequent \varphi \lor \lnot \varphi\)\quad (8から(\(\lor\)I)による)
		\item \(\lnot \paren{\varphi \lor \lnot \varphi} \sequent \bot\)\quad (\(1,9\)から(\(\lnot\)E)による)
		\item \(\sequent \lnot \lnot \paren{\varphi \lor \lnot \varphi}\)\quad (10から(\(\lnot\)I)による)
		\item \(\sequent \varphi \lor \lnot \varphi\)\quad (11から(DNF)による)
	\end{enumerate}
	よって\(\varphi \lor \lnot \varphi\)は証明可能である.
\end{proof}

\Cref{Thm:lawofexcludedmiddle}の証明で提示したシークエントの導出過程は,
通常の数学における次のような証明に対応していると考えることができる.

\begin{naiveproof}
	\(\lnot \paren{\varphi \lor \lnot \varphi}\)だと仮定して矛盾を導く.
	\(\varphi\)であるとすると\(\varphi \lor \lnot \varphi\)となり,
	\(\lnot \paren{\varphi \lor \lnot \varphi}\)と矛盾する.
	従って\(\lnot \varphi\)でなければならない.
	しかしこの場合も\(\varphi \lor \lnot \varphi\)となり
	\(\lnot \paren{\varphi \lor \lnot \varphi}\)と矛盾する.
	よって\(\lnot \lnot \paren{\varphi \lor \lnot \varphi}\),
	すなわち\(\varphi \lor \lnot \varphi\)となる.
\end{naiveproof}

構造規則の関係で多少人工的な操作が見られたものの,
\Cref{Thm:lawofexcludedmiddle}の証明はその下に述べた通常の数学における証明とうまく対応しているように見える.
それでいて形式的記号列の世界で議論しているおかげで各々の導出の根拠に一切のあいまいさが存在しない.

\Cref{Thm:lawofexcludedmiddle}と同じように考えれば,
数学における論理でよく知られてきた結果がこの自然演繹式シークエント計算においても導くことができる.

\begin{Thm} \label[Thm]{Thm:equivlogicalexpression}
	\(\varphi, \psi, \chi\)を論理式,\(x\)を変数記号とする.
	以下に述べる論理式のペアは,それぞれを左辺,右辺とするシークエントと
	その左右を入れ替えたシークエントのいずれも導出可能である.
	(「\(\varphi\)と\(\psi\)」と書かれていれば
	\(\varphi \sequent \psi\)と\(\psi \sequent \varphi\)の両方が導出可能である.)
	\begin{enumerate}
		\item \(\varphi \to \psi\)と\(\lnot \varphi \lor \psi\)
		\item \(\lnot \paren{\varphi \lor \psi}\)と\(\lnot \varphi \land \lnot \psi\)
		\item \(\lnot \paren{\varphi \land \psi}\)と\(\lnot \varphi \lor \lnot \psi\)
		\item \(\lnot \forall x \varphi\)と\(\exists x \lnot \varphi\)
		\item \(\lnot \exists x \varphi\)と\(\forall x \lnot \varphi\)
		\item \(\varphi \land \psi\)と\(\psi \land \varphi\)
		\item \(\varphi \lor \psi\)と\(\psi \lor \varphi\)
		\item \(\varphi \land \paren{\psi \land \chi}\)と\(\paren{\varphi \land \psi} \land \chi\)
		\item \(\varphi \lor \paren{\psi \lor \chi}\)と\(\paren{\varphi \lor \psi} \lor \chi\)
		\item \(\varphi \land \paren{\psi \lor \chi }\)と\(\paren{\varphi \land \psi} \lor \paren{\varphi \land \chi}\)
		\item \(\varphi \lor \paren{\psi \land \chi }\)と\(\paren{\varphi \lor \psi} \land \paren{\varphi \lor \chi}\)
		\item \(\lnot \paren{\varphi \to \psi}\)と\(\varphi \land \lnot \psi\)
		\item \(\lnot \forall x \paren{\varphi \to \psi}\)と\(\exists x \paren{\varphi \land \lnot \psi}\)
		\item \(\varphi \to \psi\)と\(\lnot \psi \to \lnot \varphi\)
		\item \(\varphi \to \paren{\psi \to \chi}\)と\(\varphi \land \psi \to \chi\)
		\item \(\forall x \paren{\varphi \to \psi}\)と\(\exists x \varphi \to \psi\)(ただし\(x\)は\(\psi\)には自由出現しないものとする)
		\item \(\exists x \paren{\varphi \to \psi}\)と\(\forall x \varphi \to \psi\)(ただし\(x\)は\(\psi\)には自由出現しないものとする)
		\item \(\forall x \paren{\varphi \to \psi}\)と\(\varphi \to \forall x \psi\)(ただし\(x\)は\(\varphi\)には自由出現しないものとする)
		\item \(\exists x \paren{\varphi \to \psi}\)と\(\varphi \to \exists x \psi\)(ただし\(x\)は\(\varphi\)には自由出現しないものとする)
		\item \(\forall x \paren{\varphi \land \psi}\)と\(\forall x \varphi \land \forall x \psi\)
		\item \(\exists x \paren{\varphi \lor \psi}\)と\(\exists x \varphi \lor \exists x \psi\)
		\item \(\forall x \paren{\varphi \lor \psi}\)と\(\forall x \varphi \lor \psi\)(ただし\(x\)は\(\psi\)には自由出現しないものとする)
		\item \(\exists x \paren{\varphi \land \psi}\)と\(\exists x \varphi \land \psi\)(ただし\(x\)は\(\psi\)には自由出現しないものとする)
	\end{enumerate}
\end{Thm}

\begin{Que} \label[Que]{Que:sequent}
	\Cref{Thm:equivlogicalexpression}を証明せよ.
\end{Que}

\begin{Que} \label[Que]{Que:peirce}
	\(\varphi, \psi\)を論理式とするとき,Peirceの法則に対応するシークエント
	\begin{equation}
		\paren{\varphi \to \psi} \to \varphi \sequent \varphi
	\end{equation}
	を導出せよ.
\end{Que}

等号についてもいろいろなシークエントを導くことができる.

\begin{Thm} \label[Thm]{Thm:equalsign}
	\(f\)をアリティ\(n\)の関数記号,\(s_1, s_2, \dotsc s_n, t_1, t_2, \dotsc, t_n\)を項とする.
	\(i \equiv 1, 2, \dotsc, n\)に対して,シークエント
	\begin{equation}
		s_i = t_i \sequent \apply{f}{s_1, s_2, \dotsc, s_i, \dotsc, s_n} = \apply{f}{s_1, s_2, \dotsc, t_i, \dotsc, s_n}
	\end{equation}
	は導出可能である.
\end{Thm}

\begin{Thm} \label[Thm]{Thm:equalsignrelation}
	\(s, t, u\)を項とするとき,シークエント
	\begin{align}
		s = t \sequent t = s \\
		\paren{s = t} \land \paren{t = u} \sequent s = u
	\end{align}
	は導出可能である.
\end{Thm}

\begin{Que} \label[Que]{Que:equalsignrelation}
	\Cref{Thm:equalsign}と\cref{Thm:equalsignrelation}を証明せよ.
\end{Que}
\include{contents/completeness}
\chapter{1階Peano算術}
\label{chap:peanoarithmetic}

本章より前の章では1階述語論理という演繹体系そのものの性質を学ぶことを目的に,
対象となる理論を限定せずに議論を進めてきた.
本章からはGödelの不完全性定理について学ぶことを目的に,
1階Peano算術という理論に焦点を当てて議論を進める.

まずは議論の対象となる1階Peano算術やその部分体系を定義し,
その証明能力を調べる.
形式体系に関する議論ではあるが,議論の中で我々が素朴に思い浮かべる
自然数全体の集合\(\NaturalNumbers\)が多岐にわたって登場する.
その際,形式体系についての議論なのか\(\NaturalNumbers\)についての議論なのかを混同しないように注意されたい.

特に,数学的帰納法については形式体系上の数学的帰納法と\(\NaturalNumbers\)上の通常の数学的帰納法の両方が同時に登場する.
数学的帰納法は算術の体系の証明能力を決定づけるといっても過言ではなく,
制限を加えたり公理から取り除いたりする.そのため,数学的帰納法を適用する際は,
形式体系上の数学的帰納法なのか\(\NaturalNumbers\)上の通常の数学的帰納法なのかを注意深く区別しながら読み進めるとよい.

\section{1階Peano算術とその部分体系}
\label{sec:peanoarithmetic}

まずは,Peano算術を1階理論として定義しよう.

\index[sidx]{\(\PA\):1階Peano算術}
\index[sidx]{\(\symcal{L}_{\PA}\):1階Peano算術の言語}
\index[widx]{1かいPeanoさんじゅつ@1階Peano算術}
\begin{Def} \label{Def:peanoarithmetic}
	1階Peano算術の言語を\(\symcal{L}_{\PA} = \Set{\obj{+}, \obj{\cdot}, \obj{0}, \obj{1}, \obj{<}}\)とする.
	ここで,\(\obj{+}, \obj{\cdot}\)はアリティ2の関数記号,\(\obj{0}, \obj{1}\)は定数記号,\(\obj{<}\)はアリティ2の関数記号である.
	1階Peano算術\(\PA\)は,以下の公理からなる\(\symcal{L}_{\PA}\)理論である:
	\begin{description}
		\item[A1.] \(\forall \obj{x} \lnot \paren{\obj{x} \obj{+} \obj{1} \objeq \obj{0}},\)
		\item[A2.] \(\forall \obj{x} \forall \obj{y} \paren{\obj{x} \obj{+} \obj{1} \objeq \obj{y} \obj{+} \obj{1} \to \obj{x} \objeq \obj{y}},\)
		\item[A3.] \(\forall \obj{x} \paren{\obj{x} \obj{+} \obj{0} \objeq \obj{x}},\)
		\item[A4.] \(\forall \obj{x} \forall \obj{y} \paren{\obj{x} \obj{+} \paren{\obj{y} + \obj{1}} \objeq \paren{\obj{x} \obj{+} \obj{y}} \obj{+} \obj{1}},\)
		\item[A5.] \(\forall \obj{x} \paren{\obj{x} \obj{\cdot} \obj{0} \objeq \obj{0}},\)
		\item[A6.] \(\forall \obj{x} \forall \obj{y} \paren{\obj{x} \obj{\cdot} \paren{\obj{y} \obj{+} \obj{1}} \objeq \paren{\obj{x} \obj{\cdot} \obj{y}} \obj{+} \obj{1}},\)
		\item[A7.] \(\forall \obj{x} \lnot \paren{\obj{x} \obj{<} \obj{0}},\)
		\item[A8.] \(\forall \obj{x} \forall \obj{y} \paren{\obj{x} \obj{<} \obj{y} \obj{+} \obj{1} \formulaequiv \obj{x} \obj{<} \obj{y} \lor \obj{x} \objeq \obj{y}},\)
		\item[A9.] 変数記号\(\obj{x}\)が自由出現するような\(\symcal{L}_{\PA}\)論理式\(\apply{\varphi}{\obj{x}}\)に対する論理式
		      \begin{equation}
			      \apply{\varphi}{\obj{0}} \land \forall \obj{x} \paren{\apply{\varphi}{\obj{x}} \to \apply{\varphi}{\obj{x} \obj{+} \obj{1}}} \to \forall \obj{x} \apply{\varphi}{\obj{x}}
			      \label{eq:inductionscheme}
		      \end{equation}
		      の全称閉包すべて.
	\end{description}
\end{Def}

\(\PA\)の各公理の役割を述べておこう.まず,(A1), (A2)は通常の数学における「次の数」に相当する「\(\obj{x} + \obj{1}\)」に関する公理である.
(A3), (A4)は加法に関する公理であり,通常の数学における加法の帰納的定義に相当するものである.
(A5), (A6)と(A7), (A8)は,それぞれ通常の数学における乗法と大小関係の帰納的定義に相当する公理である.
(A1)から(A8)までは\(\symcal{L}_{\PA}\)の各記号の定義ともいえる公理であるが,(A9)だけは特異的である.
意味的には数学的帰納法に相当するものであるが,(A9)は単一の公理ではなく可算無限個ある論理式すべてに対する公理をまとめて書いた%
\index[widx]{こうり@公理!こうりずしき@---図式}%
\term{公理図式}である.
この公理図式のために,\(\PA\)は有限個の論理式からなる理論ではなく可算無限個の公理からなる理論になっている.

\begin{Note}
	ここで定義した理論\(\PA\)は1階Peano算術と呼ばれるものであるが,
	\(\PA\)は集合\(N\)と\(N\)の元\(o\),そして写像\(S \colon N \to N\)の対\(\pair{N, o, S}\)がPeano構造であるための条件を記述した
	いわゆる「Peanoの公理」とは異なるものである.
	いわゆる「Peanoの公理」はその記述に写像や部分集合といった集合のことばが使われており,
	そのままでは1階理論として表現することはできない.
	しかも,書き方の違いだけかといえばそうではない.質的に大きく異なる点として,例えば以下の2つを挙げることができる:
	\begin{enumerate}
		\item Peanoの公理はPeano構造を特徴づけるが,\(\PA\)は通常の意味での自然数全体の集合\(\NaturalNumbers\)(に種々の構造を入れたもの)を特徴づける公理系ではない.
		\item Peano構造上の加法は0と後者関数(「次の数」をとる関数)だけから定義できるが,1階の理論ではそれは不可能である.
	\end{enumerate}

	このうち,2.について補足しておく.アリティ2の関係記号\(\obj{<}\)とアリティ1の関数記号\(\obj{S}\)からなる言語\(\symcal{L} = \Set{\obj{<}, \obj{S}}\)を考える.
	有限順序数全体の集合%
	\footnote{%
		素朴には自然数全体の集合だととらえて差し支えない.%
	}%
	\(\omega\)に対し,\(\omega\)上の2項関係\(<\)と写像\(S \colon \omega \to \omega\)を以下のように定める:
	\begin{align*}
		\alpha < \beta \metaequivalent \alpha \in \beta, \\
		\apply{S}{\alpha} = \alpha \cup \Set{\alpha}.
	\end{align*}
	\(\obj{<}, \obj{S}\)の解釈をそれぞれ上で定義した\(\mathord{<}, S\)と定めることで,
	\(\omega\)を対象領域とする\(\symcal{L}\)構造\(\symcal{N}\)を定義することができる.
	このとき,任意の\(\alpha, \beta, \gamma \in \omega\)に対して以下を満たす\(\symcal{L}\)論理式\(\apply{\varphi}{x, y, z}\)は存在しないことが知られている:
	\[
		\symcal{N} \satisfy \apply{\varphi}{\alpha, \beta, \gamma} \metaequivalent \alpha + \beta = \gamma.
	\]
	なお,有限順序数の加法については\(\PA\)における(A3), (A4)と同じようにして帰納的に定義できる.

	用語としての「Peanoの公理」は一般の知名度自体はそれなりに高いものの,それゆえかかなり雑に使われがちなようである.
	本書ではPeano構造については取り扱わないが,世間的な「Peanoの公理」のイメージに惑わされて1階Peano算術\(\PA\)とPeano構造とを混同しないように注意されたい.
\end{Note}

\backmatter

\RenewDocumentCommand{\presectionname}{}{}
\RenewDocumentCommand{\presubsectionname}{}{}
\chapter{演習問題解答} \label{chap:answer}

\section*{\Cref{chap:formulize}}

\subsection*{\Cref{Que:termexample}} \label{answer:termexample}

\(\obj{e}\)は変数記号,\(\obj{x}\)は変数記号なので,どちらも\(\symcal{L}_1\)項である.
これと\(\obj{\ast}\)がアリティ2の関数記号であることにより,\(\apply{\obj{\ast}}{\obj{e}, \obj{x}}\)が
\(\symcal{L}_1\)項であることがわかる.
\(\obj{y}\)も変数記号であり\(\symcal{L_1}\)項なので,
\(\apply{\obj{\ast}}{\apply{\obj{\ast}}{\obj{e}, \obj{x}}, \obj{y}}\)は\(\symcal{L}_1\)項である.

また,\(\obj{e}\apply{\mathord{\obj{\ast}}}{\obj{e}, \obj{e}}\)は変数記号でも定数記号でもないため,
これが\(\symcal{L}_1\)項であるためには
\cref{Def:term}における3番目の規則を最後に適用していなければならない.
このとき最初の文字は関数記号である必要があるが,\(\obj{e}\)は定数記号であり関数記号ではない.
従って,\(\obj{e}\apply{\mathord{\obj{\ast}}}{\obj{e}, \obj{e}}\)は\(\symcal{L}_1\)項ではない.

\subsection*{\Cref{Que:logicalexpression}}

\(\obj{x}, \obj{y}\)は変数記号であり,従って\(\symcal{L}_1\)項である.
また,\(\obj{\ast}\)がアリティ2の関数記号であることから,
\(\apply{\mathord{\obj{\ast}}}{\obj{x}, \obj{y}}\)と
\(\apply{\mathord{\obj{\ast}}}{\obj{y}, \obj{x}}\)はいずれも\(\symcal{L}_1\)項である.
よって,
\(\paren{\apply{\mathord{\obj{\ast}}}{\obj{y}, \obj{x}} = \apply{\mathord{\obj{\ast}}}{\obj{y}, \obj{x}}}\)
は\(\symcal{L}_1\)論理式である.

\(\obj{\leq}\)がアリティ2の関係記号であることから,
\(\apply{\mathord{\obj{\leq}}}{\obj{x}, \obj{y}}\)は\(\symcal{L}_2\)論理式である.
ゆえに,\(\paren{\forall \obj{y}\apply{\mathord{\obj{\leq}}}{\obj{x}, \obj{y}}}\)
は\(\symcal{L}_2\)論理式である.
このことから\(\paren{\forall \obj{x}\paren{\forall \obj{y}\apply{\mathord{\obj{\leq}}}{\obj{x}, \obj{y}}}}\)
が\(\symcal{L}_2\)論理式であることも従う.

さて,\(\paren{\apply{\exists \obj{x}}{\obj{x}}}\)が\(\symcal{L}\)論理式であるとすれば,
\(\obj{x}\)は\(\symcal{L}\)論理式でなければならない.しかし,変数記号が論理式であることはないので
\(\paren{\apply{\exists \obj{x}}{\obj{x}}}\)は\(\symcal{L}\)論理式ではない.
一方で,\(\paren{\obj{x} = \obj{x}}\)は\(\symcal{L}\)論理式であるから
\(\paren{\apply{\exists \obj{x}}{\obj{x} = \obj{x}}}\)は\(\symcal{L}\)論理式である.

\(\apply{\forall \obj{x}}{\obj{x} \obj{\ast} \obj{e} = \obj{x}}\)が\(\symcal{L}_1\)論理式であるとすれば,
これまでの議論と同様にして\(\obj{x} \obj{\ast} \obj{e}\)が\(\symcal{L}_1\)項でなければならないことがわかる.
しかし,\(\obj{\ast}\)の次の文字が開きカッコ「\(\lparen\)」ではないので
これは\(\symcal{L}_1\)項とはなりえない.よって,
\(\apply{\forall \obj{x}}{\obj{x} \obj{\ast} \obj{e} = \obj{x}}\)は\(\symcal{L}_1\)論理式ではない.
\(\apply{\forall \obj{x}}{\obj{x} = \obj{x}}\)については,1文字目が開きカッコ「\(\lparen\)」でも
関係記号でもないことから\(\symcal{L}_1\)論理式でないことが従う.

\nocite{*}
\printbibliography[title=参考文献, heading=bibintoc]

\printindex[sidx]
\printindex[widx]

\newpage
\pagestyle{empty}
~
\newpage
~
\newpage

\pagestyle{empty}
%\newpage
%~
%\newpage
\vspace*{\fill}
\noindent
\begin{picture}(110,1)
	\setlength{\unitlength}{1truemm}
	\put(15,3){\Large \textbf{0から始める数理論理学入門 第2.6版}}
	\thicklines
	\put(0,1){\line(2,0){110}}
	\thinlines
	\put(0,0){\line(2,0){110}}
\end{picture}

2023年12月 初版

2024年8月 第2版

2024年11月 第2.5版

2024年11月 第2.6版

著者:野口 匠

Twitter:@Nmatician

発行:NOGUTAKU Lab

印刷:株式会社ポプルス

\begin{picture}(110,1)
	\setlength{\unitlength}{1truemm}
	\thinlines
	\put(-3,1){\line(2,0){110}}
	\thicklines
	\put(-3,0){\line(2,0){110}}
\end{picture}

\end{document}

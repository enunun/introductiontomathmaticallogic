\chapter{1階述語論理の統語論}
\label[chapter]{chap:syntax}

本章では,1階述語論理と呼ばれる体系の統語論について述べる.
統語論とは,雑に述べると「文の構造」についての理論である.
この「文」は,ここでは「数学における何らかの主張」に対応する.
つまり,ここで議論したいのは
「数学における何らかの主張はどのような要素がどう組み合わさってできているのか」
ということである.
このことを議論するための足掛かりとして,我々は\cref{chap:formulize}で述べた方針に従い
「何ら素朴的直観が関与しない形式的な記号列の世界」において
論理式という概念を構成していく.

形式的記号列の世界での定義により,何を議論の対象とし,何を議論の対象としないのかが明瞭となる.
これにより,通常の数学と同じく一般的な法則や定理を研究することが可能となる.
本章の内容はその前準備に相当する.

\section{言語} \label[seciton]{sec:language}

議論を始めるにあたり,どのような記号を使用するのかを明瞭にする必要がある.
まず,通常の数学においてはどうなっているかを振り返ろう.

\begin{Ex} \label[Ex]{Ex:informalsymbol}
	群論においては,単位元を表す記号「\(\obj{e}\)」,2項演算を表す記号
	「\(\obj{\mathord{\ast}}\)」,逆元を表す記号「\({}^{\obj{-1}}\)」が用いられる.
	また,順序の理論においては,順序を表す記号「\(\obj{\le}\)」が用いられる.

	この他,「\(\obj{x}\)」や「\(\obj{y}\)」等の変数を表す記号やカッコ「\(\lparen\)」「\(\rparen\)」や
	カンマ「\(,\)」等は共通して用いられる.
\end{Ex}

\Cref{Ex:informalsymbol}では,対象の理論に依存して必要であったりそうでなかったりする記号と
共通して用いられる記号があった.
そのような記号を\cref{tab:commonsymbol}に示し,今後逐一言及しないものとする%
\footnote{%
	ここに示す記号とは違う記号を採用する場合もあるが,それは単に流儀や議論の対象の違いである.
}%
.

\begin{table}[htbp]
	\centering
	\caption{数学理論で共通して用いられる記号}
	\label{tab:commonsymbol}
	\begin{tabular}{ccc}
		\toprule
		種別   & 一覧                                                                   & 備考              \\
		\midrule
		変数記号 & \(\obj{x}, \obj{y}, \obj{z}, \dotsc\)                                & 無限に多く存在する(可算無限) \\
		論理記号 & \(\bot, \lnot, \land, \lor, \to, \forall, \exists\)                                    \\
		等号   & \(\obj{\mathord{=}}\)                                                & オブジェクト側の意味での等号  \\
		補助記号 & \(\text{``\(\lparen\)''}, \text{``\(\rparen\)''}, \text{``\(,\)''}\)                   \\
		\bottomrule
	\end{tabular}
\end{table}

「名称」列に「変数記号」や「論理記号」等の名前が定義なしに入っているが,これは単にそういう区分けができることのみが要請される.
また,変数記号については無限に多く存在することが要請されているが,これについてはすでにいくつか記号が登場している状況下において
そのいずれとも異なる変数記号をいつでも用意できることを期待しての要請である.

さて,\cref{tab:commonsymbol}に追加で記号を付け加えることにより,各理論の特色が現れる.

\begin{Def} \label{def:language}
	記号の集合\(\symcal{L}\)が%
	\index[widx]{げんご@言語}%
	\term{言語}であるとは,\(\symcal{L}\)の元が以下の3種類に区分けされていることをいう:
	\begin{itemize}
		\item \index[widx]{ていすうきごう@定数記号}定数記号
		\item \index[widx]{かんすうきごう@関数記号}関数記号,\index[widx]{ありてぃ@アリティ}\term{アリティ}と呼ばれる正の整数\(n\)をもつ
		\item \index[widx]{かんけいきごう@関係記号}関係記号,\index[widx]{ありてぃ@アリティ}アリティと呼ばれる正の整数\(n\)をもつ
	\end{itemize}
\end{Def}

「定数記号」や「関数記号」という字面はいかにも我々の素朴的直観を呼び起こしそうであるが,
ここでは単にそういう名称で区分けできるということだけを要請しているに過ぎない.
アリティについても同様である.「この関数記号のアリティはいくつですか?」に対して「2です」のような
解答を返せることを要請しているに過ぎない.

言語については,例を述べるのがわかりやすいだろう.

\begin{Def} \label[Def]{def:languageexample}
	群論の言語\(\symcal{L}_1\)は\(\symcal{L}_1 = \set{\obj{\ast}, \obj{e}, {}^{\obj{-1}}}\)
	と与えることができる.ここで,\(\obj{\ast}\)はアリティ2の関数記号,\(\obj{e}\)は定数記号,
	\({}^{\obj{-1}}\)はアリティ1の関数記号である.
	また,順序の理論の言語\(\symcal{L}_2\)は\(\symcal{L}_2 = \set{\obj{\le}}\)と与えることができる.
	ここで,\(\obj{\le}\)はアリティ2の関係記号である.
\end{Def}

アリティというのは,素朴的直観側でいえば「引数の数」を表現しているものと考えられる.
こう考えてみれば,関数記号や関係記号にアリティが定まっていることを要請するのはごく自然であろう.


\section{項と論理式} \label[section]{sec:logicalexpression}

使う記号を定義した
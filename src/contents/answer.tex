\chapter{演習問題解答} \label{chap:answer}

\section*{\Cref{chap:formulize}}

\subsection*{\Cref{Que:termexample}} \label{answer:termexample}

\(\obj{e}\)は変数記号,\(\obj{x}\)は変数記号なので,どちらも\(\symcal{L}_1\)項である.
これと\(\obj{\ast}\)がアリティ2の関数記号であることにより,\(\apply{\obj{\ast}}{\obj{e}, \obj{x}}\)が
\(\symcal{L}_1\)項であることがわかる.
\(\obj{y}\)も変数記号であり\(\symcal{L_1}\)項なので,
\(\apply{\obj{\ast}}{\apply{\obj{\ast}}{\obj{e}, \obj{x}}, \obj{y}}\)は\(\symcal{L}_1\)項である.

また,\(\obj{e}\apply{\mathord{\obj{\ast}}}{\obj{e}, \obj{e}}\)は変数記号でも定数記号でもないため,
これが\(\symcal{L}_1\)項であるためには
\cref{Def:term}における3番目の規則を最後に適用していなければならない.
このとき最初の文字は関数記号である必要があるが,\(\obj{e}\)は定数記号であり関数記号ではない.
従って,\(\obj{e}\apply{\mathord{\obj{\ast}}}{\obj{e}, \obj{e}}\)は\(\symcal{L}_1\)項ではない.
\chapter{1階述語論理の意味論}
\label[chapter]{chap:semantics}

本章では,1階述語論理の意味論について述べる.
\Cref{chap:formulize}から繰り返し「数理論理学で扱うのは形式的記号列である」と
繰り返し述べてきたが,意味論について議論する文脈においてはこれらの記号列に「意味」を与える.
このことは,形式的記号列の世界と通常の数学の世界との橋渡しを行っていると考えることもできる.
理論のモデルや同型の概念に覚えがある読者は多いだろう.
普段使っている概念を一歩引いた形で見直すことにより得られるものも多いはずである.

\section{モデル} \label[section]{sec:model}

我々は,項や論理式を純粋な形式的記号列として導入し,
数学理論の形式化を試みた.
例として群論や順序の理論を取り扱ったが,通常の数学においては
これらは集合や写像の言葉で書かれることがほとんどである.
両者の間の関係を考えよう.
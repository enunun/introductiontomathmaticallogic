\chapter{1階Peano算術}
\label{chap:peanoarithmetic}

本章より前の章では1階述語論理という演繹体系そのものの性質を学ぶことを目的に,
対象となる理論を限定せずに議論を進めてきた.
本章からはGödelの不完全性定理について学ぶことを目的に,
1階Peano算術という理論に焦点を当てて議論を進める.

まずは議論の対象となる1階Peano算術やその部分体系を定義し,
その証明能力を調べる.
形式体系に関する議論ではあるが,議論の中で我々が素朴に思い浮かべる
自然数全体の集合\(\NaturalNumbers\)が多岐にわたって登場する.
その際,形式体系についての議論なのか\(\NaturalNumbers\)についての議論なのかを混同しないように注意されたい.

特に,数学的帰納法については形式体系上の数学的帰納法と\(\NaturalNumbers\)上の通常の数学的帰納法の両方が同時に登場する.
数学的帰納法は算術の体系の証明能力を決定づけるといっても過言ではなく,
制限を加えたり公理から取り除いたりする.そのため,数学的帰納法を適用する際は,
形式体系上の数学的帰納法なのか\(\NaturalNumbers\)上の通常の数学的帰納法なのかを注意深く区別しながら読み進めるとよい.

\section{1階Peano算術とその部分体系}
\label{sec:peanoarithmetic}

まずは,Peano算術を1階理論として定義しよう.

\index[sidx]{\(\PA\):1階Peano算術}
\index[sidx]{\(\symcal{L}_{\Ar}\):1階Peano算術の言語}
\index[widx]{1かいPeanoさんじゅつ@1階Peano算術}
\begin{Def} \label{Def:peanoarithmetic}
	1階Peano算術の言語を\(\symcal{L}_{\Ar} = \Set{\obj{+}, \obj{\cdot}, \obj{0}, \obj{1}, \obj{<}}\)とする.
	ここで,\(\obj{+}, \obj{\cdot}\)はアリティ2の関数記号,\(\obj{0}, \obj{1}\)は定数記号,\(\obj{<}\)はアリティ2の関数記号である.
	\term{1階Peano算術}\(\PA\)は,以下の公理からなる\(\symcal{L}_{\Ar}\)理論である:
	\begin{enumerate}[label=A\arabic*.,ref=A\arabic*]
		\item \label{item:PAA1} \(\forall \obj{x} \lnot \paren{\obj{x} \obj{+} \obj{1} \objeq \obj{0}},\)
		\item \label{item:PAA2} \(\forall \obj{x} \forall \obj{y} \paren{\obj{x} \obj{+} \obj{1} \objeq \obj{y} \obj{+} \obj{1} \to \obj{x} \objeq \obj{y}},\)
		\item \label{item:PAA3} \(\forall \obj{x} \paren{\obj{x} \obj{+} \obj{0} \objeq \obj{x}},\)
		\item \label{item:PAA4} \(\forall \obj{x} \forall \obj{y} \paren{\obj{x} \obj{+} \paren{\obj{y} + \obj{1}} \objeq \paren{\obj{x} \obj{+} \obj{y}} \obj{+} \obj{1}},\)
		\item \label{item:PAA5} \(\forall \obj{x} \paren{\obj{x} \obj{\cdot} \obj{0} \objeq \obj{0}},\)
		\item \label{item:PAA6} \(\forall \obj{x} \forall \obj{y} \paren{\obj{x} \obj{\cdot} \paren{\obj{y} \obj{+} \obj{1}} \objeq \paren{\obj{x} \obj{\cdot} \obj{y}} \obj{+} \obj{x}},\)
		\item \label{item:PAA7} \(\forall \obj{x} \forall \obj{y} \paren{\obj{x} \obj{<} \obj{y} \formulaequiv \exists \obj{z} \paren{\obj{z} \obj{+} \paren{\obj{x} \obj{+} \obj{1}}\objeq \obj{y}}}.\)
		\item \label{item:PAA8} 変数記号\(x\)が束縛出現しない\(\symcal{L}_{\Ar}\)論理式\(\apply{\varphi}{x}\)に対する論理式
		      \begin{equation}
			      \apply{\varphi}{\obj{0}} \land \forall x \paren{\apply{\varphi}{x} \to \apply{\varphi}{x \obj{+} \obj{1}}} \to \forall x \apply{\varphi}{x}
			      \label{eq:inductionscheme}
		      \end{equation}
		      の全称閉包すべて.
	\end{enumerate}
\end{Def}

\(\PA\)の各公理の役割を述べておこう.まず,\cref{item:PAA1}, \cref{item:PAA2}は通常の自然数における「次の数」に相当する「\(\obj{x} + \obj{1}\)」に関する公理である.
\cref{item:PAA3}, \cref{item:PAA4}と\cref{item:PAA5}, \cref{item:PAA6}は,
それぞれ通常の自然数における自然数の加法と乗法の帰納的定義に相当する公理である.
また,\cref{item:PAA7}は通常の自然数における大小関係の定義に相当する公理である.
\cref{item:PAA1}から\cref{item:PAA7}までは\(\symcal{L}_{\Ar}\)の各記号の定義ともいえる公理であるが,\cref{item:PAA8}だけは特異的である.
意味的には数学的帰納法に相当するものであるが,\cref{item:PAA8}は単一の公理ではなく可算無限個ある論理式すべてに対する公理をまとめて書いた%
\index[widx]{こうり@公理!こうりずしき@---図式}%
\term{公理図式}である.
この公理図式のために,\(\PA\)は有限個の論理式からなる理論ではなく可算無限個の公理からなる理論になっている.

\begin{Note}
	ここで定義した理論\(\PA\)は1階Peano算術と呼ばれるものであるが,
	\(\PA\)は集合\(N\)と\(N\)の元\(o\),そして写像\(S \colon N \to N\)の対\(\pair{N, o, S}\)がPeano構造であるための条件を記述した
	いわゆる「Peanoの公理」とは異なるものである.
	いわゆる「Peanoの公理」はその記述に写像や部分集合といった集合のことばが使われており,
	そのままでは1階理論として表現することはできない.
	しかも,書き方の違いだけかといえばそうではない.質的に大きく異なる点として,例えば以下の2つを挙げることができる:
	\begin{enumerate}
		\item いわゆる「Peanoの公理」はPeano構造を特徴づける公理系であるが,
		      \(\PA\)は通常の意味での自然数全体の集合\(\NaturalNumbers\)(に通常の意味での演算や大小関係の構造を入れたもの)を特徴づける公理系ではない.
		\item Peano構造上の加法は0と後者関数(「次の数」をとる関数)だけから定義できるが,1階の理論ではそれは不可能である.
	\end{enumerate}

	このうち,2.について補足しておく.アリティ2の関係記号\(\obj{<}\)とアリティ1の関数記号\(\obj{S}\)からなる言語\(\symcal{L} = \Set{\obj{<}, \obj{S}}\)を考える.
	有限順序数全体の集合%
	\footnote{%
		素朴には自然数全体の集合だととらえて差し支えない.%
	}%
	\(\omega\)に対し,\(\omega\)上の2項関係\(<\)と写像\(S \colon \omega \to \omega\)を以下のように定める:
	\begin{align*}
		\alpha < \beta \metaequivalent \alpha \in \beta \quad & (\alpha, \beta \in \omega), \\
		\apply{S}{\alpha} = \alpha \cup \Set{\alpha} \quad    & (\alpha \in \omega).
	\end{align*}
	\(\obj{<}, \obj{S}\)の解釈をそれぞれ上で定義した\(\mathord{<}, S\)と定めることで,
	\(\omega\)を対象領域とする\(\symcal{L}\)構造\(\symcal{N}\)を定義することができる.
	このとき,任意の\(\alpha, \beta, \gamma \in \omega\)に対して以下を満たす\(\symcal{L}\)論理式\(\apply{\varphi}{x, y, z}\)は存在しないことが知られている:
	\[
		\symcal{N} \satisfy \apply{\varphi}{\alpha, \beta, \gamma} \metaequivalent \alpha + \beta = \gamma.
	\]
	なお,有限順序数の加法については\(\PA\)における\cref{item:PAA3}, \cref{item:PAA4}と同じようにして帰納的に定義できる.
\end{Note}

\(\PA\)の公理図式\cref{item:PAA8}は,(変数記号\(x\)が束縛出現しない)すべての\(\symcal{L}_{\Ar}\)論理式に対して要請されるものである.
ここに制限を加えることによって\(\PA\)の部分体系を得ることを考えよう.

\index[sidx]{\(\Robinson\):Robinson算術}
\index[widx]{Robinsonさんじゅつ@Robinson算術}
\begin{Def} \label{Def:robinsonarithmetic}
	\term{Robinson算術}\(\Robinson\)は,\(\PA\)における公理\cref{item:PAA1}から\cref{item:PAA7}に
	以下の公理を加えた8個の閉論理式からなる\(\symcal{L}_{\Ar}\)理論である:
	\begin{enumerate}[label=A\arabic*.,ref=A\arabic*,start=9]
		\item \label{item:PAA9} \(\forall \obj{x} \paren{\obj{x} \objeq \obj{0} \lor \exists \obj{y} \paren{\obj{x} \objeq \obj{y} \obj{+} \obj{1}}},\)
	\end{enumerate}
\end{Def}

\begin{Lemma} \label[Lemma]{lemma:paa9redudant}
	\(\PA\)において\cref{item:PAA9}は証明できる.すなわち,
	\[
		\PA \provable \text{\cref{item:PAA9}}
	\]
	が成り立つ.
\end{Lemma}

\begin{proof}
	\(\symcal{M}\)を\(M\)を対象領域とする\(\PA\)の任意のモデルとする.
	完全性定理により,\(\symcal{M} \satisfy \text{\cref{item:PAA9}}\)を示せば十分である.
	\(\obj{x}\)を変数記号とし,論理式\(\apply{\varphi}{\obj{x}}\)を\(\obj{x} \objeq \obj{0} \lor \exists \obj{y} \paren{\obj{x} \objeq \obj{y} \obj{+} \obj{1}}\)と定める.
	\(\interpretation{\symcal{M}}{\obj{0}} = \interpretation{\symcal{M}}{\obj{0}}\)だから
	\(\symcal{M} \satisfy \obj{0} \objeq \obj{0}\)であり,
	\(\symcal{M} \satisfy \apply{\varphi}{\obj{0}}\)が成り立つ.
	\(x \in M\)を任意にとり,その名前を\(c_x\)とする.
	\(x \mathbin{\interpretation{\symcal{M}}{\obj{+}}} \interpretation{\symcal{M}}{\obj{1}} = x \mathbin{\interpretation{\symcal{M}}{\obj{+}}} \interpretation{\symcal{M}}{\obj{1}}\)だから
	\(\symcal{M} \satisfy c_x \obj{+} \obj{1} \objeq c_x \obj{+} \obj{1}\)となり,
	\(\symcal{M} \satisfy \exists \obj{y} \paren{c_x \obj{+} \obj{1} \objeq \obj{y} \obj{+} \obj{1}}\)である.
	よって\(\symcal{M} \satisfy \apply{\varphi}{c_x \obj{+} \obj{1}}\)となる.従って
	\[
		\symcal{M} \satisfy \apply{\varphi}{\obj{0}} \land \forall \obj{x} \paren{\apply{\varphi}{\obj{x}} \to \apply{\varphi}{\obj{x} \obj{+} \obj{1}}}
	\]
	となり,\cref{item:PAA8}から\(\symcal{M} \satisfy \forall \obj{x} \apply{\varphi}{\obj{x}}\)を得る.
\end{proof}

\index[widx]{もでる@モデル!ひょうじゅんもでる@標準---}
\index[widx]{もでる@モデル!ちょうじゅんもでる@超準---}
\index[sidx]{\(\standardmodel\):\(\PA\)の標準モデル}
\begin{Def} \label{Def:standardstructure}
	自然数全体の集合\(\NaturalNumbers\)を対象領域とし,\(\symcal{L}_{\Ar}\)の各記号\(\obj{+}, \obj{\cdot}, \obj{0}, \obj{1}, \obj{<}\)の解釈をそれぞれ
	自然数における通常の意味での\(\mathord{+}, \mathord{\cdot}, 0, 1, \mathord{<}\)定めることにより得られる\(\symcal{L}_{\Ar}\)構造は\(\PA\)のモデルである.
	このモデル(およびこのモデルと同型なモデル)を\(\PA\)の\term{標準モデル}といい,本書では
	\begin{equation}
		\standardmodel
		\label{eq:standardmodel}
	\end{equation}
	と表す.
	また,\(\standardmodel\)と同型ではない\(\PA\)のモデルを
	\(\PA\)の\term{超準モデル}という.

	\(\PA\)の標準モデルは,通常の意味での自然数全体の集合(に通常の意味での演算や大小関係の構造を入れたもの)のことである.
\end{Def}

\begin{Note}
	とくに誤解のおそれがないときは,標準モデル\(\standardmodel\)と自然数全体の集合\(\NaturalNumbers\)を区別しないことも多い.
	これは,\(\NaturalNumbers\)に通常の意味での演算や大小関係の構造が入っていないという文脈は考えにくいからである.
\end{Note}

\begin{Note}
	\Cref{lemma:paa9redudant}の証明のように,形式体系において具体的な定理を証明する際は完全性定理を使ったモデル経由の証明の方が書きやすい.
	表記を簡素化するため,以下のように記述を省略して書くことにする:
	\begin{itemize}
		\item \(\symcal{L}_{\Ar}\)の各記号の構造\(\symcal{M}\)による解釈はその記号をタイプライタ体でない通常の書体で書くことによって表現する.
		      例えば,\(\symcal{L}_{\Ar}\)の記号\(\obj{0}, \obj{+}\)の\(\symcal{M}\)による
		      解釈\(\interpretation{\symcal{M}}{\obj{0}}, \interpretation{\symcal{M}}{\obj{+}}\)をそれぞれ単に\(0, +\)と表記する.
	\end{itemize}

	この簡素化やのちに述べる数項や算術化により,メタとオブジェクトの区別がますますつけづらくなる.
	どちらの立場で議論しているのかはこまめに確認することを推奨する.
	また,メタ側の場合には標準モデルでの議論なのかそうでないのかの区別も重要である.
\end{Note}



\index[sidx]{\(\numeral{n}\):数項}
\index[widx]{こう@項!すうこう@数---}
\begin{Def} \label{Def:numeral}
	\(n\)を自然数とする.\(n\)に対する\(\symcal{L}_{\Ar}\)項\(\numeral{n}\)を以下のように帰納的に定義する:
	\begin{align}
		\numeral{0}     & = \obj{0},
		\label{eq:numeral0}                                                                    \\
		\numeral{n + 1} & = \paren{\numeral{n} \obj{+} \obj{1}} \quad (n \in \NaturalNumbers).
		\label{eq:numeralnext}
	\end{align}
	ここで,\cref{eq:numeral0}において,左辺の\(0\)はメタ側の(つまり通常の意味での)自然数\(0\)であり,右辺の\(\obj{0}\)は\(\symcal{L}_{\Ar}\)の定数記号\(\obj{0}\)である.
	\(\numeral{0}\)は\(\obj{0}\)と同一の\(\symcal{L}_{\Ar}\)項であることにも注意しよう.
	\cref{eq:numeralnext}において,左辺の\(n, \mathord{+}, 1\)はそれぞれメタ側の(つまり通常の意味での)自然数\(n\),加法\(+\),および自然数\(1\)であり,
	左辺の\(\numeral{n}, \obj{+}, \obj{1}\)はそれぞれ\(\symcal{L}_{\Ar}\)項\(\numeral{n}\),関数記号\(\obj{+}\),および定数記号\(\obj{1}\)であることに注意しよう.

	自然数\(n\)に対する\(\numeral{n}\)を,\(n\)に対する\term{数項}と呼ぶ.
\end{Def}

\begin{Note}
	\NewDocumentCommand{\Num}{}{\operatorname{Num}}
	\Cref{Def:numeral}における「帰納的定義」を正当化するのは,メタ側の(つまり通常の意味での自然数上での)帰納的定義である.
	フォーマルに書けば,\(\standardmodel\)における帰納的定義を使って
	\begin{align*}
		\apply{\Num}{0}      & = \obj{0},                                                              \\
		\apply{\Num}{n  + 1} & = \paren{\apply{\Num}{n} \obj{+} \obj{1}} \quad (n \in \NaturalNumbers)
	\end{align*}
	を満たす写像\(\Num \colon \NaturalNumbers \to \symcal{T}\)を定義したことに相当する(\cref{Def:numeral}における\(\numeral{n}\)は上記\(\apply{\Num}{n}\)に相当する).
	ここで,\(\symcal{T}\)は\(\symcal{L}_{\Ar}\)項全体の集合である.
\end{Note}

\begin{Ex} \label[Ex]{Ex:numeral}
	数項\(\numeral{1}, \numeral{2}, \numeral{3}\)の定義はそれぞれ
	\begin{align*}
		\numeral{1} & = \paren{\obj{0} \obj{+} \obj{1}},                                                \\
		\numeral{2} & = \paren{\paren{\obj{0} \obj{+} \obj{1}} \obj{+} \obj{1}},                        \\
		\numeral{3} & = \paren{\paren{\paren{\obj{0} \obj{+} \obj{1}} \obj{+} \obj{1}} \obj{+} \obj{1}}
	\end{align*}
	である.
\end{Ex}

\begin{Note}
	数項は,\(\symcal{L}_{\Ar}\)項の略記表現である.従って「意味的には」同じであろうと思われる項であっても項として(つまり記号列として)等しいとは限らない.
	例えば,自然数1に対する数項\(\numeral{1}\)と\(\symcal{L}_{\Ar}\)の定数記号\(\obj{1}\)は項としては等しくない.
	つまり\(\numeral{1} \neq \obj{1}\)である.
	一方で,
	\begin{equation}
		\Robinson \provable \numeral{1} \objeq \obj{1}
		\label{eq:robinson1numeric}
	\end{equation}
	は成り立っている.「Robinson算術\(\Robinson\)という体系内では」自然数1に対する数項\(\numeral{1}\)と\(\symcal{L}_{\Ar}\)の定数記号\(\obj{1}\)は等しくなる.
\end{Note}

\begin{Que} \label[Que]{Que:robinson1numeric}
	\Cref{eq:robinson1numeric}を示せ.
\end{Que}

Robinson算術\(\Robinson\)は数学的帰納法を公理として含まない.
しかし,メタ側の(つまり通常の意味での自然数上での)数学的帰納法はそもそも\(\Robinson\)内での主張でも\(\PA\)内の主張でもない.
このことに注意して,以下のことを証明してみよう.

\begin{Lemma} \label{Lemma:Robinsonnumerallemma}
	\(n\)を自然数とする.\(\Robinson\)において以下が成り立つ.
	\begin{align}
		\Robinson & \provable \forall \obj{x} \lnot \paren{\obj{x} \obj{<} \obj{0}},
		\label{eq:Robinsonnumerallemmaminimum}                                                                                                                           \\
		\Robinson & \provable \forall \obj{x} \forall \obj{y} \paren{\obj{x} \obj{+} \obj{y} \objeq \obj{0} \to \obj{x} \objeq \obj{0} \land \obj{y} \objeq \obj{0}},
		\label{eq:Robinsonnumerallemmaadd0}                                                                                                                              \\
		\Robinson & \provable \forall \obj{x} \forall \obj{y} \paren{\obj{x} \obj{\cdot} \obj{y} \objeq \obj{0} \to \obj{x} \objeq \obj{0} \lor \obj{y} \objeq \obj{0}},
		\label{eq:Robinsonnumerallemmacdot0}                                                                                                                             \\
		\Robinson & \provable \forall \obj{x} \paren{\obj{x} \objeq \obj{0} \lor \obj{0} \obj{<} \obj{x}},
		\label{eq:Robinsonnumerallemmaorder}                                                                                                                             \\
		\Robinson & \provable \forall \obj{x} \paren{\obj{x} \obj{+} \obj{1} \obj{<} \numeral{n + 1} \to \obj{x} \obj{<} \numeral{n}},
		\label{eq:Robinsonnumerallemmaordern}                                                                                                                            \\
		\Robinson & \provable \forall \obj{x} \paren{\paren{\obj{x} \obj{+} \obj{1}} \obj{+} \numeral{n} \objeq \obj{x} \obj{+} \numeral{n + 1}},
		\label{eq:Robinsonnumerallemmalinerorderbase}                                                                                                                    \\
		\Robinson & \provable \obj{0} \obj{+} \numeral{n} \objeq \numeral{n},
		\label{eq:Robinsonnumerallemmaadd0commutativelaw}                                                                                                                \\
		\Robinson & \provable \forall \obj{x} \paren{\numeral{n} \obj{<} \obj{x} \to \obj{x} \objeq \numeral{n + 1} \lor \numeral{n + 1} \obj{<} \obj{x}}.
		\label{eq:Robinsonnumerallemmalinerorderbase2}
	\end{align}
\end{Lemma}

\begin{proof}
	\(\symcal{M}\)を\(\Robinson\)の任意のモデルとする.\(\symcal{M}\)が右辺の論理式を充足することを示せば十分である.
	\(\symcal{M}\)の対象領域を\(M\)とおく.

	\Cref{eq:Robinsonnumerallemmaminimum}を示そう.\(x \in M\)が\(x < 0\)を満たすと仮定する.
	このとき,\cref{item:PAA7}から\(y + \paren{x + 1} = 0\)となる\(y \in M\)が存在する.
	よって\cref{item:PAA4}から\(\paren{y + x} + 1 = 0\)となるが,これは\cref{item:PAA1}と矛盾する.

	\Cref{eq:Robinsonnumerallemmaadd0}を示そう.\(x, y \in M\)が\(x + y = 0\)を満たすとする.
	\(y \neq 0\)であるとすると,\cref{item:PAA9}から\(y = z + 1\)となる\(z \in M\)が存在する.
	このとき,\cref{item:PAA4}から
	\(x + y = x + \paren{z + 1} = \paren{x + z} + 1 = 0\)
	となり\cref{item:PAA1}に矛盾する.
	よって\(y = 0\)でなければならない.さらに,\(x \neq 0\)であると仮定すると\cref{item:PAA9}から\(x = w + 1\)となる\(w \in M\)が存在する.
	\Cref{item:PAA3}, \cref{item:PAA4}から
	\(x + y = x + 0 = x = w + 1 = 0\)
	だから\cref{item:PAA1}に矛盾する.
	従って\(x = 0\)である.

	\Cref{eq:Robinsonnumerallemmacdot0}を示す.\(x, y \in M\)が\(x \cdot y = 0\)を満たすとする.
	\(x \neq 0\)かつ\(y \neq 0\)であるとすると,\(x = w + 1\)かつ\(y = z + 1\)となる\(w, z \in M\)が存在する.
	\Cref{item:PAA4}, \cref{item:PAA6}から
	\begin{align*}
		x \cdot y & = \paren{w + 1} \cdot \paren{z + 1}             \\
		          & = \paren{\paren{w + 1} \cdot z} + \paren{w + 1} \\
		          & = \paren{\paren{\paren{w + 1} \cdot z} + w} + 1 \\
		          & = 0
	\end{align*}
	となり\cref{item:PAA1}に矛盾する.

	\Cref{eq:Robinsonnumerallemmaorder}は\cref{item:PAA9}と\cref{item:PAA7}から明らかである.

	\Cref{eq:Robinsonnumerallemmaordern}を示す.
	任意の\(x \in M\)に対し,\(x + 1 < \interpretation{\symcal{M}}{\paren{\numeral{n + 1}}}\)であるとする.
	このとき,\(y + \paren{\paren{x + 1} + 1} = \interpretation{\symcal{M}}{\paren{\numeral{n + 1}}}\)となる\(y \in M\)が存在する.
	\Cref{item:PAA4}と\cref{eq:numeralnext}から
	\(\paren{y + \paren{x + 1}} + 1 = \interpretation{\symcal{M}}{\numeral{n}} + 1\)である.
	従って\cref{item:PAA2}から\(y + \paren{x + 1} = \interpretation{\symcal{M}}{\numeral{n}}\)なので\(x < \interpretation{\symcal{M}}{\numeral{n}}\)である.

	\Cref{eq:Robinsonnumerallemmalinerorderbase}を示そう.
	任意の自然数\(n\)に対して
	\(\paren{x + 1} + \interpretation{\symcal{M}}{\numeral{n}} = x + \interpretation{\symcal{M}}{\paren{\numeral{n + 1}}}\)が
	成り立つことを\(n\)に関する数学的帰納法によって示す.
	\(n = 0\)のとき,\cref{item:PAA3}, \cref{item:PAA4}と\cref{eq:numeral0}から
	\begin{align*}
		\paren{x + 1} + \interpretation{\symcal{M}}{\numeral{0}}
		 & = \paren{x + 1} + 0                            \\
		 & = x + 1,                                       \\
		x + \interpretation{\symcal{M}}{\paren{\numeral{0 + 1}}}
		 & = x + \interpretation{\symcal{M}}{\numeral{1}} \\
		 & = x + \paren{0 + 1}                            \\
		 & = \paren{x + 0} + 1                            \\
		 & = x + 1
	\end{align*}
	だから\(\paren{x + 1} + \interpretation{\symcal{M}}{{\numeral{0}}} = x + \interpretation{\symcal{M}}{\paren{\numeral{0 + 1}}}\)である.
	各\(n\)について,
	\(\paren{x + 1} + \interpretation{\symcal{M}}{\numeral{n}} = x + \interpretation{\symcal{M}}{\paren{\numeral{n + 1}}}\)が成り立つと仮定する.
	このとき
	\begin{align*}
		\paren{x + 1} + \interpretation{\symcal{M}}{\paren{\numeral{n + 1}}}
		 & = \paren{x + 1} + \paren{\interpretation{\symcal{M}}{\numeral{n}} + 1}  \\
		 & = \paren{\paren{x + 1} + \interpretation{\symcal{M}}{\numeral{n}}} + 1  \\
		 & = \paren{x + \interpretation{\symcal{M}}{\paren{\numeral{n + 1}}}} + 1  \\
		 & = x + \paren{\interpretation{\symcal{M}}{\paren{\numeral{n + 1}}} + 1}  \\
		 & = x + \interpretation{\symcal{M}}{\paren{\numeral{\paren{n + 1} + 1}}}.
	\end{align*}
	従って,数学的帰納法によって任意の自然数\(n\)に対して
	\(\paren{x + 1} + \interpretation{\symcal{M}}{\numeral{n}} = x + \interpretation{\symcal{M}}{\paren{\numeral{n + 1}}}\)が成り立つ.
	\begin{align*}
		\interpretation{\symcal{M}}{\paren{\paren{x + \obj{1}} + \numeral{n}}} & = \paren{x + 1} + \interpretation{\symcal{M}}{\numeral{n}}, \\
		\interpretation{\symcal{M}}{\paren{x + \numeral{n + 1}}}               & = x + \interpretation{\symcal{M}}{\paren{\numeral{n + 1}}}
	\end{align*}
	だから\cref{eq:Robinsonnumerallemmalinerorderbase}を得る.

	\Cref{eq:Robinsonnumerallemmaadd0commutativelaw}について,
	任意の自然数\(n\)に対して\(0 + \interpretation{\symcal{M}}{\numeral{n}} = \interpretation{\symcal{M}}{\numeral{n}}\)が成り立つことを
	\(n\)に関する数学的帰納法によって示す.
	\(\numeral{0}\)は定数記号\(\obj{0}\)のことなので,\(n = 0\)の場合は\cref{item:PAA3}から従う.
	各\(n\)について,\(0 + \interpretation{\symcal{M}}{\numeral{n}} = \interpretation{\symcal{M}}{\numeral{n}}\)が成り立つと仮定すると
	\cref{item:PAA4}から
	\begin{align*}
		0 + \interpretation{\symcal{M}}{\paren{\numeral{n + 1}}}
		 & = 0 + \interpretation{\symcal{M}}{\paren{\numeral{n} + 1}} \\
		 & = \paren{0 + \interpretation{\symcal{M}}{\numeral{n}}} + 1 \\
		 & = \interpretation{\symcal{M}}{\numeral{n}} + 1             \\
		 & = \interpretation{\symcal{M}}{\paren{\numeral{n + 1}}}
	\end{align*}
	だから\(0 + \interpretation{\symcal{M}}{\paren{\numeral{n + 1}}} = \interpretation{\symcal{M}}{\paren{\numeral{n + 1}}}\)が成り立つ.
	よって数学的帰納法により\cref{eq:Robinsonnumerallemmaadd0commutativelaw}を得る.

	最後に\cref{eq:Robinsonnumerallemmalinerorderbase2}を示そう.
	任意の\(x \in M\)に対し,\(\interpretation{\symcal{M}}{\numeral{n}} < x\)かつ\(x \neq \interpretation{\symcal{M}}{\paren{\numeral{n + 1}}}\)ならば
	\(\interpretation{\symcal{M}}{\paren{\numeral{n + 1}}} < x\)であることを示せばよい.
	\(\interpretation{\symcal{M}}{\numeral{n}} < x\)だから\cref{item:PAA9}により
	\(y + \paren{\interpretation{\symcal{M}}{\numeral{n}} + 1} = x\)となる\(y \in M\)が存在する.
	\(y = 0\)であると仮定すると,\cref{eq:Robinsonnumerallemmaadd0commutativelaw}から
	\(x = \interpretation{\symcal{M}}{\numeral{n}} + 1 = \interpretation{\symcal{M}}{\paren{\numeral{n + 1}}}\)となって矛盾する.
	従って\(y \neq 0\)でなければならない.
	よって\cref{item:PAA9}から\(y = z + 1\)となる\(z \in M\)がとれて,
	\cref{eq:Robinsonnumerallemmaadd0commutativelaw}から\(z + \paren{\interpretation{\symcal{M}}{\paren{\numeral{n + 1}}} + 1} = x\)となる.
	ゆえに\cref{eq:Robinsonnumerallemmalinerorderbase2}は成り立つ.
\end{proof}

\begin{Note}
	\Cref{Lemma:Robinsonnumerallemma}の証明における\(M\)の元としての\(0, 1\)や\(M\)上の二項演算および二項関係\(+, <\)は,
	通常の自然数のものと同じとは限らない.
	あくまで表記を簡素化するために通常の自然数と同じ表記にしているだけであることに注意しよう.
	表記が同じだけなので,通常の自然数において成り立っていた種々の性質がそのまま成り立つとは限らない.
	\(M\)の元としての\(0,1\)が通常の自然数としての\(0,1\)であるとは限らないし,
	\(M\)上の二項演算および二項関係\(+, <\)が通常の自然数における加法,大小関係であるとも限らない.
	また,自然数\(n\)に対する数項\(\numeral{n}\)の\(\symcal{M}\)による解釈\(\interpretation{\symcal{M}}{\numeral{n}}\)が\(n\)であるとも限らない.
	使えるのは,\(\symcal{M}\)は\(\Robinson\)のモデルであるという仮定のみである.
\end{Note}

\begin{Thm} \label[Thm]{Thm:Robinsonnumeraltheorem}
	\(n, m\)を自然数とする.このとき,\(\Robinson\)において以下が成り立つ.
	\begin{align}
		\Robinson & \provable \numeral{n} \obj{+} \numeral{m} \objeq \numeral{n + m},
		\label{eq:Robinsonnumeraltheoremaddition}                                                                                                                                                     \\
		\Robinson & \provable \numeral{n} \obj{\cdot} \numeral{m} \objeq \numeral{n \cdot m},
		\label{eq:Robinsonnumeraltheoremmultiplication}                                                                                                                                               \\
		\Robinson & \provable \lnot \paren{\numeral{n} \objeq \numeral{m}} \quad (n \neq m),
		\label{eq:Robinsonnumeraltheoremnumeralterm}                                                                                                                                                  \\
		\Robinson & \provable \forall \obj{x} \paren{\obj{x} < \numeral{n + 1} \formulaequiv \obj{x} \objeq \numeral{0} \lor \obj{x} \objeq \numeral{1} \lor \dotsb \lor \obj{x} \objeq \numeral{n}},
		\label{eq:Robinsonnumeraltheoremorder}                                                                                                                                                        \\
		\Robinson & \provable \forall \obj{x} \paren{\obj{x} \obj{<} \numeral{n} \lor \numeral{n} < \obj{x} \lor \obj{x} \objeq \numeral{n}}.
		\label{eq:Robinsonnumeraltheoremlinerorder}
	\end{align}
\end{Thm}

\begin{proof}
	\(\symcal{M}\)を\(\Robinson\)の任意のモデルとする.\(\symcal{M}\)が右辺の論理式を充足することを示せば十分である.
	\(\symcal{M}\)の対象領域を\(M\)とする.

	\cref{eq:Robinsonnumeraltheoremaddition}を示す.
	自然数\(n\)を任意に1つとったとき,
	任意の自然数\(m\)に対して
	\(
	\interpretation{\symcal{M}}{\paren{\numeral{n} + \numeral{m}}} = \interpretation{\symcal{M}}{\paren{\numeral{n + m}}}
	\)
	が成り立つことを\(m\)に関する数学的帰納法によって示す.
	\(m = 0\)のとき,\(\numeral{0}\)は\(\obj{0}\)のことなので,\cref{item:PAA3}から
	\(
	\interpretation{\symcal{M}}{\paren{\numeral{n} + \numeral{0}}} = \interpretation{\symcal{M}}{\numeral{n}}
	\)
	が得られる.
	各\(m\)に対して,
	\(
	\interpretation{\symcal{M}}{\paren{\numeral{n} + \numeral{m}}} = \interpretation{\symcal{M}}{\paren{\numeral{n + m}}}
	\)
	が成り立つと仮定する.
	\cref{item:PAA4}から
	\begin{align*}
		\interpretation{\symcal{M}}{\paren{\numeral{n} \obj{+} \numeral{m + 1}}}
		 & = \interpretation{\symcal{M}}{\numeral{n}} + \interpretation{\symcal{M}}{\paren{\numeral{m + 1}}}       \\
		 & = \interpretation{\symcal{M}}{\numeral{n}} + \interpretation{\symcal{M}}{\paren{\numeral{m} + \obj{1}}} \\
		 & = \interpretation{\symcal{M}}{\numeral{n}} + \paren{\interpretation{\symcal{M}}{\numeral{m}} + 1}       \\
		 & = \paren{\interpretation{\symcal{M}}{\numeral{n}} + \interpretation{\symcal{M}}{\numeral{m}}} + 1       \\
		 & = \interpretation{\symcal{M}}{\paren{\numeral{n + m}}}  + 1                                             \\
		 & = \interpretation{\symcal{M}}{\paren{\numeral{n + \paren{m + 1}}}}
	\end{align*}
	となる.
	以上より,数学的帰納法によって任意の自然数\(n, m\)に対して
	\(
	\interpretation{\symcal{M}}{\paren{\numeral{n} + \numeral{m}}} = \interpretation{\symcal{M}}{\paren{\numeral{n + m}}}
	\)
	となることがわかるので,\cref{eq:Robinsonnumeraltheoremaddition}が得られる.
	\Cref{eq:Robinsonnumeraltheoremmultiplication}も同様である.

	\Cref{eq:Robinsonnumeraltheoremnumeralterm}を示そう.
	自然数\(n\)を任意に1つとったとき,
	\(n \neq m\)を満たす任意の自然数\(m\)に対して
	\(
	\interpretation{\symcal{M}}{\numeral{n}} \neq \interpretation{\symcal{M}}{\numeral{m}}
	\)
	が成り立つことを\(m\)に関する数学的帰納法によって示す.
	\(m = 0\)のとき,\(n \neq 0\)なので\(n = n_0 + 1\)となる自然数\(n_0\)がとれる.
	すると\cref{item:PAA1}から
	\(
	\interpretation{\symcal{M}}{\numeral{n}} = \interpretation{\symcal{M}}{\paren{\numeral{n_0} \obj{+} \obj{1}}} \neq \interpretation{\symcal{M}}{\numeral{0}}
	\)
	が成り立つので
	\(
	\interpretation{\symcal{M}}{\numeral{n}} \neq \interpretation{\symcal{M}}{\numeral{0}}
	\)
	は成り立つ.
	各\(m\)について,\(n \neq m\)ならば
	\(
	\interpretation{\symcal{M}}{\numeral{n}} \neq \interpretation{\symcal{M}}{\numeral{m}}
	\)
	が成り立つと仮定する.
	\(n \neq m + 1\)とすると,\(n = 0\)のときには\cref{item:PAA1}から
	\(
	\interpretation{\symcal{M}}{\paren{\numeral{m} \obj{+} \obj{1}}} \neq \interpretation{\symcal{M}}{\numeral{0}}
	\)
	が得られる.よって
	\(
	\interpretation{\symcal{M}}{\paren{\numeral{0}}} \neq \interpretation{\symcal{M}}{\paren{\numeral{m + 1}}}\)となる.
	\(n \neq 0\)のとき,\(n = n_0 + 1\)となる自然数\(n_0\)がとれる.\(n \neq m + 1\)より\(n_0 \neq m\)だから,
	帰納法の仮定により
	\(
	\interpretation{\symcal{M}}{\numeral{n_0}} = \interpretation{\symcal{M}}{\numeral{m}}
	\)
	となる.
	従って\cref{item:PAA2}(の対偶に相当するもの)によって
	\(
	\interpretation{\symcal{M}}{\numeral{n}} \neq \interpretation{\symcal{M}}{\paren{\numeral{m + 1}}}
	\)
	が成り立つ.
	以上より,数学的帰納法によって\(n \neq m\)を満たす任意の自然数\(n, m\)に対して
	\(
	\interpretation{\symcal{M}}{\numeral{n}} \neq \interpretation{\symcal{M}}{\numeral{m}}
	\)
	となることがわかるので,\cref{eq:Robinsonnumeraltheoremnumeralterm}が得られる.

	\Cref{eq:Robinsonnumeraltheoremorder}を示そう.
	以下の主張が任意の自然数\(n\)に対して成り立つことを
	\(n\)に関する数学的帰納法によって示す:
	すべての\(x \in M\)に対し,\(x < \interpretation{\symcal{M}}{\paren{\numeral{n + 1}}}\)が成り立つことと
	\begin{align*}
		x & = \interpretation{\symcal{M}}{\numeral{0}}, \\
		x & = \interpretation{\symcal{M}}{\numeral{1}}, \\
		  & \dots                                       \\
		x & = \interpretation{\symcal{M}}{\numeral{n}}
	\end{align*}
	のいずれかが成り立つことが同値である.
	\(n = 0\)のとき,\cref{eq:Robinsonnumerallemmaadd0}などによって
	\begin{align*}
		                & x < \interpretation{\symcal{M}}{\numeral{1}}                  \\
		\metaequivalent & \text{\(y + \paren{x + 1} = \numeral{1}\)となる\(y \in M\)が存在する} \\
		\metaequivalent & \text{\(\paren{y + x} + 1 = 0 + 1\)となる\(y \in M\)が存在する}       \\
		\metaequivalent & \text{\(y + x = 0\)となる\(y \in M\)が存在する}                       \\
		\metaequivalent & \text{\(y = 0\)かつ\(x = 0\)となる\(y \in M\)が存在する}                \\
		\metaequivalent & x = 0
	\end{align*}
	だからこの主張は成り立つ.
	この主張が\(n\)について成り立つと仮定する.
	まず,
	\begin{align*}
		x & = \interpretation{\symcal{M}}{\numeral{0}},            \\
		x & = \interpretation{\symcal{M}}{\numeral{1}},            \\
		  & \dots                                                  \\
		x & = \interpretation{\symcal{M}}{\numeral{n}},            \\
		x & = \interpretation{\symcal{M}}{\paren{\numeral{n + 1}}}
	\end{align*}
	のいずれかが成り立つと仮定すると,\(x = \interpretation{\symcal{M}}{m}\)かつ\(m < n + 2\)となる自然数\(m\)が存在する.
	このとき,\cref{eq:Robinsonnumeraltheoremaddition}から
	\(\interpretation{\symcal{M}}{\paren{\numeral{n - m + 2}}} + x = \interpretation{\symcal{M}}{\paren{\numeral{n + 2}}}\)
	だから\cref{item:PAA7}によって
	\(x < \interpretation{\symcal{M}}{\paren{\numeral{\paren{n + 1} + 1}}}\)が成り立つ.
	逆に,\(x < \interpretation{\symcal{M}}{\paren{\numeral{\paren{n + 1} + 1}}}\)が成り立つと仮定する.
	もし\(x = 0\)ならば\(x = \interpretation{\symcal{M}}{\numeral{0}}\)が成り立っている.
	\(x \neq 0\)ならば,\cref{item:PAA9}から\(x = y + 1\)となる\(y \in M\)が存在する.
	このとき,\cref{eq:Robinsonnumerallemmaordern}から\(y < \interpretation{\symcal{M}}{\numeral{n + 1}}\)となる.
	帰納法の仮定から,
	\begin{align*}
		y & = \interpretation{\symcal{M}}{\numeral{0}}, \\
		y & = \interpretation{\symcal{M}}{\numeral{1}}, \\
		  & \dots                                       \\
		y & = \interpretation{\symcal{M}}{\numeral{n}}
	\end{align*}
	のいずれかが成り立つ.\(x = y + 1\)だから,
	\begin{align*}
		x & = \interpretation{\symcal{M}}{\numeral{1}},            \\
		x & = \interpretation{\symcal{M}}{\numeral{2}},            \\
		  & \dots                                                  \\
		x & = \interpretation{\symcal{M}}{\paren{\numeral{n + 1}}}
	\end{align*}
	のいずれかが成り立つ.
	ゆえに数学的帰納法により,すべての自然数\(n\)に対して上記主張が成り立つことがわかり,\cref{eq:Robinsonnumeraltheoremorder}が得られる.

	最後に\cref{eq:Robinsonnumeraltheoremlinerorder}を示そう.
	すべての\(x \in M\)に対し,
	\(x < \interpretation{\symcal{M}}{\numeral{n}}\), \(\interpretation{\symcal{M}}{\numeral{n}} < x\), \(x = \interpretation{\symcal{M}}{\numeral{n}}\)
	のいずれかが成り立つことを\(n\)に関する数学的帰納法によって示す.
	\(n = 0\)のとき,\cref{eq:Robinsonnumerallemmaorder}からこの主張は従う.
	この主張が\(n\)のとき成り立つと仮定する.
	すべての\(x \in M\)について,帰納法の仮定により
	\(x < \interpretation{\symcal{M}}{\numeral{n}}\), \(\interpretation{\symcal{M}}{\numeral{n}} < x\), \(x = \interpretation{\symcal{M}}{\numeral{n}}\)
	のいずれかが成り立つ.
	\(x < \interpretation{\symcal{M}}{\numeral{n}}\)のとき,
	\cref{eq:Robinsonnumeraltheoremaddition}と\cref{eq:Robinsonnumeraltheoremorder}から
	\(x < \interpretation{\symcal{M}}{\paren{\numeral{n + 1}}}\)である.
	\(\interpretation{\symcal{M}}{\numeral{n}} < x\)のとき,
	\cref{eq:Robinsonnumerallemmalinerorderbase2}から\(x = \interpretation{\symcal{M}}{\numeral{n + 1}}\)か
	\(\interpretation{\symcal{M}}{\paren{\numeral{n + 1}}} < x\)が成り立つ.
	\(x = \interpretation{\symcal{M}}{\numeral{n}}\)のとき,\cref{eq:Robinsonnumerallemmaadd0commutativelaw}から
	\(0 + \paren{x + 1} = \interpretation{\symcal{M}}{\paren{\numeral{n + 1}}}\)なので\(x < \interpretation{\symcal{M}}{\paren{\numeral{n + 1}}}\)である.
	ゆえに数学的帰納法から,任意の自然数\(n\)に対して上記主張が成り立つことがわかり,\cref{eq:Robinsonnumeraltheoremlinerorder}が得られる.
\end{proof}


Robinson算術\(\Robinson\)は数学的帰納法を公理として含まない理論である.
以降,公理図式\cref{item:PAA8}を一部だけ含む\(\PA\)の部分体系を考えよう.
その前に,論理式の分類を考える.

\begin{Def} \label{Def:boundedformula}
	\(\symcal{L}\)をアリティ2の関係記号\(<\)をもつ言語とする.
	変数記号\(x\)と\(x\)を含まない\(\symcal{L}\)項\(t\),および\(\symcal{L}\)論理式\(\varphi\)に対し,
	\(\forall x \paren{x < t \to \varphi}\)と\(\exists x \paren{x < t \land \varphi}\)をそれぞれ
	\begin{align}
		\forall x < t \varphi,
		\label{eq:boundedformulaforall} \\
		\exists x < t \varphi
		\label{eq:boundedformulaexists}
	\end{align}
	と略記する.
	\(\symcal{L}\)論理式\(\varphi\)が%
	\index[widx]{ろんりしき@論理式!ゆうかいろんりしき@有界---}%
	\(\symcal{L}\)\term{有界論理式}であるとは,
	\(\varphi\)に含まれる量化子\(\forall, \exists\)がすべて\(\forall x < t\)か\(\exists x < t\)
	の形で現れていることをいう%
	\footnote{%
		この定義も帰納的定義によって比較的容易に精密化できる.
	}%
	.
	言語\(\symcal{L}\)が明らかな場合,もしくは誤解のおそれがない場合には,\(\symcal{L}\)有界論理式のことを
	単に有界論理式と呼ぶことがある.

	量化子\(\forall, \exists\)を含まない論理式,すなわち開論理式はすべて有界論理式でもある.
\end{Def}

\index[sidx]{\(\SigmaFormula_n\):\(\SigmaFormula_n\)論理式全体の集合}
\index[sidx]{\(\PiFormula_n\):\(\PiFormula_n\)論理式全体の集合}
\index[sidx]{\(\DeltaFormula_n\):\(\PiFormula_n\)論理式全体の集合}
\index[widx]{ろんりしき@論理式!\(\SigmaFormula_n\)ろんりしき@\(\SigmaFormula_n\)---}
\index[widx]{ろんりしき@論理式!\(\PiFormula_n\)ろんりしき@\(\PiFormula_n\)---}
\index[widx]{ろんりしき@論理式!\(\DeltaFormula_n\)ろんりしき@\(\DeltaFormula_n\)---}
\begin{Def}
	自然数\(n\)に対して,\(\SigmaFormula_n\)論理式,\(\PiFormula_n\)論理式を以下のように帰納的に定義する:
	\begin{enumerate}
		\item \(\symcal{L}_{\Ar}\)有界論理式はすべて\(\SigmaFormula_0\)論理式であり,かつ\(\PiFormula_0\)論理式でもある.
		\item \(\varphi\)が\(\SigmaFormula_n\)論理式であるとき,正の整数\(k\)と変数記号\(x_1, x_2, \dots, x_k\)に対する論理式
		      \[
			      \forall x_1 \forall x_2 \dotsb \forall x_k \varphi
		      \]
		      と論理的に同値な\(\symcal{L}_{\Ar}\)論理式は\(\PiFormula_{n+1}\)論理式である.
		\item \(\varphi\)が\(\PiFormula_n\)論理式であるとき,正の整数\(k\)と変数記号\(x_1, x_2, \dots, x_k\)に対する論理式
		      \[
			      \exists x_1 \exists x_2 \dotsb \exists x_k \varphi
		      \]
		      と論理的に同値な\(\symcal{L}_{\Ar}\)論理式は\(\SigmaFormula_{n+1}\)論理式である.
	\end{enumerate}
	\(\SigmaFormula_n\)論理式全体の集合,\(\PiFormula_n\)論理式全体の集合をそれぞれ\(\SigmaFormula_n, \PiFormula_n\)と表す.

	\(\symcal{L}_{\Ar}\)論理式\(\varphi\)が\(\SigmaFormula_n\)論理式であり,かつ\(\PiFormula_n\)論理式でもある場合,
	\(\varphi\)は\(\DeltaFormula_n\)論理式であるという.
	\(\DeltaFormula_n\)論理式全体の集合を\(\DeltaFormula_n\)と表す.

	また,\(\symcal{L}_{\Ar}\)開論理式全体の集合を\(\OpenFormulaSet\)と表す.
\end{Def}

\begin{Ex} \label[Ex]{Ex:primenumberformula}
	「\(x\)は素数である」という主張は,以下の\(\SigmaFormula_0\)論理式によって表現できる:
	\[
		\lnot \exists \obj{d} < \obj{x} \exists \obj{e} < \obj{x} \paren{\obj{d} \obj{\cdot} \obj{e} \objeq \obj{x}} \land \lnot \paren{\obj{x} \objeq 0} \land \lnot \paren{\obj{x} \objeq 1}.
	\]
\end{Ex}

\begin{Ex} \label[Ex]{Ex:dividedformula}
	「\(x\)は\(y\)を割り切る」という主張は,以下の\(\SigmaFormula_0\)論理式によって表現できる:
	\[
		\exists \obj{z} < \obj{x} \paren{\obj{x} \obj{\cdot} \obj{z} \objeq \obj{y}}
	\]
\end{Ex}

\begin{Lemma} \label{Lemma:formulahierarchy}
	以下が成り立つ.
	\begin{align}
		\SigmaFormula_0 \subset \SigmaFormula_1 \subset \SigmaFormula_2 \subset \dotsb \subset \SigmaFormula_n \subset \SigmaFormula_{n + 1} \subset \dotsb,
		\label{eq:Sigmaformulahierarchy} \\
		\PiFormula_0 \subset \PiFormula_1 \subset \PiFormula_2 \subset \dotsb \subset \PiFormula_n \subset \PiFormula_{n+1} \subset \dotsb
		\label{eq:Piformulahierarchy}
	\end{align}
\end{Lemma}

\begin{proof}
	\(\SigmaFormula_n\)論理式\(\varphi\)に対し,\(\varphi\)に出現しない変数記号\(x, y\)をとれば,論理式\(\exists x \forall y \varphi\)は\(\varphi\)と
	論理的に同値な\(\SigmaFormula_{n + 1}\)論理式である.よって\(\varphi\)は\(\SigmaFormula_{n + 1}\)論理式でもあり,\cref{eq:Sigmaformulahierarchy}を得る.
	\Cref{eq:Piformulahierarchy}も同様である.
\end{proof}


\index[sidx]{I\(\symup{\Gamma}\):\(\PA\)の部分体系}
\begin{Def} \label{Def:paa8restriction}
	\(\PA\)の公理のうち,公理図式\cref{item:PAA8}における論理式\(\varphi\)を論理式の集合\(\symup{\Gamma}\)の元に制限して得られる\(\symcal{L}_{\Ar}\)理論を
	\(\InductionLimitedTheory{\symup{\Gamma}}\)と表す.
\end{Def}

\begin{Thm} \label[Thm]{Thm:IOpenInduction}
	\(\OpenFormulaInduction\)において,以下が成り立つ:
	\begin{align}
		\OpenFormulaInduction & \provable \forall \obj{x} \forall \obj{y}
		\paren{\obj{x} \obj{+} \obj{y} \objeq \obj{y} \obj{+} \obj{x}},
		\label{eq:iopenadditioncommutativelaw}                                                                                                                                                            \\
		\OpenFormulaInduction & \provable \forall \obj{x} \forall \obj{y} \forall \obj{z} \paren{\obj{x} \obj{+} \paren{\obj{y} \obj{+} \obj{z}} \objeq \paren{\obj{x} \obj{+} \obj{y}} \obj{+} \obj{z}},
		\label{eq:iopenadditionassociativelaw}                                                                                                                                                            \\
		\OpenFormulaInduction & \provable \forall \obj{x} \forall \obj{y}
		\paren{\obj{x} \obj{\cdot} \obj{y} \objeq \obj{y} \obj{\cdot} \obj{x}},
		\label{eq:iopenmultcommutativelaw}                                                                                                                                                                \\
		\OpenFormulaInduction & \provable \forall \obj{x} \forall \obj{y} \forall \obj{z}
		\paren{\obj{x} \obj{\cdot} \paren{\obj{y} \obj{\cdot} \obj{z}} \objeq \paren{\obj{x} \obj{\cdot} \obj{y}} \obj{\cdot} \obj{z}},
		\label{eq:iopenmultassociativelaw}                                                                                                                                                                \\
		\OpenFormulaInduction & \provable \forall \obj{x} \forall \obj{y} \forall \obj{z}
		\paren{\obj{x} \obj{\cdot} \paren{\obj{y} \obj{+} \obj{z}} \objeq \paren{\obj{x} \obj{\cdot} \obj{y}} \obj{+} \paren{\obj{x} \obj{\cdot} \obj{z}}},
		\label{eq:iopendistributivelaw}                                                                                                                                                                   \\
		\OpenFormulaInduction & \provable \forall \obj{x} \forall \obj{y} \forall \obj{z}
		\paren{\obj{x} \obj{+} \obj{y} \objeq \obj{x} \obj{+} \obj{z} \to \obj{y} \objeq \obj{z}},
		\label{eq:iopendifference}                                                                                                                                                                        \\
		\OpenFormulaInduction & \provable \forall \obj{x}
		\lnot \paren{\obj{x} \obj{<} \obj{x}},
		\label{eq:iopennonreflexivityraw}                                                                                                                                                                 \\
		\OpenFormulaInduction & \provable \forall \obj{x} \forall \obj{y} \forall \obj{z}
		\paren{\obj{x} \obj{<} \obj{y} \land \obj{y} \obj{<} \obj{z} \to \obj{x} \obj{<} \obj{z}},
		\label{eq:iopentransitionraw}                                                                                                                                                                     \\
		\OpenFormulaInduction & \provable \forall \obj{x} \forall \obj{y}
		\paren{\obj{x} \obj{<} \obj{y} \lor \obj{x} \objeq \obj{y} \lor \obj{y} \obj{<} \obj{x}},
		\label{eq:iopenlinerorder}                                                                                                                                                                        \\
		\OpenFormulaInduction & \provable \forall \obj{x}
		\paren{\obj{x} \objeq \obj{0} \lor \obj{0} \obj{<} \obj{x}},
		\label{eq:iopen0minimum}                                                                                                                                                                          \\
		\OpenFormulaInduction & \provable \forall \obj{x} \forall \obj{y} \forall \obj{z}
		\paren{\obj{x} \obj{<} \obj{y} \formulaequiv \obj{x} \obj{+} \obj{z} \obj{<} \obj{y} \obj{+} \obj{z}},
		\label{eq:iopencalcorderaddition}                                                                                                                                                                 \\
		\OpenFormulaInduction & \provable \forall \obj{x} \forall \obj{y} \forall \obj{z}
		\paren{\lnot \paren{\obj{z} \objeq \obj{0}} \to \paren{\obj{x} \obj{<} \obj{y} \formulaequiv \obj{x} \obj{\cdot} \obj{z} \obj{<} \obj{y} \obj{\cdot} \obj{z}}},
		\label{eq:iopencalcordermult}                                                                                                                                                                     \\
		\OpenFormulaInduction & \provable \forall \obj{x} \forall \obj{y} \forall \obj{z}
		\paren{\lnot \paren{\obj{z} \objeq \obj{0}} \land \obj{x} \obj{\cdot} \obj{z} \objeq \obj{y} \obj{\cdot} \obj{z} \to \obj{x} \objeq \obj{y}}.
		\label{eq:iopendivision}
	\end{align}
\end{Thm}

\begin{proof}
	\(\symcal{M}\)を\(\OpenFormulaInduction\)の任意のモデルとする.\(\symcal{M}\)が右辺の論理式を充足することを示せば十分である.
	\(\symcal{M}\)の対象領域を\(M\)とする.

	\Cref{eq:iopenadditioncommutativelaw}を示そう.
	まず,論理式\(\apply{\varphi}{\obj{x}}\)を\(\obj{0} \obj{+} \obj{x} \objeq \obj{x}\)と定めると,\(\apply{\varphi}{\obj{x}}\)は開論理式である.
	そこで,\(\obj{x}\)に着目して\cref{item:PAA8}を適用する,すなわち\(x\)に関する帰納法によって任意の\(x \in M\)に対して\(0 + x = x\)となることを示そう.
	\(\apply{\varphi}{\obj{0}}\)は\(\obj{0} \obj{+} \obj{0} \objeq \obj{0}\)のことであり,\cref{item:PAA3}から\(0 + 0 = 0\)は成り立つ.
	各\(x \in M\)について,\(0 + x = x\)であると仮定すると,帰納法の仮定と\cref{item:PAA4}などにより
	\(0 + \paren{x + 1} = \paren{0 + x} + 1 = x + 1\)となる.
	よって\cref{item:PAA8}から任意の\(x \in M\)に対して\(0 + x = 0\)となることがわかる.
	次に,論理式\(\apply{\psi}{\obj{y}}\)を\(\paren{\obj{x} \obj{+} \obj{1}} \obj{+} \obj{y} \objeq \paren{\obj{x} \obj{+} \obj{y}} \obj{+} \obj{1}\)と定めると,
	\(\apply{\psi}{\obj{y}}\)は開論理式である.
	そこで,\(y\)に関する帰納法によって任意の\(x, y \in M\)に対して\(\paren{x + 1} + y = \paren{x + y} + 1\)が成り立つことを示す.
	\(x \in M\)を任意に1つとって固定しておく.
	まず,\cref{item:PAA3}から\(\paren{x + 1} + 0 = \paren{x + 0} + 1\)が従う.
	各\(y \in M\)に対し,\(\paren{x + 1} + y = \paren{x + y} + 1\)であると仮定する.
	このとき,帰納法の仮定と\cref{item:PAA4}から
	\(\paren{x + 1} + \paren{y + 1} = \paren{\paren{x + 1} + y} + 1 = \paren{\paren{x + y} + 1} + 1 = \paren{x + \paren{y + 1}} + 1\)となる.
	よって\cref{item:PAA8}から任意の\(x, y \in M\)に対して\(\paren{x + 1} + y = \paren{x + y} + 1\)が成り立つことがわかる.
	最後に,論理式\(\apply{\chi}{\obj{y}}\)を\(\obj{x} \obj{+} \obj{y} \objeq \obj{y} \obj{+} \obj{x}\)と定めると,
	\(\apply{\chi}{\obj{y}}\)は開論理式である.
	そこで,\(y\)に関する帰納法によって任意の\(x, y \in M\)に対して\(x + y = y + x\)となることを示そう.
	\(x \in M\)を1つとって固定しておく.
	\(x + 0 = 0 + x\)であることは,\cref{item:PAA3}と上で示した\(0 + x = 0\)から従う.
	各\(y \in M\)に対し,\(x + y = y + x\)であると仮定する.
	このとき,\cref{item:PAA4}や上で示した\(\paren{x + 1} + y = \paren{x + y} + 1\)と帰納法の仮定により
	\(x + \paren{y + 1} = \paren{x + y} + 1 = \paren{y + x} + 1 = \paren{y + 1} + 1\)となる.
	ゆえに\cref{item:PAA8}から任意の\(x, y \in M\)に対して\(x + y = y + x\)となることがわかる.
	以上より\cref{eq:iopenadditioncommutativelaw}が得られる.

	\Cref{eq:iopenadditionassociativelaw}については,
	任意の\(x, y, z\)に対して\(x + \paren{y + z} = \paren{x + y} + z\)が成り立つことを\(z\)に関する帰納法によって示せばよい.
	\Cref{eq:iopenmultcommutativelaw},  \cref{eq:iopenmultassociativelaw}, \cref{eq:iopendistributivelaw}, \cref{eq:iopendifference}
	もすべて同様に,\(\obj{z}\)に着目して\cref{item:PAA8}を適用することによって得られる.

	\Cref{eq:iopennonreflexivityraw}は\(x\)についての帰納法で示せる.

	\Cref{eq:iopentransitionraw}を示そう.任意の\(x, y, z \in M\)に対して,
	\(x < y\)かつ\(y < z\)であると仮定する.
	このとき,\cref{item:PAA7}から\(y = u + \paren{x + 1}\)となる\(u \in M\)と\(z = v + \paren{y + 1}\)となる\(v \in M\)が存在する.
	よって,\cref{item:PAA4}や\cref{eq:iopenadditionassociativelaw}
	\(z = \paren{v + \paren{u + 1}} \paren{x + 1}\)が得られる.よって\(x < z\)である.

	\Cref{eq:iopenlinerorder}についても,\(\obj{y}\)に着目して\cref{item:PAA8}を適用することで得られる.
	具体的な証明も\cref{eq:Robinsonnumeraltheoremlinerorder}と同様にできる.
	\Cref{eq:Robinsonnumeraltheoremlinerorder}を導く過程で\cref{Lemma:Robinsonnumerallemma}や\cref{Thm:Robinsonnumeraltheorem}の
	\cref{eq:Robinsonnumeraltheoremlinerorder}以外の式を利用しているが,
	そこで利用した各論理式の自然数\(n\)に対する数項\(\numeral{n}\)を使用している部分を変数記号\(\obj{y}\)に,
	数項\(\numeral{n + 1}\)を使用している部分は\(\obj{y} \obj{+} \obj{1}\)にそれぞれ置き換えて得られる論理式を\(\OpenFormulaInduction\)上で証明すればよい.
	数項がなくなるため通常の自然数に関する数学的帰納法は使えないが,
	代わりに\cref{item:PAA8}を利用すればまったく同様にして証明できる.

	\Cref{eq:iopen0minimum}は\(x\)に関する帰納法によって示すことができる.

	\Cref{eq:iopencalcorderaddition}を示す.\(x, y, z \in M\)を任意にとる.
	\(x < y\)と仮定すると,\(y = w + \paren{x + 1}\)となる\(w \in M\)が存在する.
	このとき,\cref{eq:iopenadditioncommutativelaw}や\cref{eq:iopenadditionassociativelaw}などから
	\(y + z = w + \paren{\paren{x + z} + 1}\)となるから\(x + z < y + z\)である.
	\(x + z < y + z\)ならば\(x < y\)であることについては,\(z\)に関する帰納法によって示せる.

	\Cref{eq:iopencalcordermult}を示す.
	\(z\)に関する帰納法により,\(z \neq 0\)となる任意の\(x, y, z\)に対し,\(x < y\)ならば\(x \cdot z < y \cdot z\)となること示そう.
	\(z = 0\)の場合は仮定と矛盾するのでこの主張は成り立つ.
	\(z \neq 0\)となる各\(z\)について,\(x < y\)ならば\(x \cdot z < y \cdot z\)であると仮定する.
	\(x < y\)と仮定すると,\cref{item:PAA7}により\(y = u + \paren{x + 1}\)となる\(u \in M\)が存在する.また,
	\(x \cdot z < y \cdot z\)だから\cref{item:PAA7}により\(y \cdot z = v + \paren{\paren{x \cdot z} + 1}\)となる\(v \in M\)がとれる.
	このとき,
	\begin{align*}
		y \cdot \paren{z + 1}
		 & = \paren{y \cdot z} + y                                                 \\
		 & = \paren{v + \paren{\paren{x \cdot z} + 1}} + \paren{u + \paren{x + 1}} \\
		 & = x \cdot \paren{z + 1} + \paren{\paren{u + v} + 1}
	\end{align*}
	となるから\(x \cdot z < y \cdot z\)である.
	この逆となる\(z \neq 0\)となる任意の\(x, y, z\)に対し,\(x \cdot z < y \cdot z\)ならば\(x < y\)となることについては,
	\cref{eq:iopenlinerorder}から\(x \cdot z < y \cdot z\)のとき\(x = y\)でも\(y < x\)でもないことを示せばよい.
	\(x = y\)ならば\(x \cdot z = y \cdot z\)となり\cref{eq:iopennonreflexivityraw}に矛盾し,
	\(y < x\)ならば\(y \cdot z < x \cdot z\)となり\cref{eq:iopentransitionraw}から\(x \cdot z < x \cdot z\)
	が得られるのでやはり\cref{eq:iopennonreflexivityraw}に矛盾する.

	\Cref{eq:iopendivision}を示そう.
	対偶をとって任意の\(x, y, z\)に対して\(x \neq y\)ならば\(z = 0\)または\(x \cdot z \neq y \cdot z\)であることを示す.
	\(x \neq y\)かつ\(z \neq 0\)ならば\(x \cdot z \neq y \cdot z\)であることを示せばよい.
	\cref{eq:iopenlinerorder}から\(x < y\)または\(y < x\)である.
	よって\cref{eq:iopencalcordermult}から\(x \cdot z < y \cdot z\)または\(y \cdot z < x \cdot z\)である.
	ゆえに\cref{eq:iopencalcordermult}を示したときと同様にして\(x \cdot z \neq y \cdot z\)が従う.
\end{proof}

\begin{Lemma} \label{Lemma:iopenPAA9redudant}
	\(\OpenFormulaInduction\)において,以下が成り立つ.
	\begin{equation}
		\OpenFormulaInduction \provable \text{\cref{item:PAA9}}.
		\label{eq:iopenPAA9redudant}
	\end{equation}
\end{Lemma}

\begin{proof}
	\(\symcal{M}\)を\(\OpenFormulaInduction\)の任意のモデルとする.\(\symcal{M}\)が右辺の論理式を充足することを示せば十分である.
	\(\symcal{M}\)の対象領域を\(M\)とする.

	任意の\(x \in M\)に対し,\(x \neq 0\)とすると\cref{eq:iopen0minimum}から\(0 < x\)である.
	ゆえに\cref{item:PAA7}から\(x = y + \paren{0 + 1}\)となる\(y \in M\)が存在する.
	このとき,\cref{item:PAA3}や\cref{item:PAA4}から\(x = \paren{y + 0} + 1 = y + 1\)となる.
\end{proof}


\(\Robinson\)よりもさらに弱い体系として,以下の理論が知られている.

\index[sidx]{\(\RArithmetic\):Mostowski--Robinson--Tarskiの体系}
\index[widx]{Mostowski--Robinson--Tarskiのたいけい@Mostowski--Robinson--Tarskiの体系}
\begin{Def} \label{Def:RArithmetic}
	\term{Mostowski--Robinson--Tarskiの体系}\(\RArithmetic\)は,
	以下の無限個の公理からなる\(\symcal{L}_{\Ar}\)理論である.
	ここで,\(m, n\)は任意の自然数である.

	\begin{enumerate}[label=R\arabic*.,ref=R\arabic*]
		\item \label{item:RR1} \(\forall \obj{x} \lnot \paren{\obj{x} \obj{<} \obj{0}}\),
		\item \label{item:RR2} \(\numeral{n} \obj{+} \numeral{m} \objeq \numeral{n + m}\),
		\item \label{item:RR3} \(\numeral{n} \obj{\cdot} \numeral{m} \objeq \numeral{n \cdot m}\),
		\item \label{item:RR4} \(\lnot \paren{\numeral{n} \objeq \numeral{m}} \quad (n \neq m)\),
		\item \label{item:RR5} \(\forall \obj{x} \paren{\obj{x} < \numeral{n + 1} \formulaequiv \obj{x} \objeq \numeral{0} \lor \obj{x} \objeq \numeral{1} \lor \dotsb \lor \obj{x} \objeq \numeral{n}}\),
		\item \label{item:RR6} \(\forall \obj{x} \paren{\obj{x} \obj{<} \numeral{n} \lor \numeral{n} < \obj{x} \lor \obj{x} \objeq \numeral{n}}.\)
	\end{enumerate}
\end{Def}


\begin{Thm} \label{Thm:TheoryHierarchy}
	以下で,理論\(T_1, T_2\)に対する\(T_1 \subset T_2\)は
	「\(T_1\)の定理はすべて\(T_2\)の定理でもある」ことを意味するものとする.
	このとき,以下が成り立つ.
	\begin{equation}
		\RArithmetic \subset \Robinson \subset \OpenFormulaInduction \subset \SigmaFormulaInduction_0 \subset \SigmaFormulaInduction_1 \subset \dotsb \subset \PA.
		\label{eq:TheoryHierarchy}
	\end{equation}
\end{Thm}

\begin{proof}
	\Cref{Note:equivalentTheory}も踏まえれば,定理の包含関係としての\(T_1 \subset T_2\)が成り立つためには,
	\(T_1\)の各公理が\(T_2\)の定理であることを示せば十分である.
	\(\OpenFormulaInduction\)以降の体系の包含関係については,論理式の集合としての包含関係が成り立っていることから明らか.
	定理の包含関係としての\(\RArithmetic \subset \Robinson\)は\cref{Lemma:Robinsonnumerallemma}と\cref{Thm:Robinsonnumeraltheorem}から,
	\(\Robinson \subset \OpenFormulaInduction\)については\cref{Lemma:iopenPAA9redudant}からそれぞれ従う.
\end{proof}
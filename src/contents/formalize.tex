\chapter{数学理論の形式化}
\label[chapter]{chap:formulize}

数理論理学が数学の一分野として成功を収めた要因のひとつとして,
素朴的直観を伴わない形式的な記号列を主役に据えたことが挙げられる.
ともすれば,このことは「数学で用いる論理について研究する分野」
という一般的な認識と矛盾するように見えることだろう.
実際,この「形式的な記号列」に関する議論の結果を根拠に
我々が普段使っている「論理」について何かを主張したい場合,
これらの間の橋渡しを行うのは主張したい当人の責任であり,
数理論理学の諸定理がその橋渡しについて何かを保証してくれることはない.
これは,理論物理学で得られた結果から現実世界について言及したり,
統計モデルの性質をもとに現実で得られたデータ(あるいはその生成元)に
ついて言及したりする営みに非常によく似ている.

もし読者にこのような「類似していると期待されるがそれそのものではない概念の性質をもとに,
目的の対象について議論する」
という営みに親しみがあれば,本章の内容は極めて身近に思えるに違いない.
そして,この営みが科学においていかに強力であるかを知っていれば,
数理論理学の手法がいかに強力であるかも予想できるだろう.
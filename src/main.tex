\documentclass[10pt, a5j, ]{ltjsbook} % lualatex

\usepackage{caption}
\usepackage{subcaption}
\renewcommand{\thesubfigure}{\alph{subfigure}}

\makeatletter
\captionsetup{compatibility=false}
\captionsetup{%
	format = plain,
	labelsep = quad,
	font = small,
	skip = .2\baselineskip,
	width = .85\linewidth,
	subrefformat = parens,
	skip = 5\jsc@mpt
}
\captionsetup[subfigure]{%
	labelformat = parens,
	labelsep = space
}
\makeatother
%%%%%%%%=========package=============%%%%%%%%%%%%%
%
\usepackage[no-math]{luatexja-fontspec}
\usepackage{fontspec}
\usepackage{unicode-math}
\setmainfont[Ligatures=TeX]{Libertinus Serif}
\setsansfont[Ligatures=TeX]{Libertinus Sans}
\setmathfont{Libertinus Math}[math-style=ISO, bold-style=ISO]
\setmathfont{TeX Gyre Pagella Math}[range=bb]
\setmathfont{TeX Gyre Pagella Math}[range={\DashVDash}]
\setmathfont{TeX Gyre Pagella Math}[range={\dashVdash}]
\setmathfont{Garamond Math}[range={\Coloneq}]
\usepackage[HaranoAji]{luatexja-preset}
\ltjsetparameter{jacharrange={-2,-3}}
\usepackage{amsthm}
\usepackage[draft]{graphicx}
\usepackage{booktabs}
\usepackage{enumitem}
\usepackage{multienum}
\usepackage{index}
\usepackage{pxrubrica}
\usepackage{luatexja-ruby}
\usepackage{tikz}
\usepackage{xcolor}
\usepackage{braket}
\usepackage{bussproofs}
\usepackage{multicol}
\usepackage{framed}
\usepackage{numbersets}

%\delimitershortfall=-.3pt

\graphicspath{{./fig/}}

%
%%%%%%%%%%%%%%%%%=============macro=======================%%%%%%%%%%%%%%
%

\NewDocumentCommand{\paren}{sm}{%
	\IfBooleanTF{#1}{%
	\mathopen{} (#2) \mathclose{}%
	}%
	{%
	\mathopen{} \left( #2 \right) \mathclose{}%
	}%
}

\NewDocumentCommand{\apply}{smm}{%
	\IfBooleanTF{#1}{#2\paren*{#3}%
	}%
	{%
	#2\paren{#3}%
	}%
}

\NewDocumentCommand{\sequent}{}{\Rightarrow}

\NewDocumentCommand{\term}{m}{\textbf{#1}}
\RenewDocumentCommand{\emph}{m}{\textbf{#1}}

\NewDocumentCommand{\obj}{m}{\symtt{#1}}

\DeclareMathOperator{\FV}{FV}
\DeclareMathOperator{\BV}{BV}
\DeclareMathOperator{\Var}{Var}
\DeclareMathOperator{\depth}{depth}
\DeclareMathOperator{\size}{size}
\DeclareMathOperator{\Th}{Th}
\DeclareMathOperator{\eDiag}{eDiag}
\DeclareMathOperator{\Diag}{Diag}

\NewDocumentCommand{\subst}{smm}{%
\IfBooleanTF{#1}{#2[{#3}]%
	}%
	{%
	#2\mathopen{} \left[ #3 \right] \mathclose{}%
	}%
}

\NewDocumentCommand{\LK}{}{LK}

\NewDocumentEnvironment{naiveproof}{}{\oframed}{\endoframed}

\NewDocumentCommand{\pair}{m}{\langle #1 \rangle}

\NewDocumentCommand{\provable}{}{\vdash}
\NewDocumentCommand{\interderivable}{}{\dashVdash}

\NewDocumentCommand{\KleeneClosure}{m}{{#1}^{\ast}}

\NewDocumentCommand{\vertbracket}{sm}{%
	\IfBooleanTF{#1}{%
		\mathopen{} \lvert #2\rvert \mathclose{}%
	}%
	{%
		\mathopen{} \left\lvert #2 \right\rvert \mathclose{}%
	}%
}

\NewDocumentCommand{\length}{sm}{%
	\IfBooleanTF{#1}{%
		\vertbracket*{#2}
	}%
	{%
		\vertbracket{#2}%
	}%
}

\NewDocumentCommand{\squarebrackets}{sm}{%
	\IfBooleanTF{#1}{%
		\mathopen{} [#2] \mathclose{}%
	}%
	{%
		\mathopen{} \left[ #2 \right] \mathclose{}%
	}%
}

\NewDocumentCommand{\equivclass}{sm}{%
	\IfBooleanTF{#1}{%
		\squarebrackets*{#2}
	}%
	{%
		\squarebrackets{#2}
	}%
}

\NewDocumentCommand{\polynomial}{smm}{%
	\IfBooleanTF{#1}{%
		#2 \squarebrackets*{#3}
	}%
	{%
		#2 \squarebrackets{#3}
	}%
}

\NewDocumentCommand{\anglebrackets}{sm}{%
	\IfBooleanTF{#1}{%
		\mathopen{} \langle #2\rangle \mathclose{}%
	}%
	{%
		\mathopen{} \left\langle #2\right\rangle \mathclose{}%
	}%
}

\NewDocumentCommand{\emptystring}{}{\varepsilon}

\NewDocumentCommand{\restriction}{}{\upharpoonright}

\NewDocumentCommand{\objeq}{}{\equiv}

\NewDocumentCommand{\formulaequiv}{}{\leftrightarrow}

\NewDocumentCommand{\uexists}{}{{\exists !}}

\NewDocumentCommand{\powerset}{m}{\apply{\symfrak{P}}{#1}}

\NewDocumentCommand{\interpretation}{mm}{{#2}^{#1}}

\NewDocumentCommand{\languagewithname}{mm}{\apply{#1}{#2}}

\NewDocumentCommand{\metaimplies}{}{\implies}

\NewDocumentCommand{\metaequivalent}{}{\iff}

\NewDocumentCommand{\satisfy}{}{\vDash}

\NewDocumentCommand{\logicallyequivalent}{}{\DashVDash}

\NewDocumentCommand{\isomorphic}{}{\cong}

\NewDocumentCommand{\elementarilyequivalent}{}{\equiv}

\NewDocumentCommand{\GP}{}{\symup{GP}}

\NewDocumentCommand{\Ring}{}{\symup{Ring}}

\NewDocumentCommand{\CRing}{}{\symup{CRing}}

\NewDocumentCommand{\OrderedRing}{}{\symup{OR}}

\NewDocumentCommand{\ProjectiveGeometryPart}{}{\symup{B}}

\NewDocumentCommand{\ZF}{}{\symup{ZF}}

\NewDocumentCommand{\ZFC}{}{\symup{ZFC}}

\NewDocumentCommand{\RArithmetic}{}{\symup{R}}

\NewDocumentCommand{\Robinson}{}{\symup{Q}}

\NewDocumentCommand{\PA}{}{\symup{PA}}

\NewDocumentCommand{\POSET}{}{\symup{POSET}}

\NewDocumentCommand{\TOSET}{}{\symup{TOSET}}
%
%%%%%%%%%%%===================enumerate==================%%%%%%%%%%%
%
\setlist[description, 1]{%
	style = nextline%
}
%
%%%%%%%%%%%%%%=======================theorem==================%%%%%%%%%%%%%
%
%
\theoremstyle{definition}
\newtheorem{Def}{定義}[chapter]
\newtheorem{Thm}[Def]{定理}
\newtheorem{Ex}[Def]{例}
\newtheorem{Que}{演習}[chapter]
\newtheorem{Lemma}[Def]{補題}
\newtheorem{Corollary}[Def]{系}
\newtheorem{Note}[Def]{注意}
%
%
%%%%%%%%%%%%%=====================index==================================%%%%%%%%%%%%%%
%
\newindex{sidx}{sidx}{sind}{記号索引}
\newindex{widx}{widx}{wind}{用語索引}
%
%
%%%%%%%%%===================foot note numbering=================%%%%%%%%%%%%%%%%%%%%
%
\makeatletter
\@addtoreset{footnote}{section}
\makeatother

%%
%%%%%%%%%%%%%%========hyperref============%%%%%%%%%%
%
\usepackage[unicode, pdfusetitle, hidelinks, draft = false]{hyperref} % hyperlink

\hypersetup{% setting hyperref
	bookmarksnumbered=true,
	bookmarksopen=true,
	bookmarkstype=toc,
	pdfborder={0 0 0},
	colorlinks = true,
}
\usepackage{footnotebackref}
%
%
%%%%%%%==============crossreference================%%%%%%%%%%
%
\usepackage[nameinlink]{cleveref}
\crefname{figure}{図}{図}
\Crefname{figure}{図}{図}
\crefname{equation}{式}{式}
\Crefname{equation}{式}{式}
\crefformat{chapter}{#2第#1章#3}
\Crefformat{chapter}{#2第#1章#3}
\crefrangeformat{chapter}{#3第#1#4章から#5第#2#6章}
\Crefrangeformat{chapter}{#3第#1#4章から#5第#2#6章}
\crefname{page}{p.}{p.}
\Crefname{page}{p.}{p.}
\crefname{Ex}{例}{例}
\Crefname{Ex}{例}{例}
\Crefname{Def}{定義}{定義}
\crefname{Thm}{定理}{定理}
\Crefname{Thm}{定理}{定理}
\crefname{table}{表}{表}
\Crefname{table}{表}{表}
\Crefname{Que}{演習}{演習}
\crefname{Que}{演習}{演習}
\Crefname{Lemma}{補題}{補題}
\crefname{Lemma}{補題}{補題}
\Crefname{Corollary}{系}{系}
\crefname{Corollary}{系}{系}
\Crefname{Note}{注意}{注意}
\crefname{Note}{注意}{注意}
\crefname{section}{\presectionname}{\presectionname}
\Crefname{section}{\presectionname}{\presectionname}
\crefname{subsection}{\presubsectionname}{\presubsectionname}
\Crefname{subsection}{\presubsectionname}{\presubsectionname}
\crefrangeformat{section}{#3\presectionname#1#4から#5\presectionname#2#6}
\Crefrangeformat{section}{#3\presectionname#1#4から#5\presectionname#2#6}
\crefrangeformat{subsection}{#3\presubsectionname#1#4から#5\presubsectionname#2#6}
\Crefrangeformat{subsection}{#3\presubsectionname#1#4から#5\presubsectionname#2#6}

%
%
%%%%%%%======biblatex=============%%%%%%%%%%
%
\usepackage[%
	backend = biber,%
	url = false,%
	doi = false,%
	eprint = false,%
	isbn = false,%
	sorting = none,%
	style = numeric-comp%
]{biblatex} % use "biblatex"
\DeclareFieldFormat*{title}{\mkbibquote{#1}}
\DeclareDelimFormat{labelnamepunct}{\addcomma\addspace}
\DeclareFieldFormat{url}{\url{#1}}
\addbibresource{reference/book.bib}
%\addbibresource{reference/online.bib}

\makeatletter
\renewcommand{\section}{%
	\if@slide\clearpage\fi
	\@startsection{section}{1}{\z@}%
	{\Cvs \@plus.5\Cdp \@minus.2\Cdp}% 前アキ
	{.5\Cvs \@plus.3\Cdp}% 後アキ
	%   {\normalfont\Large\headfont\@secapp}}
	{\normalfont\Large\headfont\presectionname\raggedright}%
}
\renewcommand{\subsection}{%
	\@startsection{subsection}{2}{\z@}%
	{\Cvs \@plus.5\Cdp \@minus.2\Cdp}% 前アキ
	{.5\Cvs \@plus.3\Cdp}% 後アキ
	{\normalfont\large\headfont\presubsectionname}%
}
\makeatother
\renewcommand{\presectionname}{\S}
\newcommand{\presubsectionname}{\S\S}
\setcounter{tocdepth}{2}
%
%


%%%%%%%%%%%%%%%===========index============%%%%%%%%%%%

\usepackage{needspace}

\makeatletter
\NewDocumentCommand\idxhead{m}{%
	\needspace{2\baselineskip}%
	\vspace{\baselineskip}%
	\DeclareDocumentCommand\hrulefill{}{\leavevmode\leaders\hrule height 0.8pt\hfill\kern\z@}%
	\hbox to \columnwidth{\hfil%
	\normalsize%
	\textcolor{gray}{\raisebox{.15ex}{■}}\hspace{.5\zw}\textsf{#1}\hspace{.5\zw}\textcolor{gray}{\raisebox{.15ex}{■}}\hfil}\vspace{-4.5mm}\hrulefill\par%
	\nopagebreak%
}
\makeatother
%

\begin{document}

\begin{titlepage}
	\title{0から始める数理論理学入門 第2版}
	\author{野口匠}
	\date{2024年11月14日}
	\maketitle
\end{titlepage}

\frontmatter

\chapter{はじめに}

数理論理学は,ゲーデルの不完全性定理の存在もあってか世間一般の認知度は比較的高い.
しかしながら,数理論理学の基本コンセプトである「論理を形式化してその性質を探っていく」という考え方は広く浸透しているとはいいがたい.
特にゲーデルの不完全性定理は,なんとなく日常用語として理解できそうな言い回しや字面のインパクトの強さからか
「ゲーデルは数学が万能とはなりえないことを証明した」だの「完璧な数学理論は存在しない」だのと意味不明な誤解が発生しがちである.
とはいえ,最近では「これは誤解である」という認識自体は広まっており,状況は改善しつつあるといえる.

一方で,いわゆる「通常の数学理論」と比較すると数理論理学はあまり広く学ばれてはいないのが現状である.
現在ではわかりやすい和書も多数出版されており,学ぶハードルはそれなりに低いけども,学んでいる人の数はほかの数学理論と比較すれば少ないと言わざるを得ないのではないだろうか.
実際,数学をそれなりに学んでいる人であっても数理論理学については何も知らないという人が少なくない.数学を専門とせず,道具として使っている物理学や統計学を学んでいる人であればなおさらである.

しかし,特に数学を学んでいる人にとっては「論理記号」については身近だと感じる人は多い.彼らにとって論理記号は普段使っている数学記号と同じく「なんらかの数学的主張を表現したもの」であり,
その認識のままで大きな問題が生じることはない.
むろん数理論理学を学んだ人からすればその認識は厳密には誤りである.
誤っている部分が数理論理学にとっても些事であればよいのだが,残念なことに論理を形式化するという数理論理学の基本的な方法について理解できていない致命的な誤りである.

本書はそのような誤解を払拭することを第一の目標とした.
すなわち「論理を形式化する」ことがいったいどういうことなのかを実感をもって学ぶことが目標である.
そのため,入門書で多く取り上げられているであろう数理論理学に関する結果の多くは取り上げない.
特に不完全性定理については取り上げないので,それをめあてにして本書を手に取るとがっかりするであろう.
その代わり,通常の数学理論との接点を多く紹介することで「論理の形式化」についての理解を深めたい.

とりわけ「\(A \land B\)は\(A\)かつ\(B\)という意味」\emph{ではない}というジャーゴンの意味が理解できれば目標達成である.
このことが理解できれば,数理論理学の世界に飛び込む準備は完全に整ったといってよい.要するに,本書は数理論理学入門の前段階という位置づけで活用するとよい.
本書の後,あるいは並行して読むことになるであろう1冊目の入門書としては前原\cite{maehara2005}や鹿島\cite{kashima2009}がおすすめである.
数理論理学入門としての色が強いのは後者であるが,やや難しいと感じた場合には前者を読んでみるのもよいだろう.
また,数理論理学が数学の一分野である以上,集合や写像といった「道具」の修得は欠かせない.本書でも説明抜きに用いている.
自信がない人は嘉田\cite{kada2008}を手に取るとよい.
戸次\cite{bekki2012}は複数の形式的体系に触れることができるという点において読む価値が高い.
また,ユークリッド幾何学の基礎づけについて気になる人は足立\cite{adachi2019}を,数の体系について知りたい人は田中\cite{tanaka2019}を読むとよい.

本書の原稿やソースファイルは,以下のGitHubリポジトリにて閲覧可能である.

\begin{center}
	\url{https://github.com/enunun/introductiontomathmaticallogic}
\end{center}

執筆時間の関係で書ききれなかった内容については,ここに随時追加予定である.

\begin{flushright}
	2024年8月7日
\end{flushright}

\tableofcontents

\mainmatter

\chapter{論理の形式化}
\label[chapter]{chap:formulize}

数理論理学が数学の一分野として成功を収めた要因のひとつとして,
素朴的直観を伴わない形式的な記号列を主役に据えたことが挙げられる.
ともすれば,このことは「数学で用いる論理について研究する分野」
という一般的な認識と矛盾するように見えることだろう.
実際,この「形式的な記号列」に関する議論の結果を根拠に
我々が普段使っている「論理」について何かを主張したい場合,
これらの間の橋渡しを行うのは主張したい当人の責任であり,
数理論理学の諸定理がその橋渡しについて何かを保証してくれることはない.
これは,理論物理学で得られた結果から現実世界について言及したり,
統計モデルの性質をもとに現実で得られたデータ(あるいはその生成元)に
ついて言及したりする営みに非常によく似ている.

もし読者にこのような「それそのものではないが何らかの意味で
関連性をもつと期待される概念の性質をもとに,目的の対象について議論する」
という営みに親しみがあれば,本章の内容は極めて身近に思えるに違いない.
そして,この手法が科学においていかに強力であるかを知っていれば,
数理論理学の手法がいかに強力であるかも予想できるだろう.


\section{数学における論理}
\label[section]{sec:logic}

数学が世界的に広く学ばれていることが示すように,
数学(および算数)を学ぶことは有益であるというのが一般的な認識である.
「なぜ数学を学ぶのか」という問いの答えは個々人によってさまざまだろうが,
少なくとも日本の数学教育学の分野では,
数学教育の目的は
\begin{enumerate}
	\item 陶冶的目的
	\item 実用的目的
	\item 文化的目的
\end{enumerate}
の3つの観点から論じられるのが一般的である%
\footnote{%
	念のため述べておくが,これは観点別評価とはまったく別の話である.
}%
.
このうち,実用的目的と文化的目的については字面から容易に想像できる通りの意味である.
すなわち,実用的目的は教科教育の内容そのものの修得と実社会における活用を志向したものであり,
文化的目的は文化としての学問を継承・発展させていくことを志向したものである.

一方で,1つ目の陶冶的目的についてはやや聞きなれない言葉である.
ここでの「陶冶」という言葉には「人の性質や能力を円満に育てること」という意味である.
すなわち,陶冶的目的というのは人間形成や価値観・(教科教育の内容以外での)能力を養成することを志向したものである.
具体例を挙げていけばキリがないが,「論理的思考力の養成」という目的は数学教育に明るくない人からでも
頻繁に挙がるものである%
\footnote{%
	ちなみに,価値観的な側面では「合理性を重んじる態度の養成」が挙げられる.
}%
.
なぜ数学を学ぶことで論理的思考力を養成できるのだと考えられているかといえば,
「数学」と「論理」の間に切っても切れない深いつながりがあるからに他ならない.
数学では,ある言明が「正しい」と主張したいとき,
なぜそうなるのかということの説明,すなわち「証明」が求められる.

\begin{Ex} \label[Ex]{Ex:simpleLogic}
	「6は偶数である」という言明が正しいと主張したいとする.
	この言明は以下のように「証明」できる:
	\begin{enumerate}
		\item 「偶数」とは「2の倍数,すなわち2の整数倍として表される整数」のことである.
		\item \(6 = 2 \times 3\)と表わされる.
		\item 3は整数である.
		\item 6は2の整数倍として表わされる.
		\item 従って,6は偶数である.
	\end{enumerate}
\end{Ex}

我々は\cref{Ex:simpleLogic}のような議論でもって「6は偶数である」という主張を「正しい」と認識する.
そして\cref{Ex:simpleLogic}のような議論ができないとき,我々は「それはおそらく正しくないのだろう」と認識する%
\footnote{%
	「誰がどうやってもこのような議論はできないのだ」ということを主張したければ,そのこともまた「証明」が求められる.
}%
.
また,数学においてはこの「証明」からあいまいさを極力排除することを求めるのも特徴的である.

\begin{Ex} \label[Ex]{Ex:ambiguousLogic}
	「円と楕円は位相同型である」という言明が正しいを主張したいとする.
	しかし,以下のような議論は通常「証明」とは認められない:
	\begin{enumerate}
		\item 2つの図形が位相同型であるというのは,一方を連続的に変形して他方と一致させることができることをいう.
		\item 円を少しつぶすことで楕円と一致させることができる.
		\item よって,円と楕円は位相同型である.
	\end{enumerate}
\end{Ex}

位相同型というものがどういうものか知らずとも,\cref{Ex:ambiguousLogic}のような議論が「うさんくさい」ことに気づくであろう%
\footnote{%
	このような体験をもとにして「数学では厳格な証明のみが許容されるのだ」などとは思ってはいけない.
	むしろ,\cref{Ex:ambiguousLogic}のような素朴的直観を精密化することによって厳格な証明を与えることも多い.
	許容されないのは,このようなラフな議論によって正しさの検証が完全に完結したかのように考えることである.
}%
.
例えば,
\begin{itemize}
	\item 「連続的に変形」とはいったい何をどうすることなのか
	\item 「少しつぶす」とはいったい何をどうすることなのか
\end{itemize}
あたりであろう.いずれも議論の中で使われている言葉の定義にあいまいさがあることに起因している.

言葉の定義にあいまいさがあること以外に,数学においては不適切であるとみなされる「証明」の例も挙げよう.

\begin{Ex} \label[Ex]{Ex:insufficientLogic}
	「すべての整数\(n\)に対して,整数\(n^2\)を3で割った余りは0か1である」という言明が正しいと主張したいとする.
	しかし,以下のような議論は通常「証明」とは認められない:
	\begin{enumerate}
		\item \(n = 2\)とする.\(n^2 = 4\)を3で割った余りは1である.
		\item \(n = 11\)とする.\(n^2 = 121\)を3で割った余りは1である.
		\item \(n = 30\)とする.\(n^2 = 900\)を3で割った余りは0である.
		\item 以上より,「すべての整数\(n\)に対して,整数\(n^2\)を3で割った余りは0か1である」ことが確かめられた.
	\end{enumerate}
\end{Ex}

\Cref{Ex:insufficientLogic}での議論では,それぞれの場面で言葉の定義があいまいさがある場所はなかった.
この議論が不適切であるとみなされるのは,ひとえに検証が不十分であることが要因である.
整数というのは無限に多く存在するのにもかかわらず,\(2, 11, 30\)の3つでしか検証していない.
残りの整数に対して一切言及していないにもかかわらず,
あたかもすべての整数に対して検証が終わったかのように議論を進めていることが問題である.
これらの実験は,もとの言明の「正しさの根拠」とはなりえない%
\footnote{%
	当然のことであるが,正しさの根拠にならないからといって「これらの実験は無価値である」などと思ってはいけない.
	このような実験は,数学という広大な世界を渡り歩いていくうえで学術的・教育的に極めて高い価値を有する.
	学習者にとって対象が未知であればなおさらである.
}%
.
すべての整数に対してもれなく検証を終えて初めて正しさが検証されたといえる%
\footnote{%
	愚直に行うのは当然不可能なので,検証には別の方法を考える必要がある.
	ポピュラーなのは,特定の整数に限定しない一般的な整数\(n\)を「任意に」とって,
	この\(n\)に対してだけ主張の正しさを検証することである.
}.

以上のように数学と論理との関係について振り返ってみると,例えば次のような疑問が浮かび上がってくる:
\begin{enumerate}
	\item 我々は,数学において「正しい」ことと「証明できる」ことを自然に同一視してしまっているが,それは適切なのだろうか?
	\item 我々は,数学における議論の進め方に適切なものとそうでないものがあることを知っている.その境界になっているものは何か?
	\item 数学における論理について,通常の数学と同じように何か一般的な法則や定理を見いだせないだろうか?
\end{enumerate}
これらの問いに完全な解答を与えるのは極めて困難であろう.
何を主張しても「そういう意見もあるよね」程度の立ち位置に落ち着いてしまいそうである.
「証明」や「正しさ」の意味するところがあいまいであることが解答の難しさに拍車をかけている.

数理論理学では,このあいまいさに対して一定の解決策を見出すことができる.
それは「論理そのものに対して直接議論することはせず,代わりに形式的な記号列について議論すること」である.
これがいったいどういうことなのかを次節以降で学んでいく.

\section{記号論理学} \label[section]{sec:symbolicLogic}

数学では,議論したい対象を表現するために記号を用いることが多い.
そして,記号化は単なる略記というだけでなくそれが意味するところを明確にするという役割も担うことがある.
典型的なのは文字である.

\begin{Ex} \label[Ex]{Ex:numericsymbol}
	「2」や「3」などのような特定の整数に対してではなく一般の整数に対して議論したいとき,
	「整数\(n\)」のように記号を用いて対象の整数を表現することが多い.
	「2」や「3」のような具体的な整数を表現する「数字」ではなく
	特に取り決めのない「文字」を使用することにより,「いまは特定の整数に限らない一般論を展開しているのだ」
	という意図がはっきりする.
	むろん「どんな整数に対しても」のように記号を使わず言葉で述べてもよいが,
	文章を書くのが相当に面倒になることは想像に難くない.
\end{Ex}

\Cref{Ex:numericsymbol}と同じことを数学における論理でも行うことを考える.
すなわち,「正しい」であったり「証明できる」という言葉の代わりに何らかの記号を用いるのである.
しかし,これでは単なる略記にしかなっておらず,あいまいさに対する解決策にはなっていない.
「何らかの主張\(A\)が正しいことを〇〇を表す」とか「\(\Gamma\)から\(B\)が証明できることを~と表す」
などと書いたところであいまいさが何一つ解消されてはいない.
これでは\cref{sec:logic}の最後で述べた問いへの解答とはなりえない.

ここで視点を変えて,「その記号は何らかの意味を有しているとは考えず,ただそこにあるのみである」
と考えてみよう.「2」や「3」という「数字」は我々にとって具体的な「数」を表すための記号であるが,
それはそれとして単に「2」や「3」のような形をした記号であるととらえることもできる.

\begin{Ex} \label[Ex]{Ex:objectsum}
	3つの記号\(2,3,+\)を書き並べた「\(2 + 3\)」という記号列を考える.
	素朴にはこの記号列を「これは2と3の和で5を表す」のように言いたくなるが,
	これを単なる記号列であると考える文脈においてはそのようなことはいえない.
	そもそも「5」という記号すら登場してはいないのである.
	この文脈においては,『記号「5」は記号列「\(2 + 3\)」の略記であると定義する』
	というように明示的に定義する必要がある.
\end{Ex}

\Cref{Ex:objectsum}のような議論を数学における論理についても適用してみよう.

\begin{Ex} \label[Ex]{Ex:objectLogic}
	「\(\Gamma\)から\(A\)を証明できる」と解釈できることを期待して,
	「\(\Gamma \sequent A\)」という記号列を導入する.
	この時点では,「\(\Gamma \sequent A\)」という記号列は単にそういう記号列であるというのみであり,
	「\(\Gamma\)から\(A\)を証明できる」などという意味は有していない.
	ただ単に我々がそう期待しているだけである.
\end{Ex}

このような記号の中で,数学的主張を構成するために使用される記号を特に%
\index[widx]{ろんりきごう@論理記号}%
\term{論理記号}と呼ぶことがある%
\footnote{%
	「論理記号」という言葉は「論理に関連した記号」程度の意味合いで雑に使われがちな言葉であり,
	数学者の間で共通認識があるわけではないようである.
	実際,\(\forall\)と\(\exists\)は%
	\index[widx]{りょうかし@量化子}%
	\term{量化子}と呼ばれることもある.
	「論理記号」の定義がないと困る場合にはその場で定義して使うようにすればよい.
}%
.
本書で扱う論理記号の一覧を\cref{tab:logicalsymbol}に示す.

\index[sidx]{\(\bot\):矛盾}
\index[sidx]{\(\lnot\):否定}
\index[sidx]{\(\land\):連言(かつ)}
\index[sidx]{\(\lor\):選言(または)}
\index[sidx]{\(\to\):含意(ならば)}
\index[sidx]{\(\forall\):全称(すべて)}
\index[sidx]{\(\exists\):存在}
\begin{table}[htbp]
	\centering
	\caption{本書で登場する論理記号の一覧とその記号についての素朴的直観}
	\label{tab:logicalsymbol}
	\begin{tabular}{ccc}
		\toprule
		記号          & 素朴的直観   & 通常の数学における使用例                                   \\
		\midrule
		\(\bot\)    & 矛盾      & \(\bot\):矛盾する                                  \\
		\(\lnot\)   & 否定      & \(\lnot A\):\(A\)でない                           \\
		\(\land\)   & 連言(かつ)  & \(A \land B\):\(A\)かつ\(B\)                     \\
		\(\lor\)    & 選言(または) & \(A \lor B\):\(A\)または\(B\)                     \\
		\(\to\)     & 含意(ならば) & \(A \to B\):\(A\)ならば\(B\)                      \\
		\(\forall\) & 全称(すべて) & \(\forall x \varphi\):すべての\(x\)に対して\(\varphi\) \\
		\(\exists\) & 存在      & \(\exists x \varphi\):\(\varphi\)となる\(x\)が存在する \\
		\bottomrule
	\end{tabular}
\end{table}

\(\forall\)や\(\exists\)のような論理記号は,
現代的にはこのような直感を捨てて純粋な記号列に対する議論であることを明示するために導入される.
従って,『「\(\forall x \varphi\)」は「すべての\(x\)に対して\(\varphi\)」という意味である』のような言明は
厳密には誤りであり,しかもその誤りは本質的なものである%
\footnote{%
	特に数理論理学と関係ない分野では,そのように導入したところで不都合は生じない.
	リアルタイム性が要求されるセミナーや講義の場等では書く文字数が少ない論理記号は便利である.
	一方で,数理論理学がこれだけ市民権を得ていることを考えれば,
	数学を専門とする人くらいは論理記号が単なる略記表現でないことを認識しておくべきであろう.
	その上で,時間の節約という目的であえて濫用するのであれば,そのことについてはまったく問題ないと考えられる.
}%
.



このような立場に立つと,数学で使う論理にかかわるさまざまな概念が
素朴的直観の伴わない形式的な記号列やそれに対する操作として「翻訳」できることに気づく.
そして,そのような「記号列への操作ゲーム」が有する性質を調べるのは数学が得意とするところである.
数理論理学も発展して久しく,現在ではさまざまな流儀や理論が存在するが,
本書では数学で使う論理にかかわるさまざまな概念を形式的な記号列やそれに対する操作として
「翻訳」する作業を体験することを目的とし,次章以降では話題を相当に絞って解説する.
数理論理学の広大な世界については,他の本を参照されたい.



\section{メタとオブジェクト} \label[section]{sec:metaobject}

数理論理学では,形式的な記号列やその操作に関する数学的性質を研究する.
このとき,集合や写像といった数学での道具は通常通り使用する.
これは,形式的記号列の世界の中で構成した「集合論」の性質を研究する場合でも変わらない.
循環論法になっているかのように思えるが,研究対象は集合論そのものではなくそれを模した
形式的記号列への操作ゲームなのだから,循環論法になっているわけではない.
一方で,そのような議論の中では自分が今どちらの立場なのか混乱しがちである.

本書では形式的記号列の世界での「集合論」は取り扱わないが,等号「\(=\)」については取り扱う.
形式的記号列の世界でも等号が登場するので,容易に区別をつけるために我々が普段使っている等号の方を%
\index[sidx]{\(\equiv\):メタ側における等号}%
\begin{align}
	\equiv
	\label{eq:equiv}
\end{align}
のように書き表しておくこととする.2つの記号列\(s, t\)に対する「\(s \equiv t\)」は
「\(s\)と\(t\)が(順序も含めて)まったく同じ記号列である」ことを示す.

上記のように,議論の対象として登場する理論や言語を%
\index[widx]{オブジェクト@オブジェクト}%
\term{オブジェクト}側であるといい,
議論のために用いている理論や言語を%
\index[widx]{メタ@メタ}%
\term{メタ}側であるなどということがある.
形式的記号列の世界での「集合論」を扱う文脈では,
その形式的記号列の世界での「集合論」がオブジェクト側であり,
研究のための道具として用いている集合論がメタ側である.

以降,本書では扱う記号がメタ側なのかオブジェクト側なのかを明示することを目的として,
オブジェクト側での記号はタイプライタ体で「%
\index[sidx]{\(\obj{x}\):オブジェクト側の記号}%
\(\obj{x}\)」のように表し,
メタ側での記号は通常通り「%
\index[sidx]{\(x\):メタ側の記号}%
\(x\)」のようにイタリック体で表す.

\section{整礎な半順序と整礎帰納法}

ここで少し話題を変えて,数学的準備として整礎な半順序と整礎帰納法について述べておく.
以後,\(x, y\)からなる順序対\(\Set{\Set{x}, \Set{x, y}}\)を
\(\pair{x, y}\)と表すことにする.

\begin{Def} \label{Def:preorderedset}
	\(X\)を集合,\(\mathord{\prec} \subset X \times X\)を\(X\)上の二項関係とする.
	\(\pair{x, y} \in \mathord{\prec}\)であることを\(x \prec y\)と表記する.
	このとき,\(\mathord{\prec}\)が\(X\)上の%
	\index[widx]{はんじゅんじょ@半順序}%
	\term{半順序}であるとは,
	以下の条件がすべて満たされることをいう:
	\begin{enumerate}
		\item すべての\(x \in X\)に対して\(x \not \prec x\)である.ただし,\(x \not \prec x\)とは\(x \prec x\)が成り立たないことをいう.
		\item すべての\(x, y, z \in X\)に対して,\(x \prec y\)かつ\(y \prec z\)ならば\(x \prec z\)である.
	\end{enumerate}
\end{Def}

\begin{Def} \label{Def:well-foundedset}
	\(X\)を集合,\(\mathord{\prec}\)を\(X\)上の半順序とする.
	このとき,\(A \subset X\)について,\(A\)の元\(a\)が\(A\)の%
	\index[widx]{きょくしょうげん@極小元}%
	\term{極小元}であるとは,
	\(x \prec a\)となる\(x \in A\)が存在しないことをいう.
	また,\(\prec\)が\(X\)上で%
	\index[widx]{せいそ@整礎}%
	\term{整礎}であるとは,\(X\)の空でない任意の部分集合\(A\)が必ず極小元\(a \in A\)をもつことをいう.
\end{Def}

整礎な二項関係においては,以下に述べる整礎帰納法が重要である.

\begin{Thm}[整礎帰納法] \label[Thm]{Thm:well-foundedinduction}
	\(X\)を集合,\(\mathord{\prec}\)を\(X\)上の整礎な半順序であるとする.
	このとき,\(A \subset X\)が
	「\(x \in X\)を1つとるとき,\(y \prec x\)を満たすすべての\(y \in X\)に対して\(y \in A\)を満たすのであれば\(x \in A\)である」
	という条件を満たすのであれば\(A = X\)である.
\end{Thm}

\begin{proof}
	定理の主張が成り立たない,すなわち
	「\(x \in X\)を1つとるとき,\(y \prec x\)を満たすすべての\(y \in X\)に対して\(y \in A\)を満たすのであれば\(x \in A\)である」
	という条件を満たすにもかかわらず\(A \neq X\)であるとする.
	このとき,差集合\(X \setminus A\)は空でないため,その極小元\(x \in X \setminus A\)がとれる.
	\(x\)が\(X \setminus A\)の極小元であることから,\(y \prec x\)を満たす\(y\)はすべて\(A\)の元である.
	よって\(x \in A\)でなければならないが,これは\(x \in X \setminus A\)に矛盾する.
\end{proof}

\Cref{Thm:well-foundedinduction}において,集合\(A\)を「\(x \in X\)が与えられた性質を満たすときに\(x \in A\)とする」
のように決めておけば,\cref{Thm:well-foundedinduction}は\(X\)の元に関する何らかの性質の証明に活用することができる.

帰納法とくれば次はもちろん帰納的定義である.

\index[widx]{きのうてきていぎ@(整礎な半順序における)帰納的定義}
\begin{Thm}[帰納的定義] \label[Thm]{Thm:inductivedifinition}
	集合\(X, Y\)と\(X\)上の整礎な半順序\(\prec\)が与えられ,さらに
	写像\(G \colon \bigcup _{x \in X} Y^{\Set{y | y \prec x}}\)が与えられたとする.
	このとき,写像\(f \colon X \to Y\)で任意\(x \in X\)に対して
	\begin{equation}
		\apply{f}{x} = \apply{G}{f \restriction \Set{y | y \prec x}}
		\label{eq:inductivedifinition}
	\end{equation}
	を満たすものがただ1つ存在する.
\end{Thm}

\section{形式言語} \label{sec:formallanguage}

\Cref{sec:symbolicLogic}で述べたように,
今後我々が議論の対象とするのは「意味」をもたない形式的な記号列である.
まずはこの「形式的な記号列」の数学的定式化を述べておく.
以後,「自然数」といえば0も含むので注意されたい.

\begin{Def} \label{Def:formallanguage}
	記号の集合\(\Sigma\)が与えられたとする.\(w\)が\(\Sigma\)上の%
	\index[widx]{もじれつ@文字列|see{記号列}}%
	\term{文字列}%
	,あるいは%
	\index[widx]{きごうれつ@記号列}%
	\term{記号列}%
	であるとは,自然数\(n\)が存在して,\(w\)が
	集合\(\Set{0, 1, \dots, n-1}\)から\(\Sigma\)への写像であることをいう%
	\footnote{%
		集合\(\Set{0,1,\dots, n-1}\)は,集合論的には自然数\(n\)そのものである.
	}%
	.
	ただし,\(w\)が\(\Sigma\)への空写像%
	\footnote{%
		任意の集合\(A\)に対して,始集合が空集合\(\emptyset\)で終集合が\(A\)であるような
		写像\(f \colon \emptyset \to A\)はただひとつ存在する.
		この写像\(f\)を\(A\)への空写像と呼ぶ.
	}%
	である場合も含むものとする(このとき,上でいう自然数\(n\)は0とする).
	このような自然数\(n\)は存在すれば一意である.

	また,記号列\(w \colon \Set{0, 1, \dots, n-1} \to \Sigma\)について,\(n\)を\(w\)の%
	\index[widx]{ながさ@(記号列の)長さ}%
	\term{長さ}%
	といい,%
	\index[sidx]{\(\length{w}\):(記号列の)長さ}%
	\begin{equation}
		\length{w}
		\label{eq:lengthforstring}
	\end{equation}
	と表す.
	\(\Sigma\)上の長さ\(n\)の記号列\(w\)が
	\(\apply{w}{0} = s_0, \apply{w}{1} = s_1, \dots, \apply{w}{n-1} = s_{n-1}\)
	と定義されている場合,\(w\)を\(s_0, s_1, \dots, s_{n-1}\)を書き並べて%
	\index[sidx]{\(s_0 s_1 \dots s_{n-1}\):記号列の表記}%
	\begin{equation}
		s_0 s_1 \dots s_{n-1}
	\end{equation}
	と表すことが多い.
	\(n = 0\)の場合,この表記法では何も書けないが,便宜上%
	\index[sidx]{\(\emptystring\):長さ0の記号列}%
	\begin{equation}
		\emptystring
		\label{eq:emptystring}
	\end{equation}
	と表記することとする.

	さらに,\(\Sigma\)上の記号列の全体\(\bigcup_{n \in \Natural} \Sigma^{\Set{0,1,\dots, n-1}}\)を
	\(\Sigma\)の%
	\index[widx]{Kleeneへいほう@Kleene閉包}%
	\term{Kleene閉包}%
	といい,%
	\index[sidx]{\(\KleeneClosure{\Sigma}\):Kleene閉包}%
	\begin{equation}
		\KleeneClosure{\Sigma}
		\label{eq:KleeneClosure}
	\end{equation}
	と表す.
\end{Def}

\begin{Ex} \label[Ex]{Ex:formalstring}
	\(\Sigma\)を以下の3つからなる集合とする:
	\begin{enumerate}
		\item 数字:\(\obj{0}, \obj{1}, \obj{2}, \dots\)
		\item 加法を表す演算子記号:「\(\mathord{\obj{+}}\)」
		\item 左カッコと右カッコ:「\(\lparen\)」「\(\rparen\)」
	\end{enumerate}
	このとき,以下はすべて\(\Sigma\)上の記号列である:
	\begin{align*}
		\obj{0},                                                                 \\
		\obj{\lparen 1 + 2 \rparen},                                             \\
		\obj{\lparen \lparen \lparen 4 \obj+ 1 \rparen + 5 \rparen + 6 \rparen}, \\
		\obj{\rparen 4 + + \lparen \rparen 4 5 \lparen +}
	\end{align*}
\end{Ex}

\Cref{Ex:formalstring}からもわかるように,
単に「記号列」といっただけでは数学的に何らかの意味をもつと期待されるような記号列以外の記号列も含む.
そのため,数学における「論理」を対象に議論する際には単に使う記号を列挙するだけでは不十分で,
どのような記号列を「認める」のかを宣言する必要がある.
要するに,用意した記号に対して「文法」を定義して議論の対象となる文字列を確定させる必要がある.
このときによく使われる手法が帰納的定義である.例を挙げよう.

\begin{Ex} \label[Ex]{Ex:recursivedefinition}
	\(\Sigma\)を\cref{Ex:formalstring}で述べた記号の集合とする.
	このとき,「数式」を以下のように定義する:
	\begin{enumerate}
		\item \(\Sigma\)の元のうち,数字はすべて「数式」である.
		\item \(t_1, t_2\)が数式であるならば,記号列
		      \begin{equation*}
			      \obj{\lparen} t_1 \obj{+} t_2 \obj{\rparen}
		      \end{equation*}
		      は「数式」である.
		\item 以上の規則を有限回適用して得られるもののみが「数式」である.
	\end{enumerate}
	このとき,\cref{Ex:formalstring}で述べた記号列のうち1つ目から3つ目までは「数式」であるが,4つ目は「数式」ではない.
\end{Ex}

\Cref{Ex:recursivedefinition}で特徴的なのは,『「数式」とは〇〇であるものをいう』などのような直接的な定義ではなく
『\(\Sigma\)に属する数字からどのようにして「数式」全体が得られるか』という構成方法を述べていることである.
何が「数式」で何が「数式」でないかが明確に定まっているため,この定義は定義として正当なものである.
\Cref{Ex:recursivedefinition}で行ったような「ベースとなる対象から定義したい対象がどのように構成されるかを述べる」
形式でなされる定義を%
\index[widx]{きのうてきていぎ@帰納的定義}%
\term{帰納的定義},
あるいは%
\index[widx]{さいきてきていぎ@再帰的定義|see{帰納的定義}}%
\term{再帰的定義}という.
最後に「以上の規則を有限回適用して得られるもののみが」と宣言することで,
上記の手続きで得られないものはすべてその定義からは外れていることを宣言し,
対象の範囲を確定させている.この宣言がないと,
実際に定義したいものよりも広い範囲の対象が定義から外れることが明文化されない
\footnote{%
	例えば,偶数全体の集合\(A\)を「\(0 \in A\)であり,\(n \in A\)ならば\(n + 2 \in A\)かつ\(n - 2 \in A\)である」などとして
	定義しようとしたとき,整数全体の集合\(\Zahlen\)もカギカッコ内の条件自体は満たしている.
}%
.
とはいえ,帰納的定義を行う上でこの宣言が必要なくなることはありえないので,自明であるとして省略されることも多い.

\begin{Que} \label{Que:recursivedefinition}
	\Cref{Ex:recursivedefinition}で述べた「数式」の定義を集合論のことばで正当化してみよう.
	次のような集合\(L \subset \KleeneClosure{\Sigma}\)の存在を示せば十分である:
	\begin{enumerate}
		\item \(t \in \Sigma\)が数字ならば\(t \in L\)である.
		\item 任意の\(t_1, t_2 \in L\)に対して\(\obj{\lparen} t_1 \obj{+} t_2 \obj{\rparen} \in L\)である.
		\item \(L\)は上記の性質を満たす\(\KleeneClosure{\Sigma}\)の部分集合として包含関係に関して最小のものである.
	\end{enumerate}
	1, 2, 3をすべて満たす集合\(L\)の存在を示せ.
\end{Que}
\chapter{1階述語論理の統語論}
\label[chapter]{chap:syntax}

本章では,1階述語論理と呼ばれる体系の統語論について述べる.
統語論とは,雑に述べると「文の構造」についての理論である.
この「文」は,ここでは「数学における何らかの主張」に対応する.
つまり,ここで議論したいのは
「数学における何らかの主張はどのような要素がどう組み合わさってできているのか」
ということである.
このことを議論するための足掛かりとして,我々は\cref{chap:formulize}で述べた方針に従い
「何ら素朴的直観が関与しない形式的な記号列の世界」において
論理式という概念を構成していく.

形式的記号列の世界での定義により,何を議論の対象とし,何を議論の対象としないのかが明瞭となる.
これにより,通常の数学と同じく一般的な法則や定理を研究することが可能となる.
本章の内容はその前準備に相当する.

\newpage

\section{言語} \label[section]{sec:language}

議論を始めるにあたり,どのような記号を使用するのかを明瞭にする必要がある.
まず,通常の数学においてはどうなっているかを振り返ろう.

\begin{Ex} \label[Ex]{Ex:informalsymbol}
	群論においては,単位元を表す記号「\(\obj{e}\)」,2項演算を表す記号
	「\(\obj{\mathord{\ast}}\)」,逆元を表す記号「\({}^{\obj{-1}}\)」が用いられる.
	また,順序の理論においては,順序を表す記号「\(\obj{\le}\)」が用いられる.

	この他,「\(\obj{x}\)」や「\(\obj{y}\)」等の変数を表す記号やカッコ「\(\lparen\)」「\(\rparen\)」や
	カンマ「\(,\)」等は共通して用いられる.
\end{Ex}

\Cref{Ex:informalsymbol}では,対象の理論に依存して必要であったりそうでなかったりする記号と
共通して用いられる記号があった.
そのような記号を\cref{tab:commonsymbol}に示し,今後逐一言及しないものとする%
\footnote{%
	ここに示す記号とは違う記号を採用する場合もあるが,それは単に流儀や議論の対象の違いである.
	その違いによって質的に大きな差異が生じることもあればそうでないこともある.
}%
.

\begin{table}[htbp]
	\centering
	\caption{数学理論で共通して用いられる記号}
	\label{tab:commonsymbol}
	\begin{tabular}{ccc}
		\toprule
		種別   & 一覧                                                                   & 備考              \\
		\midrule
		変数記号 & \(\obj{x}, \obj{y}, \obj{z}, \dotsc\)                                & 無限に多く存在する(可算無限) \\
		論理記号 & \(\bot, \lnot, \land, \lor, \to, \forall, \exists\)                                    \\
		等号   & \(\obj{\mathord{\objeq}}\)                                           & オブジェクト側の意味での等号  \\
		補助記号 & \(\text{``\(\lparen\)''}, \text{``\(\rparen\)''}, \text{``\(,\)''}\) & カッコやカンマ         \\
		\bottomrule
	\end{tabular}
\end{table}

「名称」列に「変数記号」や「論理記号」等の名前が定義なしに入っているが,これは単にそういう区分けができることのみが要請される.
また,変数記号については無限に多く存在することが要請されているが,これについてはすでにいくつか記号が登場している状況下において
そのいずれとも異なる変数記号をいつでも用意できることを期待しての要請である.

さて,\cref{tab:commonsymbol}に追加で記号を付け加えることにより,各理論の特色が現れる.

\begin{Def} \label{def:language}
	記号の集合\(\symcal{L}\)が%
	\index[widx]{げんご@言語}%
	\term{言語}であるとは,\(\symcal{L}\)の元が以下の3種類に区分けされていることをいう:
	\begin{itemize}
		\item \index[widx]{ていすうきごう@定数記号}定数記号
		\item \index[widx]{かんすうきごう@関数記号}関数記号,\index[widx]{ありてぃ@アリティ}\term{アリティ}と呼ばれる正の整数\(n\)をもつ
		\item \index[widx]{かんけいきごう@関係記号}関係記号,\index[widx]{ありてぃ@アリティ}アリティと呼ばれる正の整数\(n\)をもつ
	\end{itemize}
\end{Def}

「定数記号」や「関数記号」という字面はいかにも我々の素朴的直観を呼び起こしそうであるが,
ここでは単にそういう名称で区分けできるということだけを要請しているに過ぎない.
アリティについても同様である.「この関数記号のアリティはいくつですか?」に対して「2です」のような
解答を返せることを要請しているに過ぎない.

言語については,例を述べるのがわかりやすいだろう.

\begin{Ex} \label[Ex]{Ex:languageexample}
	群論の言語\(\symcal{L}_1\)は\(\symcal{L}_1 = \set{\obj{\ast}, \obj{e}, {}^{\obj{-1}}}\)
	と与えることができる.ここで,\(\obj{\ast}\)はアリティ2の関数記号,\(\obj{e}\)は定数記号,
	\({}^{\obj{-1}}\)はアリティ1の関数記号である.
	また,順序の理論の言語\(\symcal{L}_2\)は\(\symcal{L}_2 = \set{\obj{\le}}\)と与えることができる.
	ここで,\(\obj{\le}\)はアリティ2の関係記号である.
\end{Ex}

アリティというのは,素朴的直観における「引数の数」の対応物だと考えられる.
こう考えてみれば,関数記号や関係記号にアリティが定まっていることを要請するのはごく自然であろう.


\section{項と論理式} \label[section]{sec:logicalexpression}

使う記号を定義したことで,これらを組み合わせて「モノ」や「主張」に相当する概念を構築していくことができる.
これは言語に対してある種の「文法」を定めることに相当する.
文法を定めることにより,形式的記号の集合でしかなかった言語が一気に「数学っぽい」性格を帯びてくる.

\index[widx]{こう@項}
\begin{Def} \label[Def]{Def:term}
	言語\(\symcal{L}\)に対し,\(\symcal{L}\)項を以下のように帰納的に定義する:
	\begin{enumerate}
		\item 変数記号は\(\symcal{L}\)項である.
		\item 定数記号は\(\symcal{L}\)項である.
		\item \(f\)が\(\symcal{L}\)におけるアリティ\(n\)の関数記号であり%
		      \footnote{%
			      ここで\(f\)がイタリック体なのは,「\(f\)」という記号そのものが\(\symcal{L}\)の関数記号というわけではなく
			      \(\symcal{L}\)の関数記号のうちのどれかであるということを明示することを意図している.
		      }%
		      ,
		      \(t_1, t_2, \dotsc, t_n\)が\(\symcal{L}\)項であるならば,記号列
		      \begin{equation}
			      \apply{f}{t_1, t_2, \dotsc, t_n}
		      \end{equation}
		      は\(\symcal{L}\)項である.
		\item 以上の規則を有限回適用して得られるもののみが項である.
	\end{enumerate}
\end{Def}


\begin{Ex} \label[Ex]{Ex:groupterm}
	\Cref{Ex:languageexample}で述べた群論の言語\(\symcal{L}_1\)において,
	\(\apply{\obj{\ast}}{\obj{x}, \obj{y}}, \obj{e}, \apply{\obj{\ast}}{\apply{\obj{\ast}}{\obj{e}, \obj{x}}, \obj{y}}\)は
	いずれも\(\symcal{L}_1\)項である.ここで,\(\obj{x}, \obj{y}\)は変数記号である.
	なお,順序の理論の言語\(\symcal{L}_2\)は定数記号も関数記号ももたないため,
	\(\symcal{L}_2\)項は変数記号のみである.
\end{Ex}

\begin{Que} \label{Que:termexample}
	\Cref{Def:term}に基づき,\cref{Ex:groupterm}における
	「\(\apply{\obj{\ast}}{\apply{\obj{\ast}}{\obj{e}, \obj{x}}, \obj{y}}\)」
	が\(\symcal{L}_1\)項であることを確かめよ.
	また,記号列「\(\obj{e} \apply{\mathord{\obj{\ast}}}{\obj{e}, \obj{e}}\)」が\(\symcal{L}_1\)項でないことを確かめよ.
\end{Que}


\index[widx]{ろんりしき@論理式}
\begin{Def} \label[Def]{Def:logicalexpression}
	言語\(\symcal{L}\)に対し,\(\symcal{L}\)論理式を以下のように帰納的に定義する:
	\begin{enumerate}
		\item \(\bot\)は論理式である.
		\item \(t_1, t_2\)が\(\symcal{L}\)項であるならば,記号列
		      \begin{equation}
			      \paren{t_1 \obj{\objeq} t_2}
		      \end{equation}
		      は\(\symcal{L}\)論理式である.
		\item \(r\)が\(\symcal{L}\)におけるアリティ\(n\)の関係記号であり%
		      \footnote{%
			      \Cref{Def:term}のときと同様の理由で\(r\)はイタリック体としている.
		      }%
		      ,\(t_1, t_2, \dotsc, t_n\)が\(\symcal{L}\)項であるならば,
		      記号列
		      \begin{equation}
			      \apply{r}{t_1, t_2, \dotsc, t_n}
		      \end{equation}
		      は\(\symcal{L}\)論理式である.
		\item \(\varphi, \psi\)が論理式で\(x\)が変数記号であるならば,
		      記号列
		      \begin{align}
			      \paren{\lnot \varphi},      \\
			      \paren{\varphi \land \psi}, \\
			      \paren{\varphi \lor \psi},  \\
			      \paren{\varphi \to \psi},   \\
			      \paren{\forall x \varphi},  \\
			      \paren{\exists x \varphi}
		      \end{align}
		      はいずれも\(\symcal{L}\)論理式である.
		\item 以上の規則を有限回適用して得られるもののみが論理式である.
	\end{enumerate}
\end{Def}

\begin{Def} \label[Def]{Def:atomiclogicalexpression}
	\Cref{Def:logicalexpression}において,最後に規則1, 2, 3を適用して得られる\(\symcal{L}\)論理式,すなわち
	\begin{align*}
		\bot,                   \\
		\paren{t_1 \objeq t_2}, \\
		\apply{r}{t_1, t_2, \dotsc, t_n}
	\end{align*}
	の形の\(\symcal{L}\)論理式を%
	\index[widx]{ろんりしき@論理式!げんしろんりしき@原子---}%
	\(\symcal{L}\)\term{原子論理式}といい,
	\(\symcal{L}\)原子論理式でない\(\symcal{L}\)論理式,
	すなわち最後に規則4を適用して得られる\(\symcal{L}\)論理式を%
	\index[widx]{ろんりしき@論理式!ふくごうろんりしき@複合---}%
	\(\symcal{L}\)\term{複合論理式}という.
	さらに,
	\begin{align*}
		\forall x \varphi, \\
		\exists x \varphi
	\end{align*}
	の形の\(\symcal{L}\)論理式をそれぞれ%
	\index[widx]{ろんりしき@論理式!ぜんしょうろんりしき@全称---}%
	\(\symcal{L}\)\term{全称論理式},
	\index[widx]{ろんりしき@論理式!そんざいろんりしき@存在---}%
	\(\symcal{L}\)\term{存在論理式}という.
\end{Def}

\begin{Def} \label[Def]{Def:universalclosure}
	\(\symcal{L}\)を言語とし,\(\varphi\)を\(\symcal{L}\)論理式,\(\varphi\)に自由出現する変数記号全体が
	\[
		\apply{\FV}{\varphi} = \Set{x_1, x_2, \dots, x_n}
	\]
	であるとする.このとき,\(\symcal{L}\)論理式
	\[
		\forall x_1 \forall x_2 \dotsb \forall x_n \varphi
	\]
	は\(\symcal{L}\)閉論理式となる.この論理式を\(\varphi\)の%
	\index[widx]{ぜんしょうへいほう@全称閉包}%
	\term{全称閉包}という.
\end{Def}

\begin{Ex} \label[Ex]{Ex:logicalexpression}
	群論の言語\(\symcal{L}_1\)において,
	\(\paren{\apply{\obj{\ast}}{\obj{x}, \obj{y}} \obj{\objeq} \apply{\obj{\ast}}{\obj{y}, \obj{x}}}\)は
	\(\symcal{L}_1\)論理式である.また,順序の理論の言語\(\symcal{L}_2\)において,
	\(\paren{ \forall \obj{y} \mathord{\obj{\le}}\paren{ \obj{x}, \obj{y}} }\)や
	\(\paren{ \forall \obj{x} \paren{ \forall \obj{y}\apply{\mathord{\obj{\le}}}{\obj{x}, \obj{y}}} }\)は
	いずれも\(\symcal{L_2}\)論理式である.ここで,\(\obj{x}, \obj{y}\)は変数記号である.
\end{Ex}

論理式でない記号列の例も挙げておこう.

\begin{Ex} \label[Ex]{Ex:nologicalexpression}
	言語\(\symcal{L}\)において,\(\obj{x}\)が変数記号であるとき,
	記号列\(\paren{\exists \obj{x} \paren{\obj{x}}}\)は\(\symcal{L}\)論理式ではない.
	一方で,\(\paren{\exists \obj{x} \paren{\obj{x} \obj{\objeq} \obj{x}}}\)は\(\symcal{L}\)論理式である.
	また,群論の言語\(\symcal{L}_1\)において,\(\paren{\forall \obj{x} \paren{\obj{x} \obj{\ast} \obj{e} \objeq \obj{x}}}\)や
	\(\forall \obj{x} \paren{ \obj{x} \obj{\objeq} \obj{x}}\)はいずれも
	\emph{\cref{Def:logicalexpression}で述べた意味においては}\(\symcal{L}_1\)論理式ではない.
\end{Ex}

\begin{Que} \label[Que]{Que:logicalexpression}
	\Cref{Def:logicalexpression}に基づき,\cref{Ex:logicalexpression}と\cref{Ex:nologicalexpression}で
	挙げた各式について,それが実際に論理式であることやそうでないことを確かめよ.
\end{Que}

\begin{Note} \label[Note]{Note:logicalexpression}
	\Cref{Ex:nologicalexpression}の後半で「\cref{Def:logicalexpression}で述べた意味においては」
	と述べたのは,
	それなりに妥当性のある略記表現に関する約束事を適切に定めることにより,
	これらが論理式であるようにみなせるからである.
	例えば,以下のように約束することが多い:
	\begin{itemize}
		\item 誤解が生じない範囲で論理式の構成順序を表すかっこは省略してよい.
		\item 論理記号\(\lnot, \land, \lor, \to, \forall, \exists\)たちの結合の優先順位については,
		      \(\lnot, \forall, \exists\)がもっとも高く,次に\(\land, \lor\)が高く,
		      もっとも低いのが\(\to\)であると約束する.
		\item アリティ2の関数記号や関係記号については,\(\obj{x} \obj{\ast} \obj{y}\)や\(\obj{x} \le \obj{y}\)のように
		      その記号が真ん中に来るように配置して表記してよい.
	\end{itemize}
	こうすると,\(\paren{\paren{\varphi \land \paren{\lnot \psi}} \to \xi}\)は\(\varphi \land \lnot \psi \to \xi\)のように見やすくできる.
	これは,通常の数学における数の演算において,\(\mathord{\times}\)が\(\mathord{+}\)よりも優先度が高いとみなして
	\(\paren{\paren{2 \times 3} + 4} = 2 \times 3 + 4\)と略記することで可読性の向上を図るのとまったく同じである.
	これは人間が目で見る際に楽をするための約束事であって,数学的な議論の帰結ではないことに注意しておこう.
	本書でも,これらの略記表現を積極的に利用する.
\end{Note}

\begin{Note}
	例えば論理式\(\varphi, \psi, \chi\)に対する以下の2つの論理式
	\begin{align*}
		\varphi \land \paren{\psi \land \chi}, \\
		\paren{\varphi \land \psi} \land \chi
	\end{align*}
	は,それが表すと期待される主張を考えればどちらも「同じ論理式」であって,これらを
	\[
		\varphi \land \psi \land \chi
	\]
	と表記してしまってもよさそうに見える.
	本書では,これらの「素朴に考えれば問題は生じなさそう」という略記表現も積極的に使用することとする.
	確かに,素朴には問題なさそうでも本当に以降の議論に支障が生じないかどうかまでは実際に確かめるまではわからないというのは事実である.
	しかし,少なくとも本書の範囲内ではそれによって問題が生じることはまずない.
\end{Note}

\begin{Note} \label[Note]{Note:languageomission}
	ここまで「言語\(\symcal{L}\)において」とか「\(\symcal{L}\)項」のように,用いる言語を明示して議論を進めてきた.
	しかし,以下で行われるのは特定の言語に依存しない議論がほとんどである.
	そのため,特に断りがない限りは「言語\(\symcal{L}\)」の表記は省略することとする.
	「\(\symcal{L}\)項」や「\(\symcal{L}\)論理式」は単に「項」や「論理式」と呼称する.
\end{Note}


\section{変数の出現と代入可能性} \label{sec:substitution}

数学においては,一般論に具体例を当てはめることによって議論を進めることが多い.
これの形式的記号列の世界での対応物は,
項や論理式に登場する変数記号に別のものを当てはめることである.
この操作を定式化するためには,いくつかのステップを踏む必要がある.
なお,正確に書くとそれなりに議論が長くなるため,以下では相当にラフに記述していることに注意されたい.
ラフな部分はいずれも帰納的定義によって精密化できる.


\begin{Def} \label[Def]{Def:occurence}
	論理式
	\begin{align*}
		\varphi \colon \paren{\cdots \paren{\forall x \paren{\cdots x \cdots } \cdots}}, \\
		\psi \colon \paren{\cdots \paren{\exists x \paren{\cdots x \cdots } \cdots}}
	\end{align*}
	の変数記号\(x\)のように,\(\forall, \exists\)とともに出現している変数記号は
	その論理式に%
	\index[widx]{そくばくしゅつげん@束縛出現}%
	\term{束縛出現}しているといい,
	\(\forall, \exists\)をともなわずに出現している変数記号は
	その論理式に%
	\index[widx]{じゆうしゅつげん@自由出現}%
	\term{自由出現}しているという.
	また,論理式\(\varphi\)に束縛出現する変数記号全体の集合と\(\varphi\)に自由出現する変数記号全体の集合を,それぞれ%
	\index[sidx]{\(\apply{\BV}{\varphi}\):論理式に束縛出現する変数記号全体の集合}%
	\index[sidx]{\(\apply{\FV}{\varphi}\):論理式に自由出現する変数記号全体の集合}%
	\begin{align}
		\apply{\BV}{\varphi}, \\
		\apply{\FV}{\varphi}
	\end{align}
	と表す.%
	\index[sidx]{\(\apply{\Var}{t}\):項に出現する変数記号全体の集合}%
	さらに,項\(t\)に出現する変数記号全体の集合を
	\begin{equation}
		\apply{\Var}{t}
	\end{equation}
	と表す.%
	\(\apply{\FV}{\varphi} = \emptyset\)であるような論理式\(\varphi\)は%
	\index[widx]{ろんりしき@論理式!へいろんりしき@閉---}%
	\term{閉論理式},あるいは%
	\index[widx]{ぶん@文|see{閉論理式}}%
	\term{文}と呼ぶ.
	一方,\(\forall, \exists\)を含まない論理式を%
	\index[widx]{ろんりしき@論理式!かいろんりしき@開---}%
	\term{開論理式}という.
	\(\apply{\Var}{t} = \emptyset\)となる項\(t\)は%
	\index[widx]{こう@項!へいこう@閉---}%
	\term{閉項}であるという.
\end{Def}

\begin{Note} \label[Note]{Note:languagesentence}
	言語\(\symcal{L}\)を明示する文脈においては
	閉論理式や文,閉項はそれぞれ\(\symcal{L}\)閉論理式,\(\symcal{L}\)文,\(\symcal{L}\)閉項と呼ぶ.
\end{Note}

\begin{Ex} \label[Ex]{Ex:occurence}
	群論の言語\(\symcal{L} = \set{\obj{\ast}, \obj{e}, {}^{\obj{-1}}}\)において,
	論理式
	\begin{equation*}
		\varphi \colon \paren{\forall \obj{a} \paren{\forall y \paren{
					\apply{\mathord{\obj{\ast}}}{\obj{x}, \obj{y}} \obj{\objeq} \apply{\mathord{\ast}}{\obj{y}, \obj{x}}
				}}}
	\end{equation*}
	に対しては
	\begin{align*}
		\apply{\BV}{\varphi} = \set{\obj{a}, \obj{y}}, \\
		\apply{\FV}{\varphi} = \set{\obj{x}}
	\end{align*}
	が成り立つ.この\(\obj{a}\)のように,
	\(\forall, \exists\)記号の直後でのみ出現する変数記号も束縛出現するとみなす.
\end{Ex}

以上の準備のもと,代入操作を定式化したいのだが,先に代入可能性について論ずる必要がある.

\index[widx]{だいにゅう@代入}
\begin{Def} \label[Def]{Def:cansubstitution}
	\(\varphi\)を論理式,\(x\)を変数記号,\(t\)を項とする.
	以下の2条件をともに満たす変数記号\(y\)が存在するとき,
	\(t\)は\(\varphi\)中の\(x\)に%
	\index[widx]{だいにゅう@代入!だいにゅうふかのう@---不可能}%
	\term{代入不可能}であるといい,
	そのような\(y\)が存在しないとき,
	\(t\)は\(\varphi\)中の\(x\)に%
	\index[widx]{だいにゅう@代入!だいにゅうかのう@---可能}%
	\term{代入可能}%
	であるという:
	\begin{itemize}
		\item \(\varphi\)が
		      \(\paren{\cdots \paren{\forall y \paren{\cdots x \cdots} \cdots} \cdots}\)
		      か
		      \(\paren{\cdots \paren{\exists y \paren{\cdots x \cdots} \cdots} \cdots}\)
		      の形の論理式である.ただし,この\(x\)は\(\varphi\)に自由出現しているものとする.
		\item \(y \in \apply{\Var}{t}\)である.
	\end{itemize}
\end{Def}

代入可能性については,例を見るのが手っ取り早い.

\begin{Ex} \label{Ex:cansubstitution}
	群論の言語\(\symcal{L} = \set{\obj{\ast}, \obj{e}, {}^{\obj{-1}}}\)において,
	論理式
	\begin{equation*}
		\varphi \colon \paren{\forall \obj{a} \paren{\forall y \paren{
					\apply{\mathord{\obj{\ast}}}{\obj{x}, \obj{y}} \obj{\objeq} \apply{\mathord{\ast}}{\obj{y}, \obj{x}}
				}}}
	\end{equation*}
	中の\(\obj{x}\)に項\(\apply{\obj{\ast}}{\obj{y}, \obj{e}}\)は代入不可能である.
	一方で,\(\obj{x}\)に項\(\obj{e}\)は代入可能である.
	また,\(\obj{a}, \obj{y}\)だけでなく
	\(\varphi\)に出現しない変数記号すべてに対してあらゆる項が代入可能である.
\end{Ex}

\index[sidx]{\(\subst{\varphi}{t/x}\):論理式への代入}
\index[sidx]{\(\subst{t}{s/x}\):項への代入}
\begin{Def} \label[Def]{Def:substitution}
	論理式\(\varphi\)中の変数記号\(x\)に項\(t\)が代入可能であるとき,
	\(\varphi\)に自由出現している\(x\)すべてを\(t\)に置き換えて得られる論理式を
	\begin{equation}
		\subst{\varphi}{t/x}
	\end{equation}
	と表す.

	また,項\(t\)に現れる変数記号\(x\)すべてを項\(s\)に置き換えて得られる項も同様に
	\begin{equation}
		\subst{t}{s/x}
		\label{eq:substterm}
	\end{equation}
	と表す.
\end{Def}

\begin{Ex} \label[Ex]{Ex:substitution}
	\Cref{Ex:cansubstitution}における\(\varphi\)において
	\begin{equation*}
		\subst{\varphi}{\obj{e} / \obj{x}} \colon
		\paren{\forall \obj{a} \paren{\forall \obj{y} \paren{
					\apply{\mathord{\obj{\ast}}}{\obj{e}, \obj{y}} \obj{\objeq} \apply{\mathord{\ast}}{\obj{y}, \obj{e}}
				}}}
	\end{equation*}
	である.また,\(\subst{\varphi}{\obj{e} / \obj{a}}\)と\(\subst{\varphi}{\obj{e} / \obj{y}}\)はいずれも\(\varphi\)そのものである.
\end{Ex}

\begin{Note}
	\(\varphi\)を論理式,\(x_1, x_2, \dots, x_n\)を\(\varphi\)に束縛出現しない変数記号とするとき,
	\(\varphi\)のことを
	\begin{equation}
		\apply{\varphi}{x_1, x_2, \dots, x_n}
		\label{eq:freevariablefunction}
	\end{equation}
	と表すことがある.このとき,\(\varphi\)中の\(x_1, x_2, \dots, x_n\)に代入可能な項\(t_1, t_2, \dots, t_n\)を代入して得られる論理式は
	\begin{equation}
		\apply{\varphi}{t_1, t_2, \dots, t_n}
		\label{eq:freevariablefunctionsubstitution}
	\end{equation}
	と表すことが多い.
	ここで,\(t_1, t_2, \dots, t_n\)に現れる変数記号が\(\varphi\)に束縛出現する場合,その束縛出現する記号を別の変数記号に置き換え,
	\(t_1, t_2, \dots, t_n\)に現れる変数記号が\(\varphi\)に束縛出現しないようにしておく.
	このような置き換えは,しばしば暗黙的に行われる.
	そのようにしても議論に影響がないことは\cref{chap:semantics}の\cref{Thm:alphaequivalent}による.
\end{Note}

記述を簡素化するために,追加で略記表現を導入する.

\index[sidx]{\(\formulaequiv\):(論理式の)同値}
\index[sidx]{\(\uexists\):一意存在}
\begin{Def} \label{Def:AbbreviatioForFormula}
	論理式\(\varphi, \psi\)に対し,論理式
	\begin{align*}
		\paren{\varphi \to \psi} \land \paren{\psi \to \varphi}
	\end{align*}
	を
	\begin{equation}
		\varphi \formulaequiv \psi
		\label{eq:formulaequiv}
	\end{equation}
	と略記する.
	また,変数記号\(x_1, x_2\)が自由出現しない論理式\(\varphi\)に対し,論理式
	\begin{align*}
		\exists x \varphi \land \forall x \forall y \paren{\subst{\varphi}{x_1/x} \land \subst{\varphi}{x_2} \to x_1 \objeq x_2}
	\end{align*}
	を
	\begin{equation}
		\uexists x \varphi
		\label{eq:uexists}
	\end{equation}
	と略記する.

	\Cref{eq:formulaequiv}は「\(\varphi\)と\(\psi\)は同値である」
	ことを表現することが期待される論理式で,
	\cref{eq:uexists}は「\(\varphi\)となる\(x\)がただ1つ存在する」
	ことを表現することが期待される論理式である.
\end{Def}


\section{理論とその例} \label[section]{sec:Theory}

前節までは項や論理式に関する一般論を述べた.
ここでは,実際の数学理論がどう形式化されるかについて少しだけ触れる.

\begin{Def} \label[Def]{Def:theory}
	\(\symcal{L}\)を言語とするとき,\(\symcal{L}\)文からなる集合を%
	\index[widx]{りろん@理論}%
	\(\symcal{L}\)\term{理論},あるいは%
	\index[widx]{こうりけい@公理系}%
	\(\symcal{L}\)\term{公理系}
	という.
	言語\(\symcal{L}\)を明示しない文脈では,\(\symcal{L}\)理論や\(\symcal{L}\)公理系は単に理論や公理系と呼ばれる.
\end{Def}

今後,例として登場する言語や理論を先んじていくつか挙げておく.

\index[sidx]{\(\GP\):群の理論}
\index[sidx]{\(\symcal{L}_{\GP}\):群の言語}
\begin{Ex} \label[Ex]{Ex:grouptheory}
	群の言語を\(\symcal{L}_{\GP} = \set{\obj{\ast}, \obj{e}, {}^{\obj{-1}}}\)とする.
	ここで,\(\obj{\ast}\)はアリティ2の関数記号,\(\obj{e}\)は定数記号,\({}^{\obj{-1}}\)はアリティ1の関係記号である.
	群の理論\(\GP\)は,次の3つの文からなる\(\symcal{L}_{\GP}\)理論であると考えることができる:
	\begin{enumerate}
		\item \(\forall \obj{x} \forall \obj{y} \forall \obj{z}
		      \paren{\paren{\obj{x} \obj{\ast} \obj{y}} \obj{\ast} \obj{z} \objeq \obj{x} \obj{\ast} \paren{\obj{y} \obj{\ast} \obj{z}}}\),
		\item \(\forall \obj{x} \paren{\obj{e} \obj{\ast} \obj{x} \objeq \obj{x}}\),
		\item \(\forall \obj{x} \paren{\obj{x}^{\obj{-1}} \obj{\ast} \obj{x} \objeq \obj{e}}\).
	\end{enumerate}
	ここで,\(\apply{{}^{\obj{-1}}}{\obj{x}}\)を\(\obj{x} ^ {\obj{-1}}\)と略記した.
\end{Ex}

\begin{Que} \label[Que]{Que:invalidtheory}
	群の言語を\(\symcal{L} = \set{\obj{\ast}}\)だと考えて,群の理論を次のような\(\symcal{L}\)理論として書き下そうとする場合がある.
	\begin{enumerate}
		\item \(\forall \obj{x} \forall \obj{y} \forall \obj{z}
		      \paren{\paren{\obj{x} \obj{\ast} \obj{y}} \obj{\ast} \obj{z} \objeq \obj{x} \obj{\ast} \paren{\obj{y} \obj{\ast} \obj{z}}},\)
		\item \(\exists \obj{e} \forall \obj{x} \paren{\obj{e} \obj{\ast} \obj{x} \objeq \obj{x}},\)
		\item \(\forall \obj{x} \exists \obj{y} \paren{\obj{y} \obj{\ast} \obj{x} \objeq \obj{e}}.\)
	\end{enumerate}
	しかし,この書き下し方は構文的に不適切である.その理由を述べよ.
\end{Que}

\index[sidx]{\(\Ring\):環の理論}
\index[sidx]{\(\CRing\):可換環の理論}
\index[sidx]{\(\symcal{L}_{\Ring}\):環の言語}
\begin{Ex} \label[Ex]{Ex:Ring}
	環の言語を\(\symcal{L}_{\Ring} = \Set{\obj{+}, \obj{-}, \obj{\cdot}, \obj{0}, \obj{1}}\)とする.
	ここで,\(\obj{+}, \obj{\cdot}\)はアリティ2の関数記号,\(\obj{-}\)はアリティ1の関数記号,\(\obj{0}, \obj{1}\)は定数記号である.
	環の理論\(\Ring\)は,以下の文からなる\(\symcal{L}_{\Ring}\)理論であると考えることができる:
	\begin{enumerate}
		\item \(\forall \obj{x} \forall \obj{y} \forall \obj{z} \paren{\paren{\obj{x} \obj{+} \obj{y}} \obj{+} \obj{z} \objeq \obj{x} \obj{+} \paren{\obj{y} \obj{+} \obj{z}}},\)
		\item \(\forall \obj{x} \paren{\obj{0} \obj{+} \obj{x} \objeq \obj{x}},\)
		\item \(\forall \obj{x} \paren{\paren{\obj{-} \obj{x}} \obj{+} \obj{x} \objeq \obj{0}},\)
		\item \(\forall \obj{x} \forall \obj{y} \paren{\obj{x} \obj{+} \obj{y} \objeq \obj{y} \obj{+} \obj{x}},\)
		\item \(\forall \obj{x} \forall \obj{y} \forall \obj{z}
		      \paren{\paren{\obj{x} \obj{\cdot} \obj{y}} \obj{\cdot} \obj{z} \objeq \obj{x} \obj{\cdot} \paren{\obj{y} \obj{\cdot} \obj{z}}},\)
		\item \(\forall \obj{x} \paren{\obj{1} \obj{\cdot} \obj{x} \objeq \obj{x}},\)
		\item \(\forall \obj{x} \forall \obj{y} \forall \obj{z}
		      \paren{\obj{x} \obj{\cdot} \paren{\obj{y} \obj{+} \obj{z}} \objeq \paren{\obj{x} \obj{\cdot} \obj{y}} \obj{+} \paren{\obj{x} \obj{\cdot} \obj{z}}},\)
		\item \(\forall \obj{x} \forall \obj{y} \forall \obj{z}
		      \paren{ \paren{\obj{x} \obj{+} \obj{y}} \obj{\cdot} \obj{z} \objeq \paren{\obj{x} \obj{\cdot} \obj{z}} \obj{+} \paren{\obj{y} \obj{\cdot} \obj{z}}}.\)
	\end{enumerate}
	最初の文3つは,\(\symcal{L}_{\GP}\)における\(\obj{\ast}\)を\(\obj{+}\)に,\({}^{\obj{-1}}\)を\(\obj{-}\)に置き換えたものであることに注意せよ.

	なお,\(\Ring\)に閉論理式
	\[
		\forall \obj{x} \forall \obj{y} \paren{\obj{x} \obj{\cdot} \obj{y} \objeq \obj{y} \obj{\cdot} \obj{x}}
	\]
	を加えたものは可換環の理論だと考えることができる.
	この理論は\(\CRing\)と表すことにする.
\end{Ex}

\index[sidx]{\(\POSET\):狭義の半順序の理論}
\index[sidx]{\(\TOSET\):狭義の全順序の理論}
\index[sidx]{\(\symcal{L}_{\POSET}\):狭義の半順序の言語}
\begin{Ex} \label[Ex]{Ex:orderedset}
	狭義の半順序の言語を\(\symcal{L}_{\POSET} = \Set{\obj{<}}\)とする.
	ここで,\(\obj{<}\)はアリティ2の関係記号である.
	狭義の半順序の理論\(\POSET\)は,以下の2つの文からなる\(\symcal{L}_{\POSET}\)理論と考えることができる.
	\begin{enumerate}
		\item \(\forall \obj{x} \lnot \paren{x \obj{<} \obj{x}},\)
		\item \(\forall \obj{x} \forall \obj{y} \forall \obj{z}
		      \paren{\obj{x} \obj{<} \obj{y} \land \obj{y} \obj{<} \obj{z} \to \obj{x} \obj{<} \obj{z}}\)
	\end{enumerate}
	なお,理論\(\POSET\)に
	\[
		\forall \obj{x} \forall \obj{y}
		\paren{\obj{x} \obj{<} \obj{y} \lor \obj{y} \obj{<} \obj{x} \lor \obj{x} \objeq \obj{y}}
	\]
	を加えた理論は,狭義の全順序の理論を表していると考えることができる.
	この理論を\(\TOSET\)と表すこととする.
\end{Ex}

\index[sidx]{\(\OrderedRing\):順序環の理論}
\index[sidx]{\(\symcal{L}_{\OrderedRing}\):順序環の言語}
\begin{Ex} \label[Ex]{Ex:OrderedRing}
	順序環の言語を\(\symcal{L}_{\OrderedRing} = \Set{\obj{+}, \obj{-}, \obj{\cdot}, \obj{0}, \obj{1}, \obj{<}}\)とする.
	ここで,\(\obj{<}\)はアリティ2の関係記号,\(\obj{+}, \obj{\cdot}\)はアリティ2の関数記号,
	\(\obj{-}\)はアリティ1の関数記号,\(\obj{0},\obj{1}\)は定数記号である.
	順序環の理論\(\OrderedRing\)は,以下の文からなる\(\symcal{L}_{\OrderedRing}\)理論と考えることができる:
	\begin{enumerate}
		\item \Cref{Ex:Ring}で挙げた理論\(\Ring\)の8つの論理式すべて,
		\item \Cref{Ex:orderedset}で挙げた理論\(\TOSET\)の3つの論理式すべて,
		\item \(\forall \obj{x} \forall \obj{y} \paren{\obj{x} \obj{\cdot} \obj{y} \objeq \obj{y} \obj{\cdot} \obj{x}},\)
		\item \(\forall \obj{x} \forall \obj{y} \forall \obj{z}
		      \paren{\obj{x} \obj{<} \obj{y} \to \obj{x} \obj{+} \obj{z} \obj{<} \obj{y} \obj{+} \obj{z}},\)
		\item \(\forall \obj{x} \forall \obj{y} \forall \obj{z}
		      \paren{\obj{x} \obj{<} \obj{y} \land \obj{0} \obj{<} \obj{z} \to \obj{x} \obj{\cdot} \obj{z} \obj{<} \obj{y} \obj{\cdot} \obj{z}}.\)
	\end{enumerate}
\end{Ex}

\index[sidx]{\(\ProjectiveGeometryPart\):射影平面幾何の部分体系}
\index[sidx]{\(\symcal{L}_{\ProjectiveGeometryPart}\):射影平面幾何の言語}
\begin{Ex} \label[Ex]{Ex:ProjectiveGeometry}
	\NewDocumentCommand{\LiesOn}{}{\mathrel{\obj{\varepsilon}}}
	射影平面幾何の言語を\(\symcal{L}_{\ProjectiveGeometryPart}\)を\(\symcal{L}_{\ProjectiveGeometryPart} = \Set{\obj{P}, \obj{L}, \LiesOn}\)
	とする.ここで,\(\obj{P}, \obj{L}\)はアリティ1の関係記号,\(\obj{\varepsilon}\)はアリティ2の関係記号である.
	このとき,射影平面幾何の理論の部分体系\(\ProjectiveGeometryPart\)として,以下の文からなる\(\symcal{L}_{\ProjectiveGeometryPart}\)理論を考えることができる:
	\begin{enumerate}
		\item \(\forall \obj{x} \paren{\apply{\obj{P}}{\obj{x}} \formulaequiv \lnot \apply{\obj{L}}{\obj{x}}},\)
		\item \(\forall \obj{x} \forall \obj{y} \paren{\obj{x} \LiesOn \obj{y} \to \apply{\obj{P}}{\obj{x}} \land \apply{\obj{L}}{\obj{y}}},\)
		\item \(\forall \obj{x} \forall \obj{y} \paren{\apply{\obj{P}}{\obj{x}} \land \apply{\obj{P}}{\obj{y}} \land \lnot \paren{\obj{x} \objeq \obj{y}}
			      \to \uexists z \paren{\obj{x} \LiesOn \obj{z} \land \obj{y} \LiesOn \obj{z}}},\)
		\item \(\forall \obj{x} \forall \obj{y} \paren{\apply{\obj{L}}{\obj{x}} \land \apply{\obj{L}}{y} \land \lnot \paren{\obj{x} \objeq \obj{y}}
			      \to \uexists z \paren{\obj{z} \LiesOn \obj{x} \land \obj{z} \LiesOn \obj{y}}},\)
		\item \(\apply{C}{x, y, z}\)を
		      \[
			      \lnot \exists \obj{w} \paren{x \LiesOn \obj{w} \land y \LiesOn \obj{w} \land z \LiesOn \obj{w}}
		      \]
		      の略記,
		      \(\apply{\varphi}{x, y, z, w}\)を
		      \[
			      \apply{C}{x, y, z}
			      \land \apply{C}{x, y, w}
			      \land \apply{C}{x, z, w}
			      \land \apply{C}{y, z, w}
		      \]
		      の略記,
		      \(\apply{n}{a, b, c, d}\)を
		      \[
			      \lnot \paren{a \objeq b} \land \lnot \paren{a \objeq c} \land \lnot \paren{a \objeq d} \land \lnot \paren{b \objeq c}
			      \land \lnot \paren{c \objeq d}
		      \]
		      の略記としたときの
		      \[
			      \exists \obj{a} \exists \obj{b} \exists \obj{c} \exists \obj{d}
			      \paren{
				      \apply{n}{\obj{a}, \obj{b}, \obj{c}, \obj{d}}
				      \land \apply{\varphi}{\obj{a}, \obj{b}, \obj{c}, \obj{d}}
			      }.
		      \]
	\end{enumerate}
	ただし,\(\obj{P}, \obj{L}\)は\(\apply{\obj{P}}{x}, \apply{\obj{L}}{x}\)でそれぞれ「\(x\)は点である」ことと「\(x\)は直線である」
	ことを表現する意図で導入した記号であり,
	\(\LiesOn\)は\(x \LiesOn y\)で「\(x\)は\(y\)上にある」ことを表現する意図で導入した記号である.
	すると,上記の各文は順に
	\begin{enumerate}
		\item 点と直線は異なる対象である.
		\item \(x\)が\(y\)上にあるならば,\(x\)は点であり,\(y\)は直線である.
		\item 任意の相異なる2点を通る直線がただ1つ存在する.
		\item 相異なる2直線はつねに1点で交わる.
		\item どの3つも共線でないような相異なる4点が存在する.
	\end{enumerate}
	を表現することを意図した論理式であると解釈できる.
	上記\(\apply{C}{x, y, z}\)は「3点\(x, y, z\)は共線ではない」という主張に対応する論理式であると考えることができる.
\end{Ex}

\begin{Note}
	\cref{Ex:ProjectiveGeometry}では\(C, \varphi\)や\(n\)のようなもとの言語になかった記号が理論を記述するために使用されている.
	しかし,ここでの\(\apply{C}{x, y, y}\)や\(\apply{\varphi}{x, y, z, w}\)および\(\apply{n}{a, b, c, d}\)は単に既知の論理式の略記表現であり,
	それぞれのアリティ3, 4, 4の新しい関係記号が導入されたわけではないことに注意しよう.
	\(C, \varphi, n\)を使った論理式は,単に\(\apply{C}{x, y, z}, \apply{\varphi}{x, y, z, w}, \apply{n}{a, b, c, d}\)をその定義に置き換えることによって
	いつでも\(C, n\)があらわれない形に書き直すことができる.

	なお,それぞれの記号を言語に導入したと考えても本質的には変わらない.
	詳細は\cref{sec:extensionbydefinition}で述べる.
\end{Note}

\begin{Note}
	ここで導入した体系\(\ProjectiveGeometryPart\)は射影平面幾何の部分体系であり,射影平面幾何そのものではない.
	例えばこの体系\(\ProjectiveGeometryPart\)では,射影平面幾何の定理として有名なPappusの定理は証明できないことが知られている.

	また,平面幾何においては「点」と「直線」を区別して扱いたいので,対応する一階述語論理の理論においてそれを陽に提示する場合には
	\(\obj{P}\)や\(\obj{L}\)のような関係記号が必要となる.
	そのため,射影空間幾何の理論をこのやり方で作ろうとした場合は「平面」に対応する新しい記号が必要となる.
	すなわち,次元を1つ増やすごとにアリティ1の関係記号が1つ増えることになる.
\end{Note}


\begin{Que} \label[Que]{Que:ProjectiveGeometrydual}
	\NewDocumentCommand{\LiesOn}{}{\mathrel{\obj{\varepsilon}}}
	\Cref{Ex:ProjectiveGeometry}における言語\(\symcal{L}_{\ProjectiveGeometryPart}\)を考える.
	\(\sigma\)を\(\symcal{L}_{\ProjectiveGeometryPart}\)論理式とするとき,\(\sigma\)に対して以下の操作を施して得られる
	\(\symcal{L}_{\ProjectiveGeometryPart}\)論理式を\(\tilde{\sigma}\)とする%
	\footnote{%
		論理式における代入と同じように,この操作は帰納的定義によって精密化できる.%
	}%
	:
	\begin{enumerate}
		\item \(\sigma\)にあらわれる\(\obj{P}, \obj{L}\)をすべて入れ替える.
		\item \(\sigma\)にあらわれる\(x \LiesOn y\)という形の部分をすべて\(y \LiesOn x\)に変更する.
	\end{enumerate}
	この\(\tilde{\sigma}\)を\(\sigma\)の双対と呼ぶ.

	このとき,\cref{Ex:ProjectiveGeometry}で述べた体系\(\ProjectiveGeometryPart\)の各公理の双対を求めよ.
\end{Que}
\chapter{1階述語論理の意味論}
\label[chapter]{chap:semantics}

本章では,1階述語論理の意味論について述べる.
\Cref{chap:formulize}から繰り返し「数理論理学で扱うのは形式的記号列である」と
繰り返し述べてきたが,
それと同時にそれらの記号列は通常の数学において何らかの「意味」を見出すことも期待していたのであった.
\Cref{chap:syntax}で定義した項や論理式に対して数学的な「意味」を付与するというのが本章での主題である.
このことは,形式的記号列の世界と通常の数学の世界との橋渡しを行っていると考えることもできる.
理論のモデルや同型の概念に覚えがある読者は多いだろう.
普段使っている概念を一歩引いた形で見直すことにより得られるものも多いはずである.

\newpage

\section{理論とそのモデル} \label[section]{sec:model}

我々は,項や論理式を純粋な形式的記号列として導入し,
数学理論の形式化を試みた.
例として群論や順序の理論を取り扱ったが,通常の数学においては
これらは集合や写像の言葉で書かれることがほとんどである.
両者の間の関係を考えよう.

\begin{Def} \label[Def]{Def:structure}
	\(\symcal{L}\)を言語とする.このとき,空でない集合\(M\)と写像
	\(F \colon L \to M \cup \bigcup_{n > 0} \paren{\powerset{M^n} \cup M^{M^n}}\)の%
	\footnote{%
		集合\(X\)が与えられたとき,\(X\)の部分集合全体からなる集合を\(X\)の
		\index[sidx]{\(\powerset{X}\):べき集合}%
		\index[widx]{べきしゅうごう@べき集合}%
		\term{べき集合}%
		といい,\(\powerset{X}\)と表す.
	}%
	対\(\symcal{M} = \pair{M, F}\)が\(\symcal{L}\)%
	\index[widx]{こうぞう@\(\symcal{L}\)構造}%
	\term{構造}%
	であるとは,\(F\)が以下の条件をすべて満たすことをいう:
	\begin{enumerate}
		\item \(c \in L\)が定数記号ならば\(\apply{F}{c} \in M\),すなわち\(\apply{F}{c}\)は\(M\)の元である.
		\item \(f \in L\)がアリティ\(n\)の関数記号ならば\(\apply{F}{f} \in M^{M^n}\),すなわち\(\apply{F}{f}\)は\(M^n\)から\(M\)への写像である.
		\item \(r \in L\)がアリティ\(n\)の関係記号ならば\(\apply{F}{r} \in \powerset{M^n}\),すなわち\(\apply{F}{r}\)は\(M\)上の\(n\)項関係である.
	\end{enumerate}
	また\(\xi \in L\)の\(F\)による像\(\apply{F}{\xi}\)を
	\index[sidx]{\(\interpretation{\symcal{M}}{\xi}\):解釈}
	\begin{equation}
		\interpretation{\symcal{M}}{\xi}
		\label{eq:interpretation}
	\end{equation}
	と表記し,\(\xi\)の\(\symcal{L}\)構造\(\symcal{M}\)による%
	\index[widx]{かいしゃく@解釈}%
	\term{解釈}%
	と呼ぶことが多い.

	\(\symcal{M} = \pair{M, F}\)が\(\symcal{L}\)構造のとき,\(M\)を\(\symcal{M}\)の%
	\index[widx]{たいしょうりょういき@対象領域}%
	\term{対象領域}%
	と呼ぶ.
\end{Def}

\Cref{Def:structure}は,形式的記号列として与えられる言語\(L\)の元が
通常の数学においてはどういうものに相当するかを定義するものである.
項や論理式についても同様の定義をしたいのだが,まずその準備として言語の拡張を定義しておく.

\begin{Def} \label{Def:namelanguage}
	\(\symcal{L}\)を言語とし,\(\symcal{M} = \pair{M, F}\)を\(\symcal{L}\)構造とする.
	このとき,集合\(M\)の元\(a\)ごとに新たな定数記号\(c_a\)を用意し,
	これを\(\symcal{L}\)に付け加えた言語\(\symcal{L} \cup \Set{c_a | a \in M}\)を考えることができる.
	この言語を%
	\index[sidx]{\(\languagewithname{\symcal{L}}{\symcal{M}}\):構造による言語の拡張}%
	\begin{equation}
		\languagewithname{\symcal{L}}{\symcal{M}}
		\label{eq:languagewithname}
	\end{equation}
	と表すことにする.各\(a \in M\)に対する\(c_a\)を\(a\)の%
	\index[sidx]{\(c_a\):\(a\)の名前}%
	\index[widx]{なまえ@名前}%
	\term{名前}という.
\end{Def}

\begin{Note}
	\(\symcal{L}\)を言語とする.
	\(\symcal{L}\)構造\(\symcal{M}\)は,の元のうち
	\(a \in M\)の名前\(c_a\)の\(\symcal{M}\)による解釈を
	\begin{equation}
		\interpretation{\symcal{M}}{c_a} = a
		\label{eq:languagewithnameinterpretation}
	\end{equation}
	と定めることにより,自然に\(\languagewithname{\symcal{L}}{\symcal{M}}\)構造とみなせる.
	以後,\(\symcal{L}\)構造\(\symcal{M}\)はこの解釈によって
	\(\languagewithname{\symcal{L}}{\symcal{M}}\)構造でもあるとみなす.
\end{Note}

まずは項に対する解釈を定義しよう.

\begin{Def} \label{Def:interpretationforterm}
	\(\symcal{L}\)を言語とし,\(\symcal{M}\)を\(\symcal{L}\)構造とする.
	\(t\)を\(\languagewithname{\symcal{L}}{\symcal{M}}\)閉項として,\(t\)の\(\symcal{M}\)による解釈\(\interpretation{\symcal{M}}{t}\)
	を以下のように帰納的に定義する:
	\begin{enumerate}
		\item \(t\)が言語\(\languagewithname{\symcal{L}}{\symcal{M}}\)における定数記号\(c\)であるならば
		      \(\interpretation{\symcal{M}}{t} = \interpretation{\symcal{M}}{c}\)と定める.
		\item \(t\)がアリティ\(n\)の関数記号\(f\)と\(\languagewithname{\symcal{L}}{\symcal{M}}\)閉項
		      \(t_1, t_2, \dots, t_n\)を用いて\(\apply{f}{t_1, t_2, \dots, t_n}\)と表されるならば
		      \begin{equation}
			      \interpretation{\symcal{M}}{t}
			      = \apply{\interpretation{\symcal{M}}{f}}{
				      \interpretation{\symcal{M}}{t_1},
				      \interpretation{\symcal{M}}{t_2},
				      \dots,
				      \interpretation{\symcal{M}}{t_n}
			      }
			      \label{eq:terminterpretation}
		      \end{equation}
		      と定める.
	\end{enumerate}
\end{Def}

論理式においても\cref{Def:structure}や\cref{Def:interpretationforterm}
と同じような定義をしたい.しかし,与えられた集合の元や写像,関係として定式化できる
言語や項とは違い,論理式に対応する通常の数学における概念は「数学的な主張」である.
これはすでにあいまいさなく定式化されているとはいいがたいので,代わりに充足関係と呼ばれる関係を定義する.

\begin{Def} \label{Def:semanticimplies}
	\(\symcal{L}\)を言語とし,\(\symcal{M} = \pair{M, F}\)を\(\symcal{L}\)構造とする.
	このとき,\(\languagewithname{\symcal{L}}{\symcal{M}}\)閉論理式\(\varphi\)に対する
	\index[sidx]{\(\symcal{M} \satisfy \varphi\):充足関係}
	\begin{equation}
		\symcal{M} \satisfy \varphi
		\label{eq:structuresatisfy}
	\end{equation}
	を%
	\footnote{%
		この記号「\(\satisfy\)」は「ダブルターンスタイル記号」だとか「ゲタ記号」だとか呼ばれているようである.
	}%
	,以下のように帰納的に定義する:
	\begin{enumerate}
		\item \(t_1, t_2\)を\(\languagewithname{\symcal{L}}{\symcal{M}}\)閉項とするとき,
		      \[
			      \symcal{M} \satisfy t_1 \objeq t_2                                       \metaequivalent \interpretation{\symcal{M}}{t_1} = \interpretation{\symcal{M}}{t_2}
		      \]
		      とする.
		\item \(r \in \symcal{L}\)をアリティ\(n\)の関係記号,
		      \(t_1, t_2, \dots, t_n\)を\(\languagewithname{\symcal{L}}{\symcal{M}}\)閉項とするとき,
		      \(\symcal{M} \satisfy \apply{r}{t_1, t_2, \dots, t_n}
		      \metaequivalent \pair{
			      \interpretation{\symcal{M}}{t_1},
			      \interpretation{\symcal{M}}{t_2},
			      \dots,
			      \interpretation{\symcal{M}}{t_n}
		      } \in \interpretation{\symcal{M}}{r}
		      \)
		      とする.
		\item \(\varphi, \psi\)を\(\languagewithname{\symcal{L}}{\symcal{M}}\)閉論理式とするとき,
		      \[
			      \symcal{M} \satisfy \lnot \varphi      \metaequivalent \text{\(\symcal{M} \satisfy \varphi\)でない}
		      \]
		      とする.
		\item \(\varphi, \psi\)を\(\languagewithname{\symcal{L}}{\symcal{M}}\)閉論理式とするとき,
		      \[
			      \symcal{M} \satisfy \varphi \lor \psi  \metaequivalent \text{\(\symcal{M} \satisfy \varphi\)または\(\symcal{M} \satisfy \psi\)}
		      \]
		      とする.
		\item \(\varphi, \psi\)を\(\languagewithname{\symcal{L}}{\symcal{M}}\)閉論理式とするとき,
		      \[
			      \symcal{M} \satisfy \varphi \land \psi \metaequivalent \text{\(\symcal{M} \satisfy \varphi\)かつ\(\symcal{M} \satisfy \psi\)}
		      \]
		      とする.
		\item \(\varphi, \psi\)を\(\languagewithname{\symcal{L}}{\symcal{M}}\)閉論理式とするとき,
		      \[
			      \symcal{M} \satisfy \varphi \to \psi \metaequivalent \text{\(\symcal{M} \satisfy \lnot \varphi\)または\(\symcal{M} \satisfy \psi\)}
		      \]
		      とする%
		      \footnote{%
			      ここでは\(\symcal{M} \satisfy \varphi\)ならば\(\symcal{M} \satisfy \psi\)と書きたくなるところだが,あえてそうしていない.
			      これは\(\symcal{M} \satisfy \varphi\)でないばあいに\(\symcal{M} \satisfy \varphi \to \psi\)といえるかどうかに
			      若干のあいまいさが残ってしまうことを避けるためである.
			      通常の数学では,ある主張\(A\)が成り立たない場合は「\(A\)ならば\(B\)」は成り立つものとして考えるためそれでも問題ないと思われるが,
			      ここではあいまいさを極力排することを意図してこのような記述とした.
			      言い換えれば『\(A\)が成り立たない場合は「\(A\)ならば\(B\)」は成り立つと決めた』ということとなる.
		      }%
		      .
		\item \(\varphi\)を\(\languagewithname{\symcal{L}}{\symcal{M}}\)論理式,\(x\)を変数記号とするとき,
		      \(\exists x \varphi\)が\(\languagewithname{\symcal{L}}{\symcal{M}}\)閉論理式(つまり\(\varphi\)に自由出現する変数記号が\(x\)以外にない)であれば
		      \[
			      \symcal{M} \satisfy \exists x \varphi  \metaequivalent \text{\(\symcal{M} \satisfy \subst{\varphi}{c_a/x}\)となる\(a \in M\)が存在する}
		      \]
		      とする.ただし,\(c_a\)は\(a \in M\)の名前である.
		\item \(\varphi\)を\(\languagewithname{\symcal{L}}{\symcal{M}}\)論理式,\(x\)を変数記号とするとき,
		      \(\forall x \varphi\)が\(\languagewithname{\symcal{L}}{\symcal{M}}\)閉論理式(つまり\(\varphi\)に自由出現する変数記号が\(x\)以外にない)であれば
		      \[
			      \symcal{M} \satisfy \forall x \varphi  \metaequivalent \text{任意の\(a \in M\)に対して\(\symcal{M} \satisfy \subst{\varphi}{c_a/x}\)となる}
		      \]
		      とする.ただし,\(c_a\)は\(a \in M\)の名前である.
	\end{enumerate}

	\(\symcal{L}\)構造\(\symcal{M}\)と\(\symcal{L}\)論理式\(\varphi\)について,
	\(\symcal{M} \satisfy \varphi\)であるとき,
	\(\symcal{M}\)は\(\varphi\)を%
	\index[widx]{じゅうそく@充足}%
	\term{充足}するという.
	論理式\(\varphi\)が与えられたとき,\(\symcal{M} \satisfy \varphi\)となる\(\symcal{L}\)構造\(\symcal{M}\)が存在するならば,
	\(\varphi\)は%
	\index[widx]{じゅうそく@充足!じゅうそくかのう@---可能}%
	\term{充足可能}であるという.

\end{Def}

\begin{Note}
	\(a \in M\)の名前\(c_a\)は\(\apply{\Var}{c_a} = \emptyset\)を満たすので,
	任意の論理式中の任意の変数記号に代入可能である.
\end{Note}

\begin{Ex} \label[Ex]{Ex:satisfy}
	\(\symcal{L}\)を\cref{Ex:grouptheory}で述べた群論の言語とする.
	整数全体の集合\(\Integers\)に対し,\(\Integers\)を対象領域とする\(\symcal{L}\)構造\(\symcal{M}\)を以下によって定義する:
	\begin{enumerate}
		\item \(\interpretation{\symcal{M}}{\obj{e}} = 0,\)
		\item \(\interpretation{\symcal{M}}{\obj{\ast}} = \mathord{+},\)
		\item \(\interpretation{\symcal{M}}{\obj{{}^{-1}}} = \mathord{-}.\)
	\end{enumerate}
	ここで,3つ目の式における右辺の「\(\mathord{-}\)」は\(x \mapsto -x\)で定義される写像である.

	この\(\symcal{L}\)構造\(\symcal{M}\)は\cref{Ex:grouptheory}で挙げた3つの\(\symcal{L}\)文をすべて充足する.
	一方で,\(\symcal{M}\)による解釈のうち\(\symcal{M}\)による\(\obj{e}\)の解釈を\(\interpretation{\symcal{M}}{\obj{e}} = 1\)
	と置き換えて得られる\(\symcal{L}\)構造\(\symcal{M}'\)は,
	\cref{Ex:grouptheory}で挙げた3つの\(\symcal{L}\)文のうち1つ目のみを充足する.
\end{Ex}

\begin{Que} \label{Que:satisfy}
	\Cref{Ex:satisfy}で述べた事実が正しいことを確かめよ.
\end{Que}

\begin{Def} \label{Def:model}
	\(\symcal{L}\)を言語,\(T\)を\(\symcal{L}\)理論とする.
	このとき,\(\symcal{L}\)構造\(\symcal{M}\)が\(T\)の%
	\index[widx]{もでる@モデル}%
	\term{モデル}であるとは,任意の\(\varphi \in T\)に対して
	\(\symcal{M} \satisfy \varphi\)
	が成り立つことをいう.\(\symcal{M}\)が\(T\)のモデルであることを%
	\index[sidx]{\(\symcal{M} \satisfy T\):モデル}%
	\begin{equation}
		\symcal{M} \satisfy T
		\label{eq:model}
	\end{equation}
	と表す.
	また,\(T\)がモデルをもつとき,\(T\)は%
	\index[widx]{じゅうそく@充足!じゅうそくかのう@---可能}%
	\term{充足可能}であるという.

	また,\(\varphi\)を\(\symcal{L}\)閉論理式とするとき,\(\symcal{L}\)理論\(T\)のすべてのモデルが
	\(\varphi\)を充足するならば,\(\varphi\)は\(T\)の%
	\index[widx]{ろんりてききけつ@論理的帰結}%
	\term{論理的帰結}である,または\(\varphi\)は\(T\)における%
	\index[widx]{ていり@定理}%
	\term{定理}であるといい,
	\begin{equation}
		T \satisfy \varphi
		\label{eq:logicalconsequence}
	\end{equation}
	と表す.特に,\(T = \emptyset\)のときには
	\begin{equation}
		\satisfy \varphi
		\label{eq:nulllogicalconsequence}
	\end{equation}
	と表記する.
\end{Def}

\begin{Ex} \label[Ex]{Ex:model}
	\Cref{Ex:grouptheory}で述べた3つの\(\symcal{L}\)文からなる\(\symcal{L}\)理論を\(T\)とする.
	このとき,\cref{Ex:satisfy}で述べた\(\symcal{L}\)構造\(\symcal{M}\)は\(T\)のモデルである.
	一方で,\cref{Ex:satisfy}で述べた\(\symcal{L}\)構造\(\symcal{M}'\)は\(T\)のモデルではない.
\end{Ex}

\begin{Note}
	通常の通学では,\cref{Ex:model}における「\(\symcal{M}\)が\(T\)のモデルである」
	ことを「\(\symcal{M}\)が群である」と表現している.
	通常の数学では理論そのものと構造がその理論のモデルであるという主張を区別する必要性は薄いのだが,
	数理論理学の文脈では当然厳格に区別する必要がある.
\end{Note}

\begin{Que} \label{Que:model}
	\Cref{Ex:grouptheory}で挙げた群の理論\(T\)について,次の論理式を\(\varphi\)とする:
	\[
		\forall \obj{x} \forall \obj{y} \paren{\obj{x} \obj{\ast} \obj{y} = \obj{y} \obj{\ast} \obj{x}}.
	\]
	このとき,\(\varphi\)を充足する\(T\)のモデルと\(\varphi\)を充足しない\(T\)のモデルを1つずつ挙げよ.
\end{Que}

\begin{Que} \label{Que:grouptheoryaxiomize}
	\Cref{Que:invalidtheory}に関連して,アリティ2の関数記号1つだけからなる
	言語\(\symcal{L}_1 = \Set{\obj{\ast}}\)に関する以下の
	3つの論理式からなる\(\symcal{L}_1\)理論を\(T_1\)とする:
	\begin{align*}
		\varphi_1 \colon & \forall \obj{x} \forall \obj{y} \forall \obj{z}
		\paren{\paren{\obj{x} \obj{\ast} \obj{y}} \obj{\ast} \obj{z} \objeq \obj{x} \obj{\ast} \paren{\obj{y} \obj{\ast} \obj{z}}}, \\
		\varphi_2 \colon & \exists \obj{e'} \forall \obj{x} \paren{\obj{e'} \obj{\ast} \obj{x} \objeq \obj{x}},                     \\
		\varphi_3 \colon & \exists \obj{e'}\forall \obj{x} \exists \obj{y} \paren{\obj{y} \obj{\ast} \obj{x} \objeq \obj{e'}}.
	\end{align*}
	このとき,\cref{Ex:grouptheory}で挙げた群の言語\(\symcal{L}\)に対して\(\symcal{L}_1 \subset \symcal{L}\)が成り立つので,
	\(T_1\)は自然に\(\symcal{L}\)理論ともみなせる.
	さて,\(\symcal{L}_1\)理論\(T_1\)のモデルだが,\(\symcal{L}_1\)が有していない記号\(\obj{e}, {}^{\obj{-1}}\)
	の解釈をどのように定めても\cref{Ex:grouptheory}で挙げた
	群の理論\(T\)のモデルとみなすことはできない\(\symcal{L}_1\)構造の例を挙げよ.

	上記のような例が存在することは,\(\symcal{L}_1\)に記号を追加して
	言語を拡張しても,\(T_1\)から\(T\)を自然に得ることはできないことを意味している.
\end{Que}



\section{構造の同型} \label{sec:isomorphic}

\Cref{sec:model}では,構造やモデルについて基本的な定義を行った.
\Cref{sec:model}は理論と構造との間の関係に着目した議論であったが,ここでは構造同士の関係について考えよう.

\begin{Def} \label{Def:isomorphic}
	\(\symcal{L}\)を言語とし,\(\symcal{M}, \symcal{N}\)を対象領域が
	それぞれ\(M, N\)であるような\(\symcal{L}\)構造とする.
	このとき,写像\(\sigma \colon M \to N\)が\(\symcal{M}\)から\(\symcal{N}\)への%
	\index[widx]{どうけい@同型!じゅうどうけい@準---写像}%
	\term{準同型写像}であるとは,
	\(\sigma\)が以下の条件を満たすことをいう:
	\begin{enumerate}
		\item すべての定数記号\(c \in L\)に対して
		      \begin{equation}
			      \apply{\sigma}{\interpretation{\symcal{M}}{c}} = \interpretation{\symcal{N}}{c}
			      \label{eq:constanthomorphism}
		      \end{equation}
		      となる.
		\item アリティ\(n\)の任意の関数記号\(f\)と任意の\(a_1, a_2, \dots, a_n \in M\)に対して
		      \begin{equation}
			      \apply{\sigma}{\apply{\interpretation{\symcal{M}}{f}}{a_1, a_2, \dots, a_n}}
			      = \apply{\interpretation{\symcal{N}}{f}}{
				      \apply{\sigma}{a_1}, \apply{\sigma}{a_2}, \dots, \apply{\sigma}{a_n}
			      }
			      \label{eq:functionhomorphism}
		      \end{equation}
		      が成り立つ.
		\item アリティ\(n\)の任意の関係記号\(r\)と任意の\(a_1, a_2, \dots, a_n\)に対して
		      \begin{equation}
			      \pair{a_1, a_2, \dots, a_n} \in \interpretation{\symcal{M}}{r}
			      \metaimplies \pair{\apply{\sigma}{a_1}, \apply{\sigma}{a_2}, \dots, \apply{\sigma}{a_n}} \in \interpretation{\symcal{N}}{r}
			      \label{eq:relationhomorphism}
		      \end{equation}
		      が成り立つ.
	\end{enumerate}

	また,準同型写像\(\sigma \colon M \to N\)が\(\symcal{M}\)から\(\symcal{N}\)への%
	\index[widx]{うめこみ@埋め込み}%
	\term{埋め込み}であるとは,\(\sigma\)が写像として単射であって,さらに
	\begin{equation}
		\pair{a_1, a_2, \dots, a_n} \in \interpretation{\symcal{M}}{r}
		\metaequivalent \pair{\apply{\sigma}{a_1}, \apply{\sigma}{a_2}, \dots, \apply{\sigma}{a_n}} \in \interpretation{\symcal{N}}{r}
		\label{eq:relationembedding}
	\end{equation}
	が成り立つことをいう.

	埋め込み\(\sigma \colon M \to N\)が写像として全単射である場合,\(\sigma\)を\(\symcal{M}\)から\(\symcal{N}\)への
	\index[widx]{どうけい@同型!どうけいしゃぞう@---写像}%
	\term{同型写像}と呼ぶ.
	2つの構造\(\symcal{M}, \symcal{N}\)について,\(\symcal{M}\)から\(\symcal{N}\)への同型写像が存在する場合,
	\(\symcal{M}\)と\(\symcal{N}\)は%
	\index[widx]{どうけい@同型}%
	\term{同型}であるといい,%
	\index[sidx]{\(\symcal{M} \isomorphic \symcal{N}\):同型}%
	\begin{equation}
		\symcal{M} \isomorphic \symcal{N}
		\label{eq:isomorphic}
	\end{equation}
	と表す.
\end{Def}

簡単な議論によって,以下の補題が成り立つことがわかる.

\begin{Lemma} \label[Lemma]{Lemma:equivrelationforisomorphic}
	\(\symcal{L}\)を言語とし,\(\symcal{M_1}, \symcal{M_2}, \symcal{M_3}\)を\(\symcal{L}\)構造とする.
	このとき
	\begin{enumerate}
		\item \(\symcal{M_1} \isomorphic \symcal{M_1}\)
		\item \(\symcal{M_1} \isomorphic \symcal{M_2} \metaimplies \symcal{M_2} \isomorphic \symcal{M_1}\)
		\item \(\text{\(\symcal{M_1} \isomorphic \symcal{M_2}\)かつ\(\symcal{M_2} \isomorphic \symcal{M_3}\)}
		      \metaimplies \symcal{M_1} \isomorphic \symcal{M_3}\)
	\end{enumerate}
	が成り立つ.
\end{Lemma}

\begin{proof}
	1は恒等写像が同型写像であることから,
	2は同型写像の逆写像もまた同型写像であることから,
	3は同型写像の合成もまた同型写像であることからそれぞれ従う.
\end{proof}

2つの\(\symcal{L}\)構造\(\symcal{M}, \symcal{N}\)が同型であることは,
定数記号,関数記号,関係記号のすべてが同型写像\(\sigma\)を通して「翻訳」できることを主張している.
通常の数学では,2つの構造が同型であることは「その理論の内部ではその2つの構造は同一視できる」
ことと解釈される.この認識が適切であることを以下で検証しよう.

まずは「その理論の内部ではその2つの構造は同一視できる」ことの定式化から行う.
この主張は「その2つの構造はまったく同じ性質を満たす」ここと解釈できる.
今までの議論の中で類似した概念として充足関係が登場している.
そこで,以下のように定義しよう.

\begin{Def} \label{Def:elementarilyequivalent}
	\(\symcal{L}\)を言語とし,\(\symcal{M}, \symcal{N}\)を\(\symcal{L}\)構造とする.
	このとき,
	任意の\(\symcal{L}\)閉論理式\(\varphi\)に対して
	\begin{equation}
		\symcal{M} \satisfy \varphi \metaequivalent \symcal{N} \satisfy \varphi
		\label{eq:elementarilyequivalentcondition}
	\end{equation}
	が成り立つとき,\(\symcal{M}\)と\(\symcal{N}\)は
	\index[widx]{しょとうどうち@初等同値}%
	\term{初等同値}であるといい,%
	\index[sidx]{\(\symcal{M} \elementarilyequivalent \symcal{N}\):初等同値}
	\begin{equation}
		\symcal{M} \elementarilyequivalent \symcal{N}
		\label{eq:elementarilyequivalent}
	\end{equation}
	と表す%
	\footnote{%
		オブジェクト側の等号と記号が被っているが,文脈上混乱することはまずないと考えられる.
	}%
	.
\end{Def}

以下の補題は\cref{Def:elementarilyequivalent}から容易に得られる.

\begin{Lemma} \label[Lemma]{Lemma:equivrelationforelementarilyequivalent}
	\(\symcal{L}\)を言語とし,\(\symcal{M_1}, \symcal{M_2}, \symcal{M_3}\)を\(\symcal{L}\)構造とする.
	このとき
	\begin{enumerate}
		\item \(\symcal{M_1} \elementarilyequivalent \symcal{M_1}\)
		\item \(\symcal{M_1} \elementarilyequivalent \symcal{M_2} \metaimplies \symcal{M_2} \elementarilyequivalent \symcal{M_1}\)
		\item \(\text{\(\symcal{M_1} \elementarilyequivalent \symcal{M_2}\)かつ\(\symcal{M_2} \elementarilyequivalent \symcal{M_3}\)}
		      \metaimplies \symcal{M_1} \elementarilyequivalent \symcal{M_3}\)
	\end{enumerate}

	が成り立つ.
\end{Lemma}

容易に予想できるように,同型である2つの構造は初等同値である.このことを検証しよう.

\begin{Lemma} \label[Lemma]{Lemma:isomorphicforterm}
	\(\symcal{L}\)を言語とし,\(\symcal{M}, \symcal{N}\)は対象領域がそれぞれ\(M, N\)であるような
	同型である\(\symcal{L}\)構造とし,\(\sigma \colon M \to N\)を同型写像とする.
	このとき,任意の\(\languagewithname{\symcal{L}}{\symcal{M}}\)閉項\(t\)に対して
	\begin{align}
		\apply{\sigma}{
			\interpretation{\symcal{M}}{t}
		}
		= \interpretation{\symcal{N}}{
			\subst{t}{\sigma}
		}
		\label{eq:isomorphicforterm}
	\end{align}
	が成り立つ.ここで,\(\subst{t}{\sigma}\)は\(t\)に出現する各\(a \in M\)の名前をすべて\(\apply{\sigma}{a} \in N\)
	の名前で置き換えて得られる\(\languagewithname{\symcal{L}}{\symcal{N}}\)閉項とする.
\end{Lemma}

\begin{proof}
	\(t\)の構成に関する帰納法によって示す.
	\(t\)が\(\symcal{L}\)における定数記号であるとき,\cref{eq:constanthomorphism}から\cref{eq:isomorphicforterm}が従う.
	\(t\)が\(M\)の元\(a\)の名前\(c_a\)であるとき,\cref{eq:isomorphicforterm}は
	\[
		\apply{\sigma}{\interpretation{\symcal{M}}{c_a}} = \interpretation{\symcal{N}}{c_{\apply{\sigma}{a}}}
	\]
	と書き換えられ,どちらも\(\apply{\sigma}{a}\)に等しいことから\cref{eq:isomorphicforterm}の成立がわかる.
	次に,\(t\)がアリティ\(n\)の関数記号\(f\)と,
	示すべき主張が成り立つ\(\languagewithname{\symcal{L}}{\symcal{M}}\)閉項\(t_1, t_2, \dots, t_n\)を用いて
	\[
		t = \apply{f}{t_1, t_2, \dots, t_n}
	\]
	と書ける場合を考える.このとき,
	\begin{align*}
		\apply{\sigma}{
			\interpretation{\symcal{M}}{t}
		}
		 & =
		\apply{\sigma}{
			\apply{\interpretation{\symcal{M}}{f}}{
				\interpretation{\symcal{M}}{t_1},
				\interpretation{\symcal{M}}{t_2},
				\dots,
				\interpretation{\symcal{M}}{t_n}
			}
		}    \\
		 & =
		\apply{\interpretation{\symcal{N}}{f}}{
			\apply{\sigma}{\interpretation{\symcal{M}}{t_1}},
			\apply{\sigma}{\interpretation{\symcal{M}}{t_2}},
			\dots,
			\apply{\sigma}{\interpretation{\symcal{M}}{t_n}}
		}    \\
		 & =
		\apply{\interpretation{\symcal{N}}{f}}{
			\interpretation{\symcal{N}}{\subst{t_1}{\sigma}},
			\interpretation{\symcal{N}}{\subst{t_2}{\sigma}},
			\dots,
			\interpretation{\symcal{N}}{\subst{t_n}{\sigma}}
		}    \\
		 & =
		\interpretation{\symcal{N}}{\paren{\apply{f}{
					\subst{t_1}{\sigma},
					\subst{t_2}{\sigma},
					\dots,
					\subst{t_n}{\sigma},
		}}}  \\
		 & =
		\interpretation{\symcal{N}}{\subst{t}{\sigma}}
	\end{align*}
	となるので\cref{eq:isomorphicforterm}が成り立つ.

	以上より,任意の\(\languagewithname{\symcal{L}}{\symcal{M}}\)閉項\(t\)に対してもとの主張が成り立つことがわかった.
\end{proof}

\begin{Thm} \label[Thm]{Thm:isomorphicforlogicalexpression}
	\(\symcal{L}\)を言語,\(\symcal{M}, \symcal{N}\)を\(\symcal{L}\)構造とする.
	このとき,\(\symcal{M}\)と\(\symcal{N}\)が同型であるならば,
	\(\symcal{M}\)と\(\symcal{N}\)は初等同値である.
\end{Thm}

\begin{proof}
	\(\symcal{M}, \symcal{N}\)の対象領域をそれぞれ\(M, N\)とし,
	\(\sigma \colon M \to N\)を同型写像とする.
	\Cref{Lemma:equivrelationforelementarilyequivalent}により,
	\(\symcal{M} \satisfy \varphi\)のとき\(\symcal{N} \satisfy \varphi\)となることを示せば十分.

	\(\varphi\)が\(\symcal{L}\)閉項\(t_1, t_2\)に対する\(t_1 \objeq t_2\)の形のとき,
	\(\symcal{M} \satisfy \varphi\)とすると\(\interpretation{\symcal{M}}{t_1} = \interpretation{\symcal{M}}{t_2}\)である.
	よって\cref{Lemma:isomorphicforterm}から任意の\(\symcal{L}\)閉項\(t\)に対して
	\[
		\apply{\sigma}{\interpretation{\symcal{M}}{t}}
		=
		\interpretation{\symcal{N}}{\subst{t}{\sigma}}
		=
		\interpretation{\symcal{N}}{t}
	\]
	なので,\(\apply{\sigma}{t_1} = \apply{\sigma}{t_2}\)から
	\(\interpretation{\symcal{N}}{t_1} = \interpretation{\symcal{N}}{t_2}\)が得られるため
	\(\symcal{N} \satisfy t_1 \objeq t_2\)となる.

	\(\varphi\)がアリティ\(n\)の関係記号\(n\)と\(\symcal{L}\)閉項\(t_1, t_2, \dots, t_n\)に対する
	\(\apply{r}{t_1, t_2, \dots, t_n}\)の形のとき,
	\(\symcal{M} \satisfy \varphi\)とすると
	\(\pair{
		\interpretation{\symcal{M}}{t_1},
		\interpretation{\symcal{M}}{t_2},
		\dots,
		\interpretation{\symcal{M}}{t_n}
	} \in \interpretation{\symcal{M}}{r}\)となる.
	よって,\cref{eq:functionhomorphism}と\cref{Lemma:isomorphicforterm}から
	\(\symcal{N} \satisfy \varphi\)が従う.

	次に,\(\varphi\)が定理の主張を満たす\(\symcal{L}\)閉論理式\(\psi\)に対する
	\(\lnot \psi\)の形の場合を考える.
	\(\symcal{M} \satisfy \varphi\)とすると\(\symcal{M} \satisfy \psi\)ではない.
	ここで,\(\symcal{N} \satisfy \psi\)と仮定すると,帰納法の仮定によって
	\(\symcal{M} \satisfy \psi\)となって矛盾する.
	よって\(\symcal{N} \satisfy \psi\)ではないので\(\symcal{N} \satisfy \varphi\)である.

	\(\varphi\)が定理の主張を満たす\(\symcal{L}\)閉論理式\(\varphi_1, \varphi_2\)に対する
	\(\varphi_1 \lor \varphi_2\)の形の場合を考える.
	\(\symcal{M} \satisfy \varphi\)だとすると\(\symcal{M} \satisfy \varphi_1\)または
	\(\symcal{M} \satisfy \varphi_2\)である.
	どちらの場合も帰納法の仮定によって
	「\(\symcal{N} \satisfy \varphi_1\)または\(\symcal{N} \satisfy \varphi_2\)」
	が成り立つので\(\symcal{N} \satisfy \varphi\)である.

	残りの場合も上と同様にして導ける(\cref{Def:semanticimplies}の全パターンを網羅すればよい).
\end{proof}

\begin{Ex} \label[Ex]{Ex:isomorphic}
	\Cref{Ex:grouptheory}で挙げた群の理論\(T\)の2つのモデル\(\symcal{M}, \symcal{N}\)が同型であることは,
	よく知られた群の同型に関する標準的な定義
	(\(M, N\)をそれぞれ\(\symcal{M}, \symcal{N}\)の対象領域とするとき,
	全単射\(f \colon M \to N\)で任意の\(x, y \in M\)に対して
	\(\apply{f}{x \interpretation{\symcal{M}}{\ast} y}
	= \apply{f}{x} \interpretation{\symcal{N}}{\ast} \apply{f}{y}\)が成り立つものが存在すること)と一致する.
	よって「群として同型」であるような2つの群は,\(\symcal{L}\)閉論理式で表現できる性質に差異はない.
	例えば「可換群である」という主張は\(\symcal{L}\)閉論理式として表現できる(\cref{Que:model}を参照)ので,
	同型な2つの群はどちらも可換であるか,どちらも可換でないかのいずれかである.
\end{Ex}
\chapter{1階述語論理の証明論}
\label[chapter]{chap:proof}

この章では,いよいよ「証明」という概念の形式化を試みる.
\Cref{chap:syntax}で定義された論理式は,数学における何らかの主張の
形式的記号列の世界での対応物であった.
数学における「証明」は,
形式的記号列の世界においてはある論理式から別の論理式を得る操作に対応する.
どのような操作を妥当なものとして認め,それをどのような記法で書き下すかによって
多種多様な体系が得られる.本書では,その中から自然演繹式シークエント計算と呼ばれる体系を取り上げる.

\newpage


\section{シークエント} \label[section]{sec:sequent}

自然演繹式シークエント計算において基本的なのは,シークエントと呼ばれる記号列である.

\begin{Def} \label[Def]{Def:sequent}
	0個以上の論理式の有限列\(\Gamma\)と論理式\(\varphi\)に対して,記号列%
	\index[sidx]{\(\Gamma \sequent \varphi\):シークエント}%
	\begin{equation}
		\Gamma \sequent \varphi
	\end{equation}
	を%
	\index[widx]{しーくえんと@シークエント}%
	\term{シークエント}という.
	このとき,\(\Gamma\)をこのシークエントの左辺,\(\varphi\)をこのシークエントの右辺という.
\end{Def}

\begin{Ex} \label[Ex]{Ex:sequent}
	論理式\(\varphi\)に対して,
	\begin{equation*}
		\varphi \sequent \varphi
	\end{equation*}
	という記号列はシークエントである.この形のシークエントを%
	\index[widx]{ししき@始式}%
	\term{始式}と呼ぶ.
	また,左辺に何もない
	\begin{equation*}
		\sequent \varphi, \\
	\end{equation*}
	といった記号列もシークエントである.
\end{Ex}

シークエント「\(\Gamma \sequent \varphi\)」は,
通常の数学における「\(\Gamma\)という仮定によって\(\varphi\)を証明することができる」という
主張の形式的記号列の世界での対応物であることを期待して導入されたものである.
左辺に何もないシークエント「\(\sequent \varphi\)」は,
何も仮定せずとも\(\varphi\)が証明できること,すなわち「\(\varphi\)は証明できる」ことの
形式的記号列の世界での対応物であることが期待される.
以下で行われるのは,このシークエントを操作していくルールをうまく定義して
「証明っぽいもの」を作り上げていくことである.


\section{公理と推論規則} \label[section]{sec:axiomandrule}

通常の数学においては,議論の出発点となる主張,すなわち公理を用意し,
そこから演繹的推論を重ねていくことによってさまざまな結果を得る.
自然演繹式シークエント計算においては,
出発点となるシークエントから特定の操作を行うことによって
さまざまな結果を得ることになる.
このとき,出発点となるシークエントのことを%
\index[widx]{こうり@公理}%
\term{公理}と呼ぶ%
\footnote{%
	ここでいう「公理」は自然演繹式シークエント計算という証明体系そのものの出発点となるシークエントであり,
	個々の理論の基礎となる主張のことではない.通常の数学における「公理」に対応するのは
	閉論理式の集合である.
}%
.
また,
あるシークエントから別のシークエントを得る操作をいくつか列挙し,
それを使って議論を進めていく.
この時に列挙した操作のことを%
\index[widx]{すいろんきそく@推論規則}%
\term{推論規則}
と呼ぶ.
以下に自然演繹式シークエント計算で用いる公理と推論規則の一覧を述べる.

公理や推論規則は,シークエントを横や縦に並べた
\begin{prooftree}
	\AxiomC{\(\Gamma \sequent \varphi \)}
	\AxiomC{\(\Delta \sequent \psi\)}
	\LeftLabel{(name)}
	\BinaryInfC{\(\Sigma \sequent \xi\)}
\end{prooftree}
のような形式で記述される.「(name)」は規則名である.
この記述は「上側に記述されているシークエントがすべて得られたら下段のシークエントを得てよい」
という意味ととらえてよい.
上段に何も書かれていない場合もあり,それが公理である.

\begin{Def}[公理] \label[Def]{Def:axiom}
	論理式\(\varphi\)と項\(t\)に対して,以下は自然演繹式シークエント計算における公理である:
	\begin{multicols}{2}
		\begin{prooftree}
			\AxiomC{}
			\LeftLabel{(ID)}
			\UnaryInfC{\(\varphi \sequent \varphi\)}
		\end{prooftree}
		\columnbreak
		\begin{prooftree}
			\AxiomC{}
			\LeftLabel{(REFL)}
			\UnaryInfC{\(t = t\)}
		\end{prooftree}
	\end{multicols}
\end{Def}

推論規則名になっている「ID」と「REFL」は同一律(law of identity)と反射律(reflexive law)に由来する.

推論規則の数はそれなりに多いので,いくつかのグループに分けて述べる.
なお,\cref{Def:structuralrule}における推論規則名「w」「e」「c」はそれぞれ
弱化(weakening),交換(exchange),縮約(contraction)に由来する.

\begin{Def}[構造規則] \label[Def]{Def:structuralrule}
	論理式の有限列\(\Gamma, \Delta\)と論理式\(\varphi, \psi, \xi\)に対し,
	下記の3つは自然演繹式シークエント計算における推論規則である.
	\begin{multicols}{3}
		\begin{prooftree}
			\AxiomC{\(\Gamma \sequent \xi\)}
			\LeftLabel{(w)}
			\UnaryInfC{\(\varphi, \Gamma \sequent \xi\)}
		\end{prooftree}
		\columnbreak
		\begin{prooftree}
			\AxiomC{\(\Gamma, \psi, \varphi, \Delta \sequent \xi\)}
			\LeftLabel{(e)}
			\UnaryInfC{\(\Gamma, \varphi, \psi, \Delta \sequent \xi\)}
		\end{prooftree}
		\columnbreak
		\begin{prooftree}
			\AxiomC{\(\varphi, \varphi, \Gamma \sequent \xi\)}
			\LeftLabel{(c)}
			\UnaryInfC{\(\varphi, \Gamma \sequent \xi\)}
		\end{prooftree}
	\end{multicols}
\end{Def}

論理記号に関する推論規則を述べる.
推論規則名の「I」や「E」はそれぞれ
導入(Introduction)と除去(Elimination)に由来する.
例えば,「\(\land\)I」は右辺に\(\land\)記号を新しく導入する推論規則であり,
「\(\land\)E」は右辺から\(\land\)記号を除去するような推論規則である.
すべての推論規則が導入と除去の対になっていることに着目しよう.
我々の素朴的直観において導入規則と除去規則はそれぞれ
「その論理記号を含む論理式に対応する主張を結論として得るためのルール」
「その論理記号を含む論理式に対応する主張が得られた場合に行える推論を表すルール」
に対応する.


\begin{Def}[論理規則その1] \label[Def]{Def:logicalrule1}
	論理式の有限列\(\Gamma, \Delta, \Sigma\)と論理式\(\varphi, \psi, \xi\)に対して,
	下記は自然演繹式シークエント計算における推論規則である.
	\begin{multicols}{2}
		\begin{prooftree}
			\AxiomC{\(\varphi, \Gamma \sequent \psi\)}
			\LeftLabel{(\(\to\)I)}
			\UnaryInfC{\(\Gamma \sequent \varphi \to \psi\)}
		\end{prooftree}
		\columnbreak
		\begin{prooftree}
			\AxiomC{\(\Gamma \sequent \varphi\)}
			\AxiomC{\(\Delta \sequent \varphi \to \psi\)}
			\LeftLabel{(\(\to\)E)}
			\BinaryInfC{\(\Gamma, \Delta \sequent \psi\)}
		\end{prooftree}
	\end{multicols}
	\begin{multicols}{2}
		\begin{prooftree}
			\AxiomC{\(\Gamma \sequent \varphi\)}
			\AxiomC{\(\Delta \sequent \psi\)}
			\LeftLabel{(\(\land\)I)}
			\BinaryInfC{\(\Gamma, \Delta \sequent \varphi \land \psi\)}
		\end{prooftree}
		\columnbreak
		\begin{prooftree}
			\AxiomC{\(\Gamma \sequent \varphi_1 \land \varphi_2\)}
			\LeftLabel{(\(\land\)E)}
			\RightLabel{(\(i \equiv 1,2\))}
			\UnaryInfC{\(\Gamma \sequent \varphi_i\)}
		\end{prooftree}
	\end{multicols}
	\begin{prooftree}
		\AxiomC{\(\Gamma \sequent \varphi_i\)}
		\LeftLabel{(\(\lor\)I)}
		\RightLabel{(\(i \equiv 1,2\))}
		\UnaryInfC{\(\Gamma \sequent \varphi_1 \lor \varphi_2\)}
	\end{prooftree}
	\begin{prooftree}
		\AxiomC{\(\Gamma \sequent \varphi \lor \psi\)}
		\AxiomC{\(\varphi, \Delta \sequent \xi\)}
		\AxiomC{\(\psi, \Sigma \sequent \xi\)}
		\LeftLabel{(\(\lor\)E)}
		\TrinaryInfC{\(\Gamma, \Delta, \Sigma \sequent \xi\)}
	\end{prooftree}
\end{Def}

\begin{Def}[論理規則その2] \label[Def]{Def:quantiferrule}
	論理式の有限列\(\Gamma, \Delta\)と論理式\(\varphi, \psi\)および変数記号\(x, a\)に対して,
	下記の4つは自然演繹式シークエント計算における推論規則である.
	ここで,\(a\)は\(\varphi\)中の\(x\)に代入可能であるものとする.
	\begin{multicols}{2}
		\begin{prooftree}
			\AxiomC{\(\Gamma \sequent \subst{\varphi}{a/x}\)}
			\LeftLabel{(\(\forall\)I)}
			\UnaryInfC{\(\Gamma \sequent \forall x \varphi\)}
		\end{prooftree}
		ただし,\(a\)は\(\Gamma\)中の各論理式や\(\varphi\)に自由出現しない.
		\columnbreak
		\begin{prooftree}
			\AxiomC{\(\Gamma \sequent \forall x \varphi\)}
			\LeftLabel{(\(\forall\)E)}
			\UnaryInfC{\(\Gamma \sequent \subst{\varphi}{a/x}\)}
		\end{prooftree}
	\end{multicols}
	\begin{multicols}{2}
		\begin{prooftree}
			\AxiomC{\(\Gamma \sequent \subst{\varphi}{a}{x}\)}
			\LeftLabel{(\(\exists\)I)}
			\UnaryInfC{\(\Gamma \sequent \exists x \varphi\)}
		\end{prooftree}
		\columnbreak
		\begin{prooftree}
			\AxiomC{\(\Gamma \sequent \exists x \varphi\)}
			\AxiomC{\(\subst{\varphi}{a/x}, \Delta \sequent \psi\)}
			\LeftLabel{(\(\exists\)E)}
			\BinaryInfC{\(\Gamma, \Delta \sequent \psi\)}
		\end{prooftree}
		ただし,\(a\)は\(\Gamma, \Delta\)中の各論理式と\(\exists x \varphi, \psi\)のいずれにも自由出現しない.
	\end{multicols}
\end{Def}

\begin{Def}[論理規則その3] \label{Def:lnot}
	論理式の有限列\(\Gamma\)と論理式\(\varphi\)について,
	下記は自然演繹式シークエント計算における推論規則である.
	\begin{multicols}{2}
		\begin{prooftree}
			\AxiomC{\(\varphi, \Gamma \sequent \bot\)}
			\LeftLabel{(\(\lnot\)I)}
			\UnaryInfC{\(\Gamma \sequent \lnot \varphi\)}
		\end{prooftree}
		\columnbreak
		\begin{prooftree}
			\AxiomC{\(\Gamma \sequent \varphi\)}
			\AxiomC{\(\Gamma \sequent \lnot \varphi\)}
			\LeftLabel{(\(\lnot\)E)}
			\BinaryInfC{\(\Gamma \sequent \bot\)}
		\end{prooftree}
	\end{multicols}
\end{Def}

等号に関する推論規則も必要である.
ここで,規則名「SUBST」は代入(substitution)に由来する.

\begin{Def}[等号に関する推論規則] \label[Def]{Def:SUBST}
	\(\varphi\)を論理式,\(x\)を変数記号,\(s,t\)を\(\varphi\)中の\(x\)に代入可能な項とするとき,
	以下は自然演繹式シークエント計算における推論規則である.
	\begin{prooftree}
		\AxiomC{\(s = t\)}
		\AxiomC{\(\subst{\varphi}{t / x}\)}
		\LeftLabel{(SUBST)}
		\BinaryInfC{\(\subst{\varphi}{s / x}\)}
	\end{prooftree}
\end{Def}

最後に,いわゆる背理法の基礎となる推論規則を導入しておく.
規則名「DNE」は2重否定除去(Double Negation Elimination)に由来する.

\begin{Def}[2重否定除去] \label[type]{Def:DNE}
	論理式の有限列\(\Gamma\)と論理式\(\varphi\)に対し,
	下記は自然演繹式シークエント計算における推論規則である.
	\begin{prooftree}
		\AxiomC{\(\Gamma \sequent \lnot \lnot \varphi\)}
		\LeftLabel{(DNE)}
		\UnaryInfC{\(\Gamma \sequent \varphi\)}
	\end{prooftree}
\end{Def}

以上で公理と推論規則の準備が整った.あとは「証明」っぽく見えるようにいくつか定義を行うだけである.

\begin{Def} \label[Def]{Def:provable}
	シークエントが%
	\index[widx]{どうしゅつかのう@(シークエントが)導出可能}%
	導出可能であることを,以下のように定義する:
	\begin{enumerate}
		\item 任意の論理式\(\varphi\)に対し,始式\(\varphi \sequent \varphi\)は導出可能である.
		\item これまでに挙げた各推論規則における上段のシークエントがすべて導出可能であるならば,下段のシークエントも導出可能である.
		\item 以上の規則を有限回適用して得られるシークエントのみが導出可能である.
	\end{enumerate}
\end{Def}

あるシークエントが導出可能であることを
確かめる作業のことをそのシークエントの「導出」と呼ぶことがある.
「シークエント\(\Gamma \sequent \varphi\)を導出する」といった言い回しである.

さて,シークエントの導出可能性の次は論理式の証明可能性である.

\begin{Def} \label[Def]{Def:logicalexpressionprovable}
	\(\varphi\)を論理式とする.シークエント
	\begin{equation}
		\sequent \varphi
	\end{equation}
	が導出可能であるとき,\(\varphi\)は%
	\index[widx]{しょうめいかのう@(論理式が)証明可能}%
	\term{証明可能}であるという.
\end{Def}


ここで定義した体系が我々の素朴的直観における「証明」を模倣できているかどうかは,
ここから導かれるさまざまな結果を見るよりほかはない.
例を見てみよう.

\begin{Thm}[背理法] \label[Thm]{Thm:RAA}
	\(\Gamma\)を論理式の有限列,\(\varphi\)を論理式とするとき,
	シークエント
	\(\lnot \varphi, \Gamma \sequent \bot\)
	が導出可能であるならば,シークエント
	\(\Gamma \sequent \varphi\)
	は導出可能である.
\end{Thm}

\begin{proof}
	シークエント\(\Gamma \sequent \varphi\)は次のように得られる.
	\begin{prooftree}
		\AxiomC{\(\lnot \varphi, \Gamma \sequent \bot \)}
		\LeftLabel{(\(\lnot\)I)}
		\UnaryInfC{\(\Gamma \sequent \lnot \lnot \varphi\)}
		\LeftLabel{(DNF)}
		\UnaryInfC{\(\Gamma \sequent \varphi\)}
	\end{prooftree}
	よって,シークエント\(\Gamma \sequent \varphi\)は導出可能である.
\end{proof}

\begin{Thm} \label[Thm]{Thm:DNFsequent}
	\(\varphi\)を論理式とするとき,シークエント\(\lnot \lnot \varphi \sequent \varphi\)と
	\(\varphi \sequent \lnot \lnot \varphi\)はいずれも導出可能である.
\end{Thm}

\begin{proof}
	シークエント\(\lnot \lnot \varphi \sequent \varphi\)は次のようにして得られる.
	\begin{prooftree}
		\AxiomC{}
		\LeftLabel{(ID)}
		\UnaryInfC{\(\lnot \lnot \varphi \sequent \lnot \lnot \varphi\)}
		\LeftLabel{(DNF)}
		\UnaryInfC{\(\lnot \lnot \varphi \sequent \varphi\)}
	\end{prooftree}
	次に,シークエント\(\varphi \sequent \lnot \lnot \varphi\)を導出する.
	\begin{prooftree}
		\AxiomC{}
		\LeftLabel{(ID)}
		\UnaryInfC{\(\lnot \varphi \sequent \lnot \varphi\)}
		\AxiomC{}
		\LeftLabel{(ID)}
		\UnaryInfC{\(\varphi \sequent \varphi\)}
		\LeftLabel{(\(\lnot\)E)}
		\BinaryInfC{\(\lnot \varphi, \varphi \sequent \bot\)}
		\LeftLabel{(\(\lnot\)I)}
		\UnaryInfC{\(\varphi \sequent \lnot \lnot \varphi\)}
	\end{prooftree}
	よって,シークエント\(\lnot \lnot \varphi \sequent \varphi\)と
	\(\varphi \sequent \lnot \lnot \varphi\)はいずれも導出可能である.
\end{proof}

シークエントの導出過程を示すとき,\cref{Thm:RAA}の証明のように各推論規則の適用過程を
上から下に進む木構造として書き下すのが普通である.
体系がもつ一般的な性質を研究する上ではこの記法は非常に便利なのだが,
個々のシークエントの導出を人間の目にわかりやすい形で行うには扱いずらい.
そこで,推論規則の適用過程は
導出可能であることが分かったシークエントを順に書き並べる形で示すこととする.
次の\cref{Thm:lawofexcludedmiddle}の証明で実例を見せるとしよう.

\begin{Thm}[排中律] \label[Thm]{Thm:lawofexcludedmiddle}
	\(\varphi\)を論理式とするとき,論理式
	\begin{equation}
		\varphi \lor \lnot \varphi
	\end{equation}
	は証明可能である.
\end{Thm}

\begin{proof}
	シークエント\(\sequent \varphi \lor \lnot \varphi\)
	を以下のようにして導く.
	\begin{enumerate}
		\item \(\lnot \paren{\varphi \lor \lnot \varphi} \sequent \lnot \paren{\varphi \lor \lnot \varphi}\)\quad (ID)
		\item \(\varphi \sequent \varphi\)\quad (ID)
		\item \(\varphi \sequent \varphi \lor \lnot \varphi\)\quad (2から(\(\lor\)I)による)
		\item \(\varphi, \lnot \paren{\varphi, \lnot \varphi} \sequent \lnot \paren{\varphi \lor \lnot \varphi}\)\quad (1から(w)による)
		\item \(\lnot \paren{\varphi \lor \lnot \varphi}, \varphi \sequent \varphi \lor \lnot \varphi\)\quad (3から(w)による)
		\item \(\varphi,\lnot \paren{\varphi \lor \lnot \varphi} \sequent \varphi \lor \lnot \varphi\)\quad (4から(e)による)
		\item \(\varphi, \lnot \paren{\varphi \lor \lnot \varphi} \sequent \bot\)\quad (\(4,6\)から(\(\lnot\)E)による)
		\item \(\lnot \paren{\varphi \lor \lnot \varphi} \sequent \lnot \varphi\)\quad (7から(\(\lnot\)I)による)
		\item \(\lnot \paren{\varphi \lor \lnot \varphi} \sequent \varphi \lor \lnot \varphi\)\quad (8から(\(\lor\)I)による)
		\item \(\lnot \paren{\varphi \lor \lnot \varphi} \sequent \bot\)\quad (\(1,9\)から(\(\lnot\)E)による)
		\item \(\sequent \lnot \lnot \paren{\varphi \lor \lnot \varphi}\)\quad (10から(\(\lnot\)I)による)
		\item \(\sequent \varphi \lor \lnot \varphi\)\quad (11から(DNF)による)
	\end{enumerate}
	よって\(\varphi \lor \lnot \varphi\)は証明可能である.
\end{proof}

\Cref{Thm:lawofexcludedmiddle}の証明で提示したシークエントの導出過程は,
通常の数学における次のような証明に対応していると考えることができる.

\begin{naiveproof}
	\(\lnot \paren{\varphi \lor \lnot \varphi}\)だと仮定して矛盾を導く.
	\(\varphi\)であるとすると\(\varphi \lor \lnot \varphi\)となり,
	\(\lnot \paren{\varphi \lor \lnot \varphi}\)と矛盾する.
	従って\(\lnot \varphi\)でなければならない.
	しかしこの場合も\(\varphi \lor \lnot \varphi\)となり
	\(\lnot \paren{\varphi \lor \lnot \varphi}\)と矛盾する.
	よって\(\lnot \lnot \paren{\varphi \lor \lnot \varphi}\),
	すなわち\(\varphi \lor \lnot \varphi\)となる.
\end{naiveproof}

構造規則の関係で多少人工的な操作が見られたものの,
\Cref{Thm:lawofexcludedmiddle}の証明はその下に述べた通常の数学における証明とうまく対応しているように見える.
それでいて形式的記号列の世界で議論しているおかげで各々の導出の根拠に一切のあいまいさが存在しない.

\Cref{Thm:lawofexcludedmiddle}と同じように考えれば,
数学における論理でよく知られてきた結果がこの自然演繹式シークエント計算においても導くことができる.

\begin{Thm} \label[Def]{Thm:equivlogicalexpression}
	\(\varphi, \psi, \chi\)を論理式,\(x\)を変数記号とする.
	以下に述べる論理式のペアは,それぞれを左辺,右辺とするシークエントと
	その左右を入れ替えたシークエントのいずれも導出可能である.
	(「\(\varphi\)と\(\psi\)」と書かれていれば
	\(\varphi \sequent \psi\)と\(\psi \sequent \varphi\)の両方が導出可能である.)
	\begin{enumerate}
		\item \(\varphi \to \psi\)と\(\lnot \varphi \lor \psi\)
		\item \(\lnot \paren{\varphi \lor \psi}\)と\(\lnot \varphi \land \lnot \psi\)
		\item \(\lnot \paren{\varphi \land \psi}\)と\(\lnot \varphi \lor \lnot \psi\)
		\item \(\lnot \forall x \varphi\)と\(\exists x \lnot \varphi\)
		\item \(\lnot \exists x \varphi\)と\(\forall x \lnot \varphi\)
		\item \(\varphi \land \psi\)と\(\psi \land \varphi\)
		\item \(\varphi \lor \psi\)と\(\psi \lor \varphi\)
		\item \(\varphi \land \paren{\psi \land \chi}\)と\(\paren{\varphi \land \psi} \land \chi\)
		\item \(\varphi \lor \paren{\psi \lor \chi}\)と\(\paren{\varphi \lor \psi} \lor \chi\)
		\item \(\varphi \land \paren{\psi \lor \chi }\)と\(\paren{\varphi \land \psi} \lor \paren{\varphi \land \chi}\)
		\item \(\varphi \lor \paren{\psi \land \chi }\)と\(\paren{\varphi \lor \psi} \land \paren{\varphi \lor \chi}\)
		\item \(\lnot \paren{\varphi \to \psi}\)と\(\varphi \land \lnot \psi\)
		\item \(\lnot \forall x \paren{\varphi \to \psi}\)と\(\exists x \paren{\varphi \land \lnot \psi}\)
		\item \(\varphi \to \psi\)と\(\lnot \psi \to \lnot \varphi\)
		\item \(\varphi \to \paren{\psi \to \chi}\)と\(\varphi \land \psi \to \chi\)
		\item \(\forall x \paren{\varphi \to \psi}\)と\(\exists x \varphi \to \psi\)(ただし\(x\)は\(\psi\)には自由出現しないものとする)
		\item \(\exists x \paren{\varphi \to \psi}\)と\(\forall x \varphi \to \psi\)(ただし\(x\)は\(\psi\)には自由出現しないものとする)
		\item \(\forall x \paren{\varphi \to \psi}\)と\(\varphi \to \forall x \psi\)(ただし\(x\)は\(\varphi\)には自由出現しないものとする)
		\item \(\exists x \paren{\varphi \to \psi}\)と\(\varphi \to \exists x \psi\)(ただし\(x\)は\(\varphi\)には自由出現しないものとする)
		\item \(\forall x \paren{\varphi \land \psi}\)と\(\forall x \varphi \land \forall x \psi\)
		\item \(\exists x \paren{\varphi \lor \psi}\)と\(\exists x \varphi \lor \exists x \psi\)
		\item \(\forall x \paren{\varphi \lor \psi}\)と\(\forall x \varphi \lor \psi\)(ただし\(x\)は\(\psi\)には自由出現しないものとする)
		\item \(\exists x \paren{\varphi \land \psi}\)と\(\exists x \varphi \land \psi\)(ただし\(x\)は\(\psi\)には自由出現しないものとする)
	\end{enumerate}
\end{Thm}

\begin{Que} \label[Que]{Que:sequent}
	\Cref{Thm:equivlogicalexpression}を証明せよ.
\end{Que}

\begin{Que} \label[Que]{Que:peirce}
	\(\varphi, \psi\)を論理式とするとき,Peirceの法則に対応するシークエント
	\begin{equation}
		\paren{\varphi \to \psi} \to \varphi \sequent \varphi
	\end{equation}
	を導出せよ.
\end{Que}

等号についてもいろいろなシークエントを導くことができる.

\begin{Thm} \label[Thm]{Thm:equalsign}
	\(f\)をアリティ\(n\)の関数記号,\(s_1, s_2, \dotsc s_n, t_1, t_2, \dotsc, t_n\)を項とする.
	\(i \equiv 1, 2, \dotsc, n\)に対して,シークエント
	\begin{equation}
		s_i = t_i \sequent \apply{f}{s_1, s_2, \dotsc, s_i, \dotsc, s_n} = \apply{f}{s_1, s_2, \dotsc, t_i, \dotsc, s_n}
	\end{equation}
	は導出可能である.
\end{Thm}

\begin{Thm} \label[Thm]{Thm:equalsignrelation}
	\(s, t, u\)を項とするとき,シークエント
	\begin{align}
		s = t \sequent t = s \\
		\paren{s = t} \land \paren{t = u} \sequent s = u
	\end{align}
	は導出可能である.
\end{Thm}

\begin{Que} \label[Que]{Que:equalsignrelation}
	\Cref{Thm:equalsign}と\cref{Thm:equalsignrelation}を証明せよ.
\end{Que}
\chapter{数理論理学入門}
\label[chapter]{chap:advanced}

これまでの章では,
1階述語論理を例として,数学における「論理」がどのようにして形式化され,その上に意味論や証明論がどのようにして構築されるのかを見た.
いくつかの結果は得られたものの,定義の例や成立がほぼ自明なものに終始しており,
数理論理学の成果として掲げるには物足りないものばかりであった.

この章では,これまでの議論をもとにしてそれほど証明が難しくない数理論理学におけるいくつかの結果を示す.
本書の最終章ではあるが,数理論理学という学問分野としては本章の内容がスタート地点といえる.

\newpage

\section{健全性} \label{sec:soundness}

\Cref{chap:semantics}では論理的帰結という概念を,
\cref{chap:proof}では証明可能性という概念をそれぞれ導入した.
どちらの定義も独立に導入されたものであったが,これら2つの概念の通常の数学における対応物は
等価であるものとみなされるのが普通である.
すなわち,証明されたものは論理的に正しく,論理的に正しいものは必ず証明できるはずだとみなされている.
本書で導入した形式的体系においてはどうか考えよう.

\begin{Thm}[健全性定理] \label[Thm]{Thm:soundness}
	\(\symcal{L}\)を言語とし,\(\Gamma\)を\(\symcal{L}\)理論とする.
	このとき,任意の\(\symcal{L}\)閉論理式\(\varphi\)に対して
	\begin{equation}
		\Gamma \provable \varphi \metaimplies \Gamma \satisfy \varphi
		\label{eq:soundness}
	\end{equation}
	が成り立つ.
\end{Thm}

\begin{proof}
	\(\Gamma \provable \varphi\)なので,\(\Gamma\)の元からなる有限列\(\varphi_1, \varphi_2, \dots, \varphi_n\)で
	シークエント
	\begin{equation}
		\varphi_1, \varphi_2, \dots, \varphi_n \sequent \varphi
		\label{eq:targetsequent}
	\end{equation}
	が導出可能であるものが存在する.
	このシークエントの導出可能性に関する帰納法(\cref{Def:provable}に付随する整礎関係に基づく帰納法)によって示す.

	\Cref{eq:targetsequent}が最後に(ID)を適用することによって得られたとき,
	\(\Gamma = \Set{\varphi}\)なので\(\Gamma\)のすべてのモデルが\(\varphi\)を充足することは明らかである.
	次に,\cref{eq:targetsequent}が最後に(REFL)を適用することで得られたとする.
	このとき,\(\varphi\)は項\(t\)に対する\(t \objeq t\)であり,
	\(\Gamma = \emptyset\)である.
	任意の\(\symcal{L}\)構造\(\symcal{M}\)に対して
	\(\interpretation{\symcal{M}}{t} = \interpretation{\symcal{M}}{t}\)が成り立つので,
	\(\symcal{M} \satisfy t \obj t\)が得られるので\(\Gamma \satisfy \varphi\)となる.

	\Cref{eq:targetsequent}が最後に構造規則を適用することによって得られたとし,上段のシークエントについては
	定理の主張が成り立つものとする.このとき,下段のシークエントについても主張が成り立つことは,
	上段のシークエントの左辺と下段のシークエントの左辺をともに集合とみなしたときには同じ集合であることから明らか.

	\Cref{eq:targetsequent}が最後に(\(\to\)I)を適用することによって得られたとし,上段のシークエントについては
	定理の主張が成り立つものとする.このとき,上段のシークエント\(\psi, \Delta \sequent \chi\)について
	論理式の集合としては\(\Delta \cup \Set{\psi} \satisfy \chi\)となっている.
	\(\Delta \satisfy \psi\)であるとする.\(\Delta\)の任意のモデル\(\symcal{M}\)に対して
	\(\symcal{M} \satisfy \psi\)であるため\(\symcal{M}\)は\(\Delta \cup \Set{\psi}\)のモデルでもあるから
	\(\symcal{M} \satisfy \chi\)となる.よって,\(\Delta \satisfy \chi\)
	\(\Delta \satisfy \chi\)であるから\(\Delta \satisfy \psi \to \chi\)となっている.
	\(\symcal{M} \satisfy \psi\)でない場合は\(\Delta \satisfy \lnot \psi\)なので,やはり\(\Delta \satisfy \psi \to \chi\)となる.
	いずれの場合も\(\Delta \satisfy \psi \to \chi\)となる.
	\(\Delta\)が\(\Gamma\)の元からなる有限列で,\(\varphi\)が\(\psi \to \chi\)であることに注意しよう.

	\Cref{eq:targetsequent}が最後に(\(\forall\)E)を適用することによって得られたとし,
	上段のシークエントについては定理の主張が成り立つものとする.
	上段のシークエント\(\Gamma \sequent \forall x \psi\)について,
	論理式の集合としては\(\Gamma \satisfy \forall x \psi\)となっている.
	よって,\(\Gamma\)の任意のモデル\(\symcal{M}\)に対して
	\(\symcal{M} \satisfy \forall x \psi\)である.
	任意の\(\symcal{L}\)閉項\(t\)に対して,\(\interpretation{\symcal{M}}{t} \in M\)
	の名前を\(c\)とすると,\(\symcal{M} \satisfy \forall x \psi\)から
	\(\symcal{M} \satisfy \subst{\varphi}{c/x}\)が従う.
	\(c\)と\(t\)の\(\symcal{M}\)による解釈は一致するので
	\(\symcal{M} \satisfy \subst{\varphi}{t/x}\)となる.
	このことから\(\Gamma \satisfy \subst{\varphi}{t/x}\)が得られる.

	残りの場合も同様にして導ける(すべての推論規則を網羅して検証すればよい).
\end{proof}

健全性定理からの帰結として,無矛盾性が挙げられる.まずは無矛盾性を定義しよう.

\begin{Def} \label{Def:consistency}
	\(\symcal{L}\)を言語,\(\Gamma\)を\(\symcal{L}\)論理式からなる集合とする.
	\(\Gamma \provable \bot\)であるならば,\(\Gamma\)は%
	\index[widx]{むじゅん@矛盾}%
	\term{矛盾}しているという.
	\(\Gamma\)が矛盾していないとき,\(\Gamma\)は
	\index[widx]{むじゅん@矛盾!むむじゅん@無---}%
	\term{無矛盾}%
	であるという.
\end{Def}

\begin{Lemma} \label[Lemma]{Lemma:inconsistency}
	\(\symcal{L}\)を言語,\(\Gamma\)を\(\symcal{L}\)論理式からなる集合とする.
	このとき,以下の3条件はすべて同値である.
	\begin{enumerate}
		\item \(\Gamma\)は矛盾している.
		\item すべての\(\symcal{L}\)論理式\(\varphi\)に対して\(\Gamma \provable \varphi\)となる.
		\item \(\Gamma \provable \varphi\)かつ\(\Gamma \provable \lnot \varphi\)となる\(\symcal{L}\)論理式\(\varphi\)が存在する.
	\end{enumerate}
\end{Lemma}

\begin{proof}
	\(1 \metaimplies 2\)は(w)規則と(\(\lnot\)I)規則,および(DNE)規則から従う.
	\(2 \metaimplies 3\)は明らか.\(3 \metaimplies 1\)は(\(\lnot\)E)規則から従う.
\end{proof}

\begin{Corollary} \label[Corollary]{coro:consistencyfrommodel}
	\(\symcal{L}\)を言語とし,\(\Gamma\)を\(\symcal{L}\)理論とする.
	\(\Gamma\)がモデルをもつならば,\(\Gamma\)は無矛盾である.
\end{Corollary}

\begin{proof}
	\(\Gamma\)がモデルをもつが矛盾していると仮定し,\(\symcal{M}\)を\(\Gamma\)のモデルとする.
	\(\Gamma\)が矛盾しているので,\cref{Lemma:inconsistency}より
	\(\Gamma \provable \varphi\)かつ\(\Gamma \provable \lnot \varphi\)となる\(\symcal{L}\)論理式\(\varphi\)が存在する.
	このとき,\cref{Thm:soundness}によって
	\(\Gamma \satisfy \varphi\)かつ\(\Gamma \satisfy \lnot \varphi\)
	となり,\(\symcal{M} \satisfy \varphi\)かつ\(\symcal{M} \satisfy \lnot \varphi\)となるが,
	\(\symcal{M} \satisfy \lnot \varphi\)の定義からこれはありえない.
\end{proof}

\Cref{coro:consistencyfrommodel}において\(\Gamma = \emptyset\)とすることで,
以下の主張が得られる(\(\Gamma\)のモデルとしては適当な\(\symcal{L}\)構造を1つとればよい).

\begin{Corollary}[自然演繹式シークエント計算の無矛盾性] \label[Corollary]{coro:consistency}
	\begin{equation}
		\provable \bot
		\label{eq:contradictionforsystem}
	\end{equation}
	となることはない.
\end{Corollary}

\section{完全性} \label{sec:completeness}

\Cref{Thm:soundness}は,いわば「証明できるものは論理的に正しい」という主張であると解釈することができる.
この逆となる「論理的に正しいものは証明できる」に相当する主張も成立するであろうことが予想される.
実際これは成立するが,\cref{Thm:soundness}よりは長い議論となる.いくつかの準備の下で示すとしよう.

また,簡単のため,本節においては言語\(\symcal{L}\)は高々可算集合であるとする.
\(\symcal{L}\)が非可算である場合は以下で(通常の自然数に関する)
帰納法や帰納的定義を行っている部分を整列集合に関する超限帰納法や超限再帰に置き換えることによってほとんど同様に議論できる.
このとき,言語\(\symcal{L}\)上に整列順序が入ることは選択公理(と同値な整列可能定理)による.

\begin{Lemma} \label[Lemma]{Lemma:consistencyor}
	\(\symcal{L}\)を言語とし,\(\varphi\)を\(\symcal{L}\)論理式,\(\Gamma\)を\(\symcal{L}\)論理式からなる集合とする.
	このとき,\(\Gamma \cup \Set{\varphi}\)
	と\(\Gamma \cup \Set{\lnot \varphi}\)がともに矛盾するならば,\(\Gamma\)も矛盾する.
\end{Lemma}

\begin{proof}
	\(\Gamma \cup \Set{\varphi}\)と\(\Gamma \cup \Set{\lnot \varphi}\)がともに矛盾していると仮定すると,
	(\(\lnot\)I)規則と(DNE)規則によって\(\Gamma \provable \varphi\)かつ\(\Gamma \provable \lnot \varphi\)
	であることがわかる.よって\cref{Lemma:inconsistency}によって\(\Gamma\)が矛盾することがわかる.
\end{proof}

\begin{Lemma} \label[Lemma]{Lemma:existsextend}
	\(\symcal{L}\)を言語,\(\Gamma\)を論理式からなる集合,\(\varphi\)を\(\symcal{L}\)論理式,
	\(x\)を変数記号とする.
	このとき,\(\Gamma\)の元となっている論理式と\(\varphi\)のいずれにも出現しない定数記号\(c\)に対し,
	\(\Gamma\)が無矛盾ならば\(\Gamma \cup \Set{\exists x \varphi \to \subst{\varphi}{c/x}}\)も無矛盾となる.
\end{Lemma}

\begin{proof}
	\(\Gamma \cup \Set{\exists x \varphi \to \subst{\varphi}{c/x}}\)が矛盾すると仮定する.
	\(\Gamma \provable \lnot \paren{\exists x \varphi \to \subst{\varphi}{c/x}}\)
	なので\cref{Thm:prooflnotforallexistscontradiction}の\cref{eq:lnotto}
	(と(\(\to\)I)規則や(\(\to\)E)規則)により\(\Gamma \provable \exists x \varphi \land \lnot \subst{\varphi}{c/x}\)
	となる.このとき,(\(\land\)I)の形から\(\Gamma \provable \exists x \varphi\)かつ\(\Gamma \provable \lnot \subst{\varphi}{c/x}\)
	となる.すると,\(\Gamma \provable \lnot \subst{\varphi}{c/x}\)から
	(\(\forall\)I)規則によって\(\Gamma \provable \forall x \lnot \varphi\)
	が得られるが,\cref{Thm:proofdemorganlaw}の\cref{eq:existsdemorgan}から
	\(\Gamma \provable \lnot \exists x \varphi\)が得られるので\(\Gamma\)の無矛盾性に反する.
\end{proof}

\begin{Def} \label{Def:maximumconsistency}
	\(\symcal{L}\)を言語とし,\(\Gamma\)を\(\symcal{L}\)理論とする.
	このとき,\(\Gamma\)が%
	\index[widx]{きょくだいむむじゅんしゅうごう@極大無矛盾集合}%
	\term{極大無矛盾集合}%
	であるとは,
	任意の\(\symcal{L}\)論理式\(\varphi\)に対して
	\(\varphi \in \Gamma\)か\(\lnot \varphi \in \Gamma\)の
	いずれか一方のみが必ず成り立つことをいう.

	\Cref{Lemma:inconsistency}から,極大無矛盾集合は無矛盾である.
\end{Def}

\begin{Lemma} \label[Lemma]{Lemma:Henkintheory}
	\(\symcal{L}\)を言語,\(\Gamma\)を\(\symcal{L}\)理論とする.
	このとき,言語\(\symcal{L}_{\infty} \supset \symcal{L}\)と
	無矛盾な\(\symcal{L}_\infty\)理論\(\Gamma_\infty \supset \Gamma\)で,
	\(\exists x \varphi\)という形の任意の\(\symcal{L}_\infty\)閉論理式に対して
	以下の条件を満たすものが存在する:
	\begin{equation}
		\Gamma_\infty \provable \exists x \varphi
		\metaimplies \text{\(\Gamma_\infty \provable \subst{\varphi}{c/x}\)となる\(c \in \symcal{L}_\infty\)が存在する}
		\label{eq:Henkinextend}
	\end{equation}



\end{Lemma}

\begin{proof}
	まず,\(\exists x \varphi\)という形の\(\symcal{L}\)閉論理式1つにつき
	新しい定数記号\(c_{\exists x \varphi}\)を1つ用意し,
	言語を\(\symcal{L} \cup \Set{c_{\exists x \varphi}}\)に拡張する.
	次に,以下の\(\symcal{L} \cup \Set{c_{\exists x \varphi}}\)閉論理式を新しい公理として\(\Gamma\)に加える:
	\begin{equation}
		H_{\exists x \varphi} \colon \exists x \varphi \to \subst{\varphi}{c/x}
		\label{eq:Henkinaxiom}
	\end{equation}
	このときの\(c_{\exists x \varphi}\)を%
	\index[widx]{Henkinていすう@Henkin定数}%
	\term{Henkin定数},
	論理式\(H_{\exists x \varphi}\)を%
	\index[widx]{Henkinこうり@Henkin公理}%
	\term{Henkin公理}と呼ぶ.
	\Cref{Lemma:existsextend}により,\(\Gamma\)にHenkin公理を加えて得られる
	\(\symcal{L} \cup \Set{c_{\exists x \varphi}}\)理論は無矛盾である.

	さて,自然数\(n\)に対する言語\(\symcal{L}_n\)と\(\symcal{L}_n\)閉論理式からなる
	Henkin公理の集合\(H_n\),
	および\(\symcal{L}_n\)理論\(\Gamma_n\)を以下のように帰納的に定義する:
	まず,\(\symcal{L}_0 = \symcal{L}\)とし,
	\(\exists x \varphi\)という形の\(\symcal{L}\)論理式に対するHenkin定数の全体を\(C_0\),
	Henkin公理の全体を\(H_0\)とする.
	そして,\(\exists x \varphi\)という形の\(\symcal{L}_n\)論理式に対するHenkin定数の全体を\(C_n\),
	Henkin公理の全体を\(H_n\)として,\(\symcal{L}_{n+1} = \symcal{L}_n \cup C_n\),\(\Gamma_n = \Gamma \cup H_n\)とする.
	\cref{Lemma:existsextend}と帰納法によってすべての自然数\(n\)に対して\(\Gamma_n\)が無矛盾であることを示すことができる.
	\(\symcal{L}_\infty = \bigcup_{n \in \NaturalNumbers} \symcal{L}_n\),\(H_\infty = \bigcup_{n \in \NaturalNumbers} H_n\)
	とし,\(\Gamma_\infty = \bigcup_{n \in \NaturalNumbers} \Gamma_n = \Gamma \cup H_\infty\)とする.
	\(\symcal{L}_0 \subset \symcal{L}_1 \subset \symcal{L}_2 \subset \dotsb\)や
	\(\Gamma_0 \subset\Gamma_1 \subset \Gamma_2 \subset \dotsb\)が成り立っている.

	\(\Gamma_\infty\)が無矛盾であることを示そう.\(\Gamma_\infty\)が矛盾していると仮定すると,
	有限集合\(\Delta \subset \Gamma_\infty\)で\(\Delta \provable \bot\)となるものが存在する.
	このとき,\(\Delta\)が有限集合なので\(\Delta \subset \Gamma_n\)となる自然数\(n\)が存在する.
	\(\Gamma_n\)が無矛盾なので\(\Delta\)も無矛盾である必要があるが,これは\(\Delta \provable \bot\)に矛盾する.

	次に,\(\exists x \varphi\)という形の任意の\(\symcal{L}_\infty\)閉論理式に対して以下が成り立つことを示そう.
	\begin{equation}
		\Gamma_\infty \provable \exists x \varphi
		\metaimplies \text{\(\Gamma_\infty \provable \subst{\varphi}{c_{\exists x \varphi}/x}\)}.
		\label{eq:henkincondition}
	\end{equation}
	\(\varphi\)は\(\symcal{L}_\infty\)論理式であって
	その記号列としての長さは有限である.また,\(\symcal{L}_0 \subset \symcal{L}_1 \dotsb\)なので,
	\(\varphi\)が\(\symcal{L}_n\)論理式になるような自然数\(n\)が存在する.
	\(\Gamma_\infty \provable \exists x \varphi\)だから\(\symcal{L}_\infty\)論理式の
	有限集合\(\Delta \subset \Gamma_\infty\)で\(\Delta \provable \exists x \varphi\)となるものが存在する.
	\(\Delta\)は有限集合なので,\(\Delta \subset \Gamma_m\)となる自然数\(m\)がとれる.
	\(l = \max \Set{n, m} + 1\)とおくと,\(\varphi\)は\(\symcal{L}_l\)論理式であり,
	かつ\(\Delta \subset \Gamma_n \subset \Gamma_l\)なので
	\(\Gamma_l \provable \exists x \varphi\)となる.
	\(H_{\exists x \varphi} \in H_l\)より\(\Gamma_l \provable H_{\exists x \varphi}\)なので,
	(\(\to\)E)によって\(\Gamma_l \provable \subst{\varphi}{c_{\exists x \varphi}/x}\)を得る.
	よって\(\Gamma_\infty \provable \subst{\varphi}{c_{\exists x \varphi}/x}\)となる.
	また,\(c_{\exists x \varphi} \in C_l \subset \symcal{L}_\infty\)
	だから,この\(c_{\exists x \varphi}\)は\(\symcal{L}_\infty\)における定数記号である.
\end{proof}

\begin{Lemma}[Henkin拡大] \label[Lemma]{Lemma:Henkinextension}
	\(\symcal{L}\)を言語,\(\Gamma\)を\(\symcal{L}\)理論とする.
	このとき,言語\(\symcal{L}_{\infty} \supset \symcal{L}\)と
	極大無矛盾集合であるような\(\symcal{L}_\infty\)理論\(\Gamma_\infty \supset \Gamma\)で,
	\(\exists x \varphi\)という形の任意の\(\symcal{L}_\infty\)閉論理式に対して
	\cref{eq:Henkinextend}を満たすものが存在する.

	ここで存在が保証された\(\symcal{L}_\infty\)理論\(\Gamma_\infty\)を
	\(\Gamma\)の%
	\index[widx]{Henkinかくだい@Henkin拡大}%
	\term{Henkin拡大}という.
\end{Lemma}

\begin{proof}
	\Cref{Lemma:Henkintheory}における\(\Gamma_\infty\)を極大無矛盾集合に拡張する.
	\(\symcal{L}\)が高々可算集合であることを仮定したので,\(\symcal{L}\)論理式全体の集合は可算集合である.
	よって\(\symcal{L}_\infty\)閉論理式全体の集合も可算集合となるので,それらすべてを
	\(\varphi_0, \varphi_1, \dots\)のように列として表すこととする.
	自然数\(n\)に対する\(\symcal{L}_\infty\)理論\(T_n\)を以下のように帰納的に定義する:
	まず,\(T_0 = \Gamma_\infty\)とする.
	各自然数\(n\)に対し,\(T_n \cup \Set{\varphi_n} \provable \bot\)であれば\(T_{n + 1} = T_n \cup \Set{ \lnot \varphi_n}\)
	とし,そうでなければ\(T_{n + 1} = T_n \cup \Set{\varphi_n}\)とする.
	このように定義された自然数\(n\)に対する\(T_n\)を用いて\(T_\infty = \bigcup_{n \in \NaturalNumbers} T_n\)と定義する.

	\(\Gamma_\infty\)の無矛盾性と帰納法により,各\(n\)に対して\(T_n\)が無矛盾であることが従う.
	また,\(\Gamma_\infty\)の無矛盾性と同様にして\(T_\infty\)の無矛盾性が従う.
	さらに,任意の自然数\(n\)に対して\(\varphi_n \in T_n\)か\(\lnot \varphi_n \in T_n\)
	のうちいずれか一方のみが成り立つので,\(T_\infty\)が極大無矛盾集合であることもわかる.
	\(T_\infty\)が\(\exists x \varphi\)という形の任意の\(\symcal{L}_\infty\)閉論理式に対して
	\cref{eq:Henkinextend}を満たすことは\(\Gamma_\infty \subset T_\infty\)であることから明らか.
\end{proof}

\begin{Lemma} \label[Lemma]{Lemma:Henkinextensionhasmodel}
	\(\symcal{L}\)を言語,\(\Gamma\)を\(\symcal{L}\)理論とする.
	このとき,\(\Gamma\)のHenkin拡大はモデルをもつ.
\end{Lemma}

\begin{proof}
	\(\Gamma\)のHenkin拡大を\(\Gamma_\infty\)とし,
	\(\Gamma_\infty\)は\cref{Lemma:Henkinextension}における
	言語\(\symcal{L}_\infty\)閉論理式からなる集合とする.
	\(\symcal{L}_\infty\)閉項全体の集合を\(X\)とし,\(X\)上の同値関係\(\sim\)を以下のように定める:
	\begin{equation*}
		x \sim y \metaequivalent x \objeq y \in \Gamma_\infty.
	\end{equation*}
	\(\sim\)が同値関係になることは(REFL)規則と\cref{Thm:equalsignrelation}からわかる.

	この\(X\)の\(\sim\)による商集合を\(M = X / {\sim}\)とする.
	\(t\)を代表元とする同値類を\(\equivclass{t}\)と表すことにする.
	以下,\(M\)を対象領域とする\(\symcal{L}_\infty\)構造を定義する.

	定数記号\(c\)の\(\symcal{M}\)による解釈は\(\interpretation{\symcal{M}}{c} = \equivclass{c}\)とする.
	\(f\)がアリティ\(n\)の関数記号であるとき,\(f\)の\(\symcal{M}\)による解釈は
	\(t_1, t_2, \dots, t_n\)を\(\symcal{L}_\infty\)閉項として
	\[
		\apply{\interpretation{\symcal{M}}{f}}{\equivclass{t_1}, \equivclass{t_2}, \dots, \equivclass{t_n}}
		=
		\equivclass{\apply{f}{t_1, t_2, \dots, t_n}}
	\]
	によって定める.
	\(r\)がアリティ\(n\)の関係記号のとき,\(r\)の\(\symcal{M}\)による解釈は
	\(t_1, t_2, \dots, t_n\)を\(\symcal{L}_\infty\)閉項として
	\[
		\pair{\equivclass{t_1}, \equivclass{t_2}, \dots, \equivclass{t_n}} \in \interpretation{\symcal{M}}{r}
		\metaequivalent
		\apply{r}{t_1, t_2, \dots, t_n} \in \Gamma_\infty
	\]
	で定める.
	これらの定義の正当性(代表元のとり方に依存しないこと)は,(SUBST)規則や\cref{Thm:equalsign}などによる.

	以上のように定義した\(\symcal{M}\)が任意の\(\symcal{L}_\infty\)閉論理式\(\varphi\)に対して
	\begin{equation}
		\symcal{M} \satisfy \varphi \metaequivalent \varphi \in \Gamma_\infty
		\label{eq:henkintarget}
	\end{equation}
	を満たすことを\(\varphi\)の構造による帰納法によって示す.
	これが示されれば\(\symcal{M}\)が\(\Gamma_\infty\)のモデルになっていることがわかる.

	\(\varphi\)が\(\bot\)のとき,
	\(\symcal{M} \satisfy \varphi\)も\(\varphi \in \Gamma_\infty\)も成り立たないので
	\cref{eq:henkintarget}は成り立つ.
	\(\varphi\)が\(s \objeq t\)の形のとき,
	\begin{align*}
		 & \symcal{M} \satisfy \varphi            \\
		 & \metaequivalent s \sim t               \\
		 & \metaequivalent s \objeq t \in \Gamma.
	\end{align*}
	\(\varphi\)がアリティ\(n\)の関係記号\(r\)と\(\symcal{L}_\infty\)閉項\(t_1, t_2, \dots, t_n\)によって
	\(\apply{r}{t_1, t_2, \dots, t_n}\)と表されている場合
	\begin{align*}
		 & \symcal{M} \satisfy \varphi                                        \\
		 & \metaequivalent \pair{
			\interpretation{\symcal{M}}{t_1},
			\interpretation{\symcal{M}}{t_2},
			\dots,
			\interpretation{\symcal{M}}{t_n}
		}
		\in \interpretation{\symcal{M}}{r}                                    \\
		 & \metaequivalent \apply{r}{t_1, t_2, \dots, t_n} \in \Gamma_\infty.
	\end{align*}

	\(\varphi\)が\(\exists x \psi\)の形をしている場合を考える.
	まず,\(\varphi \in \Gamma_\infty\)と仮定すると,
	\(\exists x \psi \to \subst{\psi}{c_{\exists x \psi}/x} \in \Gamma_\infty\)である.
	よって\(\varphi \in \Gamma_\infty\)から(\(\to\)E)規則によって
	\(\subst{\psi}{c_{\exists x \psi}/x} \in \Gamma_\infty\)を得る.
	帰納法の仮定によって\(\symcal{M} \satisfy \subst{\psi}{c_{\exists x \psi}/x}\)なので
	\(\symcal{M} \satisfy \varphi\)となる.
	次に,\(\symcal{M} \satisfy \varphi\)と仮定すると,\(\symcal{M} \satisfy \subst{\psi}{c_a/x}\)
	となる\(a \in M\)が存在する.ここで,\(c_a\)は\(a\)の名前である.
	\(a\)は\(\symcal{L}_\infty\)閉項\(t\)を用いて\(a = \equivclass{t}\)と表されるので,
	\(\interpretation{\symcal{M}}{t} = a\)だから\(\symcal{M} \satisfy \subst{\psi}{t/x}\)となる.
	帰納法の仮定により\(\subst{\psi}{t/x} \in \Gamma_\infty\)だから(\(\exists\)I)規則に
	よって\(\varphi \in \Gamma_\infty\)が従う.

	残りの場合も同様にして証明できる.
\end{proof}

\index[widx]{Henkinのていり@Henkinの定理}
\begin{Thm}[Henkinの定理] \label[Thm]{Thm:HenkinTheorem}
	\(\symcal{L}\)を言語,\(\Gamma\)を\(\symcal{L}\)理論とする.
	\(\Gamma\)が無矛盾であるならば,\(\Gamma\)はモデルをもつ.
\end{Thm}

\begin{proof}
	\Cref{Lemma:Henkinextensionhasmodel}により,
	\(\Gamma\)のHenkin拡大\(\Gamma_\infty\)のモデル\(\symcal{M}\)が存在する.
	このとき,任意の\(\symcal{L}_\infty\)閉論理式\(\varphi\)に対して
	\[
		\symcal{M} \satisfy \varphi \metaequivalent \varphi \in \Gamma_\infty
	\]
	となる.
	\(\Gamma \subset \Gamma_\infty\)なので,\(\varphi \in \Gamma\)となる
	任意の\(\symcal{L}\)閉論理式\(\varphi\)に対して
	\(\symcal{M} \satisfy \varphi\)が成り立つので,\(\symcal{M}\)は\(\Gamma\)のモデルであることがわかる.
\end{proof}

\Cref{coro:consistencyfrommodel}と\cref{Lemma:Henkinextensionhasmodel}により,以下の帰結が得られる.

\begin{Corollary} \label{coro:completeness}
	\(\Gamma\)を理論とする.このとき,\(\Gamma\)が無矛盾であることと\(\Gamma\)がモデルをもつことは同値である.
\end{Corollary}

\index[widx]{かんぜんせいていり@完全性定理}
\begin{Thm}[完全性定理] \label[Thm]{Thm:completeness}
	\(\symcal{L}\)を言語とし,\(\Gamma\)を\(\symcal{L}\)理論とする.
	このとき,任意の\(\symcal{L}\)閉論理式\(\varphi\)に対して
	\begin{equation}
		\Gamma \satisfy \varphi \metaimplies \Gamma \provable \varphi
		\label{eq:completeness}
	\end{equation}
	が成り立つ.
\end{Thm}

\begin{Que} \label{Que:completeness}
	\Cref{Thm:HenkinTheorem}を用いて\cref{Thm:completeness}を示せ.
\end{Que}

完全性定理の応用として,次のコンパクト性定理が得られる.

\index[widx]{こんぱくとせいていり@コンパクト性定理}
\begin{Thm}[コンパクト性定理] \label[Thm]{Thm:compact}
	\(\Gamma\)を理論とする.
	このとき,\(\Gamma\)のすべての有限部分集合がモデルをもつことと
	\(\Gamma\)がモデルをもつことは同値である.
\end{Thm}

\begin{proof}
	対偶(もしくは裏)をとって,
	\(\Gamma\)がモデルをもたないことと\(\Gamma\)の有限部分集合でモデルをもたないものが存在することが同値であることを示す.
	\begin{align*}
		                & \text{\(\Gamma\)がモデルをもたない}                 \\
		\metaequivalent & \text{\(\Gamma\)は矛盾する}                     \\
		\metaequivalent & \text{\(\Gamma\)のある有限部分集合で矛盾するものが存在する}     \\
		\metaequivalent & \text{\(\Gamma\)のある有限部分集合でモデルをもたないものが存在する}
	\end{align*}
	よってもとの主張も成り立つ.
\end{proof}

\section{Herbrand構造} \label{sec:Herbrandstructure}

完全性定理の証明で経由した\cref{Lemma:Henkinextensionhasmodel}において,\(\Gamma_{\infty}\)のモデルを構成する際に
閉項全体の集合に同値関係を入れることによって対象領域を定義した.
このような手法を採用したのは,
\cref{Lemma:Henkinextensionhasmodel}の証明中ではモデルとして提示したい構造の対象領域がアプリオリに与えられていなかったためである.
単に閉項全体を対象領域として定義したのでは,理論の上では「等しい」項であっても構造の上では等しくならない.
そこで,理論の上で「等しい」項を同一視するために同値関係を入れたのであった.

この「閉項全体の集合に同値関係を入れ,それらに解釈を付与することで構造を定義する」という手法については,
完全性定理を証明するための単なる道具を超えた価値がある.そこで,以下のように定義しよう.

\index[widx]{こうぞう@構造!Herbrandこうぞう@Herbrand---}
\begin{Def} \label[Def]{Def:Herbrandstructure}
	\(\symcal{L}\)を定数記号をもつ言語とし,\(\Gamma\)を\(\symcal{L}\)理論とする.
	\(\symcal{L}\)閉項全体の集合\(X\)上の同値関係\(\sim\)を以下のように定義する:
	\begin{equation}
		x \sim y \metaequivalent \Gamma \provable x \objeq y.
		\label{eq:herbrandequivalencerelation}
	\end{equation}
	この同値関係\(\sim\)による\(X\)の商集合を\(M = X / \mathord{\sim}\)とする.
	このとき,\(M\)を対象領域とする\(\symcal{L}\)構造\(\symcal{M}\)を
	以下のように定める.

	まず,定数記号\(c\)の\(\symcal{M}\)による解釈は
	\begin{equation}
		\interpretation{\symcal{M}}{c} = \equivclass{c}
		\label{eq:herbrandinterpretationconstantsymbol}
	\end{equation}
	と定める.
	次に,アリティ\(n\)の関数記号\(f\)の\(\symcal{M}\)による解釈は,\(t_1, t_2, \dots, t_n\)を\(\symcal{L}\)閉項として
	\begin{equation}
		\apply{\interpretation{\symcal{M}}{f}}{
			\equivclass{t_1},
			\equivclass{t_2},
			\dots,
			\equivclass{t_n}
		}
		=
		\equivclass{
			\apply{f}{t_1, t_2, \dots, t_n}
		}
		\label{eq:herbrandinterpretationfunctionsymbol}
	\end{equation}
	とする.
	最後に,アリティ\(n\)の関数記号\(r\)の\(\symcal{M}\)による解釈は,\(t_1, t_2, \dots, t_n\)を\(\symcal{L}\)閉項として
	\begin{equation}
		\pair{
			\equivclass{t_1},
			\equivclass{t_2},
			\dots,
			\equivclass{t_n}
		}
		\in \interpretation{\symcal{M}}{r}
		\metaequivalent
		\Gamma \provable
		\apply{r}{t_1, t_2, \dots, t_n}
		\label{eq:herbrandinterpretationrelationsymbol}
	\end{equation}
	と定める.

	以上のように定義される\(\symcal{L}\)構造\(\symcal{M}\)を
	\(\symcal{L}\)における\(\Gamma\)の\term{Herbrand構造}という.
\end{Def}

\begin{Note}
	\Cref{Def:Herbrandstructure}においては言語に定数記号が含まれることを要請している.
	これは,構造の定義(\cref{Def:structure})において対象領域が空でないことを要請していることに基づく.
	定数記号を含まない言語におけるHerbrand構造を考えたい場合には,適当な定数記号をその言語に付け加えればよい.
\end{Note}

\begin{Que} \label[Que]{Que:Herbrandstructurewelldefineded}
	\Cref{Def:Herbrandstructure}中の議論を正当化せよ.すなわち,以下を示せ:
	\begin{enumerate}
		\item \Cref{eq:herbrandequivalencerelation}で定義される\(X\)上の二項関係\(\sim\)が同値関係であること.
		\item \Cref{eq:herbrandinterpretationconstantsymbol}から\Cref{eq:herbrandinterpretationrelationsymbol}の3つの定義が
		      いずれも\(M\)の代表元のとりかたによらないこと.
	\end{enumerate}
\end{Que}

\begin{Lemma} \label[Lemma]{lemma:herbrandsemantics}
	\(\symcal{L}\)を定数記号をもつ言語とし,\(\Gamma\)を\(\symcal{L}\)理論とする.
	\(\symcal{M}\)を\(\symcal{L}\)における\(\Gamma\)のHerbrand構造とする.
	このとき,以下が成り立つ.
	\begin{enumerate}
		\item \(t\)を\(\symcal{L}\)閉項とするとき,\(\interpretation{\symcal{M}}{t} = \equivclass{t}\)である.
		\item \(\symcal{L}\)文\(\varphi\)が\(\forall x_1 \forall x_2 \dotsb \forall x_n \apply{\psi}{x_1, x_2, \dots, x_n}\)
		      の形の論理式であるならば,\(\symcal{M} \satisfy \varphi\)とすべての\(\symcal{L}\)閉項\(t_1, t_2, \dots, t_n\)に対して
		      \(\symcal{M} \satisfy \apply{\psi}{t_1, t_2, \dots, t_n}\)であることは同値.
		\item \(\symcal{L}\)文\(\varphi\)が原子論理式であるならば,\(\Gamma \provable \varphi\)と\(\symcal{M} \satisfy \varphi\)は同値.
		\item \(\varphi\)が\(\symcal{L}\)原子論理式の全称閉包で\(\Gamma \provable \varphi\)ならば\(\symcal{M} \satisfy \varphi\)となる.
	\end{enumerate}
\end{Lemma}

\begin{Que} \label{Que:herbrandsemantics}
	\Cref{lemma:herbrandsemantics}を示せ.
\end{Que}

Herbrand構造の価値の一端については,\cref{sec:equationaltheory}で述べる.

\section{等式理論とHorn理論} \label{sec:equationaltheory}

本節では,言語や公理に特別な制約を課した場合について考察する.

\begin{Def} \label[Def]{Def:horntheory}
	原子論理式のことを%
	\index[widx]{りてらる@リテラル!せいのりてらる@正の---}%
	\term{正のリテラル}といい,
	原子論理式\(p\)の否定\(\lnot p\)のことを%
	\index[widx]{りてらる@リテラル!ふのりてらる@負の---}%
	\term{負のリテラル}という.
	正のリテラルと負のリテラルを総称して%
	\index[widx]{りてらる@リテラル}%
	\term{リテラル}という.
	また,リテラル\(p_1, p_2, \dots, p_n\)に対し,論理式
	\begin{equation*}
		p_1 \lor p_2 \lor \dotsb \lor p_n
	\end{equation*}
	を%
	\index[widx]{せつ@節}%
	\term{節}%
	といい,節に含まれる正のリテラルが高々1つである場合にはその節のことを%
	\index[widx]{せつ@節!Hornせつ@Horn---}%
	\term{Horn節}という.

	理論\(T\)の各文がすべてHorn節の全称閉包である場合,\(T\)は%
	\index[widx]{りろん@理論!Hornりろん@全称Horn---}%
	\term{全称Horn理論}であるという.
\end{Def}

全称Horn理論の特別な場合として,以下に示す等式理論も重要である.

\begin{Def} \label[Def]{Def:equationaltheory}
	\(\symcal{L}\)を言語とし,\(T\)を\(\symcal{L}\)理論とする.
	このとき,\(T\)が%
	\index[widx]{りろん@理論!とうしきりろん@等式---}%
	\term{等式理論}であるとは,\(\symcal{L}\)が関係記号をもたず,
	さらに\(T\)の各文がすべて等式(\(s \objeq t\)の形の論理式)の全称閉包になっていることをいう.
	等式理論はその各文がすべてちょうど1つの正のリテラルであるため,全称Horn理論でもある.
\end{Def}

\begin{Ex} \label[Ex]{Ex:equationaltheory}
	\Cref{Ex:grouptheory}における群の理論\(\GP\)と\cref{Ex:Ring}における環の理論\(\Ring\)はいずれも等式理論であり,
	従って全称Horn理論でもある.
\end{Ex}

\begin{Ex} \label[Ex]{Ex:horntheory}
	\Cref{Ex:orderedset}における理論\(\POSET\)は,そのままでは全称Horn理論ではない.
	1つ目の公理\(\forall \obj{x} \lnot \paren{\obj{x} \obj{<} \obj{x}}\)
	は0個の正のリテラルと1個の負のリテラルからなるHorn節の全称閉包であるが,
	2つ目の公理
	\[
		\forall \obj{x} \forall \obj{y} \forall \obj{z}
		\paren{\obj{x} \obj{<} \obj{y} \land \obj{y} \obj{<} \obj{z} \to \obj{x} \obj{<} \obj{z}}
	\]
	はHorn節の全称閉包でもなければ節の全称閉包ですらないからである.
	しかし,この公理を論理的に同値な
	\[
		\forall \obj{x} \forall \obj{y} \forall \obj{z}
		\paren{\lnot \paren{\obj{x} \obj{<} \obj{y}} \lor \lnot \paren{\obj{y} \obj{<} \obj{z}} \lor \obj{x} \obj{<} \obj{z}}
	\]
	で置き換えると,これは1個の正のリテラルと2個の負のリテラルからなるHorn節の全称閉包なので,
	この置き換えによって得られる理論は全称Horn理論である.
\end{Ex}

\begin{Thm} \label[Thm]{Thm:HornHerbrandmodel}
	\(\symcal{L}\)を定数記号をもつ言語とする.
	\(\symcal{L}\)理論\(\Gamma\)が無矛盾な全称Horn理論であるならば,
	\(\symcal{L}\)における\(\Gamma\)のHerbrand構造は\(\Gamma\)のモデルとなる.
\end{Thm}

\begin{proof}
	\(\symcal{L}\)における\(\Gamma\)のHerbrand構造を\(\symcal{M}\)とする.
	\(\varphi \in \Gamma\)を任意にとる.
	\(\varphi\)が正のリテラルを1つもつ場合を考える.
	このとき,\(\varphi\)は原子論理式\(p_1, p_2, \dots, p_{n-1}, p_n\)を用いた以下の論理式\(\apply{\psi}{x_1, \dots, x_m}\)の全称閉包とみなせる:
	\[
		\lnot \apply{p_1}{x_1, \dots, x_m} \lor \lnot \apply{p_2}{x_1, \dots, x_m} \lor \dotsb \lor \lnot \apply{p_{n-1}}{x_1, \dots, x_m}
		\lor \apply{p_{n}}{x_1, \dots, x_m}.
	\]

	さて,\(\symcal{M} \satisfy \varphi\)が成り立たないものとして矛盾を導く.
	\Cref{lemma:herbrandsemantics}から閉項\(t_1, \dots, t_m\)で
	\(\symcal{M} \satisfy \apply{\psi}{t_1, \dots, t_m}\)が成り立たないものが存在する.
	よって\(i = 1, 2, \dots, n-1\)に対して\(\symcal{M} \satisfy \apply{p_i}{t_1, \dots, t_m}\)であり,
	かつ\(\symcal{M} \satisfy \apply{p_n}{t_1, \dots, t_m}\)は成り立たない.
	従って,\Cref{lemma:herbrandsemantics}から
	\(i = 1,2,\dots, n-1\)については\(\Gamma \provable \apply{p_i}{t_1, \dots, t_m}\)であり,かつ
	\(\Gamma \provable \apply{p_n}{t_1, \dots, t_m}\)は成り立たないことがわかる.
	また,\(\varphi \in \Gamma\)と(\(\forall\)E)規則によって
	\(\Gamma \provable \apply{\psi}{t_1, \dots, t_m}\)が得られる.
	ここで,
	\[
		\apply{\psi}{t_1, \dots, t_m} \interderivable \apply{p_1}{t_1, \dots, t_m} \to \paren{\apply{p_2}{t_1, \dots, t_m} \to \paren{\dotsb \to \apply{p_n}{t_1, \dots, t_m}}}
	\]
	であるから
	\[
		\Gamma \provable \apply{p_1}{t_1, \dots, t_m} \to \paren{\apply{p_2}{t_1, \dots, t_m} \to \paren{\dotsb \to \apply{p_n}{t_1, \dots, t_m}}}
	\]
	となる.よって(\(\to\)E)規則を繰り返し適用することで
	\[
		\Gamma \provable \apply{p_n}{t_1, \dots, t_m}
	\]
	が得られ,矛盾する.
	よって\(\symcal{M} \satisfy \varphi\)となる.
	\(\varphi\)が正のリテラルをもたない場合も同様である(上記で\(p_n\)の代わりに\(\lnot p_n\)を考えればよい).

	以上の議論により,\(\symcal{M} \satisfy \Gamma\)が得られる.
\end{proof}


等式理論はすべて全称Horn理論である.よって,代数学で扱う多くの理論が全称Horn理論となる.
\Cref{Thm:HornHerbrandmodel}の代数学への応用として,
与えられた記号を使って定義される理論や構造を考えよう.

\begin{Ex} \label[Ex]{Ex:generatorforgrouptheory}
	記号\(a, b\)が与えられたとする.
	このとき,群論の言語\(\symcal{L}_{\GP}\)に\(a,b\)を加えた言語
	\(\symcal{L} = \symcal{L}_{\GP} \cup \Set{a, b}\)を考える.
	\(\symcal{L}\)理論\(\Gamma\)として,\(\GP\)に対して
	以下の3つの閉論理式を加えたものを考える:
	\begin{align*}
		\paren{a \obj{*} a} \obj{*} a & \objeq \obj{e},                      \\
		b \obj{*} b                   & \objeq \obj{e},                      \\
		b \obj{*} a                   & \objeq \paren{a \obj{*} a} \obj{*} b
	\end{align*}
	\(\symcal{L}\)における\(\Gamma\)のHerbrand構造\(\symcal{M}\)を考える.
	\(\GP\)の公理はすべて等式の全称閉包になっていることと\(\GP \subset \Gamma\)であることにより,
	任意の\(\varphi \in \GP\)に対して\(\Gamma \provable \varphi\)となる.
	従って\cref{lemma:herbrandsemantics}によって\(\symcal{M}\)は\(\GP\)の3つの公理すべてを充足する.
	よって\(\symcal{M}\)は\(\GP\)のモデルである.
	なお,\(\symcal{M}\)は上で与えた3つの基本関係式に従う2つの生成元\(a, b\)によって
	生成される位数6の二面体群となる.
\end{Ex}

\begin{Ex} \label[Ex]{Ex:polynomialringherbrandmodel}
	記号\(x_1, x_2, \dots, x_n\)が与えられたとする.
	環の理論\(\Ring\)のモデル\(\symcal{R}\)をひとつとる.
	環論の言語\(\symcal{L}_{\Ring}\)に\(x_1, x_2, \dots, x_n\)を加えて得られる
	言語\(\symcal{L} = \symcal{L}_{\Ring} \cup \Set{x_1, x_2, \dots, x_n}\)を考える.
	\(\languagewithname{\symcal{L}}{\symcal{R}}\)理論\(\Gamma\)を\(\Gamma = \Ring \cup \apply{\Diag}{\symcal{R}}\)と定めるとき,
	\(\languagewithname{\symcal{L}}{\symcal{R}}\)における\(\Gamma\)のHerbrand構造は通常
	\(x_1, x_2, \dots, x_n\)を不定元とする\(\symcal{R}\)上の多項式環と呼ばれ,
	\(\polynomial{\symcal{R}}{x_1, x_2, \dots, x_n}\)と表記される.
	\(\Ring\)も無矛盾な全称Horn理論であり,\(\symcal{R} \satisfy \Ring \cup \apply{\Diag}{\symcal{R}}\)
	だから\(\Gamma\)は無矛盾である.さらに,\(\apply{\Diag}{\symcal{R}}\)の元はすべて等式だから
	\(\Gamma\)は無矛盾な全称Horn理論である.
	ゆえに\cref{Thm:HornHerbrandmodel}によって
	\(\polynomial{\symcal{R}}{x_1, x_2, \dots, x_n}\)が\(\Gamma\)のモデルであることがわかる.
	\(\Ring \subset \Gamma\)だから,\(\polynomial{\symcal{R}}{x_1, x_2, \dots, x_n}\)は\(\Ring\)のモデルでもある.
\end{Ex}

\begin{Que} \label{Que:polynomialringherbrandmodel}
	\Cref{Ex:polynomialringherbrandmodel}で多項式環を定義する際,
	\(\Gamma\)として\(\Ring\)でも\(\Ring \cup \apply{\eDiag}{\symcal{R}}\)でもなく
	\(\Ring \cup \apply{\Diag}{\symcal{R}}\)を考えた.
	実数全体の集合を\(\RealNumbers\)を環とみなして得られる環上の多項式環や
	剰余環\(\Integers/2\Integers\)上の多項式環の計算を思い出し,
	なぜ\(\Ring\)だけではなく\(\apply{\Diag}{\symcal{R}}\)を加える必要があったのか,
	そしてなぜ\(\apply{\eDiag}{\symcal{R}}\)を加えなかったのかを考えよ.
\end{Que}


\section{定義による拡大} \label{sec:extensionbydefinition}

数学では,理論の途中で関数や関係を表した新しい記号を定義して表記の簡素化を行うことがある.
数理論理学においては,言語に関数記号や関係記号を追加してその記号の定義となる主張を新しく公理として組み込むことに相当する.
すると,項や論理式が指すものに追加した記号を使用したものが追加されることになり,
その理論が有する定理も追加した記号に関するものが登場するようになる.
しかし,このような記号の導入によっては「理論は変わらない」と考えるのが普通であろう.
本節では,このような「定義による拡大」によって「理論が本質的には変わらない」ということを検証することとする.

まずは理論の拡大について定義しよう.

\begin{Def} \label{Def:extensiontheory}
	\(\symcal{L}, \symcal{L}'\)を言語とし,\(\symcal{L} \subset \symcal{L}'\)であるとする.
	このとき,\(\symcal{L}'\)理論\(\Gamma\)が\(\symcal{L}\)理論\(\Gamma\)の%
	\index[widx]{かくだい@(理論の)拡大}%
	\term{拡大}であるとは,任意の\(\symcal{L}\)論理式\(\varphi\)に対して
	\begin{equation}
		\Gamma \satisfy \varphi \metaimplies \Gamma' \satisfy \varphi
		\label{eq:extensiontheory}
	\end{equation}
	が成り立つことをいう.
	また,\cref{eq:extensiontheory}に加えてその逆である
	\begin{equation}
		\Gamma' \satisfy \varphi \metaimplies \Gamma \satisfy \varphi
		\label{eq:conservationextensiontheoryoneside}
	\end{equation}
	も成り立つ場合,\(\Gamma'\)は\(\Gamma\)の%
	\index[widx]{かくだい@(理論の)拡大!ほぞんかくだい@保存---}%
	\term{保存拡大}という.すなわち,\(\Gamma'\)が\(\Gamma\)の保存拡大であるとは,
	任意の\(\symcal{L}\)論理式\(\varphi\)に対して
	\begin{equation}
		\Gamma \satisfy \varphi \metaequivalent \Gamma' \satisfy \varphi
		\label{eq:conservationextensiontheory}
	\end{equation}
	が成り立つことをいう.
\end{Def}

\begin{Def} \label{Def:extensionbydefinition}
	\(\symcal{L}\)を言語とし,\(\Gamma\)を\(\symcal{L}\)理論とする.
	\(\symcal{L} \subset \symcal{L}'\)を満たす言語\(\symcal{L}'\)を考える.

	\(r \in \symcal{L}' \setminus \symcal{L}\)をアリティ\(n\)の関係記号とする.
	\(x_1, x_2, \dots, x_n\)以外の変数記号が自由出現しない
	\(\symcal{L}\)論理式\(\apply{\varphi}{x_1, x_2, \dots, x_n}\)に対する
	\(\symcal{L}'\)論理式
	\begin{equation}
		\forall x_1 \forall x_2 \dotsb \forall x_n \paren{\apply{\varphi}{x_1, x_2, \dots, x_n} \formulaequiv \apply{r}{x_1, x_2, \dots, x_n}}.
		\label{eq:extensionbydefinitionrelationalsymbol}
	\end{equation}
	を,\(\symcal{L}, \Gamma\)における\(r\)の
	\index[widx]{ていぎ@(理論の拡大における)定義}%
	\term{定義}という.

	\(c \in \symcal{L}' \setminus \symcal{L}\)を定数記号とする.
	\begin{equation}
		\Gamma \provable \uexists y \apply{\theta}{y}
		\label{eq:definitionconstantsymbol}
	\end{equation}
	を満たす\(y\)以外の変数記号が自由出現しない
	\(\symcal{L}\)論理式\(\apply{\theta}{y}\)に対する
	\(\symcal{L}'\)論理式
	\begin{equation}
		\apply{\theta}{c}
		\label{eq:extensionbydefinitionconstantsymbol}
	\end{equation}
	を,\(\symcal{L}, \Gamma\)における\(c\)の%
	\index[widx]{ていぎ@(理論の拡大における)定義}%
	\term{定義}という.

	\(f \in \symcal{L}' \setminus \symcal{L}\)をアリティ\(n\)の関数記号とする.
	\begin{equation}
		\Gamma \provable \forall x_1 \forall x_2 \dotsb \forall x_n \uexists y \apply{\theta}{y}
		\label{eq:definitionfunctionsymbol}
	\end{equation}
	を満たす\(x_1, x_2, \dots, x_n, y\)以外の変数記号が自由出現しない
	\(\symcal{L}\)論理式\(\apply{\theta}{x_1, x_2, \dots, x_n, y}\)に対する
	\(\symcal{L}'\)論理式
	\begin{equation}
		\forall x_1 \forall x_2 \dotsb \forall x_n \apply{\theta}{x_1, x_2, \dots, x_n, \apply{f}{x_1, x_2, \dots, x_n}}
		\label{eq:extensionbydefinitionfunctionsymbol}
	\end{equation}
	を,\(\symcal{L}, \Gamma\)における\(f\)の%
	\index[widx]{ていぎ@(理論の拡大における)定義}%
	\term{定義}という.

	\(\symcal{L}' \setminus \symcal{L}\)に属するすべての記号の定義を\(\Gamma\)に
	加えて得られる\(\symcal{L}'\)理論を\(\Gamma\)の%
	\index[widx]{かくだい@(理論の)拡大!ていぎによるかくだい@定義による---}%
	\term{定義による拡大}という.
\end{Def}

\begin{Thm} \label[Thm]{Thm:extensionbydefinition}
	\(\symcal{L}, \symcal{L}'\)を言語とし,\(\symcal{L} \subset \symcal{L}'\)であるとする.
	また,\(\symcal{L}'\)理論\(\Gamma'\)を\(\symcal{L}\)理論\(\Gamma\)の定義による拡大とする.
	以下,論理式\(\varphi\)の全称閉包を\(\forall \forall \varphi\)と表記することとする.
	このとき,以下が成り立つ:
	\begin{enumerate}
		\item \(\Gamma'\)は\(\Gamma\)の保存拡大である.
		\item \(t\)を任意の\(\symcal{L}'\)項とするとき,
		      \(t\)がもつ変数記号とそれらとは異なるもう1つの変数記号\(y\)が自由出現する
		      \(\symcal{L}\)論理式\(\apply{\theta}{y}\)が存在して,
		      \(\Gamma \provable \forall \forall \uexists y \apply{\theta}{y}\)かつ
		      \(\Gamma' \provable \forall \forall \apply{\theta}{t}\)となる.
		\item 任意の\(\symcal{L}'\)論理式\(\varphi\)に対して,\(\varphi\)と自由出現する変数記号が同じ
		      \(\symcal{L}\)論理式\(\hat{\varphi}\)が存在して,
		      \(\Gamma' \provable \forall \forall \paren{\varphi \formulaequiv \hat{\varphi}}\)が成り立つ.
	\end{enumerate}
\end{Thm}

\begin{proof}
	まず1.を示す.\(\Gamma\)が矛盾している場合には明らか.
	\(\Gamma \subset \Gamma'\)なので,\(\Gamma'\)は\(\Gamma\)の拡大である.
	\(\Gamma' \provable \varphi\)を満たすが\(\Gamma \provable \varphi\)が成り立たない
	\(\symcal{L}\)論理式\(\varphi\)が存在したと仮定する.
	このとき,\(\Gamma\)理論\(\Gamma \cup \Set{\lnot \varphi}\)は無矛盾であるからHenkinの定理によって
	そのモデル\(\symcal{M}\)が存在する.
	\(\symcal{M}\)の対象領域を\(M\)とする.
	いま,\(\symcal{L}' \setminus \symcal{L}\)の各記号の解釈を与え,
	\(\symcal{L}\)構造\(\symcal{M}\)を\(\symcal{L}'\)構造\(\symcal{M}'\)に拡張する.

	\(r \in \symcal{L}' \setminus \symcal{L}\)がアリティ\(n\)の関係記号の場合を考える.
	\(r\)の\(\symcal{M}'\)による解釈は,
	\(\symcal{L}, \Gamma\)における\(r\)の定義となる\cref{eq:extensionbydefinitionrelationalsymbol}における
	論理式\(\apply{\varphi}{x_1, x_2, \dots, x_n}\)を用いて
	\[
		\pair{t_1, t_2, \dots, t_n} \in \interpretation{\symcal{M}'}{r} \metaequivalent \symcal{M} \satisfy \apply{\varphi}{t_1, t_2, \dots, t_n}
	\]
	によって定める.

	\(c \in \symcal{L}' \setminus \symcal{L}\)が定数記号の場合を考える.
	このとき,\(\symcal{L}, \Gamma\)における\(c\)の定義となる\cref{eq:extensionbydefinitionconstantsymbol}における
	論理式\(\apply{\theta}{y}\)について,\cref{eq:definitionconstantsymbol}から
	\(\symcal{M} \satisfy \uexists y \apply{\theta}{y}\)だから\(\symcal{M} \satisfy \apply{\theta}{c_a}\)を満たす\(a \in M\)がただ1つ存在する.
	ここで,\(c_a\)は\(a\)の名前である.
	この\(a\)を用いて,\(c\)の\(\symcal{M}'\)による解釈は
	\[
		\interpretation{\symcal{M}'}{c} = a
	\]
	と定める.

	\(f \in \symcal{L}' \setminus \symcal{L}\)がアリティ\(n\)の関数記号の場合を考える.
	このとき,\(\symcal{L}, \Gamma\)における\(f\)の定義となる\cref{eq:extensionbydefinitionfunctionsymbol}における
	論理式\(\apply{\theta}{x_1, x_2, \dots, x_n, y}\)について,\cref{eq:definitionfunctionsymbol}から
	\[
		\symcal{M} \satisfy \forall x_1 \forall x_2 \dotsb \forall x_n \uexists y \apply{\theta}{x_1, x_2, \dots, x_n, y}
	\]
	である.
	ゆえに,任意の\(a_1, a_2, \dots, a_n\)に対して
	\[
		\symcal{M} \satisfy \apply{\theta}{c_1, c_2, \dots, c_n, c_b}
	\]
	を満たす\(b \in M\)がただ1つ存在する.ここで,\(i = 1, 2, \dots, n\)に対して\(c_i\)は\(a_i\)の名前であり,
	\(c_b\)は\(b\)の名前である.
	\(f\)の\(\symcal{M}\)による解釈は,\(a_1, a_2, \dots, a_n \in M\)を上記\(b \in M\)に対応させる写像とする.

	上記によって定義された\(\symcal{M}'\)は\(\Gamma'\)のモデルであり,しかも\(\symcal{M}' \satisfy \lnot \varphi\)を満たす.
	しかし\(\Gamma' \provable \varphi\)から健全性定理によって\(\symcal{M}' \satisfy \varphi\)が得られるので矛盾である.
	ゆえに,\(\Gamma'\)は\(\Gamma\)の保存拡大である.

	2.については項\(t\)の構成に関する帰納法により,3.については論理式\(\varphi\)の構成に関する帰納法によってそれぞれ示すことができる.
\end{proof}

\begin{Que} \label[Que]{Que:extensionbydefinition}
	\Cref{Thm:extensionbydefinition}の2.と3.を証明せよ.
\end{Que}

\backmatter

\RenewDocumentCommand{\presectionname}{}{}
\RenewDocumentCommand{\presubsectionname}{}{}
\chapter{演習問題解答} \label{chap:answer}

\section*{\Cref{chap:formulize}}

\subsection*{\Cref{que:transitiveclosure}}

\(X\)上の二項関係として直積集合\(X \times X\)を考えると,\(X \times X\)は\(\prec\)を含む推移的な\(X\)上の二項関係である.
よって,\(\prec\)を含む推移的な\(X\)上の二項関係の全体を\(\symcal{R}\)とおくと,\(\symcal{R}\)は空でない.
\(\mathord{\prec'} = \bigcap \symcal{R}\)は,\(\prec\)を含む推移的な\(X\)上の二項関係で包含関係に対して最小のものである.これを示そう.

まず,\(\prec'\)が\(\prec\)を含む推移的な\(X\)上の二項関係であることを示す.
\(x \prec y\)とする.このとき,すべての\(R \in \symcal{R}\)に対して
\(x \mathrel{R} y\)となるので\(x \prec' y\)となる.
\(\prec' \subset X \times X\)であることもあわせて\(\prec'\)は\(\prec\)を含む\(X\)上の二項関係であることがわかる.
\(\prec'\)の推移性を示す.\(x \prec' y\)かつ\(y \prec' z\)とする.
任意の\(R \in \symcal{R}\)に対し,\(x \mathrel{R} y\)かつ\(y \mathrel{R} z\)となる.
\(R\)は推移的なので\(x \mathrel{R} z\)である.ゆえに\(x \prec' z\)となり,\(\prec'\)が推移的であることが従う.

最後に\(\prec'\)が\(\prec\)を含む推移的な\(X\)上の二項関係で包含関係に関して最小のものであることを示そう.
\(x \prec' y\)とする.\(\prec\)を含む推移的な\(X\)上の二項関係\(R\)を任意にとる.
このとき,\(R \in \symcal{R}\)であるから\(x \mathrel{R} y\)でなければならない.
従って,集合として\(\prec' \subset R\)が成り立つので,\(\prec'\)が\(\prec\)を含む推移的な\(X\)上の二項関係で包含関係に対して最小のものであることがわかる.

\subsection*{\Cref{que:proofinductivedefinition}}

一意性について:
任意の\(x \in X\)に対して\cref{eq:inductivedifinition}を満たす写像\(f, g \colon X \to Y\)がとれたとして,
すべての\(x \in X\)に対して\(\apply{f}{x} = \apply{g}{x}\)が成り立つことを整礎帰納法によって示す.
(\Cref{Thm:well-foundedinduction}を直接適用できるような形にしたければ,
\(A = \Set{x \in X | \apply{f}{x} = \apply{g}{x}}\)とおき,\(A = X\)を示せばよい.)
\(x \in X\)を1つとり,\(y \prec x\)を満たす任意の\(y \in X\)に対して\(\apply{f}{y} = \apply{g}{y}\)が成り立つと仮定する.
このとき,\(\Set{y | y \prec x}\)上ではつねに\(\apply{f}{y} = \apply{g}{y}\)だから
\(f \restriction \Set{y | y \prec x} = g \restriction \Set{y | y \prec x}\)であり,
さらに\(f,g\)はともに\cref{eq:inductivedifinition}を満たすことから
\[
	\apply{f}{x} = \apply{G}{f \restriction \Set{y | y \prec x}} = \apply{G}{ g \restriction \Set{y | y \prec x}} = \apply{g}{x}
\]
となることから\(\apply{f}{x} = \apply{g}{x}\)が成り立つ.よって\cref{Thm:well-foundedinduction}によって
すべての\(x \in X\)に対して\(\apply{f}{x} = \apply{g}{x}\)が成り立つので\(f = g\)が従う.

存在性について:
\Cref{Thm:inductivedifinition}の証明で精密化されていないのは,
任意の\(x \in X\)に対して「写像\(f \colon \Set{y | y \prec x} \to Y \)で\(y \prec x\)となる任意の\(y \in X\)に対して
\(\apply{f}{y} = \apply{G}{f \restriction \Set{z | z \prec y}}\)となるものが存在する」
ことが成り立つことの証明である.これを整礎帰納法によって示す.
\(w \prec x\)なる任意の\(w\)においてはこの主張が成り立つものとする.
このとき,一意性の証明とまったく同様にして各\(w\)に対してこの主張を成り立たせる
写像\(f\)がただ1つであることがわかる.これを\(f_w\)と表すことにする.
写像\(f_x \colon \Set{y | y \prec x} \to Y\)を以下のように定める:
各\(y \in \Set{z | z \prec x}\)に対し,\(y \prec z\)かつ\(z \prec x\)となる\(z \in X\)が存在するならば
\(X\)の空でない部分集合
\(\Set{z \in X | \text{\(y \prec z\)かつ\(z \prec x\)}}\)の極小元を\(m\)として
\(\apply{f}{y} = \apply{f_m}{y}\)とする.
そのような\(z\)が存在しないならば\(\apply{f_x}{y} = \apply{G}{f_y}\)とする.
この\(f_x\)は\(y \prec x\)となるすべての\(y \in X\)に対して\(\apply{f}{y} = \apply{G}{f \restriction \Set{z | z \prec y}}\)
を満たす.


\subsection*{\Cref{Que:recursivedefinition}}

自然数\(n\)に対する集合\(L_n \subset \KleeneClosure{\Sigma}\)を,以下のように帰納的に定義する:
まず,\(L_0\)を数字全体の集合とする.各\(n\)について,\(L_n\)から\(L_{n+1}\)を
\begin{equation*}
	L_{n+1} = L_n \cup \Set{\obj{\lparen} t_1 \obj{+} t_2 \obj{\rparen} | t_1, t_2 \in L_n}
\end{equation*}
によって定める.

このようにして定義される列\(L_0, L_1, \dots\)は,\Cref{Que:recursivedefinition}における規則1, 2, 3をすべて満たす.
数式全体の集合\(L\)は,この列\(L_0, L_1, \dots\)を使って
\begin{equation*}
	L = \bigcup_{n \in \NaturalNumbers} L_n
\end{equation*}
と定義できる.

\subsection*{\Cref{Que:well-foundedexample}}

まず,\(\prec'\)の非反射性を示そう.
\(t \prec' t\)となる\(t \in L\)が存在すると仮定すると,\cref{eq:recursivedefinitiondepthorder}により
この\(t\)は\(\apply{\depth}{t} < \apply{\depth}{t}\)
を満たさなければならず\(<\)の非反射性に矛盾する.よって\(\prec'\)は非反射的である.

次に,\(\prec'\)の推移性を示す.
\(t_1 \prec' t_2\)かつ\(t_2 \prec' t_3\)となる\(t_1, t_2, t_3 \in L\)を任意にとる.
このとき,\cref{eq:recursivedefinitiondepthorder}により
\(\apply{\depth}{t_1} < \apply{\depth}{t_2}\)かつ\(\apply{\depth}{t_2} < \apply{\depth}{t_3}\)である.
よって\(<\)の推移性から\(\apply{\depth}{t_1} < \apply{\depth}{t_3}\)となるので,\cref{eq:recursivedefinitiondepthorder}から
\(t_1 \prec' t_3\)となることがわかる.従って,\(\prec'\)は推移的である.

\(\prec'\)が整礎であることを示そう.
\(L'\)を\(L\)の空でない部分集合とする.このとき,\(\NaturalNumbers\)の部分集合
\begin{equation*}
	N = \Set{\apply{\depth}{t} | t \in L'}
\end{equation*}
は空でない.よって,\(\NaturalNumbers\)上の通常の大小関係\(<\)の整礎性から\(N\)の極小元\(x\)をとることができる.
\(x = \apply{\depth}{t}\)を満たす\(t\)が1つとれるが,
\(x\)が\(N\)の極小元であることから\(y < x\)を満たす\(N\)の元\(y\)
は存在せず,\cref{eq:recursivedefinitiondepthorder}から\(u \prec' t\)
を満たす\(u \in L'\)も存在しない.
従って\(t\)は\(L'\)の極小元であることから,\(\prec'\)は整礎な半順序であることがわかる.

なお,上記\(t\)は\(L'\)に対して一意であるとは限らない.
例えば,\(L'\)が数字を元としてもつならば,その数字すべてが\(L'\)の極小元である.

\subsection*{\Cref{Que:recursivedefinitiondepth}}

\(t\)が数字ならば\(t \in L_0\)なので,\(\apply{\depth}{t} = 0 + 1 = 1\)であり,
\cref{eq:Que:recursivedefinitiondepth_numeral}が成り立つ.
\(t_1, t_2 \in L\)とする.
このとき,\(t_1 \in L_n\)となる\(n\)のうち最小のもの\(m = \apply{\depth}{t_1} - 1\)と,
\(t_2 \in L_n\)となる\(n\)のうち最小のもの\(l = \apply{\depth}{t_2} - 1\)がそれぞれとれる.
\(m \leq l\)ならば,\(l\)は\(t_1 \in L_n\)と\(t_2 \in L_n\)をともに満たす\(n\)のうち最小のものである.
よって\(l + 1\)は\(\obj{\lparen} t_1 \obj{+} t_2 \obj{\rparen} \in L_n\)を満たす\(n\)のうち最小のものであって,
\begin{equation*}
	\apply{\depth}{\obj{\lparen} t_1 \obj{+} t_2 \obj{\rparen}} = l + 1 + 1 = \apply{\depth}{t_2} + 1
\end{equation*}
となる.\(l < m\)の場合も同様に
\begin{equation*}
	\apply{\depth}{\obj{\lparen} t_1 \obj{+} t_2 \obj{\rparen}} = \apply{\depth}{t_1} + 1
\end{equation*}
となる.
\(m \leq l\)のときは\(\apply{\depth}{t_1} \leq \apply{\depth}{t_2}\)であり
\(l < m\)のときは\(\apply{\depth}{t_2} < \apply{\depth}{t_1}\)であるから,
\cref{eq:Que:recursivedefinitiondepth_inductionstep}が成り立つことがわかる.

\subsection*{\Cref{Que:inductivedifinition}}

\Cref{Que:well-foundedexample}の結果により,\(\prec\)の推移閉包\(\prec'\)は整礎な半順序である.
そこで,
写像\(G \colon \bigcup_{t \in L} \NaturalNumbers^{\Set{s \in L | s \prec' t }} \to \NaturalNumbers\)を次のように定める:
\(f \in \bigcup_{t \in L} \NaturalNumbers^{\Set{s \in | s \prec' t}}\)をとると,
\(f\)はある\(t \in L\)についての\(\Set{s \in | s \prec' t}\)から\(\NaturalNumbers\)への写像である.
\(t\)が極小元である場合,すなわち\(f\)が空写像である場合は
\[
	\apply{G}{f} = 0
\]
とする.
なお,\(L\)の極小元は数字に限る.
\(t\)が極小元でない場合,\(t' \prec' t\)となる\(t' \in L\)が存在する.
このとき,\(t_1 \prec t\)となる\(t_1 \in L\)もとれるので\(t_2 \prec t\)となる\(t_2 \in L\)
もとれる.このような\(t_1, t_2\)の組は一意に定まるので,
\[
	\apply{G}{f} = \apply{f}{t_1} + \apply{f}{t_2} + 1
\]
とする.
このとき,\cref{Thm:inductivedifinition}により,写像\({\size} \colon L \to \NaturalNumbers\)
で任意の\(t \in L\)に対して
\[
	\apply{\size}{t} = \apply{G}{ \size \restriction \Set{s \in L | s \prec' t}}
\]
を満たすものがただ1つ定まる.
この\(\size\)が\cref{Que:inductivedifinition}で定義しようとした数式の大きさを表す写像にほかならない.


\subsection*{\Cref{Que:structuralinductionexample}}

\(t \in L\)を1つとり,\(u \prec' t\)を満たすすべての\(u\)に対して\cref{eq:sizedepthinequality}が成り立つと仮定する.
このとき,この\(t\)についても\cref{eq:sizedepthinequality}が成り立つことを示す.

\(t\)が数字ならば,\(\apply{\size}{t}\)と\(\apply{\depth}{t}\)はともに1なので\cref{eq:sizedepthinequality}は成り立つ.
\(t\)が数字でないとき,\(t\)は\(t_1, t_2 \in L\)を用いて\(\obj{\lparen} t_1 \obj{+} t_2 \obj{\rparen}\)と表される.
このとき,\(t_1 \prec' t\)かつ\(t_2 \prec' t\)なので,\(t_1, t_2\)はともに\cref{eq:sizedepthinequality}を満たす.
よって
\begin{align*}
	\apply{\size}{t} & = \apply{\size}{t_1} + \apply{\size}{t_2} + 1                      \\
	                 & \leq 2^{\apply{\depth}{t_1}} - 1 + 2^{\apply{\depth}{t_2}} - 1 + 1 \\
	                 & = 2^{\max\Set{\apply{\depth}{t_1}, \apply{\depth}{t_2}} + 1} - 1   \\
	                 & = 2^{\apply{\depth}{t}} - 1
\end{align*}
だから,この\(t\)についても\cref{eq:sizedepthinequality}が成り立つ.

以上より,整礎帰納法によって
すべての\(t \in L\)に対して\cref{eq:sizedepthinequality}が成り立つ.


\section*{\Cref{chap:syntax}}

\subsection*{\Cref{Que:termexample}}

\(\obj{e}\)は変数記号,\(\obj{x}\)は変数記号なので,どちらも\(\symcal{L}_1\)項である.
これと\(\obj{\ast}\)がアリティ2の関数記号であることにより,\(\apply{\obj{\ast}}{\obj{e}, \obj{x}}\)が
\(\symcal{L}_1\)項であることがわかる.
\(\obj{y}\)も変数記号であり\(\symcal{L_1}\)項なので,
\(\apply{\obj{\ast}}{\apply{\obj{\ast}}{\obj{e}, \obj{x}}, \obj{y}}\)は\(\symcal{L}_1\)項である.

また,\(\obj{e}\apply{\mathord{\obj{\ast}}}{\obj{e}, \obj{e}}\)は変数記号でも定数記号でもないため,
これが\(\symcal{L}_1\)項であるためには
\cref{Def:term}における3番目の規則を最後に適用していなければならない.
このとき最初の文字は関数記号である必要があるが,\(\obj{e}\)は定数記号であり関数記号ではない.
従って,\(\obj{e}\apply{\mathord{\obj{\ast}}}{\obj{e}, \obj{e}}\)は\(\symcal{L}_1\)項ではない.

\subsection*{\Cref{Que:logicalexpression}}

\(\obj{x}, \obj{y}\)は変数記号であり,従って\(\symcal{L}_1\)項である.
また,\(\obj{\ast}\)がアリティ2の関数記号であることから,
\(\apply{\mathord{\obj{\ast}}}{\obj{x}, \obj{y}}\)と
\(\apply{\mathord{\obj{\ast}}}{\obj{y}, \obj{x}}\)はいずれも\(\symcal{L}_1\)項である.
よって,
\(\paren{\apply{\mathord{\obj{\ast}}}{\obj{y}, \obj{x}} \objeq \apply{\mathord{\obj{\ast}}}{\obj{y}, \obj{x}}}\)
は\(\symcal{L}_1\)論理式である.

\(\obj{\leq}\)がアリティ2の関係記号であることから,
\(\apply{\mathord{\obj{\leq}}}{\obj{x}, \obj{y}}\)は\(\symcal{L}_2\)論理式である.
ゆえに,\(\paren{\forall \obj{y}\apply{\mathord{\obj{\leq}}}{\obj{x}, \obj{y}}}\)
は\(\symcal{L}_2\)論理式である.
このことから\(\paren{\forall \obj{x}\paren{\forall \obj{y}\apply{\mathord{\obj{\leq}}}{\obj{x}, \obj{y}}}}\)
が\(\symcal{L}_2\)論理式であることも従う.

さて,\(\paren{\apply{\exists \obj{x}}{\obj{x}}}\)が\(\symcal{L}\)論理式であるとすれば,
\(\obj{x}\)は\(\symcal{L}\)論理式でなければならない.しかし,変数記号が論理式であることはないので
\(\paren{\apply{\exists \obj{x}}{\obj{x}}}\)は\(\symcal{L}\)論理式ではない.
一方で,\(\paren{\obj{x} \objeq \obj{x}}\)は\(\symcal{L}\)論理式であるから
\(\paren{\apply{\exists \obj{x}}{\obj{x} \objeq \obj{x}}}\)は\(\symcal{L}\)論理式である.

\(\apply{\forall \obj{x}}{\obj{x} \obj{\ast} \obj{e} \objeq \obj{x}}\)が\(\symcal{L}_1\)論理式であるとすれば,
これまでの議論と同様にして\(\obj{x} \obj{\ast} \obj{e}\)が\(\symcal{L}_1\)項でなければならないことがわかる.
しかし,\(\obj{\ast}\)の次の文字が開きカッコ「\(\lparen\)」ではないので
これは\(\symcal{L}_1\)項とはなりえない.よって,
\(\apply{\forall \obj{x}}{\obj{x} \obj{\ast} \obj{e} \objeq \obj{x}}\)は\(\symcal{L}_1\)論理式ではない.
\(\apply{\forall \obj{x}}{\obj{x} \objeq \obj{x}}\)については,1文字目が開きカッコ「\(\lparen\)」でも
関係記号でもないことから\(\symcal{L}_1\)論理式でないことが従う.

\subsection*{\Cref{Que:invalidtheory}}

2つ目の\(\symcal{L}\)論理式からは,\(\obj{e}\)が変数記号であることが伺える.
しかしそうだとすると最後の\(\symcal{L}\)論理式
\[
	\forall \obj{x} \exists \obj{y} \paren{\obj{y} \obj{\ast} \obj{x} \objeq \obj{e}}
\]
が\(\symcal{L}\)閉論理式にならない.問題文で挙げた3つの論理式からなる集合が
\(\symcal{L}\)理論になっていないという意味で構文的に不適切である.

最後の\(\symcal{L}\)論理式を
\[
	\exists \obj{e} \forall \obj{x} \exists \obj{y} \paren{\obj{y} \obj{\ast} \obj{x} \objeq \obj{e}}
\]
としてしまえば\(\symcal{L}\)閉論理式になるので,問題文で挙げた3つの\(\symcal{L}\)論理式のうち
3つ目だけをこちらに置き換えたものは\(\symcal{L}\)理論となる.
しかし,この\(\symcal{L}\)理論は「群論の公理化」としては不適切であって,
2つ目と3つ目の\(\symcal{L}\)論理式をまとめて
\[
	\exists \obj{e} \paren{
		\forall \obj{x} \paren{\obj{e} \obj{\ast} \obj{x} \objeq \obj{x}}
		\land \forall \obj{x} \exists \obj{y} \paren{\obj{y} \obj{\ast} \obj{x} \objeq \obj{e}}
	}
\]
とするのが正しい.
なぜそうしなければならないのか,
そしてなぜそうすれば「よい」のかについては,この段階ではわからない.

\subsection*{\Cref{Que:ProjectiveGeometrydual}}

\begin{enumerate}
	\NewDocumentCommand{\LiesOn}{}{\mathrel{\obj{\varepsilon}}}
	\item \(\forall \obj{x} \paren{\apply{\obj{L}}{\obj{x}} \formulaequiv \lnot \apply{\obj{P}}{\obj{x}}},\)
	\item \(\forall \obj{x} \forall \obj{y} \paren{\obj{y} \LiesOn \obj{x} \to \apply{\obj{L}}{\obj{x}} \land \apply{\obj{P}}{\obj{y}}},\)
	\item \(\forall \obj{x} \forall \obj{y} \paren{\apply{\obj{L}}{\obj{x}} \land \apply{\obj{L}}{\obj{y}} \land \lnot \paren{\obj{x} \objeq \obj{y}}
		      \to \uexists z \paren{\obj{z} \LiesOn \obj{x} \land \obj{z} \LiesOn \obj{y}}},\)
	\item \(\forall \obj{x} \forall \obj{y} \paren{\paren{\apply{\obj{P}}{\obj{x}} \land \apply{\obj{P}}{y}} \land \lnot \paren{\obj{x} \objeq \obj{y}}
		      \to \uexists z \paren{\obj{x} \LiesOn \obj{z} \land \obj{y} \LiesOn \obj{z}}},\)
	\item \(C'\)を
	      \[
		      \lnot \exists \obj{w} \paren{\obj{w} \LiesOn x \land \obj{w} \LiesOn y \land \obj{w} \LiesOn z}
	      \]
	      の略記としたときの
	      \begin{align*}
		       & \exists \obj{a} \exists \obj{b} \exists \obj{c} \exists \obj{d} \\
		       & \paren{
			      \apply{n}{\obj{a}, \obj{b}, \obj{c}, \obj{d}}
			      \land \apply{C'}{\obj{a}, \obj{b}, \obj{c}}
			      \land \apply{C'}{\obj{a}, \obj{b}, \obj{d}}
			      \land \apply{C'}{\obj{a}, \obj{c}, \obj{d}}
			      \land \apply{C'}{\obj{b}, \obj{c}, \obj{d}}
		      }.
	      \end{align*}
\end{enumerate}

双対をとる操作によって,1と2は意味が変わらず,3と4は互いに入れ替わり,5については
「どの3直線も共線でないような相異なる4直線が存在する」という主張に対応する論理式になったことに注意しよう.
上記\(\apply{C'}{x, y, z}\)は「3直線\(x, y, z\)は共点でない」という主張に対応すると考えることができる.

\section*{\Cref{chap:semantics}}

\subsection*{\Cref{Que:satisfy}}

\(\symcal{M}\)が\cref{Ex:grouptheory}で述べた3つの\(\symcal{L}\)文すべてを充足することを確かめよう.

\begin{align*}
	\varphi_1 \colon & \forall \obj{x} \forall \obj{y} \forall \obj{z}
	\paren{\paren{\obj{x} \obj{\ast} \obj{y}} \obj{\ast} \obj{z} \objeq \obj{x} \obj{\ast} \paren{\obj{y} \obj{\ast} \obj{z}}}, \\
	\varphi_2 \colon & \forall \obj{x} \paren{\obj{e} \obj{\ast} \obj{x} \objeq \obj{x}},                                       \\
	\varphi_3 \colon & \forall \obj{x} \paren{\obj{x}^{\obj{-1}} \obj{\ast} \obj{x} \objeq \obj{e}}
\end{align*}
とする.
このとき,
\begin{align*}
	\symcal{M} & \satisfy \varphi_1, \\
	\symcal{M} & \satisfy \varphi_2, \\
	\symcal{M} & \satisfy \varphi_3  \\
\end{align*}
となるための必要十分条件を考える.
まず\(\varphi_1\)については
\begin{align*}
	                & \symcal{M} \satisfy \varphi_1                                                                                                                 \\
	\metaequivalent & \text{すべての\(a \in \Integers\)に対して\(\symcal{M} \satisfy \forall \obj{y} \forall \obj{z} \subst{\varphi_1}{c_a / \obj{x}}\)}                    \\
	\metaequivalent & \text{すべての\(a, b \in \Integers\)に対して\(\symcal{M} \satisfy \forall \obj{z} \subst{\subst{\varphi_1}{c_a / \obj{x}}}{c_b / \obj{y}}\)}          \\
	\metaequivalent & \text{すべての\(a, b, c\in \Integers\)に対して\(\symcal{M} \satisfy \subst{\subst{\subst{\varphi_1}{c_a / \obj{x}}}{c_b / \obj{y}}}{c_c / \obj{z}}\)} \\
	\metaequivalent & \text{すべての\(a, b, c\in \Integers\)に対して
		\(\interpretation{\symcal{M}}{\paren{\paren{c_a \obj{\ast} c_b} \obj{\ast} c_c}}
	= \interpretation{\symcal{M}}{\paren{c_a \obj{\ast} \paren{c_b \obj{\ast} c_c}}}\)}                                                                             \\
	\metaequivalent & \text{すべての\(a, b, c \in \Integers\)に対して
		\(\interpretation{\symcal{M}}{\paren{c_a \obj{\ast} c_b}} \interpretation{\symcal{M}}{\obj{\ast}} \interpretation{\symcal{M}}{c_c}
	= \interpretation{\symcal{M}}{c_a} \interpretation{\symcal{M}}{\obj{\ast}} \interpretation{\symcal{M}}{\paren{c_b \obj{\ast} c_c}} \)}                          \\
	\metaequivalent & \text{すべての\(a, b, c \in \Integers\)に対して\(
		\paren{\interpretation{\symcal{M}}{c_a} \interpretation{\symcal{M}}{\obj{\ast}} \interpretation{\symcal{M}}{c_b}} + c
		= a + \paren{\interpretation{\symcal{M}}{c_b} \interpretation{\symcal{M}}{\obj{\ast}} \interpretation{\symcal{M}}{c_c}}
	\)}                                                                                                                                                             \\
	\metaequivalent & \text{すべての\(a, b, c \in \Integers\)に対して\(
		\paren{a + b} + c = a + \paren{b + c}
		\)}
\end{align*}
となる.\(\varphi_2, \varphi_3\)についても同様にして
\begin{align*}
	\symcal{M} \satisfy \varphi_2 & \metaequivalent \text{すべての\(a \in \Integers\)に対して\(0 + a = a\)},         \\
	\symcal{M} \satisfy \varphi_3 & \metaequivalent \text{すべての\(a \in \Integers\)に対して\(\paren{-a} + a = 0\)}
\end{align*}
となることがわかるので,\(\symcal{M}\)が\(\varphi_1, \varphi_2, \varphi_3\)のすべてを充足することがわかる.
\Cref{Ex:satisfy}で述べた\(\symcal{L}\)構造\(\symcal{M}'\)が\(\varphi_1\)のみを充足することも同様にして確かめられる.

\subsection*{\Cref{Que:model}}

\Cref{Ex:satisfy}で挙げた\(\symcal{L}\)構造\(\symcal{M}\)は\(\varphi\)を充足する\(T\)のモデルである.
また,有限集合\(I_n = \Set{1, 2, \dots, n}\)上の置換(全単射)\(\sigma \colon I_n \to I_n\)全体の集合\(S_n\)を対象領域とする
\(\symcal{L}\)構造\(\symcal{M}'\)を次のように定義する:
\begin{align*}
	\interpretation{\symcal{M}'}{\obj{e}}       & = \symup{id}, \\
	\interpretation{\symcal{M}'}{\obj{\ast}}    & = \circ,      \\
	\interpretation{\symcal{M}'}{\obj{{}^{-1}}} & = {}^{-1}
\end{align*}
ここで,\(\symup{id} \colon S_n \to S_n\)は\(S_n\)上の恒等写像(つまり\(x \mapsto x\)という写像),
\(\circ \colon S_n \to S_n\)は写像の合成をとる写像,
\({}^{-1} \colon S_n \to S_n\)は逆写像をとる写像である.
このようにして定義される\(\symcal{L}\)構造\(\symcal{M}'\)は,\(n \geq 3\)ならば\(\varphi\)を充足しない\(T\)のモデルである.

\Cref{Que:model}は,通常の数学における「可換な群とそうでない群を1つずつ挙げよ」という問題に相当することに注意せよ.

\subsection*{\Cref{Que:grouptheoryaxiomize}}

2つの元からなる集合\(M = \Set{a, b}\)に対し,\(M\)を対象領域とする\(\symcal{L}_1\)構造\(\symcal{M}\)を以下によって定義する:
写像\(\mathord{\ast} \colon M \times M \to M\)を以下の表によって定義し,
\(\interpretation{\symcal{M}}{\obj{\ast}} = \mathord{\ast}\)とする.
\begin{table}[htbp]
	\centering
	\begin{tabular}{c|cc}
		\(\ast\) & \(a\) & \(b\) \\ \hline
		\(a\)    & \(a\) & \(b\) \\
		\(b\)    & \(b\) & \(b\) \\
	\end{tabular}
\end{table}

このとき,\(\symcal{M}\)は上記\(\varphi_1, \varphi_2, \varphi_3\)のすべてを充足するので\(T_1\)のモデルである.
これは,\(\symcal{M} \satisfy \varphi_1\),\(\symcal{M} \satisfy \varphi_2\),\(\symcal{M} \satisfy \varphi_3\)がそれぞれ
\begin{enumerate}
	\item すべての\(x, y, z\)に対して\(\paren{x \ast y} \ast z = x \ast \paren{y \ast z}\)となる
	\item \(e' \in M\)が存在して,すべての\(x \in M\)に対して\(e' \ast x = x\)となる
	\item \(e' \in M\)が存在して,すべての\(x \in M\)に対して\(y \ast x = e'\)となる\(y \in M\)がとれる
\end{enumerate}
と同値になっていることからわかる(\(\symcal{M} \satisfy \varphi_2\)が成立する根拠となる\(e'\)として\(a\)が,
\(\symcal{M} \satisfy \varphi_3\)が成立する根拠となる\(e'\)として\(b\)がそれぞれとれる).
しかし,2については\(b \ast a = a\)は成り立たず,3については\(y \ast b = a\)となる\(y \in M\)
が存在しないので,上記2つの\(e'\)を一致させることはできない.
このため,\(\symcal{M}\)による\(\obj{e}, {}^{\obj{-1}}\)の解釈をどのように定めても,
\(\symcal{M}\)が\(T\)のモデルになることはない.

\subsection*{\Cref{Que:validformula}}

整数全体の集合\(\Integers\)を対象領域とする\(\symcal{L}_{\OrderedRing}\)構造\(\symcal{M}\)を考える.
ただし,\(\symcal{M}\)による\(\obj{+}, \obj{\cdot}, \obj{0}, \obj{1}\)の解釈はそれぞれ\(\Integers\)
における通常の\(\mathord{+}, \mathord{\cdot}, 0, 1\)とし,\(\obj{<}\)の\(\symcal{M}\)による解釈は
\(\Integers\)における通常の等号付き大小関係\(\leq\)とする.
このとき,閉論理式\(\varphi \colon \forall \obj{x} \lnot \paren{\obj{x} \obj{<} \obj{x}}\)について
\[
	\symcal{M} \satisfy \varphi \metaequivalent \text{すべての\(a \in \Integers\)に対して\(a \leq a\)でない}
\]
となるが,例えば\(a = 1\)のときにこれは成り立たない.
よって,\(\Integers\)はこの\(\varphi\)を充足しない.

\subsection*{\Cref{Que:truthtableforequivalent}}

\(\varphi \formulaequiv \psi\)の真理値表を以下に示す
(\(\varphi \formulaequiv \psi\)は\(\paren{\varphi \to \psi} \land \paren{\psi \to \varphi}\)の略記であったことを思い出そう).

\begin{table}[htbp]
	\centering
	\begin{tabular}{cc|ccc}
		\hline
		\(\varphi\) & \(\psi\) & \(\varphi \to \psi\) & \(\psi \to \varphi\) & \(\varphi \formulaequiv \psi\) \\ \hline
		0           & 0        & 1                    & 1                    & 1                              \\
		0           & 1        & 1                    & 0                    & 0                              \\
		1           & 0        & 0                    & 1                    & 0                              \\
		1           & 1        & 1                    & 1                    & 1                              \\
		\hline
	\end{tabular}
\end{table}

\subsection*{\Cref{Que:semanticlawofexcludedmiddle}}

\(\varphi \lor \lnot \varphi\)の真理値表を以下に示す.

\begin{table}[htbp]
	\centering
	\begin{tabular}{c|cc}
		\hline
		\(\varphi\) & \(\lnot \varphi\) & \(\varphi \lor \varphi\) \\ \hline
		0           & 1                 & 1                        \\
		1           & 0                 & 1                        \\
		\hline
	\end{tabular}
\end{table}

\(\varphi \lor \lnot \varphi\)の真理値はつねに1になるため,\(\varphi \lor \lnot \varphi\)が恒真式であることがわかる.


\section*{\Cref{chap:proof}}

\subsection*{\Cref{Que:sequent}}

\(\varphi \to \psi \sequent \lnot \varphi \lor \psi\)の導出:
\begin{enumerate}
	\item \(\varphi \to \psi \sequent \varphi \to \psi\)\quad (ID)
	\item \(\lnot \paren{\lnot \varphi \lor \psi} \sequent \lnot \paren{\lnot \varphi \lor \psi}\)\quad (ID)
	\item \(\varphi \sequent \varphi\)\quad (ID)
	\item \(\varphi, {\varphi \to \psi} \sequent \psi\)\quad (1,3から(\(\to\)E)による)
	\item \(\varphi, {\varphi \to \psi} \sequent \lnot \varphi \lor \psi\)\quad (4から(\(\lor\)I)による)
	\item \(\varphi, {\varphi \to \psi}, \lnot \paren{\lnot \varphi \lor \psi} \sequent \bot\)\quad (2, 5から(w), (e), (\(\lnot\)E)による)
	\item \(\varphi \to \psi, \lnot \paren{\lnot \varphi \lor \psi} \sequent \lnot \varphi\)\quad (6から(\(\lnot\)I)による)
	\item \(\varphi \to \psi, \lnot \paren{\lnot \varphi \lor \psi} \sequent \lnot \varphi \lor \psi\)\quad (7から(\(\lor\)I)による)
	\item \(\varphi \to \psi, \lnot \paren{\lnot \varphi \lor \psi} \sequent \bot\)\quad (2, 8から(w), (\(\lnot\)E)による)
	\item \(\varphi \to \psi \sequent \lnot \lnot \paren{\lnot \varphi \lor \psi}\)\quad (9から(e), (\(\lnot\)E)による)
	\item \(\varphi \to \psi \sequent \lnot \varphi \lor \psi\)\quad (10から(DNF)による)
\end{enumerate}

\(\lnot \varphi \lor \psi \sequent \varphi \to \psi\)の導出:
\begin{enumerate}
	\item \(\lnot \varphi \lor \psi \sequent \lnot \varphi \lor \psi\)\quad (ID)
	\item \(\varphi \sequent \varphi\)\quad (ID)
	\item \(\lnot \varphi \sequent \lnot \varphi\)\quad (ID)
	\item \(\lnot \psi, \varphi, \lnot \varphi \sequent \bot\)\quad (2, 3から(w), (e), (\(\lnot\)E)による)
	\item \(\varphi, \lnot \varphi \sequent \lnot \lnot \psi\)\quad (4から(\(\lnot\)I)による)
	\item \(\varphi, \lnot \varphi \sequent \psi\)\quad (5から(DNF)による)
	\item \(\psi \sequent \psi\)\quad (ID)
	\item \(\lnot \varphi \lor \psi, \varphi \sequent \psi\)\quad (1, 6, 7から(e), (\(\lor\)E), (c)による)
	\item \(\lnot \varphi \lor \psi \sequent \varphi \to \psi\)\quad (8から(e), (\(\to\)I)による)
\end{enumerate}

\(\lnot \paren{\varphi \lor \psi} \sequent \lnot \varphi \land \lnot \psi\)の導出:
\begin{enumerate}
	\item \(\lnot \paren{\varphi \lor \psi} \sequent \lnot \paren{\varphi \lor \psi}\)\quad (ID)
	\item \(\varphi \sequent \varphi\)\quad (ID)
	\item \(\varphi \sequent \varphi \lor \psi\)\quad (2から(\(\lor\)I)による)
	\item \(\varphi, \lnot \paren{\varphi \lor \psi} \sequent \bot\)\quad (1, 3から(w), (e), (\(\lnot\)E)による)
	\item \(\lnot \paren{\varphi \lor \psi} \sequent \lnot \varphi\)\quad (4から(\(\lnot\)I)による)
	\item \(\lnot \paren{\varphi \lor \psi} \sequent \lnot \psi\)\quad (上と同様にして導出できるので省略)
	\item \(\lnot \paren{\varphi \lor \psi} \sequent \lnot \varphi \land \lnot \psi\)\quad (5, 6から(\(\land\)I), (c)による)
\end{enumerate}

\(\lnot \varphi \land \lnot \psi \sequent \lnot \paren{\varphi \lor \psi}\)の導出:
\begin{enumerate}
	\item \(\lnot \varphi \land \lnot \psi \sequent \lnot \varphi \land \lnot \psi\)\quad (ID)
	\item \(\varphi \lor \psi \sequent \varphi \lor \psi\)\quad (ID)
	\item \(\varphi \sequent \varphi\)\quad (ID)
	\item \(\lnot \varphi \land \lnot \psi \sequent \lnot \varphi\)\quad (1から(\(\land\)E)による)
	\item \(\varphi, \lnot \varphi \land \lnot \psi \sequent \bot\)\quad (3, 4から(w), (e), (\(\lnot\)E)による)
	\item \(\psi, \lnot \varphi \land \lnot \psi \sequent \bot\)\quad (上と同様にして導出できるので省略)
	\item \(\varphi \lor \psi, \lnot \varphi \land \lnot \psi \sequent \bot\)\quad (2, 5, 6から(\(\lor\)E), (c)による)
	\item \(\lnot \varphi \land \lnot \psi \sequent \lnot \paren{\varphi \lor \psi}\)\quad (7から(\(\lnot\)I)による)
\end{enumerate}

\(\lnot \paren{\varphi \land \psi} \sequent \lnot \varphi \lor \lnot \psi\)の導出:
\begin{enumerate}
	\item \(\lnot \paren{\varphi \land \psi} \sequent \lnot \paren{\varphi \land \psi}\)\quad (ID)
	\item \(\lnot \paren{\lnot \varphi \lor \lnot \psi} \sequent \lnot \paren{\lnot \varphi \lor \lnot \psi}\)\quad (ID)
	\item \(\lnot \varphi \sequent \lnot \varphi\)\quad (ID)
	\item \(\lnot \varphi \sequent \lnot \varphi \lor \lnot \psi\)\quad (3から(\(\lor\)I)による)
	\item \(\lnot \varphi, \lnot \paren{\lnot \varphi \lor \lnot \psi} \sequent \bot\)\quad (2, 4から(w), (e), (\(\lnot\)E)による)
	\item \(\lnot \paren{\lnot \varphi \lor \lnot \psi} \sequent \lnot \lnot \varphi\)\quad (5から(\(\lnot\)I)による)
	\item \(\lnot \paren{\lnot \varphi \lor \lnot \psi} \sequent \varphi\)\quad (6から(DNF)による)
	\item \(\lnot \paren{\lnot \varphi \lor \lnot \psi} \sequent \psi\)\quad (上と同様にして導けるので省略)
	\item \(\lnot \paren{\lnot \varphi \lor \lnot \psi} \sequent \varphi \land \psi\)\quad (7, 8から(\(\land\)I), (c)による)
	\item \(\lnot \paren{\lnot \varphi \lor \lnot \psi}, \lnot \paren{\varphi \land \psi} \sequent \bot\)\quad (1, 9から(w), (e), (\(\lnot\)E)による)
	\item \(\lnot \paren{\varphi \land \psi} \sequent \lnot \lnot \paren{\lnot \varphi \lor \lnot \psi}\)\quad (10から(\(\lnot\)I)による)
	\item \(\lnot \paren{\varphi \land \psi} \sequent \lnot \varphi \lor \lnot \psi\)\quad (11から(DNF)による)
\end{enumerate}

\(\lnot \varphi \lor \lnot \psi \sequent \lnot \paren{\varphi \land \psi}\)の導出:
\begin{enumerate}
	\item \(\lnot \varphi \lor \lnot \psi \sequent \lnot \varphi \lor \lnot \psi\)\quad (ID)
	\item \(\varphi \land \psi \sequent \varphi \land \psi\)\quad (ID)
	\item \(\lnot \varphi \sequent \lnot \varphi\)\quad (ID)
	\item \(\varphi \land \psi \sequent \varphi\)\quad (2から(\(\land\)E)による)
	\item \(\lnot \varphi, \varphi \land \psi \sequent \bot\)\quad (3, 4から(w), (e), (\(\lnot\)E)による)
	\item \(\lnot \psi, \varphi \land \psi \sequent \bot\)\quad (上と同様にして導けるので省略)
	\item \(\varphi \land \varphi, \lnot \varphi \lor \lnot \psi \sequent \bot\)\quad (1, 5, 6から(w), (e), (\(\land\)E), (c)による)
	\item \(\lnot \varphi \lor \lnot \psi \sequent \lnot \paren{\varphi \land \varphi}\)\quad (7から(\(\lnot\)I)による)
\end{enumerate}

\(\lnot \forall x \varphi \sequent \exists x \lnot \varphi\)の導出:
\begin{enumerate}
	\item \(\lnot \forall x \varphi \sequent \lnot \forall x \varphi\)\quad (ID)
	\item \(\lnot \exists x \lnot \varphi \sequent \lnot \exists x \lnot \varphi\)\quad (ID)
	\item \(\lnot \subst{\varphi}{a/x} \sequent \lnot \subst{\varphi}{a/x}\)\quad ((ID),ただし\(a\)は新しい変数記号)
	\item \(\lnot \subst{\varphi}{a/x} \sequent \exists x \lnot \varphi\)\quad (3から(\(\exists\)E)による)
	\item \(\lnot \subst{\varphi}{a/x}, \lnot \exists x \lnot \varphi \sequent \bot\)\quad (2, 4から(w), (e), (\(\lnot\)E)による)
	\item \(\lnot \exists x \lnot \varphi \sequent \lnot \lnot \subst{\varphi}{a/x}\)\quad (5から(\(\lnot\)I)による)
	\item \(\lnot \exists x \lnot \varphi \sequent \subst{\varphi}{a/x}\)\quad (6から(DNF)による)
	\item \(\lnot \exists x \lnot \varphi \sequent \forall x \varphi\)\quad (7から(\(\forall\)I)による)
	\item \(\lnot \exists x \lnot \varphi, \lnot \forall x \varphi \sequent \bot\)\quad (1, 8から(w), (e), (\(\lnot\)E)による)
	\item \(\lnot \forall x \varphi \sequent \lnot \lnot \exists x \lnot \varphi\)\quad (9から(\(\lnot\)E)による)
	\item \(\lnot \forall x \varphi \sequent \exists x \lnot \varphi\)\quad (10から(DNF)による)
\end{enumerate}

\(\exists x \lnot \varphi \sequent \lnot \forall x \varphi\)の導出:
\begin{enumerate}
	\item \(\exists x \lnot \varphi \sequent \exists x \lnot \varphi\)\quad (ID)
	\item \(\forall x \varphi \sequent \forall x \varphi\)\quad (ID)
	\item \(\lnot \subst{\varphi}{a/x} \sequent \lnot \subst{\varphi}{a/x}\)\quad ((ID),ただし\(a\)は新しい変数記号)
	\item \(\forall x \varphi \sequent \subst{\varphi}{a/x}\)\quad (2から(\(\forall\)E)による)
	\item \(\lnot \subst{\varphi}{a/x}, \forall x \varphi \sequent \bot\)\quad (3, 4から(w), (e), (\(\lnot\)E)による)
	\item \(\exists x \lnot \varphi, \forall x \varphi \sequent \bot\)\quad (1, 5から(\(\exists\)E)による)
	\item \(\exists x \lnot \varphi \sequent \lnot \forall x \varphi\)\quad (6から(e), (\(\lnot\)I)による)
\end{enumerate}

\(\lnot \exists x \varphi \sequent \forall x \lnot \varphi\)の導出:
\begin{enumerate}
	\item \(\lnot \exists x \varphi \sequent \lnot \exists x \varphi\)\quad (ID)
	\item \(\subst{\varphi}{a/x} \sequent \subst{\varphi}{a/x}\)\quad ((ID),ただし\(a\)は新しい変数記号)
	\item \(\subst{\varphi}{a/x} \sequent \exists x \varphi\)\quad (3から(\(\exists\)I)による)
	\item \(\subst{\varphi}{a/x}, \lnot \exists x \varphi \sequent \bot\)\quad (1, 3から(w), (e), (\(\lnot\)E)による)
	\item \(\lnot \exists x \varphi \sequent \lnot \subst{\varphi}{a/x}\)\quad (4から(\(\lnot\)I)による)
	\item \(\lnot \exists x \varphi \sequent \forall x \lnot \varphi\)\quad (5から(\(\forall\)I)による)
\end{enumerate}

\(\forall x \lnot \varphi \sequent \lnot \exists x \varphi\)の導出:
\begin{enumerate}
	\item \(\forall x \lnot \varphi \sequent \forall x \lnot \varphi\)\quad (ID)
	\item \(\exists x \varphi \sequent \exists x \varphi\)\quad (ID)
	\item \(\subst{\varphi}{a/x} \sequent \subst{\varphi}{a/x}\)\quad ((ID),ただし\(a\)は新しい変数記号)
	\item \(\forall x \lnot \varphi \sequent \lnot \subst{\varphi}{a/x}\)\quad (1から(\(\forall\)E)による)
	\item \(\subst{\varphi}{a/x}, \forall x \lnot \varphi \sequent \bot\)\quad (3, 4から(w), (e), (\(\lnot\)E)による)
	\item \(\exists x \varphi, \forall x \lnot \varphi \sequent \bot\)\quad (2, 5から(\(\exists\)E)による)
	\item \(\forall x \lnot \varphi \sequent \lnot \exists x \varphi\)\quad (6から(\(\lnot\)I)による)
\end{enumerate}

\(\varphi \land \psi \sequent \psi \land \varphi\)の導出(逆向きのシークエントも同様である):
\begin{enumerate}
	\item \(\varphi \land \psi \sequent \varphi \land \psi\)\quad (ID)
	\item \(\varphi \land \psi \sequent \psi\)\quad (1から(\(\land\)E)による)
	\item \(\varphi \land \psi \sequent \varphi\)\quad (1から(\(\land\)E)による)
	\item \(\varphi \land \psi \sequent \psi \land \varphi\)\quad (2, 3から(\(\land\)I), (c)による)
\end{enumerate}

\(\varphi \lor \psi \sequent \psi \lor \varphi\)の導出(逆向きのシークエントも同様である):
\begin{enumerate}
	\item \(\varphi \lor \psi \sequent \varphi \lor \psi\)\quad (ID)
	\item \(\varphi \sequent \varphi\)\quad (ID)
	\item \(\varphi \sequent \psi \lor \varphi\)\quad (2から(\(\lor\)I)による)
	\item \(\psi \sequent \psi\)\quad (ID)
	\item \(\psi \sequent \psi \lor \varphi\)\quad (4から(\(\lor\)I)による)
	\item \(\varphi \lor \psi \sequent \psi \lor \varphi\)\quad (1, 3, 5から(\(\lor\)E), (c)による)
\end{enumerate}

\(\varphi \land \paren{\psi \land \chi} \sequent \paren{\varphi \land \psi} \land \chi\)の導出(逆向きのシークエントも同様である):
\begin{enumerate}
	\item \(\varphi \land \paren{\psi \land \chi} \sequent \varphi \land \paren{\psi \land \chi}\)\quad (ID)
	\item \(\varphi \land \paren{\psi \land \chi} \sequent \varphi\)\quad (1から(\(\land\)E)による)
	\item \(\varphi \land \paren{\psi \land \chi} \sequent \psi \land \chi\)\quad (1から(\(\land\)E)による)
	\item \(\varphi \land \paren{\psi \land \chi} \sequent \psi\)\quad (3から(\(\land\)E)による)
	\item \(\varphi \land \paren{\psi \land \chi} \sequent \varphi \land \psi\)\quad (2, 4から(\(\land\)I), (c)による)
	\item \(\varphi \land \paren{\psi \land \chi} \sequent \chi\)\quad (3から(\(\land\)E)による)
	\item \(\varphi \land \paren{\psi \land \chi} \sequent \paren{\varphi \land \psi} \land \chi\)\quad (5, 6から(\(\land\)I), (c)による)
\end{enumerate}

\(\varphi \lor \paren{\psi \lor \chi} \sequent \paren{\varphi \lor \psi} \lor \chi\)の導出(逆向きのシークエントも同様である):
\begin{enumerate}
	\item \(\varphi \lor \paren{\psi \lor \chi} \sequent \varphi \lor \paren{\psi \lor \chi}\)\quad (ID)
	\item \(\varphi \sequent \varphi\)\quad (ID)
	\item \(\varphi \sequent \varphi \lor \psi\)\quad (2から(\(\lor\)I)による)
	\item \(\varphi \sequent \paren{\varphi \lor \psi} \lor \chi\)\quad (3から(\(\lor\)I)による)
	\item \(\psi \lor \chi \sequent \psi \lor \chi\)\quad (ID)
	\item \(\psi \sequent \psi\)\quad (ID)
	\item \(\psi \sequent \varphi \lor \psi\)\quad (6から(\(\lor\)I)による)
	\item \(\psi \sequent \paren{\varphi \lor \psi} \lor \chi\)\quad (7から(\(\lor\)I)による)
	\item \(\chi \sequent \chi\)\quad (ID)
	\item \(\chi \sequent \paren{\varphi \lor \psi} \lor \chi\)\quad (9から(\(\lor\)I)による)
	\item \(\psi \lor \chi \sequent \paren{\varphi \lor \psi} \lor \chi\)\quad (5, 8, 10から(\(\lor\)E)による)
	\item \(\varphi \lor \paren{\psi \lor \chi} \sequent \varphi \lor \paren{\psi \lor \chi}\)\quad (1, 4, 11から(\(\lor\)E)による)
\end{enumerate}

\(\varphi \land \paren{\psi \lor \chi} \sequent \paren{\varphi \land \psi} \lor \paren{\varphi \land \chi}\)の導出:
\begin{enumerate}
	\item \(\varphi \land \paren{\psi \lor \chi} \sequent \varphi \land \paren{\psi \lor \chi}\)\quad (ID)
	\item \(\varphi \land \paren{\psi \lor \chi} \sequent \varphi\)\quad (1から(\(\land\)E)による)
	\item \(\varphi \land \paren{\psi \lor \chi} \sequent \psi \lor \chi\)\quad (1から(\(\land\)E)による)
	\item \(\psi \sequent \psi\)\quad (ID)
	\item \(\psi, \varphi \land \paren{\psi \lor \chi} \sequent \varphi \land \psi\)\quad (2, 4から(w), (e), (\(\land\)I), (c)による)
	\item \(\psi, \varphi \land \paren{\psi \lor \chi} \sequent \paren{\varphi \land \psi} \lor \paren{\varphi \lor \psi}\)\quad (5から(\(\lor\)I)による)
	\item \(\chi, \varphi \land \paren{\psi \lor \chi} \sequent \paren{\varphi \land \psi} \lor \paren{\varphi \land \chi}\)\quad (上と同様にして導けるので省略)
	\item \(\varphi \land \paren{\psi \lor \chi} \sequent \paren{\varphi \land \psi} \lor \paren{\varphi \land \chi}\)\quad (3, 6, 7から(\(\lor\)E), (c)による)
\end{enumerate}

\(\paren{\varphi \land \psi} \lor \paren{\varphi \land \chi} \sequent \varphi \land \paren{\psi \lor \chi}\)の導出:
\begin{enumerate}
	\item \(\paren{\varphi \land \psi} \lor \paren{\varphi \land \chi} \sequent \paren{\varphi \land \psi} \lor \paren{\varphi \land \chi}\)\quad (ID)
	\item \(\varphi \land \psi \sequent \varphi \land \psi\)\quad (ID)
	\item \(\varphi \land \psi \sequent \varphi\)\quad (2から(\(\land\)E)による)
	\item \(\varphi \land \psi \sequent \psi\)\quad (2から(\(\land\)E)による)
	\item \(\varphi \land \psi \sequent \psi \lor \chi\)\quad (4から(\(\lor\)I)による)
	\item \(\varphi \land \psi \sequent \varphi \land \paren{\psi \lor \chi}\)\quad (3, 5から(\(\land\)I), (c)による)
	\item \(\varphi \land \chi \sequent \varphi \land \paren{\psi \lor \chi}\)\quad (上と同様にして導けるので省略)
	\item \(\paren{\varphi \land \psi} \lor \paren{\varphi \land \chi} \sequent \varphi \land \paren{\psi \lor \chi}\)\quad (1, 6, 7から(\(\lor\)E)による)
\end{enumerate}

\(\varphi \lor \paren{\psi \land \chi} \sequent \paren{\varphi \lor \psi} \land \paren{\varphi \lor \chi}\)の導出:
\begin{enumerate}
	\item \(\varphi \lor \paren{\psi \land \chi} \sequent \varphi \lor \paren{\psi \land \chi}\)\quad (ID)
	\item \(\varphi \sequent \varphi\)\quad (ID)
	\item \(\varphi \sequent \varphi \lor \psi\)\quad (2から(\(\lor\)I)による)
	\item \(\varphi \sequent \varphi \lor \chi\)\quad (2から(\(\lor\)I)による)
	\item \(\varphi \sequent \paren{\varphi \lor \psi} \land \paren{\varphi \lor \chi}\)\quad (3, 4から(\(\land\)I), (c)による)
	\item \(\psi \land \chi \sequent \psi \land \chi\)\quad (ID)
	\item \(\psi \land \chi \sequent \psi\)\quad (6から(\(\land\)E)による)
	\item \(\psi \land \chi \sequent \varphi \lor \psi\)\quad (7から(\(\lor\)I)による)
	\item \(\psi \land \chi \sequent \chi\)\quad (6から(\(\land\)E)による)
	\item \(\psi \land \chi \sequent \varphi \lor \chi\)\quad (9から(\(\lor\)I)による)
	\item \(\psi \land \chi \sequent \paren{\varphi \lor \psi} \land \paren{\varphi \lor \chi}\)\quad (8, 10から(\(\land\)I), (e)による)
	\item \(\varphi \lor \paren{\psi \land \chi} \sequent \paren{\varphi \lor \psi} \land \paren{\varphi \lor \chi}\)\quad (1, 5, 11から(\(\lor\)E)による)
\end{enumerate}

\(\lnot \paren{\varphi \to \psi} \sequent \varphi \land \lnot \psi\)の導出:
\begin{enumerate}
	\item \(\lnot \paren{\varphi \to \psi} \sequent \lnot \paren{\varphi \to \psi}\)\quad (ID)
	\item \(\lnot \varphi \sequent \lnot \varphi\)\quad (ID)
	\item \(\lnot \varphi \sequent \lnot \varphi \lor \psi\)\quad (2から(\(\lor\)I)による)
	\item \(\lnot \varphi \lor \psi \sequent \varphi \to \psi\)\quad (\Cref{Que:sequent}の1つ目)
	\item \(\sequent \lnot \varphi \lor \psi \to \paren{\varphi \to \psi}\)\quad (4から(\(\to\)I)による)
	\item \(\lnot \varphi \sequent \varphi \to \psi\)\quad (3, 5から(\(\to\)E)による)
	\item \(\lnot \varphi, \lnot \paren{\varphi \to \psi} \sequent \bot\)\quad (1, 6から(w), (e), (\(\lnot\)E)による)
	\item \(\lnot \paren{\varphi \to \psi} \sequent \lnot \lnot \varphi\)\quad (7から(\(\lnot\)I)による)
	\item \(\lnot \paren{\varphi \to \psi} \sequent \varphi\)\quad (8から(DNF)による)
	\item \(\psi \sequent \psi\)\quad (ID)
	\item \(\varphi, \psi \sequent \psi\)\quad (10から(w)による)
	\item \(\psi \sequent \varphi \to \psi\)\quad (11から(\(\to\)I)による)
	\item \(\psi, \lnot \paren{\varphi \to \psi} \sequent \bot\)\quad (1, 12から(w), (e), (\(\lnot\)E)による)
	\item \(\lnot \paren{\varphi \to \psi} \sequent \lnot \psi\)\quad (13から(\(\lnot\)I)による)
	\item \(\lnot \paren{\varphi \to \psi} \sequent \varphi \land \lnot \psi\)\quad (9, 14から(\(\land\)I), (c)による)
\end{enumerate}

\(\varphi \land \lnot \psi \sequent \lnot \paren{\varphi \to \psi}\)の導出:
\begin{enumerate}
	\item \(\varphi \land \lnot \psi \sequent \varphi \land \lnot \psi\)\quad (ID)
	\item \(\varphi \to \psi \sequent \varphi \to \psi\)\quad (ID)
	\item \(\varphi \land \lnot \psi \sequent \varphi\)\quad (1から(\(\land\)E)による)
	\item \(\varphi \to \psi, \varphi \land \lnot \psi \sequent \psi\)\quad (2, 3から(\(\to\)E)による)
	\item \(\varphi \land \lnot \psi \sequent \lnot \psi\)\quad (1から(\(\land\)E)による)
	\item \(\varphi \to \psi, \varphi \land \lnot \psi \sequent \bot\)\quad (4, 5から(w), (\(\lnot\)E)による)
	\item \(\varphi \land \lnot \psi \sequent \lnot \paren{\varphi \to \psi}\)\quad (6から(\(\lnot\)I)による)
\end{enumerate}

\(\lnot \forall x \paren{\varphi \to \psi} \sequent \exists x \paren{\varphi \land \lnot \psi}\)の導出:
\begin{enumerate}
	\item \(\lnot \forall x \paren{\varphi \to \psi} \sequent \exists x \lnot \paren{\varphi \to \psi}\)\quad (\Cref{Que:sequent}の4つ目)
	\item \(\subst{\paren{\lnot \paren{\varphi \to \psi}}}{a/x} \sequent \subst{\paren{\varphi \land \lnot \psi}}{a/x}\)\quad (\Cref{Que:sequent}の12個目,ただし\(a\)は新しい変数記号)
	\item \(\subst{\paren{\lnot \paren{\varphi \to \psi}}}{a/x} \sequent \exists x \paren{\varphi \land \lnot \psi}\)\quad (2から(\(\exists\)I)による)
	\item \(\lnot \forall x \paren{\varphi \to \psi} \sequent \exists x \paren{\varphi \land \lnot \psi}\)\quad (1, 3から(\(\exists\)E)による)
\end{enumerate}

\(\exists x \paren{\varphi \land \lnot \psi} \sequent \lnot \forall x \paren{\varphi \to \psi}\)の導出:
\begin{enumerate}
	\item \(\exists x \paren{\varphi \land \lnot \psi} \sequent \exists x \lnot \paren{\varphi \land \lnot \psi}\)\quad (ID)
	\item \(\subst{\paren{\varphi \land \lnot \psi}}{a/x} \sequent \subst{\paren{\lnot \paren{\varphi \to \psi}}}{a/x}\)\quad (\Cref{Que:sequent}の12個目,ただし\(a\)は新しい変数記号)
	\item \(\subst{\paren{\varphi \land \lnot \psi}}{a/x} \sequent \exists x \lnot \paren{\varphi \to \psi}\)\quad (2から(\(\exists\)I)による)
	\item \(\exists x \paren{\varphi \land \lnot \psi} \sequent \exists x \lnot \paren{\varphi \to \psi}\)\quad (1, 3から(\(\exists\)E)による)
	\item \(\exists x \lnot \paren{\varphi \to \psi} \sequent \lnot \forall x \paren{\varphi \to \psi}\)\quad (\Cref{Que:sequent}の4つ目)
	\item \(\sequent \exists x \lnot \paren{\varphi \to \psi} \to \lnot \forall x \paren{\varphi \to \psi}\)\quad (5から(\(\to\)I)による)
	\item \(\exists x \paren{\varphi \land \lnot \psi} \sequent \lnot \forall x \paren{\varphi \to \psi}\)\quad (4, 6から(\(\to\)E)による)
\end{enumerate}

\(\varphi \to \psi \sequent \lnot \psi \to \lnot \varphi\)の導出:
\begin{enumerate}
	\item \(\varphi \to \psi \sequent \varphi \to \psi\)\quad (ID)
	\item \(\lnot \psi \sequent \lnot \psi\)\quad (ID)
	\item \(\varphi \sequent \varphi\)\quad (ID)
	\item \(\varphi, \varphi \to \psi \sequent \psi\)\quad (1, 3から(\(\to\)E)による)
	\item \(\varphi, \lnot \psi, \varphi \to \psi \sequent \bot\)\quad (2, 4から(w), (e), (\(\lnot\)E)による)
	\item \(\lnot \psi, \varphi \to \psi \sequent \lnot \varphi\)\quad (5から(\(\lnot\)I)による)
	\item \(\varphi \to \psi \sequent \lnot \psi \to \lnot \varphi\)\quad (6から(\(\to\)I)による)
\end{enumerate}

\(\lnot \psi \to \lnot \varphi \sequent \varphi \to \psi\)の導出:
\begin{enumerate}
	\item \(\lnot \psi \to \lnot \varphi \sequent \lnot \psi \to \lnot \varphi\)\quad (ID)
	\item \(\varphi \sequent \varphi\)\quad (ID)
	\item \(\lnot \psi \sequent \lnot \psi\)\quad (ID)
	\item \(\lnot \psi, \lnot \psi \to \lnot \varphi \sequent \lnot \varphi\)\quad (1, 3から(\(\to\)E)による)
	\item \(\lnot \psi, \varphi \lnot \psi \to \lnot \varphi \sequent \bot\)\quad (2, 4から(w), (e), (\(\lnot\)E)による)
	\item \(\varphi, \lnot \psi \to \lnot\varphi \sequent \lnot \lnot \psi\)\quad (5から(\(\lnot\)I)による)
	\item \(\varphi, \lnot \psi \to \lnot \varphi \sequent \psi\)\quad (6から(DNF)による)
	\item \(\lnot \psi \to \lnot \varphi \sequent \varphi \to \psi\)\quad (7から(\(\to\)I)による)
\end{enumerate}

\(\varphi \to \paren{\psi \to \chi} \sequent \varphi \land \psi \to \chi\)の導出:
\begin{enumerate}
	\item \(\varphi \to \paren{\psi \to \chi} \sequent \varphi \to \paren{\psi \to \chi}\)\quad (ID)
	\item \(\varphi \land \psi\)\quad (ID)
	\item \(\varphi \land \psi \sequent \varphi\)\quad (2から(\(\land\)E)による)
	\item \(\varphi \land \psi \sequent \psi\)\quad (2から(\(\land\)E)による)
	\item \(\varphi \land \psi, \varphi \to \paren{\psi \to \chi} \sequent \psi \to \chi\)\quad (1, 3から(\(\to\)E)による)
	\item \(\varphi \land \psi, \varphi \to \paren{\psi \to \chi} \sequent \chi\)\quad (4, 5から(\(\to\)E), (c)による)
	\item \(\varphi \to \paren{\psi \to \chi} \sequent \varphi \land \psi\)\quad (6から(\(\to\)I)による)
\end{enumerate}

\(\varphi \land \psi \to \chi \sequent \varphi \to \paren{\phi \to \chi}\)の導出:
\begin{enumerate}
	\item \(\varphi \land \psi \to \chi \sequent \varphi \land \psi \to \chi\)\quad (ID)
	\item \(\varphi \sequent \varphi\)\quad (ID)
	\item \(\psi \sequent \psi\)\quad (ID)
	\item \(\psi, \varphi \sequent \varphi \land \psi\)\quad (2, 3から(\(\land\)I), (e)による)
	\item \(\psi, \varphi, \varphi \land \psi \to \chi \sequent \chi\)\quad (1, 4から(\(\to\)E)による)
	\item \(\varphi, \varphi \land \psi \to \chi \sequent \psi \to \chi\)\quad (5から(\(\to\)I)による)
	\item \(\varphi \land \psi \to \chi \sequent \varphi \to \paren{\psi \to \chi}\)\quad (6から(\(\to\)I)による)
\end{enumerate}

\(\forall x \paren{\varphi \to \psi} \sequent \exists x \varphi \to \psi\)の導出:
\begin{enumerate}
	\item \(\forall x \paren{\varphi \to \psi} \sequent \forall x \paren{\varphi \to \psi}\)\quad (ID)
	\item \(\exists x \varphi \sequent \exists x \varphi\)\quad (ID)
	\item \(\subst{\varphi}{a/x} \sequent \subst{\varphi}{a/x}\)\quad ((ID),ただし\(a\)は新しい変数記号)
	\item \(\forall x \paren{\varphi \to \psi} \sequent \subst{\varphi}{a/x} \to \psi\)\quad (1から(\(\forall\)E)による)
	\item \(\subst{\varphi}{a/x}, \forall x \paren{\varphi \to \psi} \sequent \psi\)\quad (3, 4から(\(\to\)E)による)
	\item \(\exists x \varphi, \forall x \paren{\varphi \to \psi} \sequent \psi\)\quad (2, 5から(\(\exists\)E)による)
	\item \(\forall x \paren{\varphi \to \psi} \sequent \exists x \varphi \to \psi\)\quad (6から(\(\to\)I)による)
\end{enumerate}
ここで,4の導出にあたっては,\(\psi\)に\(x\)が自由出現しないことにより
\(\subst{\paren{\varphi \to \psi}}{a/x}\)と\(\subst{\varphi}{a/x} \to \psi\)が同じ論理式であることを用いた.

\(\exists x \varphi \to \psi \sequent \forall x \paren{\varphi \to \psi}\)の導出:
\begin{enumerate}
	\item \(\exists x \varphi \to \psi \sequent \exists x \varphi \to \psi\)\quad (ID)
	\item \(\subst{\varphi}{a/x} \sequent \subst{\varphi}{a/x}\)\quad ((ID),ただし\(a\)は新しい変数記号)
	\item \(\subst{\varphi}{a/x} \sequent \exists x \varphi\)\quad (2から(\(\exists\)I)による)
	\item \(\subst{\varphi}{a/x}, \exists x \varphi \to \psi \sequent \psi\)\quad (1, 3から(\(\to\)E)による)
	\item \(\exists x \varphi \to \psi \sequent \subst{\varphi}{a/x} \to \psi\)\quad (4から(\(\to\)I)による)
	\item \(\exists x \varphi \to \psi \sequent \forall x \paren{\varphi \to \psi}\)\quad (5から(\(\forall\)I)による)
\end{enumerate}
ここで,6の導出にあたっては,\(\psi\)に\(x\)が自由出現しないことにより
\(\subst{\paren{\varphi \to \psi}}{a/x}\)と\(\subst{\varphi}{a/x} \to \psi\)が同じ論理式であることを用いた.

\(\exists x \paren{\varphi \to \psi} \sequent \forall x \varphi \to \psi\)の導出:
\begin{enumerate}
	\item \(\exists x \paren{\varphi \to \psi} \sequent \exists x \paren{\varphi \to \psi}\)\quad (ID)
	\item \(\forall x \varphi \sequent \forall x \varphi\)\quad (ID)
	\item \(\subst{\paren{\varphi \to \psi}}{a/x} \sequent \subst{\paren{\varphi \to \psi}}{a/x}\)\quad (ID)
	\item \(\forall x \varphi \sequent \subst{\varphi}{a/x}\)\quad (2から(\(\forall\)E)による,ただし\(a\)は新しい変数記号)
	\item \(\subst{\paren{\varphi \to \psi}}{a/x}, \forall x \varphi \sequent \psi\)\quad (3, 4から(\(\to\)E), (e)による)
	\item \(\forall x \varphi, \exists x \paren{\varphi \to \psi} \sequent \psi\)\quad (1, 5から(\(\exists\)E), (e)による)
	\item \(\exists x \paren{\varphi \to \psi} \sequent \forall x \to \psi\)\quad (6から(\(\to\)I)による)
\end{enumerate}
ここで,5の導出にあたっては,\(\psi\)に\(x\)が自由出現しないことにより
\(\subst{\paren{\varphi \to \psi}}{a/x}\)と\(\subst{\varphi}{a/x} \to \psi\)が同じ論理式であることを用いた.

\(\forall x \varphi \to \psi \sequent \exists x \paren{\varphi \to \psi}\)の導出:
\begin{enumerate}
	\item \(\forall x \varphi \to \psi \sequent \forall x \varphi \to \psi\)\quad (ID)
	\item \(\lnot \exists x \paren{\varphi \to \psi} \sequent \lnot \exists x \paren{\varphi \to \psi}\)\quad (ID)
	\item \(\lnot \subst{\varphi}{a/x} \sequent \lnot \subst{\varphi}{a/x}\)\quad ((ID),ただし\(a\)は新しい変数記号)
	\item \(\subst{\varphi}{a/x} \sequent \subst{\varphi}{a/x}\)\quad (ID)
	\item \(\lnot \psi, \subst{\varphi}{a/x}, \lnot \subst{\varphi}{a/x} \sequent \bot\)\quad (3, 4から(w), (e), (\(\lnot\)E)による)
	\item \(\subst{\varphi}{a/x}, \lnot \subst{\varphi}{a/x} \sequent \lnot \lnot \psi\)\quad (5から(\(\lnot\)I)による)
	\item \(\subst{\varphi}{a/x}, \lnot \subst{\varphi}{a/x} \sequent \psi\)\quad (6から(DNF)による)
	\item \(\lnot \subst{\varphi}{a/x} \sequent \subst{\varphi}{a/x} \to \psi\)\quad (7から(\(\to\)I)による)
	\item \(\lnot \subst{\varphi}{a/x} \sequent \exists x \paren{\varphi \to \psi}\)\quad (8から(\(\exists\)I)による)
	\item \(\lnot \subst{\varphi}{a/x}, \lnot \exists x \paren{\varphi \to \psi} \sequent \bot\)\quad (2, 9から(w), (e), (\(\lnot\)E)による)
	\item \(\lnot \exists x \paren{\varphi \to \psi} \sequent \lnot \lnot \subst{\varphi}{a/x}\)\quad (10から(\(\lnot\)I)による)
	\item \(\lnot \exists x \paren{\varphi \to \psi} \sequent \subst{\varphi}{a/x}\)\quad (11から(DNF)による)
	\item \(\lnot \exists x \paren{\varphi \to \psi} \sequent \forall x \varphi\)\quad (12から(\(\forall\)I)による)
	\item \(\subst{\varphi}{a/x}, \forall x \varphi \to \psi, \lnot \exists x \paren{\varphi \to \psi} \sequent \psi\)\quad (1, 13から(\(\to\)E), (w)による)
	\item \(\forall x \varphi \to \psi, \lnot \exists x \paren{\varphi \to \psi} \sequent \subst{\varphi}{a/x} \to \psi\)\quad (14から(\(\to\)I)による)
	\item \(\forall x \varphi \to \psi, \lnot \exists x \paren{\varphi \to \psi} \sequent \exists x \paren{\varphi \to \psi}\)\quad (15から(\(\exists\)I)による)
	\item \(\forall x \varphi \to \psi, \lnot \exists x \paren{\varphi \to \psi} \sequent \bot\)\quad (2, 16から(w), (\(\lnot\)E)による)
	\item \(\forall x \varphi \to \psi \sequent \lnot \lnot \exists x \paren{\varphi \to \psi}\)\quad (17から(e), (\(\lnot\)I)による)
	\item \(\forall x \varphi \to \psi \sequent \exists x \paren{\varphi \to \psi}\)\quad (18から(DNF)による)
\end{enumerate}
ここで,9と15の導出にあたっては,\(\psi\)に\(x\)が自由出現しないことにより
\(\subst{\paren{\varphi \to \psi}}{a/x}\)と\(\subst{\varphi}{a/x} \to \psi\)が同じ論理式であることを用いた.

\(\forall x \paren{\varphi \to \psi} \sequent \varphi \to \forall x \psi\)の導出:
\begin{enumerate}
	\item \(\forall x \paren{\varphi \to \psi} \sequent \forall x \paren{\varphi \to \psi}\)\quad (ID)
	\item \(\varphi \sequent \varphi\)\quad (ID)
	\item \(\forall x \paren{\varphi \to \psi} \sequent \varphi \to \subst{\psi}{a/x}\)\quad (1から(\(\forall\)E)による,ただし\(a\)は新しい変数記号)
	\item \(\varphi, \forall x \paren{\varphi \to \psi} \sequent \subst{\psi}{a/x}\)\quad (2, 3から(\(\to\)E)による)
	\item \(\varphi, \forall x \paren{\varphi \to \psi} \sequent \forall x \psi\)\quad (4から(\(\forall\)I)による)
	\item \(\forall x \paren{\varphi \to \psi} \sequent \varphi \to \forall x \psi\)\quad (5から(\(\to\)I)による)
\end{enumerate}
ここで,3の導出にあたっては,\(\varphi\)に\(x\)が自由出現しないことにより
\(\subst{\paren{\varphi \to \psi}}{a/x}\)と\(\varphi \to \subst{\psi}{a/x}\)が同じ論理式であることを用いた.

\(\varphi \to \forall x \psi \sequent \forall x \paren{\varphi \to \psi}\)の導出:
\begin{enumerate}
	\item \(\varphi \to \forall x \psi \sequent \varphi \to \forall x \psi\)\quad (ID)
	\item \(\varphi \sequent \varphi\)\quad (ID)
	\item \(\varphi, \varphi \to \forall x \psi \sequent \forall x \psi\)\quad (1, 2から(\(\to\)E)による)
	\item \(\varphi, \varphi \to \forall x \psi \sequent \subst{\psi}{a/x}\)\quad (3から(\(\forall\)E)による,ただし\(a\)は新しい変数記号)
	\item \(\varphi \to \forall x \psi \sequent \paren{\varphi \to \psi}{a/x}\)\quad (4から(\(\to\)I)による)
	\item \(\varphi \to \forall x \psi \sequent \forall x \paren{\varphi \to \psi}\)\quad (5から(\(\forall\)I)による)
\end{enumerate}
ここで,4の導出にあたっては,\(\varphi\)に\(x\)が自由出現しないことにより
\(\varphi \to \subst{\psi}{a/x}\)と\(\subst{\varphi \to \psi}{a/x}\)が同じ論理式であることを用いた.

\(\exists x \paren{\varphi \to \psi} \sequent \varphi \to \exists x \psi\)の導出:
\begin{enumerate}
	\item \(\exists x \paren{\varphi \to \psi} \sequent \exists x \paren{\varphi \to \psi}\)\quad (ID)
	\item \(\varphi \to \varphi\)\quad (ID)
	\item \(\varphi \to \subst{\psi}{a/x} \sequent \varphi \to \subst{\psi}{a/x}\)\quad ((ID),ただし\(a\)は新しい変数記号)
	\item \(\varphi, \varphi \to \subst{\psi}{a/x} \sequent \subst{\psi}{a/x}\)\quad (2, 3から(\(\to\)E)による)
	\item \(\varphi, \varphi \to \subst{\psi}{a/x} \sequent \exists x \psi\)\quad (4から(\(\exists\)I)による)
	\item \(\varphi \to \subst{\psi}{a/x} \sequent \varphi \to \exists x \psi\)\quad (5から(\(\to\)I)による)
	\item \(\exists x \paren{\varphi \to \psi} \sequent \varphi \to \exists x \psi\)\quad (1, 6から(\(\exists\)E)による)
\end{enumerate}
ここで,7の導出にあたっては,\(\varphi\)に\(x\)が自由出現しないことにより
\(\subst{\paren{\varphi \to \psi}}{a/x}\)と\(\varphi \to \subst{\psi}{a/x}\)が同じ論理式であることを用いた.

\(\varphi \to \exists x \psi \sequent \exists x \paren{\varphi \to \psi}\)の導出:
\begin{enumerate}
	\item \(\varphi \to \exists x \psi \sequent \varphi \to \exists x \psi\)\quad (ID)
	\item \(\lnot \exists x \paren{\varphi \to \psi} \sequent \lnot \exists x \paren{\varphi \to \psi}\)\quad (ID)
	\item \(\varphi \sequent \varphi\)\quad (ID)
	\item \(\varphi, \varphi \to \exists x \psi \sequent \exists x \psi\)\quad (1, 3から(\(\to\)E)による)
	\item \(\subst{\psi}{a/x} \sequent \subst{\psi}{a/x}\)\quad ((ID),ただし\(a\)は新しい変数記号)
	\item \(\varphi, \subst{\psi}{a/x} \sequent \subst{\psi}{a/x}\)\quad (5から(w)による)
	\item \(\subst{\psi}{a/x} \sequent \varphi \to \subst{\psi}{a/x}\)\quad (6から(\(\to\)I)による)
	\item \(\subst{\psi}{a/x} \sequent \exists x \paren{\varphi \to \psi}\)\quad (7から(\(\exists\)I)による)
	\item \(\varphi, \varphi \to \exists x \psi \sequent \exists x \paren{\varphi \to \psi}\)\quad (4, 8から(\(\exists\)E)による)
	\item \(\lnot \subst{\psi}{b/x}, \varphi, \lnot \exists x \paren{\varphi \to \psi}, \varphi \to \exists x \psi \sequent \bot\)\quad (2, 9から(w), (e), (\(\lnot\)E)による,ただし\(b\)は新しい変数記号)
	\item \(\varphi, \lnot \exists x \paren{\varphi \to \psi}, \varphi \to \exists x \psi \sequent \lnot \lnot \subst{\psi}{b/x}\)\quad (10から(\(\lnot\)I)による)
	\item \(\varphi, \lnot \exists x \paren{\varphi \to \psi}, \varphi \to \exists x \psi \sequent \subst{\psi}{b/x}\)\quad (11から(DNF)による)
	\item \(\lnot \exists x \paren{\varphi \to \psi}, \varphi \to \exists x \psi \sequent \varphi \to \subst{\psi}{b/x}\)\quad (12から(\(\to\)I)による)
	\item \(\lnot \exists x \paren{\varphi \to \psi}, \varphi \to \exists x \psi \sequent \exists x \paren{\varphi \to \psi}\)\quad (13から(\(\exists\)I)による)
	\item \(\lnot \exists x \paren{\varphi \to \psi}, \varphi \to \exists x \psi \sequent \bot\)\quad (2, 14から(w), (e), (\(\lnot\)I)による)
	\item \(\varphi \to \exists x \psi \sequent \lnot \lnot \exists x \paren{\varphi \to \psi}\)\quad (15から(\(\lnot\)I)による)
	\item \(\varphi \to \exists x \psi \sequent \exists x \paren{\varphi \to \psi}\)\quad (16から(DNF)による)
\end{enumerate}
ここで,8, 14の導出にあたっては,\(\varphi\)に\(x\)が自由出現しないことにより
\(\subst{\paren{\varphi \to \psi}}{a/x}, \subst{\paren{\varphi \to \psi}}{b/x}\)がそれぞれ
\(\varphi \to \subst{\psi}{a/x}, \varphi \to \subst{\psi}{b/x}\)と同じ論理式であることを用いた.

\(\forall x \paren{\varphi \land \psi} \sequent \forall x \varphi \land \forall x \psi\)の導出:
\begin{enumerate}
	\item \(\forall x \paren{\varphi \land \psi} \sequent \forall x \paren{\varphi \land \psi}\)\quad (ID)
	\item \(\forall x \paren{\varphi \land \psi} \sequent \subst{\varphi}{a/x} \land \subst{\psi}{a/x}\)\quad (1から(\(\forall\)E)による,ただし\(a\)は新しい変数記号)
	\item \(\forall x \paren{\varphi \land \psi} \sequent \subst{\varphi}{a/x}\)\quad (2から(\(\land\)E)による)
	\item \(\forall x \paren{\varphi \land \psi} \sequent \forall x \varphi\)\quad (3から(\(\forall\)I)による)
	\item \(\forall x \paren{\varphi \land \psi} \sequent \forall x \psi\)\quad (上と同様にして導けるので省略)
	\item \(\forall x \paren{\varphi \land \psi} \sequent \forall x \varphi \land \forall x \psi\)\quad (4, 5から(\(\land\)I), (c)による)
\end{enumerate}

\(\forall x \varphi \land \forall x \psi \sequent \forall x \paren{\varphi \land \psi}\)の導出:
\begin{enumerate}
	\item \(\forall x \varphi \land \forall x \psi \sequent \forall x \varphi \land \forall x \psi\)\quad (ID)
	\item \(\forall x \varphi \land \forall x \psi \sequent \forall x \varphi\)\quad (1から(\(\land\)E)による)
	\item \(\forall x \varphi \land \forall x \psi \sequent \subst{\varphi}{a/x}\)\quad (2から(\(\forall\)E)による,ただし\(a\)は新しい変数記号)
	\item \(\forall x \varphi \land \forall x \psi \sequent \subst{\psi}{a/x}\)\quad (上と同様にして導けるので省略)
	\item \(\forall x \varphi \land \forall x \psi \sequent \subst{\varphi}{a/x} \land \subst{\psi}{a/x}\)\quad (3, 4から(\(\land\)I), (c)による)
	\item \(\forall x \varphi \land \forall x \psi \sequent \forall x \paren{\varphi \land \psi}\)\quad (5から(\(\forall\)I)による)
\end{enumerate}

\(\exists x \paren{\varphi \lor \psi} \sequent \exists x \varphi \lor \exists x \psi\)の導出:
\begin{enumerate}
	\item \(\exists x \paren{\varphi \lor \psi} \sequent \exists x \paren{\varphi \lor \psi}\)\quad (ID)
	\item \(\subst{\varphi}{a/x} \lor \subst{\psi}{a/x} \sequent \subst{\varphi}{a/x} \lor \subst{\psi}{a/x}\)\quad (ID)
	\item \(\subst{\varphi}{a/x} \sequent \subst{\varphi}{a/x}\)\quad (ID)
	\item \(\subst{\varphi}{a/x} \sequent \exists x \varphi\)\quad (3から(\(\exists\)I)による)
	\item \(\subst{\varphi}{a/x} \sequent \exists x \varphi \lor \exists x \psi\)\quad (4から(\(\lor\)I)による)
	\item \(\subst{\psi}{a/x} \sequent \exists x \varphi \lor \exists x \psi\)\quad (上と同様にして導けるので省略)
	\item \(\subst{\varphi}{a/x} \lor \subst{\psi}{a/x} \sequent \exists x \varphi \lor \exists x \psi\)\quad (2, 5, 6から(\(\lor\)E)による)
	\item \(\exists x \paren{\varphi \lor \psi} \sequent \exists x \varphi \lor \exists x \psi\)\quad (1, 7から(\(\exists\)E)による)
\end{enumerate}

\(\exists x \varphi \lor \exists x \psi \sequent \exists x \paren{\varphi \lor \psi}\)の導出:
\begin{enumerate}
	\item \(\exists x \varphi \lor \exists x \psi \sequent \exists x \varphi \lor \exists x \psi\)\quad (ID)
	\item \(\exists x \varphi \sequent \exists x \varphi\)\quad (ID)
	\item \(\subst{\varphi}{a/x} \sequent \subst{\varphi}{a/x}\)\quad ((ID),ただし\(a\)は新しい変数記号)
	\item \(\subst{\varphi}{a/x} \sequent \subst{\varphi}{a/x} \lor \subst{\psi}{a/x}\)\quad (3から(\(\lor\)I)による)
	\item \(\exists x \varphi \sequent \exists x \paren{\varphi \lor \psi}\)\quad (2, 4から(\(\exists\)E)による)
	\item \(\exists x \psi \sequent \exists x \paren{\varphi \lor \psi}\)\quad (上と同様にして導けるので省略)
	\item \(\exists x \varphi \lor \exists x \psi \sequent \exists x \paren{\varphi \lor \psi}\)\quad (1, 5, 6から(\(\lor\)E)による)
\end{enumerate}

\(\forall x \paren{\varphi \lor \psi} \sequent \forall x \varphi \lor \psi\)の導出:
\begin{enumerate}
	\item \(\forall x \paren{\varphi \lor \psi} \sequent \forall x \paren{\varphi \lor \psi}\)\quad (ID)
	\item \(\lnot \paren{\forall x \varphi \lor \psi} \sequent \lnot \paren{\forall x \varphi \lor \psi}\)\quad (ID)
	\item \(\forall x \paren{\varphi \lor \psi} \sequent \subst{\varphi}{a/x} \lor \psi\)\quad (1から(\(\forall\)E)による,ただし\(a\)は新しい変数記号)
	\item \(\subst{\varphi}{a/x} \sequent \subst{\varphi}{a/x}\)\quad (ID)
	\item \(\psi \sequent \psi\)\quad (ID)
	\item \(\lnot \psi \sequent \lnot \psi\)\quad (ID)
	\item \(\lnot \subst{\varphi}{a/x}, \psi, \lnot \psi \sequent \bot\)\quad (5, 6から(w), (e), (\(\lnot\)E)による)
	\item \(\psi, \lnot \psi \sequent \lnot \lnot \subst{\varphi}{a/x}\)\quad (7から(\(\lnot\)I)による)
	\item \(\psi, \lnot \psi \sequent \subst{\varphi}{a/x}\)\quad (8から(DNF)による)
	\item \(\forall x \paren{\varphi \lor \psi}, \lnot \psi \sequent \subst{\varphi}{a/x}\)\quad (3, 4, 9から(\(\lor\)E)による)
	\item \(\forall x \paren{\varphi \lor \psi}, \lnot \psi \sequent \forall x \varphi\)\quad (10から(\(\forall\)I)による)
	\item \(\forall x \paren{\varphi \lor \psi}, \lnot \psi \sequent \forall x \varphi \lor \psi\)\quad (11から(\(\lor\)I)による)
	\item \(\lnot \psi, \lnot \paren{\forall x \varphi \lor \psi}, \forall x \paren{\varphi \lor \psi} \sequent \bot\)\quad (2, 12から(\(\lnot\)E)による)
	\item \(\lnot \paren{\forall x \varphi \lor \psi}, \forall x \paren{\varphi \lor \psi} \sequent \lnot \lnot \psi\)\quad (13から(\(\lnot\)I)による)
	\item \(\lnot \paren{\forall x \varphi \lor \psi}, \forall x \paren{\varphi \lor \psi} \sequent \psi\)\quad (14から(DNF)による)
	\item \(\lnot \paren{\forall x \varphi \lor \psi}, \forall x \paren{\varphi \lor \psi} \sequent \forall x \varphi \lor \psi\)\quad (15から(\(\lor\)I)による)
	\item \(\lnot \paren{\forall x \varphi \lor \psi}, \forall x \paren{\varphi \lor \psi} \sequent \bot\)\quad (2, 16から(w), (e), (\(\lnot\)E)による)
	\item \(\forall x \paren{\varphi \lor \psi} \sequent \lnot \lnot \paren{\forall x \varphi \lor \psi}\)\quad (17から(\(\lnot\)I)による)
	\item \(\forall x \paren{\varphi \lor \psi} \sequent \forall x \varphi \lor \psi\)\quad (18から(DNF)による)
\end{enumerate}
ここで,3の導出にあたっては,\(\psi\)に\(x\)が自由出現しないことにより
\(\subst{\paren{\varphi \lor \psi}}{a/x}\)と\(\subst{\varphi}{a/x} \lor \psi\)が同じ論理式であることを用いた.

\(\forall x \varphi \lor \psi \sequent \forall x \paren{\varphi \lor \psi}\)の導出:
\begin{enumerate}
	\item \(\forall x \varphi \lor \psi \sequent \forall x \varphi \lor \psi\)\quad (ID)
	\item \(\forall x \varphi \sequent \forall x \varphi\)\quad (ID)
	\item \(\forall x \varphi \sequent \subst{\varphi}{a/x}\)\quad (2から(\(\forall\)E)による,ただし\(a\)は新しい変数記号)
	\item \(\forall x \varphi \sequent \subst{\varphi}{a/x} \lor \psi\)\quad (3から(\(\lor\)I)による)
	\item \(\forall x \varphi \sequent \forall x \paren{\varphi \lor \psi}\)\quad (4から(\(\forall\)I)による)
	\item \(\psi \sequent \forall x \paren{\varphi \lor \psi}\)\quad (上と同様にして導けるので省略)
	\item \(\forall x \varphi \lor \psi \sequent \forall x \paren{\varphi \lor \psi}\)\quad (1, 5, 6から(\(\lor\)E)による)
\end{enumerate}
ここで,5の導出にあたっては,\(\psi\)に\(x\)が自由出現しないことにより
\(\subst{\paren{\varphi \lor \psi}}{a/x}\)と\(\subst{\varphi}{a/x} \lor \psi\)が同じ論理式であることを用いた.

\(\exists x \paren{\varphi \land \psi} \sequent \exists x \varphi \land \psi\)の導出:
\begin{enumerate}
	\item \(\exists x \paren{\varphi \land \psi} \sequent \exists x \paren{\varphi \land \psi}\)\quad (ID)
	\item \(\subst{\varphi}{a/x} \land \psi \sequent \subst{\varphi}{a/x} \land \psi\)\quad ((ID),ただし\(a\)は新しい変数記号)
	\item \(\subst{\varphi}{a/x} \land \psi \sequent \subst{\varphi}{a/x}\)\quad (2から(\(\land\)E)による)
	\item \(\subst{\varphi}{a/x} \land \psi \sequent \exists x \varphi\)\quad (3から(\(\exists\)I)による)
	\item \(\subst{\varphi}{a/x} \land \psi \sequent \psi\)\quad (2から(\(\land\)E)による)
	\item \(\subst{\varphi}{a/x} \land \psi \sequent \exists x \varphi \land \psi\)\quad (4, 5から(\(\land\)I), (c)による)
	\item \(\exists x \paren{\varphi \land \psi} \sequent \exists x \varphi \land \psi\)\quad (6から(\(\exists\)E)による)
\end{enumerate}
ここで,7の導出にあたっては,\(\psi\)に\(x\)が自由出現しないことにより
\(\subst{\paren{\varphi \land \psi}}{a/x}\)と\(\subst{\varphi}{a/x} \land \psi\)が同じ論理式であることを用いた.

\(\exists x \varphi \land \psi \sequent \exists x \paren{\varphi \land \psi}\)の導出:
\begin{enumerate}
	\item \(\exists x \varphi \land \psi \sequent \exists x \varphi \land \psi\)\quad (ID)
	\item \(\exists x \varphi \land \psi \sequent \psi\)\quad (2から(\(\land\)E)による)
	\item \(\exists x \varphi \land \psi \sequent \exists x \varphi\)\quad (1から(\(\land\)E)による)
	\item \(\subst{\varphi}{a/x} \sequent \subst{\varphi}{a/x}\)\quad ((ID),ただし\(a\)は新しい変数記号)
	\item \(\subst{\varphi}{a/x}, \exists x \varphi \land \psi \sequent \subst{\varphi}{a/x} \land \psi\)\quad (2, 4から(\(\land\)I)による)
	\item \(\subst{\varphi}{a/x}, \exists x \varphi \land \psi \sequent \exists x \paren{\varphi \land \psi}\)\quad (5から(\(\exists\)I)による)
	\item \(\exists x \varphi \land \psi \sequent \exists x \paren{\varphi \land \psi}\)\quad (3, 6から(\(\exists\)E), (c)による)
\end{enumerate}
ここで,6の導出にあたっては,\(\psi\)に\(x\)が自由出現しないことにより
\(\subst{\paren{\varphi \land \psi}}{a/x}\)と\(\subst{\varphi}{a/x} \land \psi\)が同じ論理式であることを用いた.


\subsection*{\Cref{Que:peirce}}

\begin{enumerate}
	\item \(\paren{\varphi \to \psi} \to \varphi \sequent \paren{\varphi \to \psi} \to \varphi\)\quad (ID)
	\item \(\lnot \varphi \sequent \lnot \varphi\)\quad (ID)
	\item \(\varphi \sequent \varphi\)\quad (ID)
	\item \(\lnot \psi, \varphi, \lnot \varphi \sequent \bot\)\quad (2, 3から(w), (e), (\(\lnot\)E)による)
	\item \(\varphi, \lnot \varphi \sequent \lnot \lnot \psi\)\quad (4から(\(\lnot\)I)による)
	\item \(\varphi, \lnot \varphi \sequent \psi\)\quad (5から(DNF)による)
	\item \(\lnot \varphi \sequent \varphi \to \psi\)\quad (6から(\(\to\)I)による)
	\item \(\lnot \varphi, \paren{\varphi \to \psi} \to \varphi \sequent \varphi\)\quad (1, 7から(\(\to\)E)による)
	\item \(\lnot \varphi, \paren{\varphi \to \psi} \to \varphi \sequent \bot\)\quad (2, 8から(w), (e), (\(\lnot\)E)による)
	\item \(\paren{\varphi \to \psi} \to \varphi \sequent \lnot \lnot \varphi\)\quad (9から(\(\lnot\)I)による)
	\item \(\paren{\varphi \to \psi} \to \varphi \sequent \varphi\)\quad (10から(DNF)による)
\end{enumerate}

\subsection*{\Cref{Que:equalsignrelation}}

まずは\cref{Thm:equalsign}を示す.
\(s_i, t_i\)に登場しない,つまり\(s_i, t_i\)に自由出現も束縛出現もしない変数記号\(x\)をとる.
論理式
\begin{equation*}
	\apply{f}{s_1, s_2, \dots, s_i, \dots, s_n} \objeq \apply{f}{s_1, s_2, \dots, x, \dots, s_n}
\end{equation*}
を\(\varphi\)とすると,\(\varphi\)中の\(x\)に\(s_i, t_i\)は代入可能である.このとき,論理式
\begin{equation*}
	\apply{f}{s_1, s_2, \dots, s_i, \dots, s_n} \objeq \apply{f}{s_1, s_2, \dots, s_i, \dots, s_n}
\end{equation*}
は\(\subst{\varphi}{s_i /x}\)
と書きかえられ,論理式\(\subst{\varphi}{t_i / x}\)は
\begin{equation*}
	\apply{f}{s_1, s_2, \dots, s_i, \dots, s_n} \objeq \apply{f}{s_1, s_2, \dots, t_i, \dots, s_n}
\end{equation*}
を意味する.
よって,シークエント
\begin{equation*}
	s_i \objeq t_i \sequent \subst{\varphi}{t_i / x}
\end{equation*}
を導出すれば定理の証明は完了する.
シークエント\(\sequent \subst{\varphi}{s_i / x}\)は(REFL)によって導出可能であることに注意しよう.
目的のシークエントの導出は下記で与えられる:
\begin{enumerate}
	\item \(s_i \objeq t_i \sequent s_i \objeq t_i\)\quad (ID)
	\item \(\sequent \subst{\varphi}{s_i / x}\)\quad (REFL)
	\item \(s_i \objeq t_i \sequent \subst{\varphi}{t_i / x}\)\quad (1, 2から(SUBST)による)
\end{enumerate}

次に\cref{Thm:equalsignrelation}を示そう.まずは\cref{eq:equalsignsymmetry}を導出する.
ここで\(s, t\)のどちらにも登場しない変数記号\(x\)をとり,論理式\(x \objeq s\)
を\(\varphi\)とすると,\cref{eq:equalsignsymmetry}は
\begin{equation*}
	s \objeq t \sequent \subst{\varphi}{t / x}
\end{equation*}
と書きかえられる.
このシークエントは以下のように導出される:
\begin{enumerate}
	\item \(s \objeq t \sequent s \objeq t\)\quad (ID)
	\item \(\sequent \subst{\varphi}{s/x}\)\quad (REFL)
	\item \(s \objeq t \sequent \subst{\varphi}{t/x}\)\quad (1, 2から(SUBST)による)
\end{enumerate}

最後に\cref{eq:equalsigntransitivity}を導出しよう.
ここでは,\(s, t, u\)のいずれにも登場しない変数記号\(x\)をとり,論理式\(x \objeq u\)を\(\varphi\)で表す.
すると,\cref{eq:equalsigntransitivity}は
\begin{equation*}
	\paren{s \objeq t} \land \paren{t \objeq u} \sequent \subst{\varphi}{s/x}
\end{equation*}
と書きかえられる.
このシークエントは以下のように導出される:
\begin{enumerate}
	\item \(\paren{s \objeq t} \land \paren{t \objeq u} \sequent \paren{s \objeq t} \land \paren{t \objeq u}\)\quad (ID)
	\item \(\paren{s \objeq t} \land \paren{t \objeq u} \sequent s \objeq t\)\quad (1から(\(\land\)E)による)
	\item \(\sequent s \objeq t \to t \objeq s\)\quad (\Cref{eq:equalsignsymmetry}と(\(\to\)I)による)
	\item \(\paren{s \objeq t} \land \paren{t \objeq u} \sequent t \objeq s\)\quad (2, 3から(\(\to\)E)による)
	\item \(\paren{s \objeq t} \land \paren{t \objeq u} \sequent \subst{\varphi}{t/x}\)\quad (1から(\(\land\)E)による)
	\item \(\paren{s \objeq t} \land \paren{t \objeq u} \sequent \subst{\varphi}{s/x}\)\quad (4, 5から(SUBST)による)
\end{enumerate}

\section*{\Cref{chap:advanced}}

\subsection*{\Cref{Que:completeness}}

対偶,すなわち\(\Gamma \provable \varphi\)でないと仮定して\(\Gamma \satisfy \varphi\)でないことを示す.
\(\Gamma \provable \varphi\)でないので,\(\Gamma \cup \Set{\lnot \varphi}\)が無矛盾となる.
実際,\(\Gamma \cup \Set{\lnot \varphi}\)が矛盾すると仮定すると(\(\lnot\)I)規則と
(DNF)規則により\(\Gamma \provable \varphi\)となる.
\Cref{Thm:HenkinTheorem}により,\(\Gamma \cup \Set{\lnot \varphi}\)のモデル\(\symcal{M}\)が存在する.
この\(\symcal{M}\)は\(\symcal{M} \satisfy \lnot \varphi\)を満たすので\(\symcal{M} \satisfy \varphi\)
は成り立たず,従って\(\Gamma \satisfy \varphi\)ではない.

\nocite{*}
\printbibliography[title=参考文献, heading=bibintoc]

\printindex[sidx]
\printindex[widx]

\newpage
\pagestyle{empty}
~
\newpage

\pagestyle{empty}
%\newpage
%~
%\newpage
\vspace*{\fill}
\noindent
\begin{picture}(110,1)
	\setlength{\unitlength}{1truemm}
	\put(15,3){\Large \textbf{0から始める数理論理学入門 第2版}}
	\thicklines
	\put(0,1){\line(2,0){110}}
	\thinlines
	\put(0,0){\line(2,0){110}}
\end{picture}

2023年12月 初版

2024年8月 第2版

著者:野口 匠

Twitter:@Nmatician

発行:NOGUTAKU Lab

印刷:株式会社ポプルス

\begin{picture}(110,1)
	\setlength{\unitlength}{1truemm}
	\thinlines
	\put(-3,1){\line(2,0){110}}
	\thicklines
	\put(-3,0){\line(2,0){110}}
\end{picture}

\end{document}

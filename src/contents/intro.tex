\chapter{はじめに}

数理論理学は,ゲーデルの不完全性定理の存在もあってか世間一般の認知度は比較的高い.
しかしながら,数理論理学の基本コンセプトである「論理を形式化してその性質を探っていく」という考え方は広く浸透しているとはいいがたい.
特にゲーデルの不完全性定理は,なんとなく日常用語として理解できそうな言い回しや字面のインパクトの強さからか
「ゲーデルは数学が万能とはなりえないことを証明した」だの「完璧な数学理論は存在しない」だのと意味不明な誤解が発生しがちである.
とはいえ,最近では「これは誤解である」という認識自体は広まっており,状況は改善しつつあるといえる.

一方で,いわゆる「通常の数学理論」と比較すると数理論理学はあまり広く学ばれてはいないのが現状である.
現在ではわかりやすい和書も多数出版されており,学ぶハードルはそれなりに低いけども,学んでいる人の数はほかの数学理論と比較すれば少ないと言わざるを得ないのではないだろうか.
実際,数学をそれなりに学んでいる人であっても数理論理学については何も知らないという人が少なくない.数学を専門とせず,道具として使っている物理学や統計学を学んでいる人であればなおさらである.

しかし,特に数学を学んでいる人にとっては「論理記号」については身近だと感じる人は多い.彼らにとって論理記号は普段使っている数学記号と同じく「なんらかの数学的主張を表現したもの」であり,
その認識のままで大きな問題が生じることはない.
むろん数理論理学を学んだ人からすればその認識は厳密には誤りである.
誤っている部分が数理論理学にとっても些事であればよいのだが,残念なことに論理を形式化するという数理論理学の基本的な方法について理解できていない致命的な誤りである.

本書はそのような誤解を払拭することを第一の目標とした.
すなわち「論理を形式化する」ことがいったいどういうことなのかを実感をもって学ぶことが目標である.
そのため,入門書で多く取り上げられているであろう数理論理学に関する結果の多くは取り上げない.
特に不完全性定理については取り上げないので,それをめあてにして本書を手に取るとがっかりするであろう.
その代わり,通常の数学理論との接点を多く紹介することで「論理の形式化」についての理解を深めたい.

とりわけ「\(A \land B\)は\(A\)かつ\(B\)という意味」\emph{ではない}というジャーゴンの意味が理解できれば目標達成である.
このことが理解できれば,数理論理学の世界に飛び込む準備は完全に整ったといってよい.要するに,本書は数理論理学入門の前段階という位置づけで活用するとよい.
本書の後,あるいは並行して読むことになるであろう1冊目の入門書としては前原\cite{maehara2005}や鹿島\cite{kashima2009}がおすすめである.
数理論理学入門としての色が強いのは後者であるが,やや難しいと感じた場合には前者を読んでみるのもよいだろう.
また,数理論理学が数学の一分野である以上,集合や写像といった「道具」の修得は欠かせない.本書でも説明抜きに用いている.
自信がない人は嘉田\cite{kada2008}を手に取るとよい.
戸次\cite{bekki2012}は複数の形式的体系に触れることができるという点において読む価値が高い.
また,ユークリッド幾何学の基礎づけについて気になる人は足立\cite{adachi2019}を,数の体系について知りたい人は田中\cite{tanaka2019}を読むとよい.

第2版では初版に比べて意味論に関して加筆し,完全性定理の証明まで記載した.
また,Herbrand構造や定義による拡大について記載し,それらの通常の数学との接点について紹介した.
これにより,数理論理学の入門書としては最低限使えるものになったのではないかと考えられる.
また,構造帰納法に関しては他の本よりも圧倒的に詳しく記載している.
数理論理学においては構造帰納法は基本的で重要な道具ではあるが,その正当化を含めて詳細に解説されることはあまりない.
この点において,本書は独自の価値を提供できていると考えられる.

本書の原稿やソースファイルは,以下のGitHubリポジトリにて閲覧可能である.

\begin{center}
	\url{https://github.com/enunun/introductiontomathmaticallogic}
\end{center}

執筆時間の関係で書ききれなかった内容については,ここに随時追加予定である.

\begin{flushright}
	2024年11月15日
\end{flushright}
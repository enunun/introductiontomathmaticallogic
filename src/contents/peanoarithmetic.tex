\chapter{1階Peano算術}
\label{chap:peanoarithmetic}

本章より前の章では1階述語論理という演繹体系そのものの性質を学ぶことを目的に,
対象となる理論を限定せずに議論を進めてきた.
本章からはGödelの不完全性定理について学ぶことを目的に,
1階Peano算術という理論に焦点を当てて議論を進める.

まずは議論の対象となる1階Peano算術やその部分体系を定義し,
その証明能力を調べる.
形式体系に関する議論ではあるが,議論の中で我々が素朴に思い浮かべる
自然数全体の集合\(\NaturalNumbers\)が多岐にわたって登場する.
その際,形式体系についての議論なのか\(\NaturalNumbers\)についての議論なのかを混同しないように注意されたい.

特に,数学的帰納法については形式体系上の数学的帰納法と\(\NaturalNumbers\)上の通常の数学的帰納法の両方が同時に登場する.
数学的帰納法は算術の体系の証明能力を決定づけるといっても過言ではなく,
制限を加えたり公理から取り除いたりする.そのため,数学的帰納法を適用する際は,
形式体系上の数学的帰納法なのか\(\NaturalNumbers\)上の通常の数学的帰納法なのかを注意深く区別しながら読み進めるとよい.

\section{1階Peano算術とその部分体系}
\label{sec:peanoarithmetic}

まずは,Peano算術を1階理論として定義しよう.

\index[sidx]{\(\PA\):1階Peano算術}
\index[sidx]{\(\symcal{L}_{\PA}\):1階Peano算術の言語}
\index[widx]{1かいPeanoさんじゅつ@1階Peano算術}
\begin{Def} \label{Def:peanoarithmetic}
	1階Peano算術の言語を\(\symcal{L}_{\PA} = \Set{\obj{+}, \obj{\cdot}, \obj{0}, \obj{1}, \obj{<}}\)とする.
	ここで,\(\obj{+}, \obj{\cdot}\)はアリティ2の関数記号,\(\obj{0}, \obj{1}\)は定数記号,\(\obj{<}\)はアリティ2の関数記号である.
	\term{1階Peano算術}\(\PA\)は,以下の公理からなる\(\symcal{L}_{\PA}\)理論である:
	\begin{description}
		\item[A1.] \(\forall \obj{x} \lnot \paren{\obj{x} \obj{+} \obj{1} \objeq \obj{0}},\)
		\item[A2.] \(\forall \obj{x} \forall \obj{y} \paren{\obj{x} \obj{+} \obj{1} \objeq \obj{y} \obj{+} \obj{1} \to \obj{x} \objeq \obj{y}},\)
		\item[A3.] \(\forall \obj{x} \paren{\obj{x} \obj{+} \obj{0} \objeq \obj{x}},\)
		\item[A4.] \(\forall \obj{x} \forall \obj{y} \paren{\obj{x} \obj{+} \paren{\obj{y} + \obj{1}} \objeq \paren{\obj{x} \obj{+} \obj{y}} \obj{+} \obj{1}},\)
		\item[A5.] \(\forall \obj{x} \paren{\obj{x} \obj{\cdot} \obj{0} \objeq \obj{0}},\)
		\item[A6.] \(\forall \obj{x} \forall \obj{y} \paren{\obj{x} \obj{\cdot} \paren{\obj{y} \obj{+} \obj{1}} \objeq \paren{\obj{x} \obj{\cdot} \obj{y}} \obj{+} \obj{1}},\)
		\item[A7.] \(\forall \obj{x} \lnot \paren{\obj{x} \obj{<} \obj{0}},\)
		\item[A8.] \(\forall \obj{x} \forall \obj{y} \paren{\obj{x} \obj{<} \obj{y} \obj{+} \obj{1} \formulaequiv \obj{x} \obj{<} \obj{y} \lor \obj{x} \objeq \obj{y}},\)
		\item[A9.] 変数記号\(x\)が束縛出現しない\(\symcal{L}_{\PA}\)論理式\(\apply{\varphi}{x}\)に対する論理式
		      \begin{equation}
			      \apply{\varphi}{\obj{0}} \land \forall x \paren{\apply{\varphi}{x} \to \apply{\varphi}{x \obj{+} \obj{1}}} \to \forall x \apply{\varphi}{x}
			      \label{eq:inductionscheme}
		      \end{equation}
		      の全称閉包すべて.
	\end{description}
\end{Def}

\(\PA\)の各公理の役割を述べておこう.まず,(A1), (A2)は通常の数学における「次の数」に相当する「\(\obj{x} + \obj{1}\)」に関する公理である.
(A3), (A4)は加法に関する公理であり,通常の数学における加法の帰納的定義に相当するものである.
(A5), (A6)と(A7), (A8)は,それぞれ通常の数学における乗法と大小関係の帰納的定義に相当する公理である.
(A1)から(A8)までは\(\symcal{L}_{\PA}\)の各記号の定義ともいえる公理であるが,(A9)だけは特異的である.
意味的には数学的帰納法に相当するものであるが,(A9)は単一の公理ではなく可算無限個ある論理式すべてに対する公理をまとめて書いた%
\index[widx]{こうり@公理!こうりずしき@---図式}%
\term{公理図式}である.
この公理図式のために,\(\PA\)は有限個の論理式からなる理論ではなく可算無限個の公理からなる理論になっている.

\begin{Note}
	ここで定義した理論\(\PA\)は1階Peano算術と呼ばれるものであるが,
	\(\PA\)は集合\(N\)と\(N\)の元\(o\),そして写像\(S \colon N \to N\)の対\(\pair{N, o, S}\)がPeano構造であるための条件を記述した
	いわゆる「Peanoの公理」とは異なるものである.
	いわゆる「Peanoの公理」はその記述に写像や部分集合といった集合のことばが使われており,
	そのままでは1階理論として表現することはできない.
	しかも,書き方の違いだけかといえばそうではない.質的に大きく異なる点として,例えば以下の2つを挙げることができる:
	\begin{enumerate}
		\item いわゆる「Peanoの公理」はPeano構造を特徴づける公理系であるが,
		      \(\PA\)は通常の意味での自然数全体の集合\(\NaturalNumbers\)(に種々の構造を入れたもの)を特徴づける公理系ではない.
		\item Peano構造上の加法は0と後者関数(「次の数」をとる関数)だけから定義できるが,1階の理論ではそれは不可能である.
	\end{enumerate}

	このうち,2.について補足しておく.アリティ2の関係記号\(\obj{<}\)とアリティ1の関数記号\(\obj{S}\)からなる言語\(\symcal{L} = \Set{\obj{<}, \obj{S}}\)を考える.
	有限順序数全体の集合%
	\footnote{%
		素朴には自然数全体の集合だととらえて差し支えない.%
	}%
	\(\omega\)に対し,\(\omega\)上の2項関係\(<\)と写像\(S \colon \omega \to \omega\)を以下のように定める:
	\begin{align*}
		\alpha < \beta \metaequivalent \alpha \in \beta, \\
		\apply{S}{\alpha} = \alpha \cup \Set{\alpha}.
	\end{align*}
	\(\obj{<}, \obj{S}\)の解釈をそれぞれ上で定義した\(\mathord{<}, S\)と定めることで,
	\(\omega\)を対象領域とする\(\symcal{L}\)構造\(\symcal{N}\)を定義することができる.
	このとき,任意の\(\alpha, \beta, \gamma \in \omega\)に対して以下を満たす\(\symcal{L}\)論理式\(\apply{\varphi}{x, y, z}\)は存在しないことが知られている:
	\[
		\symcal{N} \satisfy \apply{\varphi}{\alpha, \beta, \gamma} \metaequivalent \alpha + \beta = \gamma.
	\]
	なお,有限順序数の加法については\(\PA\)における(A3), (A4)と同じようにして帰納的に定義できる.

	用語としての「Peanoの公理」は一般の知名度自体はそれなりに高いものの,それゆえかかなり雑に使われがちなようである.
	本書ではPeano構造については取り扱わないが,世間的な「Peanoの公理」のイメージに惑わされて1階Peano算術\(\PA\)とPeano構造とを混同しないように注意されたい.
\end{Note}

\(\PA\)の公理図式(A9)は,(変数記号\(x\)が束縛出現しない)すべての\(\symcal{L}_{\PA}\)論理式に対して要請されるものである.
ここに制限を加えることによって\(\PA\)の部分体系を得ることを考えよう.
まず最初に思いつくのは\(\PA\)から公理図式(A9)を取り除いた体系である.

\index[sidx]{\(\Robinson\):Robinson算術}
\index[widx]{Robinsonさんじゅつ@Robinson算術}
\begin{Def} \label{Def:robinsonarithmetic}
	\term{Robinson算術}\(\Robinson\)は,\(\PA\)における公理(A1)から(A8)に以下の公理を加えて得られる\(\symcal{L}_{\PA}\)理論である:
	\begin{description}
		\item[A10.] \(\forall \obj{x} \paren{\obj{x} \objeq \obj{0} \lor \exists \obj{y} \paren{\obj{x} \objeq \obj{y} \obj{+} \obj{1}}},\)
	\end{description}
\end{Def}

\begin{Lemma} \label{lemma:paa10redudant}
	\(\PA\)においては(A10)は証明できる.すなわち,
	\[
		\PA \provable \text{(A10)}
	\]
	が成り立つ.
\end{Lemma}

\begin{proof}
	\(\obj{x}\)を変数記号とし,論理式\(\apply{\varphi}{\obj{x}}\)を\(\obj{x} \objeq \obj{0} \lor \exists \obj{y} \paren{\obj{x} \objeq \obj{y} \obj{+} \obj{1}}\)と定める.
	\(\symcal{M}\)を\(M\)を対象領域とする\(\PA\)の任意のモデルとする.\(\symcal{M} \satisfy \apply{\varphi}{\obj{0}}\)は明らか.
	\(x \in M\)を任意にとり,その名前を\(c_x\)とする.
	\(\symcal{M} \satisfy \exists \obj{y} \paren{c_x \obj{+} \obj{1} \objeq \obj{y} \obj{+} \obj{1}}\)だから
	\(\symcal{M} \satisfy \apply{\varphi}{c_x \obj{+} \obj{1}}\)であり,
	\[
		\symcal{M} \satisfy \apply{\varphi}{\obj{0}} \land \forall \obj{x} \paren{\apply{\varphi}{\obj{x}} \to \apply{\varphi}{\obj{x} \obj{+} \obj{1}}}
	\]
	となる.ゆえに(A9)から\(\symcal{M} \satisfy \forall \obj{x} \apply{\varphi}{\obj{x}}\)を得る.従って完全性定理から\(\PA \provable \text{(A10)}\)となる.
\end{proof}

\begin{Note}
	\Cref{lemma:paa10redudant}の証明のように,形式体系において具体的な定理を証明する際は完全性定理を使ったモデル経由の証明の方が書きやすい.
	表記を簡素化するため,以下のように記述を省略して書くことにする:
	\begin{enumerate}
		\item 構造\(\symcal{M}\)の対象領域\(M\)の元\(x \in M\)の名前をそのまま\(x\)と表す.
		\item 構造\(\symcal{M}\)において,タイプライタ体で書かれた各記号の解釈はその記号をタイプライタ体でない通常の書体で表すことによって表現する.
		      例えば,\(\symcal{L}_{\PA}\)の定数記号\(\obj{0}\)の解釈は\(0\)と表現する.
		\item 構造は\(\symcal{M}\)は表に出さず,\(\symcal{M} \satisfy \varphi\)であることを「\(\varphi\)が成り立つ」などのように表記する.
	\end{enumerate}
\end{Note}



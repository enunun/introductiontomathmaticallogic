\chapter{演習問題解答} \label{chap:answer}

\section*{\Cref{chap:formulize}}

\subsection*{\Cref{Que:termexample}} \label{answer:termexample}

\(\obj{e}\)は変数記号,\(\obj{x}\)は変数記号なので,どちらも\(\symcal{L}_1\)項である.
これと\(\obj{\ast}\)がアリティ2の関数記号であることにより,\(\apply{\obj{\ast}}{\obj{e}, \obj{x}}\)が
\(\symcal{L}_1\)項であることがわかる.
\(\obj{y}\)も変数記号であり\(\symcal{L_1}\)項なので,
\(\apply{\obj{\ast}}{\apply{\obj{\ast}}{\obj{e}, \obj{x}}, \obj{y}}\)は\(\symcal{L}_1\)項である.

また,\(\obj{e}\apply{\mathord{\obj{\ast}}}{\obj{e}, \obj{e}}\)は変数記号でも定数記号でもないため,
これが\(\symcal{L}_1\)項であるためには
\cref{Def:term}における3番目の規則を最後に適用していなければならない.
このとき最初の文字は関数記号である必要があるが,\(\obj{e}\)は定数記号であり関数記号ではない.
従って,\(\obj{e}\apply{\mathord{\obj{\ast}}}{\obj{e}, \obj{e}}\)は\(\symcal{L}_1\)項ではない.

\subsection*{\Cref{Que:logicalexpression}}

\(\obj{x}, \obj{y}\)は変数記号であり,従って\(\symcal{L}_1\)項である.
また,\(\obj{\ast}\)がアリティ2の関数記号であることから,
\(\apply{\mathord{\obj{\ast}}}{\obj{x}, \obj{y}}\)と
\(\apply{\mathord{\obj{\ast}}}{\obj{y}, \obj{x}}\)はいずれも\(\symcal{L}_1\)項である.
よって,
\(\paren{\apply{\mathord{\obj{\ast}}}{\obj{y}, \obj{x}} = \apply{\mathord{\obj{\ast}}}{\obj{y}, \obj{x}}}\)
は\(\symcal{L}_1\)論理式である.

\(\obj{\leq}\)がアリティ2の関係記号であることから,
\(\apply{\mathord{\obj{\leq}}}{\obj{x}, \obj{y}}\)は\(\symcal{L}_2\)論理式である.
ゆえに,\(\paren{\forall \obj{y}\apply{\mathord{\obj{\leq}}}{\obj{x}, \obj{y}}}\)
は\(\symcal{L}_2\)論理式である.
このことから\(\paren{\forall \obj{x}\paren{\forall \obj{y}\apply{\mathord{\obj{\leq}}}{\obj{x}, \obj{y}}}}\)
が\(\symcal{L}_2\)論理式であることも従う.

さて,\(\paren{\apply{\exists \obj{x}}{\obj{x}}}\)が\(\symcal{L}\)論理式であるとすれば,
\(\obj{x}\)は\(\symcal{L}\)論理式でなければならない.しかし,変数記号が論理式であることはないので
\(\paren{\apply{\exists \obj{x}}{\obj{x}}}\)は\(\symcal{L}\)論理式ではない.
一方で,\(\paren{\obj{x} = \obj{x}}\)は\(\symcal{L}\)論理式であるから
\(\paren{\apply{\exists \obj{x}}{\obj{x} = \obj{x}}}\)は\(\symcal{L}\)論理式である.

\(\apply{\forall \obj{x}}{\obj{x} \obj{\ast} \obj{e} = \obj{x}}\)が\(\symcal{L}_1\)論理式であるとすれば,
これまでの議論と同様にして\(\obj{x} \obj{\ast} \obj{e}\)が\(\symcal{L}_1\)項でなければならないことがわかる.
しかし,\(\obj{\ast}\)の次の文字が開きカッコ「\(\lparen\)」ではないので
これは\(\symcal{L}_1\)項とはなりえない.よって,
\(\apply{\forall \obj{x}}{\obj{x} \obj{\ast} \obj{e} = \obj{x}}\)は\(\symcal{L}_1\)論理式ではない.
\(\apply{\forall \obj{x}}{\obj{x} = \obj{x}}\)については,1文字目が開きカッコ「\(\lparen\)」でも
関係記号でもないことから\(\symcal{L}_1\)論理式でないことが従う.

\subsection*{\Cref{Que:invalidtheory}}

2つ目の\(\symcal{L}\)論理式からは,\(\obj{e}\)が変数記号であることが伺える.
しかしそうだとすると最後の\(\symcal{L}\)論理式
\[
	\forall \obj{x} \exists \obj{y} \paren{\obj{y} \obj{\ast} \obj{x} = \obj{e}}
\]
が\(\symcal{L}\)閉論理式にならない.問題文で挙げた3つの論理式からなる集合が
\(\symcal{L}\)理論になっていないという意味で構文的に不適切である.

最後の\(\symcal{L}\)論理式を
\[
	\exists \obj{e} \forall \obj{x} \exists \obj{y} \paren{\obj{y} \obj{\ast} \obj{x} = \obj{e}}
\]
としてしまえば\(\symcal{L}\)閉論理式になるので,問題文で挙げた3つの\(\symcal{L}\)論理式のうち
3つ目だけをこちらに置き換えたものは\(\symcal{L}\)理論となる.
しかし,この\(\symcal{L}\)理論は「群論の公理化」としては不適切であって,
2つ目と3つ目の\(\symcal{L}\)論理式をまとめて
\[
	\exists \obj{e} \paren{
		\forall \obj{x} \paren{\obj{e} \obj{\ast} \obj{x} = \obj{x}}
		\land \forall \obj{x} \exists \obj{y} \paren{\obj{y} \obj{\ast} \obj{x} = \obj{e}}
	}
\]
とするのが正しい.
なぜそうしなければならないのか,
そしてなぜそうすれば「よい」のかについては,この段階ではわからない.